\documentclass{article}
\usepackage{mstyle}
\usepackage{pgfplots}
\usetikzlibrary{intersections, pgfplots.fillbetween}

\title{\begin{figure}[t!]
    \centering
    \includegraphics[trim=0 55 0 60, clip, width=0.3\textwidth]{Stemma_unipi.jpg}
  \end{figure}
  \vspace{-17.5mm}
  \textsc{\Large Università di Pisa}\\
  \textsc{\large Corso di Laurea in Matematica}\\
  \,\\
  {\large Tesi di Laurea Magistrale}\\
  \,\\
  \,\\
  Teoremi di tipo ``Wolff-Denjoy'' in più variabili complesse\\
  \,\\
  \,}
  \author{Candidato:  \hspace{215px} Relatore:\\
  \textbf{Marco Vergamini} \hfill Prof. \textbf{Marco Abate}}
  \date{}
  
  \begin{document}
  \maketitle
  \vspace*{\fill}
  \begin{center}
    22 Settembre 2023
    \par\noindent\rule{\textwidth}{0.5pt}
    \Large Anno Accademico 2022/2023
  \end{center}
  \thispagestyle{empty}
  \newpage
  \tableofcontents
  \newpage


\section*{Introduzione}
\addcontentsline{toc}{section}{Introduzione}
L'obiettivo di questa tesi è dimostrare alcune possibili generalizzazioni, in più variabili complesse o anche per varietà complesse astratte che soddisfano opportune ipotesi, del teorema di Wolff-Denjoy, dimostrato indipendentemente nel 1926 da Denjoy in \cite{D} e da Wolff in \cite{Wo}. Riportiamo l'enunciato di tale teorema.

\begin{thm}
    (Wolff-Denjoy) Sia $f$ una funzione olomorfa nel disco unitario in $\mathbb{C}$ a valori nel disco stesso. Allora vale esattamente una delle seguenti affermazioni:
    \begin{itemize}
        \item la funzione $f$ ha un punto fisso nel disco; oppure,
        \item esiste un unico punto del bordo del disco tale che la successione delle iterate di $f$ converge, uniformemente sui compatti, a quel punto.
    \end{itemize}
\end{thm}

Era già nota da tempo la generalizzazione, dovuta ad Abate (\cite[Theorem 0.5]{A2}), per domini limitati e strettamente pseudoconvessi in più variabili.\\
SCALETTA PER DOMANI (POSSIBILE INIZIO: "Più recentemente, ..."):\\
- PARLARE DI BZ1;\\
- PARLARE DI BM;\\
- PARLARE DI CMS;\\
- RIASSUMERE COSA VIENE FATTO NELLA TESI (CERCARE DI CAPIRE SE VERRANNO FATTI I DOMINI ILLIMITATI E LE ESTENSIONI AL BORDO).

\newpage

\section{Preliminari} \label{Preliminari}
\subsection{Notazioni e definizioni di base}
Introduciamo la notazione che useremo:
\begin{itemize}
    \item scriviamo $\Omega$ per indicare un \textit{dominio} di $\mathbb{C}^n$, vale a dire un aperto connesso;
    \item con \textit{varietà complessa} s'intende una varietà differenziabile reale di dimensione pari con i cambi di carta olomorfi se visti come fra aperti di $\mathbb{C}^n$;
    \item data $X$ varietà complessa e $x \in X$, indichiamo con $T_xX$ lo spazio tangente a $X$ in $x$, che nel caso dei domini è canonicamente identificato con $\mathbb{C}^n$;
    \item dati $X,Y$ spazi topologici, quando parliamo di convergenza nell'insieme $C^0(X,Y)$ delle funzioni continue da $X$ a $Y$ sottintendiamo sempre che si parla della topologia compatta-aperta, che nel caso in cui $Y$ sia uno spazio metrico coincide con la topologia della convergenza uniforme sui compatti (potrebbe capitare che commetteremo abusi di notazione in merito);
    \item date $X,Y$ varietà complesse, indichiamo con $\text{Hol}(X,Y)$ l'insieme delle funzioni olomorfe da $X$ a $Y$, con $\mathcal{O}(X)$ l'insieme delle funzioni olomorfe da $X$ in $\mathbb{C}$ e con $\text{Aut}(X)$ l'insieme delle funzioni biolomorfe da $X$ in sé;
    \item data $f \in \text{Hol}(X,Y)$, indichiamo con $Df(x)$ il differenziale di $f$ in $x \in X$;
    \item il disco unitario è $\mathbb{D}=\{z \in \mathbb{C} \mid |z|<1\}$, mentre $\mathbb{D}^n$ è il polidisco in $\mathbb{C}^n$ e $\mathbb{D}_r=\{z \in \mathbb{C} \mid |z|<r\}$ è il disco di raggio $r>0$;
    \item la palla unitaria (euclidea) in $\mathbb{C}^n$ è $\mathbb{B}^n=\{z \in \mathbb{C}^n \mid \|z\|<1\}$, dove $\|\cdot\|$ indica la norma euclidea, mentre $\mathbb{B}_r^n=\{z \in \mathbb{C}^n \mid \|z\|<r\}$ è la palla (euclidea) di raggio $r>0$;
    \item dato un dominio $\Omega\subseteq\mathbb{C}^n$ e $x \in \mathbb{C}^n$, scriviamo $\delta(x)=\displaystyle\inf_{w \in \partial\Omega}\|x-w\|$ per indicare la distanza euclidea di $x$ dal bordo di $\Omega$;
    \item ALTRO?
\end{itemize}

Ricordiamo cosa sono la metrica e la distanza di Poincaré in $\mathbb{D}$.

\begin{defn}
    La \textit{metrica di Poincaré} (o \textit{iperbolica}) su $\mathbb{D}$ è data da
    \begin{equation}
        K_{\mathbb{D}}(z;v)=\frac{1}{1-|z|^2}|v|
    \end{equation}
    per ogni $z \in \mathbb{D}$ e $v \in \mathbb{C}\cong T_z^{\mathbb{C}}\mathbb{D}$. La metrica $K_{\mathbb{D}}$ è hermitiana completa con curvatura gaussiana costante uguale a $-4$.
\end{defn}

\begin{defn}
    La \textit{distanza di Poincaré} (o \textit{iperbolica}) $k_{\mathbb{D}}$ su $\mathbb{D}$ è la forma integrata della metrica di Poincaré. Per fatti noti di geometria iperbolica, è una distanza completa la cui espressione è data da
    \begin{equation}
        k_{\mathbb{D}}(z_1,z_2)=\frac{1}{2}\log{\frac{1+\left|\frac{z_1-z_2}{1-\bar{z}_1z_2}\right|}{1-\left|\frac{z_1-z_2}{1-\bar{z}_1z_2}\right|}}
    \end{equation}
    per ogni $z_1,z_2 \in \mathbb{D}$.
\end{defn}

Oltre alla curvatura negativa costante, la metrica e la distanza di Poincaré sono tali che le funzioni olomorfe dal disco unitario in sé sono semicontrazioni rispetto ad esse (lemma di Schwarz-Pick, si veda ad esempio \cite[Chapter I, paragraph 1, Theorem 1.1]{K1}).

Quello che vogliamo fare ora è generalizzare la metrica e la distanza di Poincaré ad una qualsiasi varietà complessa mantenendo queste proprietà, in particolare quella di rendere le funzioni olomorfe delle semicontrazioni. Ci sono vari modi per farlo, noi nello specifico vedremo la (pseudo)metrica e la (pseudo)distanza di Kobayashi, introdotte nel 1967 in \cite{K2}.

\begin{defn}
    Sia $X$ una varietà; la \textit{pseudometrica di Kobayashi} su $X$ è
    \begin{equation}\begin{split}
        K_X(x;Z)=&\inf\{|v| \mid v \in \mathbb{C}, \text{ esiste }f \in \text{Hol}(\mathbb{D},X) \\
        &\text{ tale che } f(0)=x, Df(0)v=Z\}
    \end{split}\end{equation}
    per ogni $x \in X$ e $Z \in T_xX$.
\end{defn}

\begin{oss}
    Non possiamo parlare di metrica perché, ad esempio, $K_{\mathbb{C}^n}\equiv 0$; vedremo però tra poco che per i domini limitati è effettivamente una metrica. Notiamo anche che, data $f \in \text{Hol}(X,Y)$, allora dalla definizione segue che $K_Y\big(f(x);Df(x)Z\big) \le K_X(x;Z)$ per ogni $x \in X$ e $Z \in T_xX$.
\end{oss}

Definiamo adesso la (pseudo)distanza di Kobayashi; più avanti vedremo (SOLO CIT DEL RISULTATO ORIGINALE O ANCHE DIM?) com'è collegata alla pseudometrica di Kobayashi.

\begin{defn}
    Sia $X$ una varietà complessa connessa; la \textit{pseudodistanza di Kobayashi} su $X$ è data da
    \begin{equation}\begin{split}
        k_X(z,w)=&\inf\Bigg\{\sum_{j=1}^m k_{\mathbb{D}}(z_{j-1},z_j) \mid m \in \mathbb{N}, z_j \in \mathbb{D}\text{ per }j=0,\dots,m\text{ tali che}\\
        &\text{esistono }\varphi_j\in \text{Hol}(\mathbb{D},X) \text{ con } \varphi_1(z_0)=z,\varphi_m(z_m)=w\Bigg\}
    \end{split}\end{equation}
    per $z,w \in X$, dove $k_{\mathbb{D}}$ è la distanza di Poincaré.
\end{defn}

\begin{oss} \label{k_lip}
    È facile vedere che $k_X$ è una pseudodistanza, ma in generale non è una distanza, ad esempio perché, come prima, $k_{\mathbb{C}^n}\equiv 0$. Inoltre, data $f \in \text{Hol}(X,Y)$, dalla definizione segue che $k_Y\big(f(x),f(y)\big) \le k_X(x,y)$ per ogni $x,y \in X$.
\end{oss}

\begin{defn}
    Sia $X$ una varietà complessa connessa. Se $k_X$ è una distanza, diremo che $X$ è \textit{Kobayashi-iperbolica}.
\end{defn}

Il seguente risultato per le varietà Kobayashi-iperboliche verrà spesso usato implicitamente.

\begin{prop}
    (INSERIRE CIT) Una varietà complessa connessa $X$ è Kobayashi-iperbolica se e solo se $k_X$ induce su $X$ la topologia di varietà.
\end{prop}

Diamo ora delle definizioni che ci serviranno per enunciare i risultati già noti nel caso dei domini regolari.

\begin{defn}
    Una funzione continua $\mu:\mathbb{C}^n \longrightarrow [0,+\infty)$ è detta \textit{funzionale di Minkowski} se
    \begin{nlist}
        \item $\mu(Z)=0$ se e solo se $Z=0$;
        \item $\mu(\zeta Z)=|\zeta|\mu(Z)$ per ogni $Z \in \mathbb{C}^n$ e $\zeta\in\mathbb{C}$.
    \end{nlist}
    Dato $\Omega\subseteq\mathbb{C}^n$ un dominio, poniamo $\mu_{\Omega}(Z)=\inf_{w\in\mathbb{C}^n\setminus\Omega}\mu(Z-w)$.
\end{defn}

\begin{defn}
    Sia $A\subseteq\mathbb{C}$ un aperto. Una funzione $u:A \longrightarrow \mathbb{R}\cup\{-\infty\}$ semicontinua superiormente è detta \textit{subarmonica} se per ogni $a \in A$, per ogni $r>0$ tale che $\overline{D(a,r)} \subset A$ e per ogni $h$ continua in $\overline{D(a,r)}$ e armonica in $D(a,r)$, se $h\restrict{\partial D(a,r)} \ge u\restrict{\partial D(a,r)}$, allora anche $h\restrict{D(a,r)} \ge u \restrict{D(a,r)}$.

    Sia $A\subseteq\mathbb{C}^n$ un aperto. Una funzione $u:A \longrightarrow \mathbb{R}\cup\{-\infty\}$ semicontinua superiormente è detta \textit{plurisubarmonica} se per ogni $a \in A$ e per ogni $Z \in \mathbb{C}^n$ l'applicazione $\zeta \longmapsto u(a+\zeta Z)$ è subarmonica dove definita.
\end{defn}

\begin{defn}
    Un dominio $\Omega \subseteq \mathbb{C}^n$ si dice \textit{(Hartogs) pseudoconvesso} se esiste un funzionale di Minkowski $\mu$ tale che $-\log\mu_{\Omega}$ è plurisubarmonica in $\Omega$.
\end{defn}

Nel caso di domini regolari, si può dare una definizione di pseudoconvessità più operativa equivalente.

\begin{defn}
    Sia $\Omega \subseteq \mathbb{C}^n$ un dominio con bordo $C^2$, cioè esiste $\rho \in C^2(\mathbb{C}^n)$ tale che $\Omega=\{\rho(z)<0\}$ e $\diff\rho\not=0$ in ogni punto di $\partial\Omega$.
    
    Dato $p \in \partial\Omega$, lo \textit{spazio tangente complesso} a $\partial\Omega$ in $p$ è
    \begin{equation}
        H_p\partial\Omega=\{Z \in \mathbb{C}^n \mid \langle \bar{\partial}\rho(p),Z\rangle=0\}.
    \end{equation}
    Diciamo che $\Omega$ è \textit{Levi pseudoconvesso} se la \textit{forma di Levi}
    \begin{equation}
        L_{\rho}(p;Z)=\sum_{\nu,\mu=1}^n \frac{\partial^2\rho}{\partial z_\nu\partial\bar{z}_\mu}(p)Z_\nu\bar{Z}_\mu, \quad Z=(Z_1,\dots,Z_n) \in \mathbb{C}^n
    \end{equation}
      è semidefinita positiva in $H_p\partial\Omega$ per ogni $p \in \partial\Omega$. Diciamo che è \textit{strettamente pseudoconvesso} se la forma di Levi è definita positiva.
\end{defn}

Vale il seguente risultato. (INSERIRE CIT, POI: MEGLIO CITARLO COME FATTO O COME TEOREMA?)
\begin{ftt}
    Sia $\Omega \subseteq \mathbb{C}^n$ un dominio limitato con bordo $C^2$. Allora $\Omega$ è Levi pseudoconvesso se e solo se è Hartogs pseudoconvesso.
\end{ftt}

Nella prossima sottosezione citeremo alcuni risultati sulla geometria dei domini limitati strettamente pseudoconvessi dotati della distanza di Kobayashi. In particolare, vedremo che sono Gromov-iperbolici, il che permette di derivare un teorema di tipo Wolff-Denjoy per questi domini.

\begin{defn}
    Sia $(X,d)$ uno spazio metrico proprio, cioè tale che ogni sottoinsieme chiuso e limitato è compatto. Dati $x,y,w \in X$ il \textit{prodotto di Gromov} tra $x$ e $y$ con punto base $w$ è $(x,y)_w=\frac{1}{2}\big(d(x,w)+d(y,w)-d(x,y)\big)$. Dato $\delta \ge 0$, diciamo che $X$ è \textit{$\delta$-iperbolico} se
    $$(x,y)_w \ge \min\{(x,z)_w,(y,z)_w\}-\delta\text{ per ogni }x,y,z,w \in X.$$
    Se $(X,d)$ è $\delta$-iperbolico per qualche $\delta \ge 0$, diremo che è \textit{Gromov-iperbolico}.

    (SERVE DAVVERO LA DEF DEL BORDO?) Fissato $w \in X$, il \textit{bordo iperbolico} $\partial_GX$ è costruito come classi di equivalenza delle successioni $(x_i)$ che convergono a infinito, cioè tali che $\displaystyle\lim_{i,j\rightarrow \infty}(x_i,x_j)_w=\infty$; due tali successioni $(x_i), (y_i)$ sono equivalenti se $\displaystyle\lim_{i\rightarrow \infty}(x_i,y_i)_w=\infty$. Questa costruzione è indipendente dalla scelta di $w$.
\end{defn}

Prima di passare a vedere i risultati noti della teoria sulla pseudometrica e la pseudodistanza di Kobayashi e sui domini strettamente pseudoconvessi, introduciamo il concetto di varietà taut, che sarà per noi un'ipotesi importante per i teoremi che andremo a dimostrare: infatti, quest'ipotesi ci darà la dicotomia nella tesi del teorema. Vedremo anche con un esempio l'importanza di tale ipotesi. Prima di dare la definizione, ci servirà un risultato sul comportamento delle funzioni olomorfe a valori in una varietà Kobayashi-iperbolica.

\begin{prop}
    (INSERIRE CIT) Sia $X$ una varietà complessa connessa. Allora $X$ è Kobayashi-iperbolica se e solo se $\text{\normalfont{Hol}}(\mathbb{D},X)$ è relativamente compatto in $C^0(\mathbb{D},X^*)$, dove $X^*$ è la compattificazione di Alexandroff di $X$. In tal caso, $\text{\normalfont{Hol}}(Y,X)$ è relativamente compatto in $C^0(Y,X^*)$ per ogni varietà complessa $Y$.
\end{prop}

\begin{defn}
    Una varietà complessa $X$ si dice \textit{taut} se è Kobayashi-iperbolica e ogni mappa nella chiusura di $\text{Hol}(\mathbb{D},X)$ in $C^0(\mathbb{D},X^*)$ è in $\text{Hol}(\mathbb{D},X)$ oppure è la mappa costante a $\infty$.
\end{defn}

Per finire, diamo delle definizioni che ci serviranno per parlare del comportamente delle iterate di funzioni olomorfe.

\begin{defn}
    Siano $X$ e $Y$ due spazi topologici. Diciamo che una successione $\{f_{\nu}\}_{\nu \in \mathbb{N}} \subset C^0(X,Y)$ è \textit{compattamente divergente} se per ogni coppia di compatti $H\subseteq X$ e $K\subseteq Y$ esiste $\nu_0 \in \mathbb{N}$ tale che $f(H)\cap K=\emptyset$ per ogni $\nu \ge \nu_0$.

    Una famiglia $\mathcal{F} \subseteq C^0(X,Y)$ è detta \textit{normale} se ogni successione in $\mathcal{F}$ ammette una sottosuccessione che converge uniformemente sui compatti oppure è compattamente divergente.
\end{defn}
\subsection{Risultati noti della teoria} \label{risnoti}
Vediamo ora alcuni risultati noti della teoria che si saranno utili nelle nostre dimostrazioni. \\

DA FARE QUI:\\
UN SACCO DI RISULTATI, DA DIMOSTRARE E NON, SU PSEUDOMETRICA E PSEUDODISTANZA\\
UN SACCO DI RISULTATI, DA DIMOSTRARE E NON, SULLE VARIETÀ TAUT\\
ALTRE COSE CHE TI VENGONO IN MENTE VIA VIA CHE SCRIVI E CHE SONO TROPPO DI BASE PER TROVARE POSTO ALTROVE\\
NON TI SCORDARE IL TEOREMA DI MONTEL (E FORSE SERVIRÀ ANCHE WEIERSTRASS); SERVIRÀ ANCHE ASCOLI ARZELÀ? SPERIAMO DI NO

\newpage

\section{Un teorema di tipo ``Wolff-Denjoy'' per varietà taut con visibilità} \label{Un teorema di tipo ``Wolff-Denjoy'' per varietà taut con visibilità}
\subsection{Il concetto di visibilità}
Nella sezione precedente abbiamo visto come l'ipotesi di varietà taut ci permette di dire, se le orbite di una certa funzione non sono relativamente compatte, che la successione delle iterate è compattamente divergente.

Per ottenere un risultato di tipo Wolff-Denjoy, nel caso in cui le iterate siano compattamente divergenti dobbiamo dire due cose: che le iterate convergono uniformemente sui compatti a una funzione a valori nel bordo euclideo, e che in realtà tale funzione è una costante.

Per ottenere la convergenza uniforme al bordo ci basterà supporre che la varietà sia embeddata in un qualche $\mathbb{C}^d$ e limitata, dopodiché si applica il teorema di Montel. Per dire che la funzione è costante, invece, ci serviranno delle ipotesi aggiuntive di tipo geometrico: la condizione di visibilità per le simil-geodetiche. \\

Nel caso di un dominio con bordo regolare, l'ipotesi di essere strettamente pseudoconvesso permetteva di concludere la condizione geometrica di Gromov-iperbolicità; \marginpar{Capire se torna}sotto l'ipotesi di Gromov-iperbolicità, si può dimsotrare che il dominio soddisfa una condizione di visibilità per le geodetiche (si veda la discussione che precede il Theorem 1.4 in \cite{BZ1} per delle reference più approfondite).

Tuttavia, nella prossima sezione vedremo esempi di domini che soddisfano la condizione di visibilità per le simil-geodetiche ma che non sono Gromov-iperbolici.
\marginpar{Capire se è davvero questo il tipo di controesempio che ci serve}
Inoltre, non vale neanche l'altra implicazione: esistono domini Gromov-iperbolici per i quali non vale la condizione di visibilità per le circa geodetiche.

Dunque non c'è implicazione né in un senso né nell'altro, e il risultato che otteniamo è indipendente da quello per domini Gromov-iperbolici. Prima di enunciarlo, ci serviranno alcune definizioni, partendo da quella di simil-geodetica.

\begin{defn}
    Sia $X$ una varietà connessa complessa embeddata in $\mathbb{C}^d$ e limitata, e fissiamo $\lambda \ge 1$ e $\kappa \ge 0$. Sia $I\subseteq \mathbb{R}$ un intervallo; una curva $\sigma:I \longrightarrow X$ è detta una \textit{$(\lambda,\kappa)$-simil-geodetica} se
    \begin{enumerate}
        \item per ogni $s,t \in I$ si ha
        $$\frac{1}{\lambda}|t-s|-\kappa \le k_X\big(\sigma(s),\sigma(t)\big)\le\lambda|t-s|+\kappa;$$
        \item $\sigma$ è assolutamente continua (quindi $\sigma'(t)$ esiste per quasi ogni $t \in I$) e per quasi ogni $t \in I$ si ha
        $$K_X\big(\sigma(t);\sigma'(t)\big) \le \lambda.$$
    \end{enumerate}
\end{defn}

Diamo adesso la definizione di condizione di visibilità.

\begin{defn} \label{visibility}
    Sia $X$ una varietà connessa complessa embeddata in $\mathbb{C}^d$ e limitata, e fissiamo $\lambda \ge 1$ e $\kappa \ge 0$. Diciamo che $X$ ha la \textit{condizione di visibilità rispetto alle $(\lambda,\kappa)$-simil-geodetiche} se
    \begin{enumerate}
        \item ogni due punti distinti di $X$ possono essere collegati da una $(\lambda,\kappa)$-simil-geodetica;
        \item per ogni coppia di punti $p,q\in\partial X$ con $p\not=q$, esistono in $\mathbb{C}^d$ due intorni $V$ e $W$ di $p$ e $q$ rispettivamente e un compatto $K$ di $X$ tali che: $\overline{V}\cap\overline{W}=\emptyset$; ogni $(\lambda,\kappa)$-simil-geodetica in $X$ che collega un punto di $V$ a un punto di $W$ interseca $K$.
    \end{enumerate}
\end{defn}

Le simil-geodetiche sono delle curve che, a meno di costanti moltiplicative e additive, si comportano come le geodetiche, cioè le curve che minimizzano la lunghezza.

\marginpar{Qua ci va un bel disegnino in inkscape, sia del caso vietato che del caso previsto dalla definizione}
Quello che chiediamo nella Definizione \ref{visibility} è che, per andare da un punto a un altro del bordo con tali curve, allora non possiamo stare arbitrariamente vicini al bordo, ma siamo costretti a ``piegarci'' verso l'interno; in pratica, stiamo chiedendo che ci sia una sorta di curvatura negativa.

\begin{ftt}
    Il dominio $\Omega$ definito nell'Esempio \ref{servetaut} soddisfa la condizione di visibilità per le simil-geodetiche. Per vederlo, consideriamo due casi:
    \begin{nlist}
        \item uno dei due punti è l'origine. Allora basta prendere come compatto un qualsiasi insieme della forma $\{r \le |z| \le R\}$ con $0<r<R<1$ e i due intorni aperti sufficientemente piccoli;
        \item i due punti sono entrambi sulla sfera unitaria. Per \cite[Proposition 6]{NTT}, è facile vedere che, se la palla unitaria soddisfa la condizione di visibilità per simil-geodetiche, allora $\Omega$ la soddisfa in questo caso. Adesso, la palla unitaria è Gromov-iperbolica (è facile verificare, ad esempio, che è strettamente pseudoconvessa), dunque soddisfa la condizione di visibilità rispetto alle geodetiche. Usando \cite[Part III, Chapter H, Paragraph 1, Theorem 1.7]{BH} e il fatto che è uno spazio metrico proprio, possiamo facilmente concludere che soddisfa anche la condizione di visibilità per le simil-geodetiche.
    \end{nlist}
\end{ftt}
\subsection{Risultati tecnici preparatori}
Prima di andare a vedere il teorema di tipo ``Wolff-Denjoy'', dobbiamo mostrare diversi risultati preliminari. Visto che andremo a dimostrare la versione del teorema che si trova in \cite{CMS}, tali risultati sono per la maggior parte dimostrati, e il resto citati, nel suddetto articolo. \\

Cominciamo con delle stime dal basso e dall'alto per la metrica di Kobayashi, che permettono anche di ottenere la lipschitzianità delle simil-geodetiche.

\begin{lm} \label{polycpt}
    Sia $X$ una varietà complessa. Se un sottoinsieme compatto $K\subseteq X$ è contenuto nel polidisco di una carta di $X$, allora esiste una costante $C=C(K)>0$ tale che $K_X(z;Z) \le C\|Z\|_X$ per ogni $z \in K$ e $Z \in T_zX$.
\end{lm}

\begin{proof}
    Siano $n=\dim{X}$ e $D=\mathbb{D}_{r_1}\times\dots\times\mathbb{D}_{r_n}$ il polidisco che contiene $K$. Applicando la Proposizione \ref{metrdecr} all'inclusione e passando in coordinate, per ogni $z \in K$ e $Z \in T_zX$ si ha che
    $$K_X(z;Z) \le K_D(z;Z) \le \max_{j=1,\dots,n} \frac{r_j|Z_j|}{r_j^2-|z_j|^2},$$
    dove $|\cdot|$ è il modulo, cioè la norma euclidea delle coordinate della carta. Per ottenere la seconda disuguaglianza, ragioniamo al seguente modo: a meno di riscalare tutto per una costante, possiamo supporre che il membro destro sia uguale a $1$ (se fosse $0$, avremmo $Z=0$ e la tesi sarebbe immediata). Consideriamo adesso la funzione $f\in\text{Hol}(\mathbb{D},D)$ che manda $\zeta\in\mathbb{D}$ nell'elemento di $D$ che ha come $j$-esima coordinata $\dfrac{r_j\alpha_j\zeta+z_j}{1+\bar{z}_j\alpha_j\zeta/r_j}$, dove $\alpha_j=\dfrac{r_jZ_j}{r_j^2-|z_j|^2}$; allora $f(0)=z$ e si ha che $Df(0)\cdot 1=Z$ e la disuguaglianza discende dalla definizione di $K_D$. Poiché, per compattezza di $K$, la quantità $r_j^2-|z_j|^2$ è limitata dal basso da una costante positiva per $j=1,\dots,n$, esiste una costante $C_0>0$ tale che
    $$K_X(z;Z) \le C_0\max_{j=1,\dots,n}\{|Z_j|\}.$$
    
    
    Consideriamo la norma hermitiana come una funzione
    $$\|\cdot\|_X:K\times\big\{v\in\mathbb{C}^n\mid\max_{j=1,\dots,n}\{|v_j|\}=1\big\}\longrightarrow (0,+\infty),$$
    dove il secondo fattore nel dominio della funzione è visto come sottoinsieme del tangente al variare dei punti nel primo fattore. Per compattezza tale funzione ammette un minimo $c>0$; a meno di riscalare abbiamo che
    $$\|Z\|_X \ge c\max_{j=1,\dots,n}\{|Z_j|\}$$
    per ogni $z\in K$ e $Z\in T_zX$. Basta allora prendere $C=C_0/c$.
\end{proof}

\begin{prop} \label{metrica_bilip}
    Sia $X$ una varietà complessa. Allora:
    \begin{enumerate}[label={(\arabic*)}]
        \item se $X$ è connessa, è Kobayashi-iperbolica se e solo se per ogni compatto $K\subseteq X$ esiste una costante $c=c(K)>0$ tale che $c\|Z\|_X \le K_X(z;Z)$ per ogni $z \in X$ e $Z \in T_zX$;
        \item per ogni compatto $K\subseteq X$ esiste una costante $C=C(K)>0$ tale che $K_X(z;Z) \le C\|Z\|_X$ per ogni $z \in X$ e $Z \in T_zX$;
        \item se $X$ è una sottovarietà limitata di $\mathbb{C}^d$ esiste una costante $c>0$ tale che $c\|Z\| \le K_X(z;Z)$ per ogni $z \in X$ e $Z \in T_zX$.
    \end{enumerate}
\end{prop}

\begin{proof}
    (1) Mostriamo che se vale quella condizione allora $X$ è Kobayashi-iperbolica. Prendiamo due punti distinti $z_0,w_0\in X$, per cui esiste un intorno compatto $K$ di $z_0$ con $w_0\not\in K$; in particolare, ogni curva da $z_0$ a $w_0$ deve uscire da $K$. Quindi per il Teorema \ref{lung_int} si ha
    $$k_X(z_0,w_0) \ge c\cdot d_X\big(z_0,\partial_X K)>0,$$
    da cui $k_X$ è una distanza e $X$ è Kobayashi-iperbolica.

    Viceversa, supponiamo che $X$ sia Kobayashi-iperbolica. Prendiamo un punto $z_0\in X$ e fissiamo un intorno $U$ di $X$ contenuto in una carta e che sia biolomorfo a $\mathbb{B}^n$, dove $n=\dim{X}$. Consideriamo l'aperto $V\subseteq U$ corrispondente a $\mathbb{B}_{1/2}^n$ tramite il biolomorfismo, che è ancora un intorno di $z_0$. Scegliamo inoltre $\epsilon>0$ tale che $B_X(z_0,2\epsilon)\subset\subset V$ (ricordiamo che con $B_X$ si intendono le palle rispetto a $k_X$). In particolare, per la Definizione \ref{poidist}, il punto (i) della Proposizione \ref{k_polidisco} e la Proposizione \ref{semicontr}, per ogni $\varphi\in\text{Hol}(\mathbb{D},X)$ tale che $\varphi(0)\in B_X(z_0,\epsilon)$ si ha $\varphi(\mathbb{D}_{\tanh{\epsilon}})\subseteq B_X(z_0,2\epsilon)$.

    Dati $z\in B_X(z_0,\epsilon)$ e $Z\in T_zX$, siano $\varphi\in\text{Hol}(\mathbb{D},X)$ e $v\in\mathbb{C}$ tali che $\varphi(0)=z$ e $\diff_0\varphi(v)=Z$. Allora, ponendo $\psi(\zeta)=\varphi\big((\tanh{\epsilon})\zeta\big)$, abbiamo che $\psi\in\text{Hol}(\mathbb{D},V)$, $\psi(0)=z$ e $\diff_0\psi(v)=(\tanh{\epsilon})Z$. Dalla definizione della pseudometrica di Kobayashi segue che
    $$(\tanh{\epsilon})K_V(z;Z) \le k_X(z;Z)$$
    per ogni $z\in B_X(z_0,\epsilon)$ e $Z\in T_zX$. Dal punto (3), che dimostreremo indipendentemente tra poco, segue che la condizione voluta è vera in $\mathbb{B}_{1/2}^n$, e dunque in $V$, senza che la costante dipenda da un compatto, ma rispetto alla metrica euclidea. Tuttavia, considerano il compatto $\overline{\mathbb{B}_{1/2}^n}$ (motivo per cui abbiamo dovuto prendere la palla più piccola) e chiamando $E_j$ i vettori delle base canonica di $T_wX$ identificato, al variare di $w\in\overline{\mathbb{B}_{1/2}^n}$, con $\mathbb{C}^n$, si ha
    \begin{align*}
        \|Z\|_X \le \sum_{j=1}^n |a_j|\cdot\|E_j\|_X &\le \max_{\substack{w\in\overline{\mathbb{B}_{1/2}^n},\\j=1,\dots,n}}\|E_j\|_X \sum_{j=1}^n |a_j| \\
        &\le C_1\sqrt{\sum_{j=1}^n |a_j|^2}=C_1\|Z\|
    \end{align*}
    per ogni $w\in\overline{\mathbb{B}_{1/2}^n}$ e $Z\in T_wX$ della forma $Z=\displaystyle\sum_{j=1}^n a_jE_j$, e per una qualche costante $C_1>0$, per cui la condizione vale anche con la metrica $\|\cdot\|_X$. Dato allora un compatto $K$, basta ricoprirlo con un numero finito di intorni della forma $B_X(z_0,\epsilon)$ e prendere la costante più piccola al variare di tali intorni.

    (2) Per ogni $z \in K$, scegliamo un polidisco $U_z$ centrato in $z$ e contenuno in una carta di $X$; sia $U_z'\subseteq U_z$ un altro polidisco, nella stessa carta, centrato in $z$ e relativamente compatto in $U_z$ per ogni $z \in K$. Dato che $K$ è compatto, esistono $z_1,\dots,z_l$ tali che $K\subseteq\displaystyle\bigcup_{j=1}^l U_{z_j}'$. Allora, poiché $\overline{U}_{z_j}'$ è un sottoinsieme compatto di $U_{z_j}$ per $j=1,\dots,l$, per il Lemma \ref{polycpt} abbiamo
    $$K_X(z,Z) \le C_j\|Z\|_X$$
    per ogni $z \in U_{z_j}'$ e $Z\in T_zX$, dove $C_j>0$ è una costante che dipende dal compatto $\overline{U}_{z_j}'$. Basta allora porre $C(K)=\displaystyle\max_{j=1,\dots,l}\{C_j\}$.

    (1) Supponiamo per assurdo che esistano $z_j \in X$ e $Z_j \in T_{z_j}X$, con $Z_j\not=0$, tali che $\displaystyle \lim_{j \longrightarrow +\infty} K_X(z_j;Z_j)/\|Z_j\| \longrightarrow 0$. Senza perdita di generalità possiamo supporre $\|Z_j\|=1$ per ogni $j$. Per definizione di $K_X$, esistono delle funzioni $f_j\in\text{Hol}(\mathbb{D},X)$ e dei $v_j \in \mathbb{C}$ tali che $f_j(0)=z_j$ e
    $$|v_j| \le K_X(z_j;Z_j)+1/j\quad\text{ e }\quad \diff_0 f_j(v_j)=Z_j.$$

    Segue che $\displaystyle \lim_{j \longrightarrow +\infty}\|\diff_0f_j(1)\| \ge \lim_{j \longrightarrow +\infty}1/|v_j|=+\infty$. A meno di sottosuccessioni e di riordinare le coordinate, possiamo supporre che siano le prime componenti dei vettori $\diff_0f_j(1)$ a tendere a $+\infty$. Chiamiamo $g_j$ la prima componente di $f_j$, cosicché $g_j'$ è la prima componente di $\diff f_j(1)$. Le $g_j$ sono le composizioni delle $f_j$ con un embedding e una proiezione, dunque sono olomorfe; inoltre, poiché $X$ è limitata, sono equilimitate. Esiste quindi un $r>0$ tale che $g_j \in \text{Hol}(\mathbb{D},\mathbb{D}_r)$ per ogni $j$. Adesso, noi sappiamo che $\displaystyle\lim_{j\longrightarrow +\infty} |g_j'(0)|=+\infty$; basta allora applicare il lemma di Schwarz a $\dfrac{g_j-g_j(0)}{2r}$ con $j$ sufficientemente grande per ottenere una contraddizione.
\end{proof}

\begin{oss} \label{basta_herm}
    Per il punto (3) della Proposizione precedente non è necessario che la metrica sia quella euclidea, basta una metrica hermitiana qualsiasi. Per vederlo, basta usare il punto (1) e la Proposizione \ref{metrdecr}.
\end{oss}

\begin{cor} \label{dist_da_sotto}
    Siano $Y$ una varietà Kobayashi-iperbolica e $X$ una sottovarietà complessa, connessa e relativamente compatta di $Y$. Allora esiste $c>0$ tale che $c\cdot d_X(z,w) \le k_X(z,w)$ per ogni $z,w \in X$.
\end{cor}

\begin{proof}
    Per il Teorema \ref{lung_int} ci basta mostrare che $c\cdot d_X(z,w) \le l_X(\gamma)$ per ogni curva $C^1$ a tratti $\gamma:[a,b] \longrightarrow X$ tale che $\gamma(a)=z$ e $\gamma(b)=w$. Prendendo $c=c(\overline{X})>0$ dato dal punto (1) della Proposizione \ref{metrica_bilip}, ricordando che $\|\cdot\|_X=\|\cdot\|_Y\restrict{X}$ e usando la Proposizione \ref{metrdecr}, abbiamo che
    \begin{align*}
        l_X(\gamma)&=\int_a^b K_X\big(\gamma(t);\gamma'(t)\big)\diff t \ge \int_a^b K_Y\big(\gamma(t);\gamma'(t)\big)\diff t\\
        &\ge \int_a^b c\|\gamma'(t)\|_Y\diff t=\int_a^b c\|\gamma'(t)\|_X\diff t\ge c\cdot d_X(z,w),
    \end{align*}
    come voluto.
\end{proof}

\begin{prop} \label{simillip}
    Siano $Y$ una varietà Kobayashi-iperbolica e $X$ una sottovarietà complessa, connessa e relativamente compatta di $Y$. Allora per ogni $\lambda \ge 1$ esiste una costante $C=C(\lambda)>0$ tale che ogni $(\lambda,\kappa)$-simil-geodetica è $C$-lipschitziana rispetto a $d_X$.
\end{prop}

\begin{proof}
    Ogni $(\lambda,\kappa)$-simil-geodetica $\sigma:I\longrightarrow X$ è, per definizione, assolutamente continua rispetto a $d_X$. Allora, per il teorema fondamentale del calcolo integrale, per ogni $s,t\in I$ abbiamo che
    $$\sigma(t)=\sigma(s)+\int_s^t\sigma'(r)\diff r.$$
    
    Per il punto (1) della Proposizione \ref{metrica_bilip}, esiste una costante $c=c(\overline{X})>0$ tale che $c\|Z\|_Y \le K_Y(z;Z)$ per ogni $z\in X$ e $Z\in T_zX$, e per definizione di $(\lambda,\kappa)$-simil-geodetica $K_X\big(\sigma(t);\sigma'(t)\big) \le \lambda$ per quasi ogni $t\in I$. Dunque, usando anche la Proposizione \ref{metrdecr}, si ha che $\|\sigma'(t)\|_Y \le \lambda/c$ per quasi ogni $t\in I$, da cui
    $$d_X\big(\sigma(t),\sigma(s)\big) \le \int_s^t\|\sigma'(r)\|_X\diff r=\int_s^t\|\sigma'(r)\|_Y\diff r \le \frac{\lambda}{c}|t-s|,$$
    cioè $\sigma$ è $\lambda/c$-lipschitziana rispetto alla distanza euclidea.
\end{proof}

Il seguente Lemma è un fatto tecnico che ci servirà tra poco.

\begin{lm} \label{restrdis}
    Siano $X$ una varietà complessa e connessa e $\sigma:[a,b] \longrightarrow X$ una curva assolutamente continua rispetto a $d_X$. Se
    $$l_X(\sigma) \le k_X\big(\sigma(a),\sigma(b)\big)+\kappa,$$
    allora, per ogni $a \le s \le t \le b$, si ha
    $$l_X(\sigma\restrict{[s,t]}) \le k_X\big(\sigma(s),\sigma(t)\big)+\kappa.$$
\end{lm}

\begin{proof}
    Siano $s$ e $t$ come sopra. Allora
    $$l_X(\sigma\restrict{[s,t]})=l_X(\sigma)-l_X(\sigma\restrict{[a,s]})-l_X(\sigma\restrict{[t,b]}).$$

    Usando la nostra ipotesi e il punto (ii) del Teorema \ref{lung_int}, troviamo
    $$l_X(\sigma\restrict{[s,t]}) \le k_X\big(\sigma(a),\sigma(b)\big)+\kappa-k_X\big(\sigma(a),\sigma(s)\big)-k_X\big(\sigma(t),\sigma(b)\big);$$
    applicando la disuguaglianza triangolare, si ottiene la tesi.
\end{proof}

 Adesso vogliamo mostrare che le varietà Kobayashi-iperboliche sono connesse per archi simil-geodetici.

\begin{thm} \label{similgeo_conn}
    Sia $X$ una varietà Kobayashi-iperbolica. Per ogni $z,w \in X$ e ogni $\kappa>0$ esiste una $(1,\kappa)$-simil-geodetica $\sigma:[a,b]\longrightarrow X$ tale che $\sigma(a)=z$ e $\sigma(b)=w$.
\end{thm}

\begin{proof}
    Per il punto (i) del Teorema \ref{lung_int}, a meno di riparametrizzare esiste una curva $C^1$ a tratti $\gamma:[0,1] \longrightarrow X$ tale che $\gamma(0)=z,\gamma(1)=w$ e
    $$l_X(\gamma)<k_X(z,w)+\kappa;$$
    inoltre, a meno di perturbare di poco la curva, possiamo assumere che sia $C^1$ e che $\gamma'(t)\not=0$ per ogni $t\in[0,1]$. Consideriamo la funzione $f:[0,1] \longrightarrow [0,+\infty)$ data da
    $$f(t)=\int_0^t K_X\big(\gamma(r);\gamma'(r)\big)\diff r.$$

    Poiché $\gamma([0,1])$ è compatto in $X$, per i punti (1) e (2) della Proposizione \ref{metrica_bilip} esiste $C>0$ tale che
    $$\frac{1}{C}\|\gamma'(t)\|_X \le K_X\big(\gamma(t);\gamma'(t)\big) \le C\|\gamma'(t)\|_X\text{ per ogni }t\in[0,1].$$

    Dato che $\|\gamma'(t)\|_X>0$ per ogni $t\in[0,1]$ e $\gamma'$ è continua, esistono $A,B>0$ tali che $A \le \|\gamma'(t)\|_X\le B$ per ogni $t\in[0,1]$. Dunque $f$ è una funzione bilipschitziana e, di conseguenza, strettamente crescente. Sia ora $g:[0,l_X(\gamma)] \longrightarrow [0,1]$ l'inversa di $f$. Vogliamo dire che la curva $\sigma=\gamma\circ g:[0,l_X(\gamma)] \longrightarrow X$ è una $(1,\kappa)$-simil-geodetica; sostanzialmente, $\sigma$ è la riparametrizzazione per lunghezza d'arco di $\gamma$.

    Poiché $g$ è bilipschitziana (perché lo è la sua inversa) e $\gamma$ è $C^1$, abbiamo che $\sigma$ è lipschitziana, per cui anche assolutamente continua, rispetto a $d_X$; allora, per i $t$ per i quali $g'(t)$ esiste, si ha $\sigma'(t)=\gamma'\big(g(t)\big)g'(t)$. Inoltre, per tali $t$ anche $f'\big(g(t)\big)$ esiste ed è non-nullo, e $g'(t)=1/f'\big(g(t)\big)>0$. Per il teorema fondamentale del calcolo integrale, si ha che $f'$ esiste per quasi ogni $s\in[0,1]$ e $f'(s)=K_X\big(\gamma(s);\gamma'(s)\big)$. Siccome $g$ è bilipschitziana, la preimmagine degli $s\in[0,1]$ per cui $f'(s)$ esiste è un sottoinsieme di $[0,l_X(\gamma)]$ di misura piena. Visto che $\gamma'(s)\not=0$ per ogni $s\in[0,1]$, otteniamo che
    $$g'(t)=\frac{1}{K_X\Big(\gamma\big(g(t)\big);\gamma'\big(g(t)\big)\Big)}$$
    per quasi ogni $t\in[0,l_X(\gamma)]$. Per tali $t$ si ha che
    $$K_X\big(\sigma(t);\sigma'(t)\big)=K_X\Big(\gamma\big(g(t)\big);\gamma'\big(g(t)\big)g'(t)\Big)=1;$$
    quindi $l_X(\sigma)=l_X(\gamma) \le k_X(z,w)+\kappa$. Per il Lemma \ref{restrdis} si ha, per ogni $0 \le s \le t \le l_X(\gamma)$, che
    $$|t-s|=l_X(\sigma\restrict{[s,t]}) \le k_X\big(\sigma(s),\sigma(t)\big)+\kappa.$$

    Dato che $\sigma$ è assolutamente continua, per il punto (ii) del Teorema \ref{lung_int} abbiamo anche che
    $$k_X\big(\sigma(s),\sigma(t)\big) \le l_X(\sigma\restrict{[s,t]})=|s-t|$$
    per ogni $0 \le s \le t \le l_X(\gamma)$. Segue dunque che $\sigma$ è una $(1,\kappa)$-simil-geodetica.
\end{proof}

Adesso ci servirà un Lemma quasi ovvio.

\begin{lm} \label{quasiovvio}
    Sia $X$ una varietà complessa e connessa. Se $\sigma:[a,b] \longrightarrow X$ è una $(1,\kappa)$-simil-geodetica per qualche $\kappa>0$, allora per ogni $t \in [a,b]$ si ha
    $$k_X\big(\sigma(a),\sigma(t)\big)+k_X\big(\sigma(t),\sigma(b)\big) \le k_X\big(\sigma(a),\sigma(b)\big)+3\kappa.$$
\end{lm}

\begin{proof}
    È un'immediata conseguenza della definizione di $(1,\kappa)$-simil-geodetica.
\end{proof}

Il seguente Lemma, invece, ci servirà per la prossima dimostrazione. È un risultato sulla convergenza puntuale, mentre più avanti ne vedremo uno sulla convergenza uniforme sui compatti per varietà $(1,\kappa_0)$-visibili per qualche $\kappa_0>0$.

\begin{lm} \label{lemmasottosotto1}
    Sia $X$ una sottovarietà complessa e relativamente compatta di una varietà complessa e connessa $Y$, e sia $F\in\textnormal{Hol}(X,X)$. Dati una funzione strettamente crescente $\mu:\mathbb{N}\longrightarrow\mathbb{N}$ e $z\in X$, se l'insieme dei punti limite della successione $\{F^{\mu(j)}(z)\}_{j\in\mathbb{N}}$ consiste in un solo punto $\xi$ allora abbiamo che $\displaystyle\lim_{j\longrightarrow+\infty}F^{\mu(j)}(z)\longrightarrow\xi$.
\end{lm}

\begin{proof}
    Supponiamo per assurdo che non valga la tesi. Allora esistono un $\epsilon>0$ e una sottosuccessione $\{j_n\}_{n\in\mathbb{N}}$ tali che $d_Y\big(F^{\mu(j_n)}(z),\xi\big) \ge \epsilon$ per ogni $n\in\mathbb{N}$. Per relativa compattezza di $X$, a meno di prendere un'ulteriore sottosuccessione possiamo suppore che $\displaystyle\lim_{n\longrightarrow+\infty}F^{\mu(j_n)}(z)\longrightarrow\xi'$. Per ipotesi dev'essere $\xi'=\xi$, in contraddizione con il fatto che $d_Y\big(F^{\mu(j_n)}(z),\xi\big) \ge \epsilon$ per ogni $n\in\mathbb{N}$.
\end{proof}

Quello che andremo ora a dimostrare è uno dei fatti cruciali per ottenere il teorema di tipo ``Wolff-Denjoy''. Esso afferma che, sotto condizioni di visibilità per le simil-geodetiche, le sottosuccessioni di iterate di una funzione olomorfa che ``tendono a infinito'' convergono tutte, puntualmente, a un unico punto del bordo.

\begin{prop} \label{inf_impl_ugu}
    Sia $X$ una sottovarietà complessa, connessa e relativamente compatta di una varietà Kobayashi-iperbolica $Y$. Supponiamo che esista un $\kappa_0>0$ tale che $X$ sia $(1,\kappa_0)$-visibile. Data una funzione $F\in\textnormal{Hol}(X,X)$, esiste $\xi\in\partial_YX$ tale che per ogni funzione $\mu:\mathbb{N}\longrightarrow\mathbb{N}$ strettamente crescente per cui esiste $y_0 \in X$ tale che
    \begin{equation}
        \lim_{j\longrightarrow+\infty} k_X\big(F^{\mu(j)}(y_0),y_0\big)=+\infty
    \end{equation}
    si ha
    \begin{equation}
        \lim_{j\longrightarrow+\infty} F^{\mu(j)}(z)=\xi
    \end{equation}
    per ogni $z \in X$.
\end{prop}

\begin{proof}
    Se $\displaystyle\limsup_{n\longrightarrow+\infty}k_X\big(F^n(x),x\big)<+\infty$ per ogni $x\in X$ l'affermazione è vera a vuoto per ogni $\xi\in\partial_YX$. Altrimenti, esiste $x_0\in X$ tale per cui possiamo prendere una funzione strettamente crescente $\nu:\mathbb{N}\longrightarrow\mathbb{N}$ tale che:
    \begin{itemize}
        \item si ha $k_X\big(F^{\nu(j)}(x_0),x_0\big) \ge k_X\big(F^k(x_0),x_0\big)$ per ogni $j \in \mathbb{N}$ e per ogni $k \le \nu(j)$;
        \item si ha $\displaystyle\lim_{j\longrightarrow+\infty}k_X\big(F^{\nu(j)}(x_0),x_0\big)=+\infty$;
        \item la successione $\{F^{\nu(j)}(x_0)\}_{j\in\mathbb{N}}$ converge a un certo $\xi\in\partial_YX$.
    \end{itemize}
    
    Infatti, definendo induttivamente $l_0=0$ e $l_n$ come il minimo numero naturale $h>l_{n-1}$ tale che $k_X\big(F^h(x_0),x_0\big) \ge \max\left\{n,k_X\big(F^{l_{n-1}}(x_0),x_0\big)\right\}$, abbiamo $k_X\big(F^{l_n}(x_0),x_0\big) \ge k_X\big(F^k(x_0),x_0\big)$ per ogni $n$ e per ogni $k \le l_n$; inoltre, abbiamo $\displaystyle\lim_{n\longrightarrow+\infty}k_X\big(F^{l_n}(x_0),x_0\big)=+\infty$. Per limitatezza di $X$, esiste un'ulteriore sottosuccessione $l_{n_j}$ tale che $F^{l_{n_j}}(x_0)$ converge a un certo $\xi\in\overline{X}$, e in realtà $\xi\in\partial_YX$ visto che la distanza di Kobayashi da $x_0$ tende a $+\infty$; basta allora porre $\nu(j)=l_{n_j}$.\\

    Vogliamo ora mostrare la seguente asserzione.

    Siano $\{m_j\}_{j\in\mathbb{N}}$ e $\{m'_j\}_{j\in\mathbb{N}}$ due successioni strettamente crescenti di numeri naturali e $z_0,z'_0\in X$ tali che:
    \begin{enumerate}[label={(\arabic*)}]
        \item per ogni $j\in\mathbb{N}$ si ha $m_j \ge m'_j$;
        \item per ogni $j\in\mathbb{N}$ e $k \le m_j$ si ha $k_X\big(F^{m_j}(z_0),z_0\big) \ge k_X\big(F^k(z_0),z_0\big)$;
        \item si ha $\displaystyle\lim_{j\longrightarrow+\infty}k_X\big(F^{m_j}(z_0),z_0\big)=\lim_{j\longrightarrow+\infty}k_X\big(F^{m'_j}(z'_0),z_0\big)=+\infty$;
        \item le successioni $\{F^{m_j}(z_0)\}_{j\in\mathbb{N}}$ e $\{F^{m'_j}(z'_0)\}_{j\in\mathbb{N}}$ convergono, rispettivamente, a $\zeta$ e $\zeta'$ in $\partial_YX$;
    \end{enumerate}
    allora $\zeta=\zeta'$.

    Supponiamo per assurdo che $\zeta\not=\zeta'$. Per l'Osservazione \ref{kobisdist} anche $X$ è Kobayashi-iperbolica; quindi grazie al Teorema \ref{similgeo_conn} possiamo scegliere, per ogni $j\in\mathbb{N}$, una $(1,\kappa_0)$-simil-geodetica $\sigma_j:[0,T_j] \longrightarrow X$ tale che $\sigma_j(0)=F^{m_j}(z_0)$ e $\sigma_j(T_j)=F^{m'_j}(z'_0)$. Adesso, dato che abbiamo assunto che $\{F^{m_j}(z_0)\}_{j\in\mathbb{N}}$ e $\{F^{m'_j}(z'_0)\}_{j\in\mathbb{N}}$ convergano a due punti di $\partial_YX$ distinti e $X$ ha la visibilità rispetto alle $(1,\kappa_0)$-simil-geodetiche, esistono una costante $0<R<+\infty$ e, per ogni $j\in\mathbb{N}$, un $t_j\in[0,T_j]$ tali che $k_X\big(z_0,\sigma_j(t_j)\big)<R$. Per il Lemma \ref{quasiovvio} si ha dunque che
    \begin{equation} \label{3k_02R}
        \begin{aligned}
            k_X\big(F^{m_j}(z_0),F^{m'_j}(z'_0)\big) &\ge k_X\big(F^{m_j}(z_0),\sigma_j(t_j)\big)+k_X\big(\sigma_j(t_j),F^{m'_j}(z'_0)\big)-3\kappa_0\\
            &\ge k_X\big(F^{m_j}(z_0),z_0\big)-k_X\big(z_0,\sigma_j(t_j)\big)\\
            &\quad+k_X\big(F^{m'_j}(z'_0),z_0\big)-k_X\big(z_0,\sigma_j(t_j)\big)-3\kappa_0\\
            &\ge k_X\big(F^{m_j}(z_0),z_0\big)+k_X\big(F^{m'_j}(z'_0),z_0\big)-3\kappa_0-2R;
        \end{aligned}
    \end{equation}
    d'altra parte, abbiamo anche che
    \begin{equation} \label{-3k_02R}
        \begin{aligned}
            k_X\big(F^{m_j}(z_0),F^{m'_j}(z'_0)\big) &\le k_X\big(F^{m_j-m'_j}(z_0),z'_0\big) \\
            &\le k_X\big(F^{m_j-m'_j}(z_0),z_0\big)+k_X(z_0,z'_0) \\
            &\le k_X\big(F^{m_j}(z_0),z_0\big)+k_X(z_0,z'_0),
        \end{aligned}
    \end{equation}
    dove per la prima e la terza disuguaglianza abbiamo usato, rispettivamente, le condizioni (1) e (2) sulle successioni $\{m_j\}_{j\in\mathbb{N}}$ e $\{m'_j\}_{j\in\mathbb{N}}$; nella prima, abbiamo anche usato che le funzioni olomorfe sono semicontrazioni rispetto a $k_X$. Concatenando la \eqref{3k_02R} e la \eqref{-3k_02R} e riarrangiando i termini, otteniamo
    $$k_X\big(F^{m'_j}(z'_0),z_0\big) \le k_X(z_0,z'_0)+3\kappa_0+2R,$$
    che è in contraddizione con la condizione (3).\\

    Adesso che l'asserzione è stata dimostrata, possiamo concludere la dimostrazione. Usando la disuguaglianza triangolare e il fatto che le funzioni olomorfe sono semicontrazioni rispetto a $k_X$, troviamo che
    \begin{gather*}
        k_X\big(F^{\mu(j)}(z),z'\big) \ge k_X\big(F^{\mu(j)}(y_0),y_0\big)-k_X(y_0,z)-k_X(z',y_0);
    \end{gather*}
    segue che $\displaystyle\lim_{j\longrightarrow+\infty}k_X\big(F^{\mu(j)}(z),z'\big)=+\infty$ per ogni $z,z' \in X$.

     Fissiamo ora uno $z\in X$, e prendiamo $\xi'$ punto limite di $\{F^{\mu(j)}(z)\}_{j\in\mathbb{N}}$. Allora deve esistere una funzione strettamente crescente $\tau:\mathbb{N}\longrightarrow\mathbb{N}$ tale che $\displaystyle \lim_{j\longrightarrow+\infty} F^{(\mu\circ\tau)(j)}(z)=\xi'$; inoltre, poiché $\displaystyle\lim_{j\longrightarrow+\infty}k_X\big(F^{\mu(j)}(z),z\big)=+\infty$, dev'essere $\xi'\in\partial_YX$. Scegliamo una funzione strettamente crescente $\tau':\mathbb{N}\longrightarrow\mathbb{N}$ tale che $\nu\circ\tau' \ge \mu\circ\tau$ e applichiamo l'asserzione dimostrata sopra alle successioni $m_j=(\nu\circ\tau')(j), m'_j=(\mu\circ\tau)(j)$ e ai punti $z_0=x_0,z'_0=z$, per i quali si verificano facilmente le condizioni (1), (2), (3) e (4). Troviamo così $\xi'=\xi$ e si conclude grazie al Lemma \ref{lemmasottosotto1}.
\end{proof}

Anche il seguente teorema ci aiuterà nella nostra dimostrazione. Esso afferma che, sotto condizioni di visibilità per le simil-geodetiche, le successioni di funzioni olomorfe che convergono uniformemente sui compatti di $X$ devono necessariamente convergere a una costante.

\begin{thm} \label{lim_is_const}
    Sia $X$ una sottovarietà complessa, connessa e relativamente compatta di una varietà Kobayashi-iperbolica $Y$. Supponiamo che esista una costante $\kappa_0>0$ tale che $X$ sia $(1,\kappa_0)$-visibile. Consideriamo una successione $\{F_\nu\}_{\nu\in\mathbb{N}} \subseteq \textnormal{Hol}(X,X)$ che converge uniformemente sui compatti di $X$ a una $\psi\in\textnormal{Hol}(X,Y)$ con $\psi(X)\subseteq\partial_YX$. Allora $\psi$ è costante.
\end{thm}

\begin{proof}
    Supponiamo, per assurdo, che $\psi$ non sia costante. Allora, dati $x \in X$ e $\epsilon>0$, la restrizione di $\psi$ a $B_X(x,\epsilon)$, la palla aperta rispetto alla distanza di Kobayashi di centro $x$ e raggio $\epsilon$, non è costante. Infatti, tale palla è un aperto di $X$, e se $\psi$ fosse costante su di essa lo sarebbe su tutta $X$ per il principio di identità delle funzioni olomorfe, poiché $X$ è connessa. Questa, però, sarebbe una contraddizione alla nostra assunzione.

    Fissiamo $x_0 \in X$ e poniamo $\epsilon_0=\kappa_0/3$. Per quanto detto, deve esistere un $x_1 \in B_X(x_0,\epsilon_0)$ tale che $\{F_\nu(x_0)\}_{\nu\in\mathbb{N}}$ e $\{F_\nu(x_1)\}_{\nu\in\mathbb{N}}$ convergono, rispettivamente, a $\xi$ e $\eta$, con $\xi,\eta \in \partial_YX$ e $\xi\not=\eta$. Consideriamo adesso una $(1,\kappa_0/3)$-simil-geodetica $\gamma:[a,b]\longrightarrow X$ tale che $\gamma(a)=x_0$ e $\gamma(b)=x_1$, che esiste per il Teorema \ref{similgeo_conn}; da \eqref{simil-geo1} nella definizione di simil-geodetica si ha che
    $$|a-b|-\kappa_0/3 \le k_X(x_0,x_1) \implies |a-b| \le k_X(x_0,x_1)+\kappa_0/3 \le 2\kappa_0/3.$$

    Per ogni $\nu\in\mathbb{N}$, definiamo $\sigma_\nu:[a,b]\longrightarrow X$ come $\sigma_\nu=F_\nu\circ\gamma$. Mostriamo che $\sigma_\nu$ è una $(1,\kappa_0)$-simil-geodetica per ogni $\nu\in\mathbb{N}$. Per ogni $s,t \in [a,b]$ si ha
    \begin{align*}k_X\big(\sigma_\nu(s),\sigma_\nu(t)\big)&=k_X\Big(F_\nu\big(\gamma(s)\big),F_\nu\big(\gamma(t)\big)\Big)\\
        &\le k_X\big(\gamma(s),\gamma(t)\big) \le |s-t|+\kappa_0/3 \le |s-t|+\kappa_0,
    \end{align*}
    dove abbiamo usato che le funzioni olomorfe sono delle semicontrazioni rispetto alla distanza di Kobayashi e, dato che $\gamma$ è una $(1,\kappa_0/3)$-simil-geodetica, la \eqref{simil-geo1}. Inoltre, si ha che $|s-t|-\kappa_0 \le |a-b|-\kappa_0 \le -\kappa_0/3<0$ per ogni $s,t \in [a,b]$; dunque
    \begin{equation} \label{snu1}
        |s-t|-\kappa_0 \le k_X\big(\sigma_\nu(s),\sigma_\nu(t)\big) \le |s-t|+\kappa_0.
    \end{equation}

    Infine, per ogni $t$ si ha
    \begin{equation} \label{snu2}
        \begin{aligned}
            K_X\big(\sigma_\nu(t);\sigma_\nu'(t)\big)&=K_X\Big(F_\nu\big(\gamma(s)\big);\diff_{\gamma(t)}F_\nu\big(\gamma'(t)\big)\Big) \\
            &\le K_X\big(\gamma(t);\gamma'(t)\big) \le 1,
        \end{aligned}
    \end{equation}
    dove abbiamo usato che le funzioni olomorfe non aumentano la metrica di Kobayashi e, visto che $\gamma$ è una $(1,\kappa_0/3)$-simil-geodetica, la \eqref{simil-geo2}. Si ha anche che la curva $\sigma_\nu$ è assolutamente continua rispetto a $d_X$ per ogni $\nu$, poiché composizione di $\gamma$, che è assolutamente continua per definizione di $(1,\kappa_0/3)$-simil-geodetica, e $F_\nu$, che è olomorfa e dunque lipschitziana (sempre rispetto a $d_X$) sul compatto $\gamma([a,b])$. Grazie alla \eqref{snu1} e alla \eqref{snu2}, possiamo concludere che $\sigma_\nu$ è una $(1,\kappa_0)$-simil-geodetica per ogni $\nu$.
    
    Adesso, poiché $\{F_\nu\}_{\nu\in\mathbb{N}}$ converge uniformemente sui compatti di $X$ a una funzione olomorfa $\psi$ a valori in $\partial_YX$, ne consegue che per ogni compatto $K\subseteq X$ esiste un $\nu(K)\in\mathbb{N}$ tale che, per ogni $\nu\ge\nu(K)$, si ha $\sigma_\nu([a,b])\cap K=\emptyset$. Ma così troviamo una contraddizione, perché $X$ è $(1,\kappa_0)$-visibile.
\end{proof}

Concludiamo la questa sottosezione con l'analogo del Lemma \ref{lemmasottosotto1} per la convergenza uniforme sui compatti.

\begin{lm} \label{lemmasottosotto2}
    Sia $X$ una sottovarietà complessa, connessa e tautly embdedded di una varietà Kobayashi-iperbolica $Y$. Supponiamo che esista un $\kappa_0>0$ tale che $X$ sia $(1,\kappa_0)$-visibile. Consideriamo una $F\in\textnormal{Hol}(X,X)$ tale che la successione delle iterate di $F$ è compattamente divergente, e supponiamo che esistano un compatto $K$ di $X$, una funzione strettamente crescente $\mu:\mathbb{N}\longrightarrow\mathbb{N}$ e $\xi\in\partial_YX$ tali che la successione $\{F^{\mu(j)}\}_{j\in\mathbb{N}}$ converge alla costante $\xi$ uniformemente su $K$; allora converge alla costante $\xi$ uniformemente su tutti i compatti di $X$.
\end{lm}

\begin{proof}
    Supponiamo per assurdo che non valga la tesi; allora esistono un compatto $H\subseteq X$, un $\epsilon>0$, una successione di numeri naturali $\{j_n\}_{n\in\mathbb{N}}$ e una successione $\{z_n\}_{n\in\mathbb{N}}$ con $z_n\in H$ per ogni $n\in\mathbb{N}$ tali che $d_Y\big(F^{\mu(j_n)}(z_n),\xi\big)\ge\epsilon$ per ogni $n\in\mathbb{N}$. Per la Proposizione \ref{montel} $\text{Hol}(X,X)$ è relativamente compatto in $\text{Hol}(X,Y)$; dalla dimostrazione della suddetta Proposizione segue anche che $C^0(X,Y)$ è metrizzabile, per cui $\text{Hol}(X,X)$ è relativamente compatto per successioni in $\text{Hol}(X,Y)$. Possiamo allora trovare una sottosuccessione $\{j_{n_m}\}_{m\in\mathbb{N}}$ tale che la successione $\{F^{\mu(j_{n_m})}\}_{m\in\mathbb{N}}$ converge uniformemente sui compatti di $X$ a una funzione olomorfa $\tilde{F}:X \rightarrow Y$ con $\tilde{F}(X)\subseteq\overline{X}$. Poiché la successione delle iterate di $F$ è compattamente divergente, dev'essere $\tilde{F}(X)\subseteq\partial_YX$; allora per il Teorema \ref{lim_is_const} $\tilde{F}$ è costante. Siccome su $K$ la successione $\{F^{\mu(j)}\}_{j\in\mathbb{N}}$ converge alla costante $\xi$, la funzione $\tilde{F}$ dev'essere proprio tale costante; ma per costruzione la convergenza non può essere uniforme sul compatto $H$, assurdo.
\end{proof}
\subsection{Convergenza uniforme sui compatti a meno di sottosuccessioni}\label{convcpt}
In questa sezione dimostreremo alcuni risultati di convergenza di successioni di funzioni olomorfe. Cominciamo con un lemma sulla convergenza puntuale.

\begin{lm} \label{limpt-bis}
    Sia $X$ una sottovarietà complessa, connessa e relativamente compatta di una varietà complessa $Y$. Supponiamo che esista $\kappa_0>0$ tale che $X$ sia $(1,\kappa_0)$-visibile.
    
    Siano $Z$ una varietà Kobayashi-iperbolica e $\{f_n\}_{n\in\mathbb{N}}\subseteq\textnormal{Hol}(Z,X)$ una successione compattamente divergente. Allora esistono $\xi\in\partial_YX$ e una sottosuccessione $\{f_{n_j}\}_{j\in\mathbb{N}}$ tali che $f_{n_j}(z)\longrightarrow\xi$ per ogni $z\in Z$.
\end{lm}

\begin{proof}
    Sia $z_0\in Z$. Per relativa compattezza, a meno di sottosuccessioni possiamo supporre che esista $\xi_0\in\partial_YX$ tale che $f_n(z_0)\longrightarrow\xi_0$. Supponiamo per assurdo che esista $z_1\in Z$ tale che la successione $\{f_n(z_1)\}_{n\in\mathbb{N}}$ non converge a $\xi_0$. Considerando una curva continua che collega $z_0$ a $z_1$, si trova facilmente che possiamo eventualmente sostituirli con due punti tali che $k_Z(z_0,z_1)<\kappa_0/2$. Di nuovo a meno di sottosuccessioni, possiamo supporre che esista $\xi_1\in\partial_YX$ tale che $\xi_1\not=\xi_0$ e $f_n(z_1)\longrightarrow\xi_1$.

    Per il Teorema \ref{similgeo_conn} esiste una $(1,\kappa_0/2)$-simil-geodetica $\sigma:[0,T]\longrightarrow Z$ tale che $\sigma(0)=z_0$ e $\sigma(T)=z_1$. Poniamo $\sigma_n:=f_n\circ\sigma$, vogliamo dimostrare che $\sigma_n$ è una $(1,\kappa_0)$-simil-geodetica per ogni $n$. Per la Proposizione \ref{semicontr} e per la definizione di $(1,\kappa_0/2)$-simil-geodetica, abbiamo che
    $$k_X\big(\sigma_n(s),\sigma_n(t)\big) \le k_Z\big(\sigma(s),\sigma(t)\big) \le |t-s|+\kappa_0/2\le |t-s|+\kappa_0$$
    per ogni $s,t\in[0,T]$. Inoltre, sempre dalla definizione di di $(1,\kappa_0/2)$-simil-geodetica si ha che
    \begin{align*}
        |t-s|-\kappa_0 &\le |0-T|-\kappa_0/2-\kappa_0/2 \le k_Z\big(\sigma(0),\sigma(T)\big)-\kappa_0/2\\
        &=k_Z(z_0,z_1)-\kappa_0/2<0\le k_X\big(\sigma_n(s),\sigma_n(t)\big)
    \end{align*}
    per ogni $s,t\in[0,T]$. Infine, dalla Proposizione \ref{metrdecr} e dalla definizione di $(1,\kappa_0/2)$-simil-geodetica segue che
    $$K_X\big(\sigma_n(t);\sigma_n'(t)\big)\le K_Z\big(\sigma(t);\sigma'(t)\big) \le 1$$
    per ogni $t\in[0,T]$. Dunque $\sigma_n$ è una $(1,\kappa_0)$-simil-geodetica per ogni $n$, come volevamo.

    Adesso, poiché $\sigma_n(0)=f_n(z_0)\longrightarrow\xi_0$ e $\sigma_n(T)=f_n(z_1)\longrightarrow\xi_1$ e $X$ è $(1,\kappa_0)$-visibile, esiste un compatto $K\subseteq X$ tale che
    $$\emptyset\not=K\cap\sigma_n([0,T])=K\cap f_n\big(\sigma([0,T])\big)$$
    per ogni $n$, in contraddizione con l'ipotesi che la successione $\{f_n\}_{n\in\mathbb{N}}$ sia compattamente divergente.
\end{proof}

Quello che andremo a dimostrare adesso è uno dei fatti cruciali per ottenere il teorema di tipo ``Wolff-Denjoy''. Esso afferma che, sotto condizioni di visibilità per le simil-geodetiche, le sottosuccessioni di iterate di una funzione olomorfa che ``tendono a infinito'' convergono tutte, puntualmente, a un unico punto del bordo.

\begin{prop} \label{inf_impl_ugu}
    Sia $X$ una sottovarietà complessa, connessa e relativamente compatta di una varietà Kobayashi-iperbolica $Y$. Supponiamo che esista un $\kappa_0>0$ tale che $X$ sia $(1,\kappa_0)$-visibile. Data una funzione $F\in\textnormal{Hol}(X,X)$, esiste $\xi\in\partial_YX$ tale che per ogni funzione $\mu:\mathbb{N}\longrightarrow\mathbb{N}$ strettamente crescente per cui esiste $y_0 \in X$ tale che
    \begin{equation}
        \lim_{j\longrightarrow+\infty} k_X\big(F^{\mu(j)}(y_0),y_0\big)=+\infty
    \end{equation}
    si ha
    \begin{equation}
        \lim_{j\longrightarrow+\infty} F^{\mu(j)}(z)=\xi
    \end{equation}
    per ogni $z \in X$.
\end{prop}

\begin{proof}
    Se $\displaystyle\limsup_{n\longrightarrow+\infty}k_X\big(F^n(x),x\big)<+\infty$ per ogni $x\in X$ l'affermazione è vera a vuoto per ogni $\xi\in\partial_YX$. Altrimenti, scelto $x_0\in X$ tale che $\displaystyle\limsup_{n\longrightarrow+\infty}k_X\big(F^n(x_0),x_0\big)=+\infty$ possiamo costruire una funzione strettamente crescente $\nu:\mathbb{N}\longrightarrow\mathbb{N}$ tale che:
    \begin{itemize}
        \item si ha $k_X\big(F^{\nu(j)}(x_0),x_0\big) \ge k_X\big(F^k(x_0),x_0\big)$ per ogni $j \in \mathbb{N}$ e per ogni $k \le \nu(j)$;
        \item si ha $\displaystyle\lim_{j\longrightarrow+\infty}k_X\big(F^{\nu(j)}(x_0),x_0\big)=+\infty$;
        \item la successione $\{F^{\nu(j)}(x_0)\}_{j\in\mathbb{N}}$ converge a un certo $\xi\in\partial_YX$.
    \end{itemize}
    
    Infatti, definendo induttivamente $l_0=0$ e $l_n$ come il minimo numero naturale $h>l_{n-1}$ tale che $k_X\big(F^h(x_0),x_0\big) \ge \max\left\{n,k_X\big(F^{l_{n-1}}(x_0),x_0\big)\right\}$, abbiamo $k_X\big(F^{l_n}(x_0),x_0\big) \ge k_X\big(F^k(x_0),x_0\big)$ per ogni $n$ e per ogni $k \le l_n$; inoltre, abbiamo $\displaystyle\lim_{n\longrightarrow+\infty}k_X\big(F^{l_n}(x_0),x_0\big)=+\infty$. Per relativa compattezza di $X$, esiste un'ulteriore sottosuccessione $l_{n_j}$ tale che $F^{l_{n_j}}(x_0)$ converge a un certo $\xi\in\overline{X}$, e in realtà $\xi\in\partial_YX$ visto che la distanza di Kobayashi da $x_0$ tende a $+\infty$; basta allora porre $\nu(j)=l_{n_j}$.\\

    Vogliamo ora mostrare la seguente asserzione.

    Siano $\{m_j\}_{j\in\mathbb{N}}$ e $\{m'_j\}_{j\in\mathbb{N}}$ due successioni strettamente crescenti di numeri naturali e $z_0,z'_0\in X$ tali che:
    \begin{enumerate}[label={(\arabic*)}]
        \item per ogni $j\in\mathbb{N}$ si ha $m_j \ge m'_j$;
        \item per ogni $j\in\mathbb{N}$ e $k \le m_j$ si ha $k_X\big(F^{m_j}(z_0),z_0\big) \ge k_X\big(F^k(z_0),z_0\big)$;
        \item si ha $\displaystyle\lim_{j\longrightarrow+\infty}k_X\big(F^{m_j}(z_0),z_0\big)=\lim_{j\longrightarrow+\infty}k_X\big(F^{m'_j}(z'_0),z_0\big)=+\infty$;
        \item le successioni $\{F^{m_j}(z_0)\}_{j\in\mathbb{N}}$ e $\{F^{m'_j}(z'_0)\}_{j\in\mathbb{N}}$ convergono, rispettivamente, a $\zeta$ e $\zeta'$ in $\partial_YX$;
    \end{enumerate}
    allora $\zeta=\zeta'$.

    Supponiamo per assurdo che $\zeta\not=\zeta'$. Per il Corollario \ref{kobisdist} anche $X$ è Kobayashi-iperbolica; quindi grazie al Teorema \ref{similgeo_conn} possiamo scegliere, per ogni $j\in\mathbb{N}$, una $(1,\kappa_0)$-simil-geodetica $\sigma_j:[0,T_j] \longrightarrow X$ tale che $\sigma_j(0)=F^{m_j}(z_0)$ e $\sigma_j(T_j)=F^{m'_j}(z'_0)$. Adesso, dato che abbiamo assunto che $\{F^{m_j}(z_0)\}_{j\in\mathbb{N}}$ e $\{F^{m'_j}(z'_0)\}_{j\in\mathbb{N}}$ convergano a due punti di $\partial_YX$ distinti e $X$ ha la visibilità rispetto alle $(1,\kappa_0)$-simil-geodetiche, esistono una costante $0<R<+\infty$ e, per ogni $j\in\mathbb{N}$, un $t_j\in[0,T_j]$ tali che $k_X\big(z_0,\sigma_j(t_j)\big)<R$. Per il Lemma \ref{quasiovvio} si ha dunque che
    \begin{equation} \label{3k_02R}
        \begin{aligned}
            k_X\big(F^{m_j}(z_0),F^{m'_j}(z'_0)\big) &\ge k_X\big(F^{m_j}(z_0),\sigma_j(t_j)\big)+k_X\big(\sigma_j(t_j),F^{m'_j}(z'_0)\big)-3\kappa_0\\
            &\ge k_X\big(F^{m_j}(z_0),z_0\big)-k_X\big(z_0,\sigma_j(t_j)\big)\\
            &\quad+k_X\big(F^{m'_j}(z'_0),z_0\big)-k_X\big(z_0,\sigma_j(t_j)\big)-3\kappa_0\\
            &\ge k_X\big(F^{m_j}(z_0),z_0\big)+k_X\big(F^{m'_j}(z'_0),z_0\big)-3\kappa_0-2R;
        \end{aligned}
    \end{equation}
    d'altra parte, abbiamo anche che
    \begin{equation} \label{-3k_02R}
        \begin{aligned}
            k_X\big(F^{m_j}(z_0),F^{m'_j}(z'_0)\big) &\le k_X\big(F^{m_j-m'_j}(z_0),z'_0\big) \\
            &\le k_X\big(F^{m_j-m'_j}(z_0),z_0\big)+k_X(z_0,z'_0) \\
            &\le k_X\big(F^{m_j}(z_0),z_0\big)+k_X(z_0,z'_0),
        \end{aligned}
    \end{equation}
    dove per la prima e la terza disuguaglianza abbiamo usato, rispettivamente, le condizioni (1) e (2) sulle successioni $\{m_j\}_{j\in\mathbb{N}}$ e $\{m'_j\}_{j\in\mathbb{N}}$; nella prima, abbiamo anche usato che le funzioni olomorfe sono semicontrazioni rispetto a $k_X$. Concatenando la \eqref{3k_02R} e la \eqref{-3k_02R} e riarrangiando i termini, otteniamo
    $$k_X\big(F^{m'_j}(z'_0),z_0\big) \le k_X(z_0,z'_0)+3\kappa_0+2R,$$
    che è in contraddizione con la condizione (3).\\

    Adesso che l'asserzione è stata dimostrata, possiamo concludere la dimostrazione. Usando la disuguaglianza triangolare e il fatto che le funzioni olomorfe sono semicontrazioni rispetto a $k_X$, troviamo che
    \begin{gather*}
        k_X\big(F^{\mu(j)}(z),z'\big) \ge k_X\big(F^{\mu(j)}(y_0),y_0\big)-k_X(y_0,z)-k_X(z',y_0);
    \end{gather*}
    segue che $\displaystyle\lim_{j\longrightarrow+\infty}k_X\big(F^{\mu(j)}(z),z'\big)=+\infty$ per ogni $z,z' \in X$.

     Fissiamo ora uno $z\in X$, e prendiamo $\xi'$ punto limite di $\{F^{\mu(j)}(z)\}_{j\in\mathbb{N}}$. Allora deve esistere una funzione strettamente crescente $\tau:\mathbb{N}\longrightarrow\mathbb{N}$ tale che $\displaystyle \lim_{j\longrightarrow+\infty} F^{(\mu\circ\tau)(j)}(z)=\xi'$; inoltre, poiché $\displaystyle\lim_{j\longrightarrow+\infty}k_X\big(F^{\mu(j)}(z),z\big)=+\infty$, dev'essere $\xi'\in\partial_YX$. Scegliamo una funzione strettamente crescente $\tau':\mathbb{N}\longrightarrow\mathbb{N}$ tale che $\nu\circ\tau' \ge \mu\circ\tau$ e applichiamo l'asserzione dimostrata sopra alle successioni $m_j=(\nu\circ\tau')(j), m'_j=(\mu\circ\tau)(j)$ e ai punti $z_0=x_0,z'_0=z$, per i quali si verificano facilmente le condizioni (1), (2), (3) e (4). Troviamo così $\xi'=\xi$ e si conclude grazie al Lemma \ref{lemmasottosotto1}.
\end{proof}

Anche la prossima proposizione ci aiuterà nella nostra dimostrazione. Essa afferma che, sotto condizioni di visibilità per le simil-geodetiche, vale una sorta di lemma ``sotto-sotto'', cioè da ogni sottosuccessione si può estrarre una sottosottosuccessione convergente (a una costante). Prima ci serviranno un paio di lemmi.

\begin{lm} \label{nonazeroplease-bis}
    Sia $X$ una sottovarietà Kobayashi-iperbolica di una varietà complessa $Y$. Supponiamo che esista $\kappa_0>0$ tale che $X$ sia $(1,\kappa_0)$-visibile. Siano $\{x_n\}_{n\in\mathbb{N}}$ e $\{y_n\}_{n\in\mathbb{N}}$ due successioni di punti di $X$ tali che $x_n\longrightarrow\xi$ e $y_n\longrightarrow\xi'$, con $\xi,\xi'\in\partial_YX$ e $\xi\not=\xi'$.
    
    Allora non può essere che $\displaystyle\lim_{n\longrightarrow+\infty}k_X(x_n,y_n)=0$.
\end{lm}

\begin{proof}
    Supponiamo per assurdo che la tesi sia falsa, e prendiamo due successioni che la contraddicono. Per il Teorema \ref{similgeo_conn} possiamo prendere, per ogni $n\in\mathbb{N}$, una $(1,1/n)$-simil-geodetica $\sigma_n:[0,T_n]\longrightarrow X$ tale che $\sigma_n(0)=x_n$ e ${\sigma_n(T_n)=y_n}$. Per $n$ sufficientemente grande sono tutte $(1,\kappa_0)$-simil-geodetiche, per cui esistono un compatto $K\subseteq X$ e $t_n\in[0,T_n]$ tali che $\sigma_n(t_n)\in K$ per ogni $n$. Dato che $X$ è una varietà, possiamo prendere un compatto ${H\subseteq X}$ tale che $K\subseteq{\mathop H\limits^ \circ}$. Allora $\partial^\text{top}H$ e $K$ sono due compatti disgiunti, dove $\partial^\text{top}$ è il bordo topologico in $X$ (chiusura meno parte interna); dunque ${\displaystyle\inf_{\substack{x\in\partial^\text{top}H,\\y\in K}} k_X(x,y)=\epsilon>0}$.

    Poiché $\sigma_n(0)$ e $\sigma_n(T_n)$ convergono a punti del bordo, per $n$ sufficientemente grande appartengono a $X\setminus H$. Ma $\sigma_n(t_n)\in K\subseteq{\mathop H\limits^ \circ}$; quindi deve esistere $t_n'$ tale che $\sigma_n(t_n')\in\partial^\text{top}H$. Segue che
    \begin{align*}
        0&<\epsilon\le k_X\big(\sigma_n(t_n),\sigma_n(t_n')\big) \le |t_n-t_n'|+1/n\le |0-T_n|+1/n \\
        &\le k_X\big(\sigma_n(0),\sigma_n(T_n)\big)+2/n=k_X(x_n,y_n)+2/n\longrightarrow0,
    \end{align*}
    contraddizione.
\end{proof}

\begin{lm} \label{unaltrosottosotto-bis}
    Sia $X$ una sottovarietà Kobayashi-iperbolica di una varietà complessa $Y$. Supponiamo che $X$ sia relativamente compatta e che esista ${\kappa_0>0}$ tale che $X$ sia $(1,\kappa_0)$-visibile. Sia $F\in\textnormal{Hol}(X,X)$ tale che la successione $\{F^n\}_{n\in\mathbb{N}}$ sia compattamente divergente.
    
    Supponiamo che esistano un compatto $K\subseteq X$, una funzione strettamente crescente $\mu:\mathbb{N}\longrightarrow\mathbb{N}$ e $\xi\in\partial_YX$ tali che la successione $F^{\mu(j)}$ converge alla costante $\xi$ uniformemente su $K$. Allora la successione $\{F^{\mu(j)}\}_{j\in\mathbb{N}}$ converge alla costante $\xi$ uniformemente su tutti i compatti di $X$.
\end{lm}

\begin{proof}
    Supponiamo per assurdo che la tesi non sia vera; esistono dunque un compatto $H\subseteq X$ e un intorno $U$ di $\xi$ in $\overline{X}$ tali che ${F^{\mu(j)}(H)\not\subseteq U}$ frequentemente. Allora esistono una sottosuccessione $\{j_n\}_{n\in\mathbb{N}}\subseteq\mathbb{N}$ e una successione $\{z_n\}_{n\in\mathbb{N}}\subseteq H$ tali che $F^{\mu(j_n)}(z_n)\not\in U$ per ogni $n\in\mathbb{N}$.
    
    A meno di sottosuccessioni, usando il Lemma \ref{limpt-bis} e la relativa compattezza di $X$, possiamo supporre che $z_n\longrightarrow\tilde{z}\in H$, che ${F^{\mu(j_n)}(z)\longrightarrow\tilde{\xi}\in\overline{X}}$ per ogni $z\in X$ e che $F^{\mu(j_n)}(z_n)\longrightarrow\xi'\in\overline{X}$. Dato che $\{F^n\}_{n\in\mathbb{N}}$ è compattamente divergente, dev'essere $\xi'\in\partial_YX$. Siccome $F^{\mu(j_n)}(z)\longrightarrow\xi$ per ogni $z\in K$, dev'essere $\tilde{\xi}=\xi$. Visto che $F^{\mu(j_n)}(z_n)\not\in U$ per ogni $n\in\mathbb{N}$, dev'essere $\xi\not=\xi'$.

    Adesso notiamo che per la Proposizione \ref{semicontr} si ha
    $$k_X\big(F^{\mu(j_n)}(z_n),F^{\mu(j_n)}(\tilde{z})\big)\le k_X(z_n,\tilde{z})\longrightarrow0$$
    per $n\longrightarrow+\infty$. Basta allora applicare il Lemma \ref{nonazeroplease-bis} con $x_n=F^{\mu(j_n)}(z_n)$ e $y_n=F^{\mu(j_n)}(\tilde{z})$ per ottenere una contraddizione.
\end{proof}

\begin{prop} \label{estraiunasotto-bis}
    Sia $X$ una sottovarietà Kobayashi-iperbolica di una varietà complessa $Y$. Supponiamo che $X$ sia relativamente compatta e che esista $\kappa_0>0$ tale che $X$ sia $(1,\kappa_0)$-visibile. Sia $F\in\textnormal{Hol}(X,X)$ tale che la successione $\{F^n\}_{n\in\mathbb{N}}$ sia compattamente divergente.

    Per ogni funzione strettamente crescente $\mu:\mathbb{N}\longrightarrow\mathbb{N}$ esistono $\xi\in\partial_YX$ e una sottosuccessione $\{j_n\}_{n\in\mathbb{N}}\subseteq\mathbb{N}$ tale che la successione $\{F^{\mu(j_n)}\}_{n\in\mathbb{N}}$ converge alla costante $\xi$ uniformemente su tutti i compatti di $X$.
\end{prop}

\begin{proof}
    Fissiamo $z_0\in X$. Per la relativa compattezza di $X$ e la divergenza dai compatti di $\{F^n\}_{n\in\mathbb{N}}$, esistono $\xi\in\partial_YX$ e una sottosuccessione $\{j_n\}_{n\in\mathbb{N}}\subseteq\mathbb{N}$ tali che $F^{\mu(j_n)}(z_0)\longrightarrow\xi$. Allora la successione $\{F^{\mu(j_n)}\}_{n\in\mathbb{N}}$ converge alla costante $\xi$ uniformemente sul compatto $\{z_0\}$. Si conclude applicando il Lemma \ref{unaltrosottosotto-bis}.
\end{proof}
\subsection{Il teorema di tipo ``Wolff-Denjoy''}
Andiamo adesso ad enunciare e dimostrare la versione più generale di un teorema di tipo Wolff-Denjoy per varietà Kobayashi-iperboliche. Riportiamo la dimostrazione data in \cite{CMS}, ma notiamo che la strategia e le tecniche impiegate sono sostanzialmente riprese da \cite{BZ1} e \cite{BM}. Ognuno di questi articoli ha generalizzato il risultato ottenuto nel precedente.

\begin{thm}
    (\cite[Theorem 1.15]{CMS}) Sia $X$ una varietà complessa, connessa, embeddata in $\mathbb{C}^d$ e limitata. Supponiamo che $X$ sia taut e che esista un $\kappa_0>0$ tale che $X$ soddisfi la condizione di visibilità rispetto alle $(1,\kappa_0)$-simil-geodetiche.
    
    Sia $F:X \longrightarrow X$ una funzione olomorfa. Allora vale esattamente una delle seguenti affermazioni:
    \begin{itemize}
        \item le orbite di $F$ sono relativamente compatte in $X$;
        \item esiste un unico punto di $\partial X$ tale che le iterate di $F$ convergono tutte, uniformemente sui compatti, a quel punto.
    \end{itemize}
\end{thm}

\begin{proof}
    Poiché $X$ è taut, per il Teorema \ref{dicotomia} o l'orbita di $z$ tramite $F$ è relativamente compatta per ogni $z \in X$, oppure la successione delle iterate $\{F^\nu\}_{\nu\in\mathbb{N}}$ è compattamente divergente. Supponiamo che le orbite di $F$ non siano relativamente compatte in $X$; allora la successione delle iterate di $F$ è compattamente divergente.

    Consideriamo una sottosuccessione qualsiasi di $\{F^\nu\}_{\nu\in\mathbb{N}}$. Usando il teorema di Montel, possiamo trovare una sottosottosuccessione che converge uniformemente sui compatti a una funzione olomorfa $\tilde{F}:X \rightarrow \overline{X}$. Infatti, poiché $X$ è limitata allora la sottosuccessione è uniformemente limitata sui compatti; possiamo dunque fissare un ricoprimento numerabile fatto di aperti relativamente compatti contenuti in carte di $X$ ed estrarre, con un procedimento diagonale nel quale si applica il teorema di Montel ad ogni tale carta, una sottosottosuccessione che converga uniformemente sui compatti di ogni carta, in particolare sugli aperti del ricoprimento. Prendendo un'esaustione in compatti di $X$ (ad esempio, le palle chiuse rispetto a $k_X$) si può concludere facilmente che la convergenza è uniforme su tutti i compatti di $X$.

    Poiché le iterate di $F$ sono compattamente divergenti, si ha che $\tilde{F}\in\text{Hol}(X,\partial X)$. Allora, per il Teorema \ref{lim_is_const}, troviamo che $\tilde{F}$ è costante. Identifichiamo quindi
    $$\Gamma:=\overline{\{F^\nu\}}_{\nu\in\mathbb{N}}\setminus\{F^\nu\}_{\nu\in\mathbb{N}}$$
    come un insieme di punti di $\partial X$, dove la chiusura è intesa rispetto alla topologia compatta-aperta. Supponiamo, per assurdo, che $\Gamma$ contenga almeno due punti. \\

    Caso 1: esiste (e quindi per ogni) $o \in X$ tale che
    $$\limsup_{\nu\longrightarrow+\infty} k_X\big(F^\nu(o),o\big)=+\infty.$$
    Possiamo dunque scegliere una sottosuccessione $\{\nu_j\}_{j\in\mathbb{N}}$ tale che
    \begin{enumerate}[label={(\arabic*)}]
        \item per ogni $j\in\mathbb{N}$ e $k\le\nu_j$ si ha $k_X\big(F^k(o),o\big) \le k_X\big(F^{\nu_j}(o),o\big)$;
        \item $\{F^{\nu_j}\}_{j\in\mathbb{N}}$ converge, uniformemente sui compatti di $X$, a $\xi\in\partial X$.
    \end{enumerate}

    Adesso, poiché abbiamo assunto che $\Gamma$ contenga almeno due elementi, esiste una sottosuccessione $\{\mu_j\}_{j\in\mathbb{N}}$ tale che $\{F^{\mu_j}\}_{j\in\mathbb{N}}$ converge, uniformemente sui compatti di $X$, a $\eta\in\partial X$ con $\eta\not=\xi$. Segue immediatamente dalla Proposizione \ref{inf_impl_ugu} che non possiamo avere $\displaystyle\limsup_{j\longrightarrow+\infty} k_X\big(F^{\mu_j}(o),o\big)=+\infty$. Perciò dev'essere $\displaystyle\limsup_{j\longrightarrow+\infty} k_X\big(F^{\mu_j}(o),o\big)<+\infty$, quindi
    \begin{align*}
        \limsup_{h\longrightarrow+\infty}\limsup_{j\longrightarrow+\infty}\, &k_X\big(F^{\nu_h}(o),F^{\mu_j}(o)\big) \\
        &\ge \limsup_{h\longrightarrow+\infty}\limsup_{j\longrightarrow+\infty} \Big(k_X\big(F^{\nu_h}(o),o\big)-k_X\big(F^{\mu_j}(o),o\big)\Big)=+\infty.
    \end{align*}

    Fissiamo ora un $l\in\mathbb{N}$. Se applichiamo il Teorema \ref{lim_is_const} ad una sottosuccessione arbitraria di $\{F^{\mu_j-l}\}_{j\in\mathbb{N}}$ che converge uniformemente sui compatti di $X$, dal momento che sul compatto $\{F^l(o)\}$ converge a $\eta$ otteniamo (usando anche il teorema di Montel) che tutta la sottosuccessione converge, uniformemente sui compatti di $X$, a $\eta$.

    Poniamo $M_l=\displaystyle\limsup_{j\longrightarrow+\infty}k_X\big(F^{\mu_j-l}(o),o\big)$. Affermiamo che $\displaystyle\limsup_{l\longrightarrow+\infty}M_l<+\infty$. Supponiamo per assurdo che non sia così, allora esiste una sottosuccessione $\{l_m\}_{m\in\mathbb{N}}$ tale che $M_{l_m}>m$ per ogni $m$. Per definizione di $M_l$ e per quanto appena trovato sulla successione $\{F^{\mu_j-l}\}_{j\in\mathbb{N}}$, abbiamo quindi che esiste una sottosottosuccessione $\{j_m\}_{m\in\mathbb{N}}$ tale che
    \begin{enumerate}[label={(\arabic*)}]
        \item $\|F^{\mu_{j_m}-l_m}(o)-\eta\|<1/k$;
        \item $k_X\big(F^{\mu_{j_m}-l_m}(o),o\big)>k$.
    \end{enumerate}
    Per la Proposizione \ref{inf_impl_ugu} deve dunque essere $\eta=\xi$, contraddizione. Perciò segue che $\displaystyle\limsup_{l\longrightarrow+\infty}M_l<+\infty$. Allora
    \begin{align*}
        \limsup_{h\longrightarrow+\infty}\limsup_{j\longrightarrow+\infty}\, &k_X\big(F^{\nu_h}(o),F^{\mu_j}(o)\big) \\
        &\le \limsup_{h\longrightarrow+\infty}\limsup_{j\longrightarrow+\infty} k_X\big(o,F^{\mu_j-\nu_h}(o)\big)=\limsup_{h\longrightarrow+\infty}M_{\nu_h}<+\infty,
    \end{align*}
    in contraddizione con quanto trovato prima; questo conclude il Caso 1.\\

    Caso 2: esiste (e quindi per ogni) $o \in X$ tale che
    $$\limsup_{\nu\longrightarrow+\infty} k_X\big(F^\nu(o),o\big)<+\infty.$$
    Ricordiamo che abbiamo assunto che esistano due punti distinti $\xi,\eta\in\Gamma$. Poiché $X$ soddisfa la condizione di visibilità rispetto alle $(1,\kappa_0)$-simil-geodetiche, esistono $V_\xi, V_\eta$ intorni in $\mathbb{C}^d$ rispettivamente di $\xi$ e di $\eta$, con $\overline{V}_\xi\cap\overline{V}_\eta=\emptyset$, e un compatto $K$ di $X$ tali che ogni $(1,\kappa_0)$-simil-geodetica in $X$ che collega un punto di $V_\xi$ a un punto di $V_\eta$ interseca $K$.
    Adesso definiamo, per $\delta>0$ arbitrario, la funzione $G_\delta:K\times K\longrightarrow [0,+\infty)$ data da
    $$G_\delta(x_1,x_2)=\inf\{k_X\big(F^m(x_1),x_2\big)\mid m\in\mathbb{N}, \|F^m(x_1)-\xi\|<\delta\}.$$
\end{proof}

\newpage

\section{Esempi di domini con visibilità} \label{Esempi di domini con visibilità}
Dopo aver dimostrato il Corollario \ref{wd_type}, viene naturale chiedersi: esistono sottovarietà limitate di $\mathbb{C}^n$ che possiamo dimostrare essere taut e visibili per le simil-geodetiche, anche senza ipotesi di regolarità (l'esempio che già conosciamo, i domini strettamente pseudoconvessi, hanno regolarità $C^2$)?

In questa sezione andremo a vedere diversi esempi di domini limitati in $\mathbb{C}^n$ che soddisfano la condizione di visibilità: due classi di domini, la prima introdotta \cite{BZ1} e della quale studieremo una sottoclasse di esempi espliciti, e la seconda introdotta in \cite{BM} della quale costruiremo un esempio semplice; un ulteriore esempio introdotto in \cite{CMS}. Inoltre, i domini della classe introdotta in \cite{BM} sono anche taut, perciò per essi vale automaticamente il Corollario \ref{wd_type}.
\subsection{Domini Goldilocks}
Il primo esempio di una classe di domini di $\mathbb{C}^n$ con visibilità è quello, introdotto in \cite{BZ1}, dei domini Goldilocks. Prima di darne la definizione, introduciamo, dati un dominio limitato $\Omega\subseteq\mathbb{C}^n$, un sottoinsieme $U\subseteq\mathbb{C}^n$ e $r>0$, la seguente quantità:
$$M_{\Omega,U}(r):=\sup\left\{\frac{1}{K_\Omega(x;v)}\mid x\in\Omega\cap U,\delta_\Omega(x) \le r\text{ e } \|v\|=1\right\}.$$

Poniamo $M_\Omega:=M_{\Omega,\mathbb{C}^n}$. La funzione $M_{\Omega,U}$ è monotona crescente, dunque misurabile secondo Lebesgue; inoltre, segue dal punto (3) della Proposizione \ref{metrica_bilip} che è anche limitata. Perciò ha senso la definizione che stiamo per dare.

\begin{defn} \label{gold}
    Un dominio limitato $\Omega\subseteq\mathbb{C}^n$ è detto \textit{dominio Goldilocks} se:
    \begin{enumerate}[label={(\arabic*)}]
        \item esiste (e quindi per ogni) $\epsilon>0$ tale che $\displaystyle\int_0^{\epsilon} \frac{1}{r}M_\Omega(r)\diff r<+\infty$;
        \item per ogni $x_0\in\Omega$ esistono due costanti $C,\alpha>0$ (che dipendono da $x_0$) tali che $k_\Omega(x_0,x) \le C+\alpha\log{\dfrac{1}{\delta_\Omega(x)}}$ per ogni $x\in\Omega$.
    \end{enumerate}
\end{defn}

\begin{oss}
    Il nome particolare, domini Goldilocks (Riccioli d'oro, in italiano), è dovuto al fatto che, come la protagonista della fiaba, tali domini evitano due estremi ``sgradevoli'': il bordo non ha cuspidi rivolte verso l'esterno né punti in cui è piatto fino a ordine infinito. Il primo caso è escluso dalla condizione (2) nella definizione. La condizione (1) implica che il dominio è pseudoconvesso (\cite[Proposition 2.15]{BZ1}).
\end{oss}

Per vedere che i domini Goldilocks soddisfano la condizione di visibilità, iniziamo mostrando un risultato che ci servirà anche nei prossimi esempi.

\begin{thm} \label{extvis}
    (\cite[Theorem 1.9]{CMS}) Sia $\Omega$ un dominio limitato di $\mathbb{C}^n$. Sia $S\subseteq\partial\Omega$ un insieme chiuso tale che per ogni $p,q\in\partial\Omega$ con $p\not=q$ esistono $p'\in\partial\Omega$ e $r>0$ tali, detta $B(p',r)$ la palla euclidea di centro $p'$ e raggio $r$, che:
    \begin{nlist}
        \item si ha $p\in B(p',r)$ e $q\in\partial\Omega\setminus\overline{B(p',r)}$;
        \item si ha $S\cap\partial B(p',r)=\emptyset$.
    \end{nlist}
    
    Inoltre, supponiamo che per ogni $\xi\in\partial\Omega\setminus S$ esistano un intorno $U$ di $\xi$, uno $z_0\in\Omega$ e una funzione $C^1$ strettamente crescente $f:(0,+\infty)\longrightarrow\mathbb{R}$, con $f(t)\longrightarrow+\infty$ per $t\longrightarrow+\infty$, tali che:
    \begin{enumerate}[label={(\arabic*)}]
        \item si ha $k_\Omega(z_0,z) \le f\big(1/\delta_\Omega(z)\big)$ per ogni $z\in\Omega\cap U$;
        \item si ha $M_{\Omega,U}(r)\longrightarrow 0$ per $r\longrightarrow 0$;
        \item esiste $r_0>0$ tale che $\displaystyle\int_0^{r_0}\frac{M_{\Omega,U}(r)}{r^2}f'\left(\frac{1}{r}\right)\diff r<+\infty$.
    \end{enumerate}

    Allora $\Omega$ è $(\lambda,\kappa)$-visibile per ogni $\lambda \ge 1$ e $\kappa>0$.
\end{thm}

\begin{proof}
    Supponiamo per assurdo che esistano $\lambda \ge 1$ e $\kappa>0$ tali che $\Omega$ non sia $(\lambda,\kappa)$-visibile. Allora esistono $p,q\in\partial\Omega$ con $p\not=q$, due successioni $\{p_j\}_{j\in\mathbb{N}}$ e $\{q_j\}_{j\in\mathbb{N}}$, convergenti rispettivamente a $p$ e $q$, e una successione $\{\gamma_j\}_{j\in\mathbb{N}}$ di $(\lambda,\kappa)$-simil-geodetiche, con $\gamma_j:[a_j,b_j]\longrightarrow\Omega$ e $\gamma_j(a_j)=p_j$ e $\gamma_j(b_j)=q_j$ per ogni $j\in\mathbb{N}$, tali che
    $$\max_{a_j\le t\le b_j}\delta_\Omega\big(\gamma_j(t)\big)\longrightarrow 0\text{ per }j\longrightarrow+\infty.$$

    Per ipotesi esistono $p'\in\partial\Omega$ e $r>0$ tali che valgano (i) e (ii). Poiché $p_j\longrightarrow p$ e $q_j\longrightarrow q$ per $j\longrightarrow+\infty$, possiamo assumere senza perdita di generalità che $\{p_j\}_{j\in\mathbb{N}}\subseteq B(p',r)$ e $\{q_j\}_{j\in\mathbb{N}}\subseteq \Omega\setminus\overline{B(p',r)}$. Poiché $\gamma_j$ è una curva continua che collega $p_j$ e $q_j$, deve esistere $\alpha_j\in(a_j,b_j)$ tale che $\xi_j:=\gamma_j(\alpha_j)\in\partial B(p',r)$; a meno di sottosuccessioni, possiamo assumere che $\xi_j\longrightarrow\xi\in\partial\Omega\cap\partial B(p',r)$ per $j\longrightarrow+\infty$. Per (ii) si ha che $\xi\in\partial\Omega\setminus S$; allora esistono, per ipotesi, un intorno $U$ di $\xi$, uno $z_0\in\Omega$ e una funzione $C^1$ strettamente crescente $f:(0,+\infty)\longrightarrow\mathbb{R}$ tali che valgano (1), (2) e (3). Osserviamo che tali ipotesi sono ancora soddisfatte se prendiamo un intorno $V \subseteq U$ di $\xi$, per cui, a meno di prendere un intorno più piccolo, possiamo supporre che $\overline{U}\cap(S\cup\{p,q\})=\emptyset$; inoltre, di nuovo a meno di sottosuccessioni, possiamo anche supporre che $q_j\not\in\overline{U}$.

    Sia $\epsilon>0$ tale che $\overline{B(\xi,\epsilon)}\subseteq U$; poiché $\xi_j\longrightarrow \xi$ per $j\longrightarrow+\infty$, possiamo assumere senza perdita di generalità che $\{\xi_j\}_{j\in\mathbb{N}}\subseteq B(\xi,\epsilon)$. Per ogni $j\in\mathbb{N}$ poniamo
    $$\beta_j:=\inf\{t\in[\alpha_j,b_j]\mid \gamma_j(t)\in\partial B(\xi,\epsilon)\};$$
    per definizione di $\beta_j$ e per il fatto che $\partial B(\xi,\epsilon)$ è chiuso, si ha $\gamma_j(\beta_j)\in\partial B(\xi,\epsilon)$, e inoltre si ha $\gamma_j(\alpha_j)=\xi_j\not\in\partial B(\xi,\epsilon)$, per cui $a_j<\alpha_j<\beta_j<b_j$. Poniamo $\sigma_j:=\gamma_j\restrict{[\alpha_j,\beta_j]}:[\alpha_j,\beta_j]\longrightarrow\Omega$ per ogni $j\in\mathbb{N}$. Poiché $\gamma_j(\alpha_j)=\xi_j\in B(\xi,\epsilon)$, per definizione di $\beta_j$ dev'essere $\sigma_j([\alpha_j,\beta_j])\subseteq\overline{B(\xi,\epsilon)}\subseteq U$. Notiamo che, essendo la restrizione della $(\lambda,\kappa)$-simil-geodetica $\gamma_j$, anche $\sigma_j$ è una $(\lambda,\kappa)$-simil-geodetica per ogni $j\in\mathbb{N}$; inoltre, si ha
    $$\max_{\alpha_j\le t\le \beta_j}\delta_\Omega\big(\sigma_j(t)\big)\le\max_{a_j\le t\le b_j}\delta_\Omega\big(\gamma_j(t)\big)\longrightarrow 0\text{ per }j\longrightarrow+\infty.$$

    Adesso, a meno di riparametrizzare le curve $\sigma_j$, possiamo assumere senza perdita di generalità che $\alpha_j\le 0\le\beta_j$ e che
    $$\max_{\alpha_j\le t\le \beta_j}\delta_\Omega\big(\sigma_j(t)\big)=\delta_\Omega\big(\sigma_j(0)\big)$$
    per ogni $j\in\mathbb{N}$. Per la Proposizione \ref{simillip} esiste una costante $C>0$, che dipende solo da $\lambda$, tale che le $(\lambda,\kappa)$-simil-geodetiche di $\Omega$ sono $C$-lipschitziane rispetto alla distanza euclidea. Allora, applicando il teorema di Ascoli-Arzelà e passando a un'opportuna sottosuccessione con un procedimento diagonale, possiamo assumere che:
    \begin{itemize}
        \item si ha $\alpha_j\longrightarrow\alpha\in[-\infty,0]$ e $\beta_j\longrightarrow\beta\in[0,+\infty]$ per $j\longrightarrow+\infty$;
        \item la successione $\{\sigma_j\}_{j\in\mathbb{N}}$ converge uniformemente sui compatti di $(\alpha,\beta)$ a una curva continua $\sigma:(\alpha,\beta)\longrightarrow \overline{B(\xi,\epsilon)}\subseteq U$;
        \item si ha $\sigma_j(\alpha_j)=\xi_j\longrightarrow\xi$ e $\sigma_j(\beta_j)=\eta_j\longrightarrow\eta$ per $j\longrightarrow+\infty$, con ${\xi\in\partial\Omega\cap\partial B(p',r)}$ e $\eta\in\partial\Omega\cap\partial B(\xi,\epsilon)$.
    \end{itemize}

    Ovviamente dev'essere $\xi\not=\eta$; quindi, dato che $\|\sigma(\alpha_j)-\sigma(\beta_j)\| \le C(\beta_j-\alpha_j)$ per ogni $j\in\mathbb{N}$, si ha che $C(\beta-\alpha)\ge\|\xi-\eta\|>0$, per cui $\beta>\alpha$. \\

    Mostriamo adesso che $\sigma$ è costante. Vediamo innanzitutto che per ogni $j\in\mathbb{N}$ e per quasi ogni $t\in(\alpha_j,\beta_j)$ si ha
    $$\|\sigma_j'(t)\| \le \lambda M_{\Omega,U}\Big(\delta_\Omega\big(\sigma_j(t)\big)\Big).$$
    
    Sia $t\in(\alpha_j,\beta_j)$ tale che $\sigma_j'(t)$ esiste e $\|\sigma_j'(t)\|\not=0$ (altrimenti la disuguaglianza è immediata). Ricordiamo che ogni $\sigma_j$ è una $(\lambda,\kappa)$-simil-geodetica, per cui $K_\Omega\big(\sigma_j(t);\sigma_j'(t)\big)\le\lambda$; quindi, dato che $\sigma_j([\alpha_j,\beta_j])\subseteq U$, si ha
    $$\|\sigma_j'(t)\|\le \frac{\lambda}{K_\Omega\left(\sigma_j(t);\frac{\sigma_j'(t)}{\|\sigma_j'(t)\|}\right)}\le\lambda M_{\Omega,U}\Big(\delta_\Omega\big(\sigma_j(t)\big)\Big),$$
    come voluto. Adesso, poiché $\displaystyle\max_{\alpha_j\le t\le \beta_j}\delta_\Omega\big(\sigma_j(t)\big)\longrightarrow 0$ per $j\longrightarrow+\infty$, dal fatto che $M_{\Omega,U}$ è crescente e dall'ipotesi (2) si ha che $M_{\Omega,U}\Big(\delta_\Omega\big(\sigma_j(t)\big)\Big)\longrightarrow 0$ uniformemente. Ma allora, dati $\alpha<u\le w<\beta$, si ha
    \begin{align*}
        \|\sigma(u)-\sigma(w)\|&\le \lim_{j\longrightarrow+\infty}\|\sigma_j(u)-\sigma_j(w)\| \\
        =&\lim_{j\longrightarrow+\infty}\left\|\int_u^w\sigma_j'(t)\diff t\right\|\le \limsup_{j\longrightarrow+\infty} \int_u^w \|\sigma_j'(t)\|\diff t\\
        &\le \lambda\limsup_{j\longrightarrow+\infty} \int_u^w M_{\Omega,U}\Big(\delta_\Omega\big(\sigma_j(t)\big)\Big)\diff t=0;
    \end{align*}
    dunque $\sigma$ è costante su $(\alpha,\beta)$. \\

    Vogliamo ottenere una contraddizione mostrando anche che $\sigma$ non è costante. Distinguiamo due casi.

    Caso 1: sia $\alpha$ che $\beta$ sono finiti. Per ogni $j\in\mathbb{N}$, definiamo al seguente modo la curva $\tilde{\sigma}_j:[\alpha,\beta]\longrightarrow\Omega$: restringiamo $\sigma_j$ all'intervallo $[\alpha_j,\beta_j]\cap[\alpha,\beta]$ ed estendiamola ad una costante sugli intervalli $[\alpha,\alpha_j]$ e $[\beta_j,\beta]$ se $\alpha<\alpha_j$ o $\beta_j<\beta$. È facile vedere che le $\tilde{\sigma}_j$ sono ancora $C$-lipschitziane, per cui, applicando di nuovo Ascoli-Arzelà, a meno di sottosuccessioni convergono, uniformemente sui compatti, a una curva continua $\tilde{\sigma}:[\alpha,\beta]\longrightarrow\overline{\Omega}$. Questa curva estende in modo continuo $\sigma$ a tutto $[\alpha,\beta]$ e $\tilde{\sigma}(\alpha)=\xi\not=\eta=\tilde{\sigma}(\beta)$, per cui $\tilde{\sigma}$ non è costante, e di conseguenza non lo è neanche $\sigma$.

    Caso 2: $\alpha=-\infty$ o $\beta=+\infty$. Ricordiamo che $\sigma_j$ è una $(\lambda,\kappa)$-simil-geodetica per ogni $j\in\mathbb{N}$; dunque si ha
    \begin{equation}\label{disconf}
        \begin{aligned}
            \frac{1}{\lambda}|t|-\kappa&\le k_\Omega\big(\sigma_j(0),\sigma_j(t)\big)\\
            &\le k_\Omega\big(\sigma_j(0),z_0\big)+k_\Omega\big(z_0,\sigma_j(t)\big) \le 2f\left(\frac{1}{\delta_\Omega\big(\sigma_j(t)\big)}\right),
        \end{aligned}
    \end{equation}
    dove l'ultima disuguaglianza segue dall'ipotesi (1), dal fatto che per ogni $j\in\mathbb{N}$ si ha $\sigma_j([\alpha_j,\beta_j])\subseteq\Omega\cap U$ e da $\displaystyle\max_{\alpha_j\le t\le \beta_j}\delta_\Omega\big(\sigma_j(t)\big)=\delta_\Omega\big(\sigma_j(0)\big)$.

    Consideriamo il caso $\beta=+\infty$. Poiché la successione $\{\beta_j\}_{j\in\mathbb{N}}$ diverge a $+\infty$ e $f$ è continua con $\displaystyle\lim_{t\longrightarrow+\infty}f(t)=+\infty$, esistono un naturale $N\in\mathbb{N}$ e una costante $B>0$ tali che per ogni $j\ge N$ e $t\in(B,\beta_j]$ si ha $\dfrac{|t|}{2\lambda}-\dfrac{\kappa}{2}\in f\big((0,+\infty)\big)$. Usando anche la disuguaglianza \eqref{disconf} e il fatto che $f$ è strettamente crescente, troviamo che
    $$f^{-1}\left(\frac{|t|}{2\lambda}-\frac{\kappa}{2}\right) \le \frac{1}{\delta_\Omega\big(\sigma_j(t)\big)}$$
    per ogni $j\ge N$ e $t\in(B,\beta_j]$. Se $\alpha=-\infty$, ragionando allo stesso modo troviamo un intero $N'$ e una costante $A>0$ tali che
    $$f^{-1}\left(\frac{|t|}{2\lambda}-\frac{\kappa}{2}\right) \le \frac{1}{\delta_\Omega\big(\sigma_j(t)\big)}$$
    per ogni $j\ge N'$ e $t\in[\alpha_j,-A)$. Vediamo il caso $\alpha=-\infty$ e $\beta=+\infty$. Dalle due disuguaglianze appena mostrate, usando anche che $\|\sigma_j'(t)\| \le \lambda M_{\Omega,U}\Big(\delta_\Omega\big(\sigma_j(t)\big)\Big)$ e che $M_{\Omega,U}$ è crescente, troviamo che
    $$\|\sigma_j'(t)\| \le \lambda M_{\Omega,U}\left(\frac{1}{f^{-1}\left(\frac{|t|}{2\lambda}-\frac{\kappa}{2}\right)}\right)$$
    per ogni $j\ge\max\{N,N'\}$ e per quasi ogni $t\in[\alpha_j,-A)\cup(B,\beta_j]$. Usando l'ipotesi (3) e il cambio di variabile $r=\dfrac{1}{f^{-1}\left(\frac{|t|}{2\lambda}-\frac{\kappa}{2}\right)}$, si ha che esistono due costanti $c\in(-\infty,-A)$ e $d\in(B,+\infty)$ tali che
    \begin{align*}
        \lambda\int_{-\infty}^c M_{\Omega,U}\left(\frac{1}{f^{-1}\left(\frac{|t|}{2\lambda}-\frac{\kappa}{2}\right)}\right)\diff t&+\lambda\int_d^{+\infty}M_{\Omega,U}\left(\frac{1}{f^{-1}\left(\frac{|t|}{2\lambda}-\frac{\kappa}{2}\right)}\right)\diff t\\
        &<\|\xi-\eta\|.
    \end{align*}

    Allora, usando le ultime due disuguaglianze, otteniamo
    \begin{align*}
        \|\sigma(d)-\sigma(c)\|&=\lim_{j\longrightarrow+\infty}\|\sigma_j(d)-\sigma_j(c)\|\\
        &\ge\begin{aligned}[t]\liminf_{j\longrightarrow+\infty}(&\|\sigma_j(\beta_j)-\sigma_j(\alpha_j)\|-\|\sigma_j(\alpha_j)-\sigma_j(c)\|\\
        &-\|\sigma_j(\beta_j)-\sigma_j(d)\|)\end{aligned}\\
        &\ge\begin{aligned}[t]&\lim_{j\longrightarrow+\infty}\|\sigma_j(\beta_j)-\sigma_j(\alpha_j)\|-\limsup_{j\longrightarrow+\infty}\left\|\int_{\alpha_j}^c \sigma_j'(t)\diff t\right\|\\
        &-\limsup_{j\longrightarrow+\infty}\left\|\int_d^{\beta_j} \sigma_j'(t)\diff t\right\|\end{aligned}\\
        &\ge\|\xi-\eta\|-\limsup_{j\longrightarrow+\infty}\int_{\alpha_j}^c \|\sigma_j'(t)\|\diff t-\limsup_{j\longrightarrow+\infty}\int_d^{\beta_j} \|\sigma_j'(t)\|\diff t\\
        &\ge\begin{aligned}[t]\|\xi-\eta\|&-\limsup_{j\longrightarrow+\infty}\lambda\int_{\alpha_j}^c M_{\Omega,U}\left(\frac{1}{f^{-1}\left(\frac{|t|}{2\lambda}-\frac{\kappa}{2}\right)}\right)\diff t\\
        &-\limsup_{j\longrightarrow+\infty}\lambda\int_d^{\beta_j} M_{\Omega,U}\left(\frac{1}{f^{-1}\left(\frac{|t|}{2\lambda}-\frac{\kappa}{2}\right)}\right)\diff t\end{aligned}\\
        &=\begin{aligned}[t]\|\xi-\eta\|&-\lambda\int_{-\infty}^c M_{\Omega,U}\left(\frac{1}{f^{-1}\left(\frac{|t|}{2\lambda}-\frac{\kappa}{2}\right)}\right)\diff t\\
        &-\lambda\int_d^{+\infty} M_{\Omega,U}\left(\frac{1}{f^{-1}\left(\frac{|t|}{2\lambda}-\frac{\kappa}{2}\right)}\right)\diff t>0;\end{aligned}
    \end{align*}
    dunque in questo caso $\sigma$ non è costante. Se invece $a<-\infty$ e $b=+\infty$ (il caso $a=-\infty$ e $b<+\infty$ è analogo), ragionando come nel caso 1 estendiamo le $\sigma_j$ a delle $\tilde{\sigma}_j$ che, a meno di sottosuccessioni, convergono uniformemente sui compatti a una curva continua $\tilde{\sigma}:[\alpha,+\infty)\longrightarrow\overline{\Omega}$ che estende $\sigma$. Allora basta ripetere la stima precedente con $\tilde{\sigma}_j, \tilde{\sigma}$ e $\alpha$ al posto di $\sigma_j, \sigma$ e $c$ e scegliendo un $d$ opportuno, trovando così che $\tilde{\sigma}$, e di conseguenza $\sigma$, non è costante. \\

    Poiché la nostra assunzione porta a una contraddizione, dev'essere falsa, da cui la tesi.
\end{proof}

Adesso mostriamo che i domini Goldilocks sono visibili per le simil-geodetiche.

\begin{cor} \label{gold_is_vis}
    Un dominio limitato Goldilocks $\Omega\subseteq\mathbb{C}^n$ è $(\lambda,\kappa)$-visibile per ogni $\lambda\ge 1$ e $\kappa\ge 0$.
\end{cor}

\begin{proof}
    Basta applicare il Teorema \ref{extvis} con $S=\emptyset$, $U=\mathbb{C}^n$, per cui abbiamo che $f(x)=C+\alpha\log{x}$, con $C$ e $\alpha$ dati dal punto (2) della Definizione \ref{gold} per un punto fissato.
\end{proof}

Un esempio esplicito di domini Goldilocks sono i domini limitati, pseudoconvessi e di tipo finito.
\subsection{Domini di tipo finito}
Iniziamo con la definizione di dominio finito nel senso di D'Angelo, introdotta in \cite{D'A}.

\begin{defn}
    \marginpar{Controllare se questa definizione va bene}
    Data una funzione $g\in C^\infty(\mathbb{C}^n)$ e $p\in\mathbb{C}^n$, \textit{l'ordine di contatto di $g$ in $p$} è $v_p(g)$, il grado del primo polinomio non nullo dell'espansione in Taylor in $p$ di $g-g(p)$. Se la funzione è a valori in più variabili, si considera il minimo degli ordini delle componenti.

    Sia $\Omega\subseteq\mathbb{C}^n$ un dominio con bordo $C^\infty$, cioè esiste una funzione $\rho\in C^\infty(\mathbb{C}^n)$ tale che $\Omega=\{\rho(z)<0\}$ e $\diff\rho\not=0$ in ogni punto di $\partial\Omega$. Sia $\textbf{f}$ il germe di una funzione olomorfa a valori in una $1$-varietà algebrica tale che $f(0)=z_0\in\partial\Omega$. Poniamo $\Delta_1(z_0)=\displaystyle\sup_{\textbf{f}} \frac{v_0(\rho\circ f)}{v_0(f)}$.

    La quantità $\Delta_1(z_0)$ è detta il \textit{tipo} di $z_0\in\partial\Omega$. Il dominio $\Omega$ si dice \textit{di tipo finito nel senso di D'Angelo} (o più brevemente \textit{di tipo finito}) se $\Delta_1(z_0)<+\infty$ per ogni $z_0\in\partial\Omega$.
\end{defn}

Ci servirà anche il seguente fatto.

\begin{thm} \label{cho}
    (\cite[Theorem 1]{C}) Sia $\Omega\subseteq\mathbb{C}^n$ un dominio limitato. Sia $z_0\in\partial\Omega$ di tipo finito e $U$ un intorno di $z_0$ tale che $\partial\Omega\cap U$ è liscio e pseudoconvesso. Allora esiste un intorno $V\subseteq U$ di $z_0$ e due costanti $c,\epsilon>0$ tali che si ha
    $$K_\Omega(z;Z) \ge c\frac{\|Z\|}{\delta_\Omega(z)^\epsilon}$$
    per ogni $z\in\Omega$ e $Z\in T_z\Omega$.
\end{thm}

Possiamo allora dimostrare che per i domini limitati, pseudoconvessi e di tipo finito vale la condizione (1) nella definizione di dominio Goldilocks.

\begin{cor}
    Sia $\Omega$ un dominio limitato, pseudoconvesso e di tipo finito. Allora $\Omega$ soddisfa la condizione (1) nella Definizione \ref{gold}.
\end{cor}

\begin{proof}
    Poiché $\Omega$ è limitato, pseudoconvesso e di tipo finito, $\partial\Omega$ è compatto e ogni intorno di ogni suo punto soddisfa le ipotesi del Teorema \ref{cho}. Possiamo allora trovare un numero finito di aperti $V_1,\dots, V_n$ che ricoprono $\partial\Omega$ e delle costanti $c,\epsilon_1,\dots,\epsilon_N>0$ tali che $K_\Omega(x;v) \ge c\cdot\delta_\Omega(z)^{-\epsilon_j}$ per ogni $z\in\Omega\cap V_j$ e $v\in T_z\Omega$ con $\|v\|=1$.

    Basta dunque prendere $s=\min\{\epsilon_1,\dots,\epsilon_N\}$ e $r>0$ piccolo abbastanza affinché $r<1$ e $\{z\in\Omega\mid\delta_\Omega(z) \le r\}\subseteq V_1\cup\dots\cup V_N$. Segue che $M_\Omega(r) \le r^s/c$ con $s>0$ per $r$ sufficientemente piccolo, dunque la condizione (1) nella Definizione \ref{gold} è soddisfatta.
\end{proof}

Andiamo ora a dimostrare che soddisfano anche la condizione (2). Per questa, servono ipotesi meno stringenti.

\begin{prop}
    Sia $\Omega\subseteq\mathbb{C}^n$ limitato con bordo $C^2$. Allora $\Omega$ soddisfa la condizione (2) nella Definizione \ref{gold}.
\end{prop}

\begin{proof}
    Per \cite[Lemma 2.1]{BB}, $\partial\Omega$ ammette un intorno tubolare $U_\epsilon$, con $\epsilon>0$, tale che:
    \begin{nlist}
        \item si ha $U_\epsilon=\{z\in\mathbb{C}^n\mid\delta_\Omega(z)<\epsilon\}$;
        \item per ogni $z\in\Omega\cap U_\epsilon$ esiste un unico punto $\pi(z)\in\partial\Omega$ con $\|\pi(z)-z\|=\delta_\Omega(z)$;
        \item per ogni $z\in\Omega$, la fibra $\pi^{-1}\big(\pi(z)\big)$ è un sottoinsieme della normale a $\partial\Omega$ in $\pi(z)$;
        \item la mappa $z\longmapsto\big(\pi(z),\delta_\Omega(z)\big)$ è un omeomorfismo tra $U_\epsilon$ e $\partial\Omega\times(0,\epsilon)$.
    \end{nlist}
    A meno di prendere $\epsilon$ sufficientemente piccolo, per ogni $z\in\Omega$ con $\delta_\Omega(z)=\epsilon$ si ha che la palla euclidea di centro $z$ e raggio $\epsilon$ è tutta contenuta in $\Omega$. Presi allora $p,q\in\Omega$ con $\pi(p)=\pi(q)$ e $\epsilon\ge\delta_\Omega(p)\ge\delta_\Omega(q)$, poniamo $z_0$ il punto (interno a $\Omega$) della normale a $\partial\Omega$ in $\pi(p)$ tale che $\delta_\Omega(z_0)=\epsilon$. Usando la funzione $\varphi:\mathbb{D}\longrightarrow\Omega$ tale che $\varphi(\zeta)=z_0+\zeta\big(\pi(p)-z_0\big)$, troviamo $k_\Omega(p,q) \le \dfrac{1}{2}\log{\dfrac{\delta_\Omega(p)}{\delta_\Omega(q)}}$.

    Adesso, notiamo che l'insieme $K=\{z\in\Omega\mid\delta_\Omega(z)\ge\epsilon\}$ è compatto. Dato $x_0\in\Omega$, basta allora prendere $\alpha=1/2$ e $C=\frac{1}{2}\log{\epsilon}+\text{diam}_{k_\Omega}(K)$ se $x_0\in K$, altrimenti si prende $C=\max\{\frac{1}{2}\log{\epsilon},D_{k_{\Omega}}(x_0,K)\}$ dove abbiamo posto $D_{k_\Omega}(x_0,K)=\max\{k_\Omega(x_0,z)\mid z\in K\}$. Segue dunque che $\Omega$ soddisfa la condizione (2) nella Definizione \ref{gold}.
\end{proof}
\subsection{Domini Caltrop}
Il secondo esempio, introdotto in \cite{BM}, è quello dei domini Caltrop.

\begin{defn} \label{defcaltrop}
    Un dominio limitato $\Omega\subseteq\mathbb{C}^n$, con $n\ge 2$, è detto \textit{dominio Caltrop} se esiste un insieme finito di punti $\{q_1,\dots,q_N\}\subseteq\partial\Omega$ tale che:
    \begin{itemize}
        \item il sottoinsieme del bordo $\partial\Omega\setminus\{q_1,\dots,q_N\}$ è $C^2$ e $\Omega$ è strettamente pseudoconvesso in ogni punto di tale insieme;
        \item per ogni $j=1,\dots, N$ esiste un intorno aperto e connesso $V_j\ni q_j$ tale che esistono una trasformazione unitaria $\mathbb{U}^{(j)}$ e una funzione continua $\psi_j:[0,A_j]\longrightarrow[0,+\infty)$, con $A_j>0$, tali che $\mathbb{U}_j(\Omega\cap V_j)$ è dato da
        \begin{align*}
            \mathbb{U}_j(\Omega\cap V_j)=\Bigg\{&(z_1,\dots,z_n)\in\mathbb{C}^n\mid \mathfrak{Re}z_n\in (0,A_j),\\
            &\left.(\mathfrak{Im}z_n)^2+\sum_{j=1}^{n-1}|z_j|^2<\psi_j(\mathfrak{Re}z_n)^2\right\},
        \end{align*}
        dove $\mathbb{U}_j(z)=\mathbb{U}^{(j)}(z-q_j)$ per ogni $z\in\mathbb{C}^n$. Inoltre, esistono due costanti $p_j\in(1,3/2)$ e $C_j>1$ tali che $\psi_j$ ha le seguenti proprietà:
        \begin{itemize}
            \item è di classe $C^2$ su $(0,A_j)$;
            \item per ogni $x\in[0,A_j]$ si ha $(1/C_j)x^{p_j} \le \psi_j(x) \le C_jx^{p_j}$;
            \item si ha che $\psi_j$ è strettamente crescente e convessa;
            \item si ha $\displaystyle\lim_{x\longrightarrow0^+}\psi_j(x)\psi_j''(x)=0$.
        \end{itemize}
    \end{itemize}
\end{defn}

\begin{oss}
    La condizione $p_j>1$ è necessaria perché in seguito ci servirà poter definire $\alpha=\dfrac{1}{p-1}$ per $p=p_j$; la condizione $p_j<3/2$, invece, serve per una stima all'interno di una dimostrazione.
\end{oss}

\begin{oss} \label{cuspidiii}
    Il nome, che in italiano può essere tradotto come tribolo o ``piede di corvo'' (un'arma da lancio a quattro punte), rimanda al fatto che, vicino ai punti in cui non è liscio, il bordo di tali domini assume una forma simile a quella di una cuspide hölderiana non eccessivamente appuntita.
\end{oss}

\begin{figure}[h!]
    \begin{center}
        \includegraphics[width=0.7\textwidth, trim=0 4cm 0 2.5cm]{Immagini/caltrop.png} \\
        \caption{proiezione a $\mathfrak{Im}z=0$ del bordo della punta in $\mathbb{C}^2$ con coordinate $(z,w)$ corrispondente a $\psi(x)=x^{5/4}$}
    \end{center}
\end{figure}

Ci occupiamo adesso di mostrare che i domini Caltrop esistono. Vediamo l'esempio di un dominio Caltrop con una sola punta in $\mathbb{C}^2$. Siano $A,\beta>0$ e sia $\psi:[-A,\beta]\longrightarrow[0,+\infty)$ una funzione continua di classe $C^2$ su $(-A,\beta)$ tale che:
\begin{enumerate}[label={(\arabic*)}]
    \item per ogni $t\in[-A,-B)$ si ha $\psi(t)=(t+A)^p$;
    \item per ogni $t\in(0,\beta]$ si ha $\psi(t)=\sqrt{\beta^2-t^2}$,
\end{enumerate}
dove $B\in(0,A)$ e $p\in(1,3/2)$. Consideriamo
$$\Omega:=\{(z,w)\in\mathbb{C}^2\mid |z|^2+|\mathfrak{Im}w|^2<C\psi(\mathfrak{Re}w)^2,-A<\mathfrak{Re}w<\beta\},$$
dove $C>0$ è una costante che sceglieremo più avanti.

Poniamo inoltre
$$\rho(z,w):=|z|^2+|\mathfrak{Im}w|^2-C\psi(\mathfrak{Re}w)^2,$$
considerata sull'insieme $\{(z,w)\in\mathbb{C}^2\mid -A<\mathfrak{Re}w<\beta+\epsilon\}$, dove $\epsilon>0$ è fissato e $\psi^2$ è estesa a $\beta^2-x^2$ per ogni $x\in(\beta,\beta+\epsilon)$. Si verifica che $\rho$ è una funzione di classe $C^2$ avente $\partial\Omega$ come luogo di zeri. Calcoliamo le seguenti derivati parziali seconde:
\begin{gather*}
    \partial^2_{z\bar{z}}\rho\equiv 1;\\
    \partial^2_{z\bar{w}}\rho=\partial^2_{\bar{z}w}\rho\equiv 0;\\
    \partial^2_{w\bar{w}}\rho(z,w)=\frac{1}{2}-\frac{C}{2}\big(\psi''(\mathfrak{Re}w)\psi(\mathfrak{Re}w)+\psi'(\mathfrak{Re}w)^2\big).
\end{gather*}

In particolare, si ha che
$$\partial^2_{w\bar{w}}\rho(z,w)-\frac{1}{2}=-\frac{Cp(2p-1)}{2}(\mathfrak{Re}w+A)^{2(p-1)}$$
per $\mathfrak{Re}w$ sufficientemente vicino a $-A$, che tende crescendo a $0$ per $\mathfrak{Re}w$ che tende descrescendo a $-A$. Allora, essendo $\psi$ di classe $C^2$ su $(-A,\beta)$, scegliendo $C$ sufficientemente piccolo possiamo imporre che $\partial^2_{w\bar{w}}\rho(z,w)\ge\dfrac{1}{4}$ per ogni $w$ tale che $-A<\mathfrak{Re}w\le 0$. Segue che $\partial\Omega\cap\{(z,w)\in\mathbb{C}^2\mid -A<\mathfrak{Re}w\le 0\}$ è un sottoinsieme di punti strettamente pseudoconvessi del bordo di $\Omega$. Per la condizione (2) su $\psi$, anche $\partial\Omega\cap\{(z,w)\in\mathbb{C}^2\mid \mathfrak{Re}w>0\}$ lo è. Le altre proprietà di dominio Caltrop seguono dalla condizione (1) su $\psi$; la punta è in $(0,-A)$.

In \cite[Section 3.2]{BM} vengono costruiti domini Caltrop con un numero arbitrario di punte. \\

Vediamo adesso che i domini Caltrop hanno le proprietà volute. Vogliamo mostrare innanzitutto la condizione di visibilità e il fatto che non possono mai essere domini Goldilocks, per cui i domini con visibilità sono una classe più ampia che contiene sia i domini Goldilocks che i domini Caltrop. Per fare ciò, vogliamo applicare il Teorema \ref{extvis} con $U=\mathbb{C}^n$, per cui $M_{\Omega,U}=M_\Omega$, e $S=\emptyset$. Per mostrare che un dominio Caltrop $\Omega$ soddisfa le ipotesi del Teorema \ref{extvis} l'idea, spiegata in \cite[Section 6]{BM}, è la seguente: si calcola $k_D$ per un dominio planare $D$ che useremo come modello, dopodiché immergeremo copie di $D$ in $\Omega$ in maniera affine, di modo che ogni punto di $\Omega$ sufficientemente vicino al bordo sia contenuto in una di queste copie. A questo punto, useremo la Proposizione \ref{semicontr} per stimare la distanza di Kobayashi su $\Omega$. 

Per essere precisi, useremo una classe di domini con certe proprietà, che adesso andiamo a costruire. Dati $a,h>0$, poniamo
$$S_{a,h}=\{z\in\mathbb{C}\mid\mathfrak{Re}z>a\text{ e }-h<\mathfrak{Im}z<h\}.$$

Indichiamo con $T_{a,h}$ l'immagine di $S_{a,h}$ tramite la mappa $z\longmapsto 1/z$ e notiamo che
$$T_{a,h}=\left(\mathbb{C}\setminus\overline{D\left(\frac{-i}{2h},\frac{1}{2h}\right)}\right)\cap\left(\mathbb{C}\setminus\overline{D\left(\frac{i}{2h},\frac{1}{2h}\right)}\right)\cap D\left(\frac{1}{2a},\frac{1}{2a}\right).$$

Indichiamo con $\mathcal{Q}^{\alpha,a,h}$ l'immagine di $T_{a,h}$ tramite la mappa $\phi_\alpha(z)=z^{\alpha}$, dove $\alpha$ è un reale maggiore di $1$ e $a$ e $h$ sono scelti in modo che $\phi_\alpha$ sia un biolomorfismo.

\begin{figure}[h!]
    \begin{center}
        \includegraphics[width=0.45\textwidth, trim=0 1cm 0 0]{Immagini/qalphaah.pdf} \\
        \caption{Il dominio $\mathcal{Q}^{5,\frac{1}{2},\frac{1}{10}}$}\label{siriooo}
    \end{center}
\end{figure}

Osserviamo che $T_{a,h}$ ha una cuspide quadratica in $0$. Dunque esistono due costanti $c_1,c_2>0$ tali che per ogni $z\in\partial T_{a,h}$ si ha
\begin{equation} \label{cusp_estimate}
    c_1(\mathfrak{Re}z)^2 \le |\mathfrak{Im}z| \le c_2(\mathfrak{Re}z)^2
\end{equation}
per $\mathfrak{Re}z$ sufficientemente piccola. Più precisamente, per $\delta>0$ sufficientemente piccolo l'insieme $\partial T_{a,h}\cap\{z\in\mathbb{C}\mid 0 \le \mathfrak{Re}\le\delta\}$ è dato dall'unione dei grafici di $f$ e $-f$, dove, posto $z=x+iy$, si ha $f(x)=\dfrac{1}{2h}-\sqrt{\dfrac{1}{4h^2}-x^2}$, per cui abbiamo che
\begin{equation} \label{cerchio}
    f(x)=hx^2+O(x^4)
\end{equation}
per $x\longrightarrow0^+$.

\begin{oss}
    Scrivendo l'equazione esplicita per il bordo di $\mathcal{Q}^{2,a,1}$ vicino a $0$, con $a$ sufficientemente grande, troviamo una cardioide, della quale ricordiamo la cuspide proprio in $0$. Tuttavia, sebbene semplice da calcolare, il caso $\alpha=2$ non è contemplato per via di una costrizione che imporremo più avanti.
\end{oss}

Il seguente risultato sulle proprietà dei domini $\mathcal{Q}^{\alpha,a,h}$ sarà quello usato per ottenere stime sulla distanza di Kobayashi dei domini Caltrop.

\begin{prop} \label{qaah_biolo}
    Sia $\alpha>1$ e sia $\mathcal{Q}^{\alpha,a,h}$ come sopra, con $a,h>0$ scelti opportunamente. Poniamo $p=(1+\alpha)/\alpha$; allora
    \begin{enumerate}[label={(\arabic*)}]
        \item esistono delle costanti $\epsilon,C_1,C_2>0$ tali che, per ogni $z\in\partial\mathcal{Q}^{\alpha,a,h}$ con $0\le\mathfrak{Re}z\le\epsilon$, si ha che
        $$C_1(\mathfrak{Re}z)^p \le |\mathfrak{Im}z| \le C_2(\mathfrak{Re}z)^p;$$
        \item fissata una costante $M>1$ esiste $\epsilon>0$ sufficientemente piccolo tale che la disuguaglianza al punto (1) vale con $C_2=Mh\alpha$. Inoltre, fissati $\alpha>1$ e $h>0$, tale scelta di $\epsilon$ decresce al crescere di $a$;
        \item fissiamo un punto $x_0\in\mathcal{Q}^{\alpha,a,h}\cap\mathbb{R}$. Esiste una costante $C=C(x_0)>0$ tale che per ogni $x\in(0,x_0)$ si ha che
        $$k_{\mathcal{Q}^{\alpha,a,h}}(x_0,x) \le C+\frac{\pi}{4h}x^{-1/\alpha}.$$
    \end{enumerate}
\end{prop}

\begin{proof}
    Siano $c_1$ e $c_2$ le costanti date in \eqref{cusp_estimate}, e sia $f$ la funzione di \eqref{cerchio}, cioè $f(x)=\dfrac{1}{2h}-\sqrt{\dfrac{1}{4h^2}-x^2}$. Vogliamo studiare l'immagine dei grafici di $f$ e $-f$ tramite $\phi_\alpha$. Per simmetria, ci basterò studiare l'immagine del grafico di $f$. Scriviamo $z$ nel grafico di $f$ sufficientemente vicino a $0$ come $z=x+iy$, con $x \ge 0$ e $c_1x^2\le y\le c_2x^2$. Per $x>0$ sufficientemente piccolo, svolgiamo il seguente conto:
    \begin{align*}
        \phi_\alpha(z)&=(x+iy)^{\alpha}\\
        &=\begin{aligned}[t]&x^{\alpha}\left(1+\sum_{j=1}^{+\infty}\frac{(-1)^j}{(2j)!}\prod_{\nu=0}^{2j-1}(\alpha-\nu)\frac{y^{2j}}{x^{2j}}\right)\\
        &+ix^{\alpha}\left(\sum_{j=0}^{+\infty}\frac{(-1)^j}{(2j+1)!}\prod_{\nu=0}^{2j}(\alpha-\nu)\frac{y^{2j+1}}{x^{2j+1}}\right).\end{aligned}
    \end{align*}

    Usando il fatto che $c_1x^2\le y\le c_2x^2$, si vede facilmente che
    \begin{gather*}
        \mathfrak{Re}\big(\phi_\alpha(z)\big)=x^\alpha+O(x^{2+\alpha})\\
        \text{ e }\\
        c_1\alpha x^{1+\alpha}\big(1-O(x^2)\big) \le \mathfrak{Im}\big(\phi_\alpha(z)\big) \le c_2\alpha x^{1+\alpha}\big(1+O(x^2)\big)
    \end{gather*}
    per $z=x+iy$ nel grafico di $f$ e $x>0$ sufficientemente piccolo. Da queste disuguaglianze segue la tesi del punto (1).

    Il punto (2) segue dalle stime con le quali abbiamo dimostrato il punto (1), e da come l'intervallo in cui ci interessa studiare $f$ dipende, per costruzione, da $a$.

    Consideriamo il biolomorfismo $\Phi_{\alpha,a,h}$ da $\mathcal{Q}^{\alpha,a,h}$ in $\mathbb{D}$ dato da
    $$\Phi_{\alpha,a,h}=f_3\circ f_2\circ f_1\circ g\circ(\phi_\alpha\restrict{T_{a,h}})^{-1},$$
    dove
    \begin{align*}
        g(z)&=1/z\text{ per ogni }z\in T_{a,h},\\
        f_1(z)&=\frac{\pi i}{2h}(z-a)\text{ per ogni }z\in S_{a,h},\\
        f_2(z)&=\sin{z}\text{ per ogni }z\in \mathbb{C}\text{ con }-\pi/2<\mathfrak{Re}z<\pi/2\text{ e }\mathfrak{Im}z>0,\\
        f_3(z)&=\frac{z-i}{z+i}\text{ per ogni }z\in\mathbb{C}\text{ con }\mathfrak{Im}z>0.
    \end{align*}

    Osserviamo che $\Phi_{\alpha,a,h}$ manda l'intervallo chiuso e limitato $\overline{\mathcal{Q}^{\alpha,a,h}\cap\mathbb{R}}$ omeomorficamente in $[-1,1]$. Inoltre, manda il punto $o=\dfrac{1}{\left(\frac{2h}{\pi}\log(\sqrt{2}+1)+a\right)^\alpha}$ in $0$, e se $x\in\mathcal{Q}^{\alpha,a,h}\cap\mathbb{R}$ è minore di $o$ allora $\Phi_{\alpha,a,h}(x)\in(0,1)$. Allora per tali $x$ si ha che
    $$k_{\mathcal{Q}^{\alpha,a,h}}(o,x)=k_{\mathbb{D}}\big(0,\Phi_{\alpha,a,h}(x)\big)=\frac{1}{2}\log\left(\frac{1+\Phi_{\alpha,a,h}(x)}{1-\Phi_{\alpha,a,h}(x)}\right).$$

    Calcolando esplicitamente $\Phi_{\alpha,a,h}(x)$, troviamo che
    \begin{align*}
        \frac{1}{2}\log\left(\frac{1+\Phi_{\alpha,a,h}(x)}{1-\Phi_{\alpha,a,h}(x)}\right)&=\frac{1}{2}\log\left(e^{\frac{\pi}{2h}\left(\frac{1}{x^{1/\alpha}}-a\right)}-e^{-\frac{\pi}{2h}\left(\frac{1}{x^{1/\alpha}}-a\right)}\right)-\frac{\log{2}}{2}\\
        &\le \frac{1}{2}\log\left(e^{\frac{\pi}{2h}\left(\frac{1}{x^{1/\alpha}}-a\right)}\right) \le \frac{\pi}{4h}x^{-1/\alpha};
    \end{align*}
    usando anche la disuguaglianza triangolare, otteniamo così la tesi del punto (3).
\end{proof}

I prossimi risultati saranno quelli necessari a immergere affinamente copie di $\mathcal{Q}^{\alpha,a,h}$ in un dominio Caltrop nel modo voluto. Nel seguito, con $o$ indichiamo il punto introdotto nella dimostrazione della Proposizione \ref{qaah_biolo}, associato al dominio $\mathcal{Q}^{\alpha,a,h}$ che staremo trattando.

\begin{lm} \label{superadd}
    Siano $\epsilon>0$ e $\phi:[0,\epsilon)\longrightarrow\mathbb{R}$ una funzione continua, strettamente crescente e derivabile in $(0,\epsilon)$. Supponiamo che $\phi'$ sia strettamente crescente e che $\phi(0)=0$. Allora per ogni $(x,y)\in[0,+\infty)\times[0,+\infty)$ tale che $x+y<\epsilon$ si ha che $\phi(x+y) \ge \phi(x)+\phi(y)$.
\end{lm}

\begin{proof}
    È una banale conseguenza del teorema fondamentale del calcolo integrale.
\end{proof}

\begin{lm} \label{psierre}
    Siano $A>0$ e $\psi:[0,A]\longrightarrow[0,+\infty)$ una funzione continua di classe $C^2$ su $(0,A)$, e sia $p\in(1,2)$. Supponiamo inoltre che:
    \begin{itemize}
        \item esiste una costante $C>1$ tale che $x^p/C\le\psi(x)\le Cx^p$ per ogni $x\in[0,A]$;
        \item $\psi$ è strettamente crescente;
        \item $\psi'$ è strettamente crescente su $(0,A)$.
    \end{itemize}

    Poniamo $\mathcal{R}:=\left\{z\in\mathbb{C}\mid 0<\mathfrak{Re}z<A\text{ e }|\mathfrak{Im}z|<\psi\bigl(\mathfrak{Re}z\bigr)\right\}$. Allora esistono una costante $B\in(0,A)$, un compatto $K$ che interseca $\{z\in\mathbb{C}\mid\mathfrak{Re}z=A\}$ e tale che $K\setminus\{z\in\mathbb{C}\mid\mathfrak{Re}z=A\}\subsetneq\mathcal{R}$, e due costanti $a,h>0$ tali che per ogni $x+iy\in\mathcal{R}$ con $x\le B$ si ha che:
    \begin{enumerate}[label={(\arabic*)}]
        \item $\bigl(\psi^{-1}(|y|)+iy\bigr)+\mathcal{Q}^{1/(p-1),a,h}\subseteq\mathcal{R}$;
        \item $\psi^{-1}(|y|)+o>x$;
        \item $\bigl(\psi^{-1}(|y|)+iy\bigr)+o\in K$;
        \item $\delta_{\mathcal{R}}(x+iy) \le |\psi^{-1}(|y|)-x|$.
    \end{enumerate}
\end{lm}

\begin{proof}
    Per il punto (2) della Proposizione \ref{qaah_biolo}, possiamo fissare una costante $M>1$ tale che per ogni $\alpha>1$ e ogni $a,h>0$ esiste $\epsilon=\epsilon(\alpha,a,h)>0$ tale che
    \begin{align*}
        \mathcal{Q}^{\alpha,a,h}\cap\{\mathfrak{Re}w<\epsilon\}&\subseteq \{w\in\mathbb{C}\mid 0<\mathfrak{Re}w<\epsilon\text{ e }|\mathfrak{Im}w|<Mh\alpha(\mathfrak{Re}w)^{(1+\alpha)/\alpha}\}\\
        &=:S^{\alpha,a,h},
    \end{align*}
    e tale che, per $\alpha$ e $h$ fissati, $\epsilon\longrightarrow0$ per $a\longrightarrow+\infty$; notiamo anche che possiamo imporre $\mathcal{Q}^{\alpha,a,h}\cap\{\mathfrak{Re}w<\epsilon\}=\mathcal{Q}^{\alpha,a,h}$ per $a$ sufficientemente grande. Fissiamo adesso $\alpha=1/(p-1)$. Poiché, per costruzione di $\mathcal{Q}^{\alpha,a,h}$, a parte reale fissata di un punto del bordo la parte immaginaria decresce, la costante $\epsilon$ scelta non decresce al decrescere di $h$. Allora possiamo scegliere $a$ sufficientemente grande e $h$ sufficientemente piccolo, in modo che $\mathcal{Q}^{\alpha,a,h}\cap\{\mathfrak{Re}w<\epsilon\}=\mathcal{Q}^{\alpha,a,h}$ e $\epsilon,o<A/2$, e $Mh\alpha<1/C$.
    
    Adesso fissiamo una costante $B\in(0,A)$ tale che $B<\min\{o,\epsilon/2\}$. Sia $z=x+iy\in\mathcal{R}$ con $x\le B$. Consideriamo l'insieme $\bigl(\psi^{-1}(|y|)+iy\bigr)+\mathcal{Q}^{\alpha,a,h}$. Un elemento arbitrario di questo insieme è della forma $\bigl(\psi^{-1}(|y|)+s\bigr)+i(y+t)$, con $s+it\in\mathcal{Q}^{\alpha,a,h}$. Dato che $\mathcal{Q}^{\alpha,a,h}\subseteq S^{\alpha,a,h}$, si ha $0<s<\epsilon$ e $|t|<Mh\alpha s^p$. Il punto $\bigl(\psi^{-1}(|y|)+s\bigr)+i(y+t)$ sta in $\mathcal{R}$ se e solo \begin{gather*}
        0<\psi^{-1}(|y|)+s<A\\
        \text{e}\\
        |y+t|<\psi\big(\psi^{-1}(|y|)+s\big).
    \end{gather*}

    Poiché $x+iy\in\mathcal{R}$ e $\psi$ è strettamente crescente, si ha $0 \le \psi^{-1}(|y|)<x\le B$, per cui $0<\psi^{-1}(|y|)+s<\epsilon/2+\epsilon<A$. Dunque, per mostrare il punto (1), ci resta da dimostrare che $|y+t|<\psi\big(\psi^{-1}(|y|)+s\big)$. Notiamo che $\psi$ soddisfa le ipotesi del Lemma \ref{superadd}, da cui
    $$\psi\big(\psi^{-1}(|y|)+s\big)\ge |y|+\psi(s)\ge |y|+s^p/C,$$
    dove la prima disuguaglianza è data dal lemma e la seconda è vera per ipotesi. Allora si ha che
    $$|y+t| \le |y|+|t|<|y|+Mh\alpha s^p<|y|+s^p/C \le \psi\big(\psi^{-1}(|y|)+s\big),$$
    come voluto.

    Per ogni $x+iy\in\mathcal{R}$ con $x\le B$ si ha $\psi^{-1}(|y|)+o>B \ge x$ per come è stato scelto $B$, e questo dimostra il punto (2).

    Definiamo $K:=\left\{z\in\mathbb{C}\mid o \le \mathfrak{Re}z\le A\text{ e }|\mathfrak{Im}z| \le \psi\big(\mathfrak{Re}(z)-o\big)\right\}$. Per ogni $x+iy\in\mathcal{R}$ con $x\le B$ si ha
    $$o \le o+\psi^{-1}(|y|) <o+x \le o+B<2o<A.$$

    Inoltre $|y|=\psi\Big(\big(\psi^{-1}(|y|)+o\big)-o\Big)$, per cui $o+\big(\psi^{-1}(|y|)+iy\big)\in K$. Per costruzione, $K$ è un compatto che soddisfa le condizioni richieste, per cui abbiamo mostrato il punto (3).

    Infine, per ogni $x+iy\in\mathcal{R}$ con $x\le B$ si ha che $\psi^{-1}(|y|)+iy\in\partial\mathcal{R}$, per cui $\delta_{\mathcal{R}}(x+iy) \le \left|\big(\psi^{-1}(|y|)+iy\big)-(x+iy)\right|=|\psi^{-1}(|y|)-x|$; questo dimostra il punto (4).
\end{proof}

Il prossimo risultato, che è sostanzialmente una versione parametrizzata del precedente, tratta dell'immersione del dominio modello $\mathcal{Q}^{\alpha,a,h}$ in un dominio Caltrop con una punta. Per brevità, scriveremo $(z_1,\dots,z_{n-1},z_n)\in\mathbb{C}^n$ come $(z',z_n)$.

\begin{lm}\label{6punto4}
    Siano $A>0$ e $\psi:[0,A]\longrightarrow[0,+\infty)$ una funzione continua di classe $C^2$ su $(0,A)$, e sia $p\in(1,2)$. Supponiamo inoltre che:
    \begin{itemize}
        \item esiste una costante $C>1$ tale che $x^p/C\le\psi(x)\le Cx^p$ per ogni $x\in[0,A]$;
        \item $\psi$ è strettamente crescente;
        \item $\psi'$ è strettamente crescente su $(0,A)$.
    \end{itemize}

    Sia
    $$D:=\{z\in\mathbb{C}^n\mid 0<\mathfrak{Re}z_n<A\text{ e }(\mathfrak{Im}z_n)^2+\|z'\|^2<\big(\psi(\mathfrak{Re}z_n)\big)^2\}.$$

    Sia $w'\in\mathbb{C}^{n-1}$ e poniamo
    $$\mathcal{R}_{w'}:=\pi_n\left[\big((w',0)+\{0'\}\times\mathbb{C}\big)\cap D\right],$$
    dove $\pi_n$ è la proiezione sull'ultima coordinata. Sia $\alpha=1/(p-1)$. Allora esistono delle costanti $a,h,B>0$ e un compatto $K\subseteq\{z\in\mathbb{C}^n\mid\mathfrak{Re}z_n\le A\}$, che interseca $\{z\in\mathbb{C}^n\mid\mathfrak{Re}z_n=A\}$ e tale che $K\setminus\{z\in\mathbb{C}^n\mid\mathfrak{Re}z_n=A\}\subsetneq D$, tali che per ogni $w'\in\mathbb{C}^{n-1}$ con $\|w'\|<\psi(B/2)$ e ogni $\zeta\in\mathcal{R}_{w'}$ con $\mathfrak{Re}\zeta\le B$ si ha che:
    \begin{enumerate}[label={(\arabic*)}]
        \item $\Big(\psi^{-1}\big(S(\zeta,w')\big)+i\mathfrak{Im}\zeta\Big)+\mathcal{Q}^{\alpha,a,h}\subseteq\mathcal{R}_{w'}$;
        \item $\psi^{-1}\big(S(\zeta,w')\big)+o>\mathfrak{Re}\zeta$;
        \item $\Big(\psi^{-1}\big(S(\zeta,w')\big)+i\mathfrak{Im}\zeta\Big)+o\in\pi_n\left[\big((w',0)+\{0_{n-1}\}\times\mathbb{C}\big)\cap K\right]$;
        \item $\delta_D\big((w',\zeta)\big) \le \left|\mathfrak{Re}\zeta-\psi^{-1}\big(S(\zeta,w')\big)\right|$,
    \end{enumerate}
    dove $S(\zeta,w')=\sqrt{(\mathfrak{Im}\zeta)^2+\|w'\|^2}$.
\end{lm}

\begin{proof}
    Notiamo che
    \begin{align*}
        \mathcal{R}_{w'}&=\{\zeta\mid (w',\zeta)\in D\}\\
        &=\{\zeta\mid0<\mathfrak{Re}\zeta<A\text{ e }(\mathfrak{Im}\zeta)^2+\|w'\|^2<\big(\psi(\mathfrak{Re}\zeta)\big)^2\}\\
        &=\{\zeta\mid \psi^{-1}(\|w'\|)<\mathfrak{Re}\zeta<A\text{ e }(\mathfrak{Im}\zeta)^2+\|w'\|^2<\big(\psi(\mathfrak{Re}\zeta)\big)^2\},
    \end{align*}
    per cui $\mathcal{R}_{w'}\not=\emptyset$ se e solo se $\|w'\|<\psi(A)$; poiché la costante $B$ sarà scelta in $(0,A)$, nel nostro caso $\mathcal{R}_{w'}$ sarà sempre non vuoto. Notiamo inoltre che l'insieme $\mathcal{R}_{0'}$ coincide con l'insieme $\mathcal{R}$ del Lemma \ref{psierre}. Prendiamo dunque $a,h$ e $B$ date da tale lemma. Scriviamo per semplicità $c=1/C$, e osserviamo che dalla dimostrazione del Lemma \ref{psierre} discendono le seguenti disuguaglianze:
    \begin{gather}
        B+s<3A/4<A\text{ e }|t|<cs^p\text{ per ogni }s+it\in\mathcal{Q}^{\alpha,a,h};\label{qminoredia}\\
        o>B.\label{obbi}
    \end{gather}

    Supponiamo ora che $w'\not=0'$. Sia $\zeta\in\mathcal{R}_{w'}$ tale che $\mathfrak{Re}\zeta\le B$. Un elemento arbitrario dell'insieme $\Big(\psi^{-1}\big(S(\zeta,w')\big)+i\mathfrak{Im}\zeta\Big)+\mathcal{Q}^{\alpha,a,h}$ è della forma $\Big(\psi^{-1}\big(S(\zeta,w')\big)+s\Big)+i(\mathfrak{Im}\zeta+t)$ con $s+it\in\mathcal{Q}^{\alpha,a,h}$. Un tale punto appartiene a $\mathcal{R}_{w'}$ se e solo se:
    \begin{nlist}
        \item si ha $\psi^{-1}\big(S(\zeta,w')\big)+s<A$;
        \item si ha $\|w'\|^2+(\mathfrak{Im}w+t)^2<\bigg(\psi\Big(\psi^{-1}\big(S(\zeta,w'\big)+s)\Big)\bigg)^2$.
    \end{nlist}

    Poiché $\zeta\in\mathcal{R}_{w'}$ e $\mathfrak{Re}\zeta\le B$, abbiamo che
    $$(\mathfrak{Im}\zeta)^2+\|w'\|^2<\big(\psi(\mathfrak{Re}\zeta)\big)^2<\big(\psi(B)\big)^2;$$
    quindi $\psi^{-1}\big(S(\zeta,w')\big)+s<B+s<A$, dove l'ultima disuguaglianza è la prima in \eqref{qminoredia}. È così verificata la condizione (i). Vediamo ora la condizione (ii). Per il Lemma \ref{superadd} e per le ipotesi su $\psi$ si ha che
    $$\psi\Big(\psi^{-1}\big(S(\zeta,w'\big)+s)\Big)\ge S(\zeta,w')+\psi(s) \ge S(\zeta,w')+cs^p;$$
    dunque
    $$\bigg(\psi\Big(\psi^{-1}\big(S(\zeta,w'\big)+s)\Big)\bigg)^2-(\mathfrak{Im}\zeta)^2-\|w'\|^2\ge 2cs^pS(\zeta,w')+c^2s^{2p}.$$

    Allora la condizione (ii) segue se mostriamo che
    $$2t\mathfrak{Im}\zeta+t^2<2cs^pS(\zeta,w')+c^2s^{2p},$$
    ma quest'ultima disuguaglianza segue dalla seconda disuguaglianza in \eqref{qminoredia}. Abbiamo così dimostrato il punto (1).

    Adesso notiamo che
    $$\psi^{-1}\big(S(\zeta,w')\big)+o \ge \psi^{-1}(\mathfrak{Im}\zeta)+o>B\ge\mathfrak{Re}\zeta,$$
    dove la seconda disuguaglianza segue da \eqref{obbi}; questo dimostra il punto (2).

    Poniamo $K:=\{(w',\zeta)\in\mathbb{C}^n\mid o\le\mathfrak{Re}\zeta\le A\text{ e }S(\zeta,w')\le\psi(\mathfrak{Re}\zeta-o)\}$. Si verifica facilmente che $K$ è un compatto con le proprietà richieste. Sia $\zeta\in\mathcal{R}_{w'}$ tale che $\mathfrak{Re}\zeta\le B$ e poniamo $\eta:=\Big(\psi^{-1}\big(S(\zeta,w')\big)+i\mathfrak{Im}\zeta\Big)+o$; allora abbiamo che
    $$o\le\psi^{-1}(\|w'\|)+o\le\mathfrak{Re}\eta<\mathfrak{Re}\zeta+o\le B+o<A.$$

    Inoltre, si ha che $S(\eta,w')=S(\zeta,w')=\psi(\mathfrak{Re}\eta-o)$; quindi
    $$\eta\in\pi_n\left[\big((w',0)+\{0'\}\times\mathbb{C}\big)\cap K\right],$$
    e questo dimostra il punto (3).

    Per il punto (4), se $(w',\zeta)$ è preso come sopra, allora
    $$\psi^{-1}\big(S(\zeta,w')\big)+i\mathfrak{Im}\zeta\in\partial\mathcal{R}_{w'};$$
    dunque
    $$\delta_D\big((w',\zeta)\big)\le \delta_{\mathcal{R}_{w'}}(\zeta)\le\left|\mathfrak{Re}\zeta-\psi^{-1}\big(S(\zeta,w')\big)\right|,$$
    come voluto.
\end{proof}

Ci serviranno anche i seguenti risultati.

\begin{lm} \label{analisibase}
    Siano $A>0$, $p>1$ e $\psi:[0,A]\longrightarrow[0,+\infty)$ una funzione tale che:
    \begin{itemize}
        \item è di classe $C^1$ su $(0,A)$;
        \item esiste $C>1$ tale che per ogni $x\in[0,A]$ si ha $\psi(x) \le Cx^{p}$;
        \item $\psi'$ è crescente su $(0,A)$.
    \end{itemize}

    Allora $\psi$ è derivabile in $0$ e $\psi'$ è continua su $[0,A)$, per cui $\displaystyle\lim_{x\longrightarrow0^+}\psi'(x)=0$.
\end{lm}

\begin{proof}
    Che $\psi'(0)$ esista e sia uguale a $0$ segue immediatamente dalla stima su $\psi(x)$. Dunque $\psi'$ si estende a una funzione su $[0,A)$. Poiché $\psi'$ è monotona crescente, esiste $\displaystyle\lim_{x\longrightarrow0^+}\psi'(x)=l<+\infty$. Segue allora da risultati elementari di analisi in una variabile che $l=0$.
\end{proof}

\begin{lm} \label{pshestimate}
    (Conseguenza della dimostrazione di \cite[Proposition 6]{Si}) Siano $\Omega\subseteq\mathbb{C}^n$ un dominio e $p\in\Omega$. Supponiamo che esista una funzione $u$ plurisubarmonica, negativa, di classe $C^2$ in un intorno di $p$ e tale che esiste $c>0$ per cui si ha
    $$L_u(p;v) \ge c\|v\|^2$$
    per ogni $v\in\mathbb{C}^n$. Allora esiste una costante $\alpha>0$ tale che
    $$K_\Omega(p;v)\ge\left(\frac{c}{\alpha}\right)^{1/2}\frac{\|v\|}{|u(p)|^{1/2}}$$
    per ogni $v\in\mathbb{C}^n$.
\end{lm}

\begin{lm}\label{psdcvxcpt}
    (Conseguenza di \cite[Theorem B]{M}) Sia $\Omega\subseteq\mathbb{C}^n$ con $n\ge 2$ un dominio limitato. Sia $\mathcal{M}_0$ un insieme aperto in $\partial\Omega$ che sia anche un'ipersuperficie di classe $C^2$. Assumiamo che $\mathcal{M}_0$ ammetta una funzione di definizione $\phi$ di classe $C^2$ in un qualche aperto contenente $\mathcal{M}_0$, e tale che esista una costante $\delta>0$ per cui $L_\phi(\xi;v)\ge\delta\|v\|^2$ per ogni $\xi\in\mathcal{M}_0$ e $v\in\mathbb{C}^n$. Sia $\mathcal{M}_1\subsetneq\mathcal{M}_0$ un sottoinsieme compatto. Allora esistono un intorno di $\mathcal{M}_1$ in $\overline{\Omega}$, sia esso $\mathcal{V}$, e due costanti $C,c>0$ tali che $K_\Omega(z;v)\ge \big(1-C\delta_\Omega(z)^{1/2}\big)\dfrac{c\|v\|}{\delta_\Omega(z)^{1/2}}$ per ogni $z\in\mathcal{V}\cap\Omega$ e $v\in\mathbb{C}^n$.
\end{lm}

Siamo ora pronti a dimostrare quello che volevamo.

\begin{thm} \label{ilprecteo}
    (\cite[Theorem 1.4]{BM}) Sia $\Omega$ un dominio Caltrop; allora $\Omega$ è $(\lambda,\kappa)$-visibile per ogni $\lambda \ge 1$ e $\kappa>0$, ma non è un dominio Goldilocks.
\end{thm}

\begin{proof}
    Fissiamo la seguente notazione: date due funzioni non negative $F$ e $G$ che dipendono da alcuni parametri, scriviamo $G\lesssim F$ per dire che esiste una costante $C>0$, indipendente dai parametri, tale che $G\le C\cdot F$. Scriviamo $G\approx F$ per intendere $G\lesssim F$ e $F\lesssim G$.\\

    Passo 1: una stima dal basso per $K_\Omega(w;\cdot)$ per $w$ contenuto in una punta.

    Siano $\{q_1,\dots,q_N\}\subseteq\partial\Omega$ i punti della Definizione \ref{defcaltrop}, e fissiamo $q_{j^*}$ uno di tali punti. Siano $p_{j^*}\in(1,3/2)$ l'esponente, $\mathbb{U}^{(j^*)}$ la trasformazione unitaria e $\psi_{j^*}:[0,A_{j^*}]\longrightarrow[0,+\infty)$ la funzione associati a $q_{j^*}$ nella Definizione \ref{defcaltrop}. Per la Proposizione \ref{metrdecr} abbiamo che $K_\Omega$ è invariante per biolomorfismi, e $\mathbb{U}_{j^*}$ è un biolomorfismo con differenziale $\mathbb{U}^{(j^*)}$ che preserva la norma euclidea; allora possiamo assumere senza perdita di generalità $\mathbb{U}_{j^*}=\id$ e $q_{j^*}=0$, per cui, ponendo $A=A_{j^*}$ e $\psi=\psi_{j^*}$, si ha che
    $$\Omega\cap V_{j^*}=\left\{z\in\mathbb{C}^n\mid 0<\mathfrak{Re}z_n<A\text{ e }(\mathfrak{Im}z_n)^2+\|z'\|^2<\big(\psi(\mathfrak{Re}z_n)\big)^2\right\}.$$

    Poniamo $\rho(z):=(\mathfrak{Im}z_n)^2+\|z'\|^2-\big(\psi(\mathfrak{Re}z_n)\big)^2$ per $z\in\Omega\cap V_{j^*}$. Calcolando le derivate parziali seconde in modo analogo a come fatto nella costruzione dell'esempio a una punta, ricordando le proprietà di $\psi$ e applicando il Lemma \ref{analisibase}, troviamo che esiste una costante $A'\in(0,A]$ tale che
    $$L_\rho(z;v) \ge \|v'\|^2+|v_n|^2/4$$
    per ogni $z\in\Omega\cap V_{j^*}$ con $0<\mathfrak{Re}z_n<A'$ e $v\in\mathbb{C}^n$.

    Siano $U^{(\nu)}$, per $\nu=1,2,3,4$, degli intorni aperti e connessi di $0$ tali che:
    \begin{itemize}
        \item si ha $U^{(1)}\subset\subset U^{(2)}\subset\subset U^{(3)}\subset\subset U^{(4)}$;
        \item si ha $U^{(\nu)}\cap\Omega=\{z\in\Omega\cap V_{j^*}\mid0<\|z\|<\nu A'/4\}$ per $\nu=1,2,3,4$.
    \end{itemize}

    Sia $\chi_1:\mathbb{C}^n\longrightarrow[0,1]$ una funzione liscia con $\chi_1\restrict{U^{(1)}}\equiv 0$ e $\chi_1\restrict{\mathbb{C}^n\setminus U^{(2)}}\equiv 1$, e sia $\phi:[0,+\infty)\longrightarrow[0,+\infty)$ una funzione liscia con le seguenti proprietà:
    \begin{itemize}
        \item è convessa e non decrescente;
        \item è nulla in $[0,(A')^2/16]$;
        \item crece lentamente nell'intervallo $((A')^2/16,(A')^2/4]$;
        \item cresce rapidamente nell'intervallo $[9(A')^2/16,+\infty)$.
    \end{itemize}

    A breve specificheremo cosa intendiamo di preciso con le ultime due condizioni, ma per farlo ci serviranno altre funzioni che adesso definiamo. Poniamo $M_\phi:=\displaystyle\sup_{z\in\Omega}\phi(\|z\|^2)$ e $\Phi(z):=\phi(\|z\|^2)-M_\phi$ per ogni $z\in\Omega$. Abbiamo che $\Phi$ è plurisubarmonica per \cite[Proposition 2.2.6]{Kr}. Si ha allora che
    \begin{align*}
        L_{\rho+\chi_1\Phi}(z;v)&=\begin{aligned}[t]&L_\rho(z;v)+\chi_1(z)L_\Phi(z;v)\\
        &+2\mathfrak{Re}\left(\sum_{j,k=1}^n\partial_{z_j}\chi_1(z)\partial_{\bar{z}_k}\Phi(z)v_j\bar{v}_k\right)+\Phi(z)L_{\chi_1}(z;v)\end{aligned}\\
        &\ge\begin{aligned}[t]&\|v'\|^2+|v_n|^2/4-2\sum_{j,k=1}^n|\partial_{z_j}\chi_1(z)\partial_{\bar{z}_k}\Phi(z)||v_j||\bar{v}_k|\\
        &-|\Phi(z)||L_{\chi_1}(z;v)|\end{aligned}
    \end{align*}
    per ogni $z\in(U^{(2)}\setminus U^{(1)})\cap\Omega$ e $v\in\mathbb{C}^n$. Specifichiamo adesso la prima condizione di crescita di $\phi$. Sull'intervallo $((A')^2/16,(A')^2/4]$ deve crescere così lentamente che
    $$L_{\rho+\chi_1\Phi}(z;v)\ge \|v'\|^2/2+|v_n|^2/8$$
    per ogni $z\in(U^{(2)}\setminus U^{(1)})\cap\Omega$ e $v\in\mathbb{C}^n$.

    Sia adesso $\chi_2:\mathbb{C}^n\longrightarrow[0,1]$ una funzione liscia e a supporto compatto tale che $\chi_2\restrict{U^{(3)}}\equiv1$ e $\chi_2\restrict{\mathbb{C}^n\setminus{U^{(4)}}}\equiv 0$, e chiediamo che il supporto di $\chi_2$ intersecato con $\Omega$ sia contenuto in $\{z\in\Omega\mid \|z\|<7A'/8\}$. Abbiamo allora che
    \begin{align*}
        L_{\chi_2\rho+\Phi}(z;v)&\ge\begin{aligned}[t]&\phi'(\|z\|^2)\|v\|^2+\phi''(\|z\|^2)|\langle z,v\rangle|^2\\
        &-2\sum_{j,k=1}^n|\partial_{z_j}\chi_2(z)\partial_{\bar{z}_k}\rho(z)||v_j||\bar{v}_k|-|\rho(z)||L_{\chi_2}(z;v)|\end{aligned}
    \end{align*}
    per ogni $z\in(U^{(4)}\setminus U^{(3)})\cap\Omega$ e $v\in\mathbb{C}^n$. Specifichiamo adesso la seconda condizione di crescita di $\phi$. Sull'intervallo $[9(A')^2/16,+\infty)$ deve crescere così rapidamente che esiste una costante $c>0$ tale che
    $$L_{\chi_2\rho+\Phi}(z;v)\ge c\|v\|^2$$
    per ogni $z\in(U^{(4)}\setminus U^{(3)})\cap\Omega$ e $v\in\mathbb{C}^n$.

    Poniamo ora $u(z):=\chi_1(z)\Phi(z)+\chi_2(z)\rho(z)$. Ricordiamo che $\Phi$ è plurisubarmonica; allora:
    \begin{itemize}
        \item per il principio del massimo (\cite[Corollary 2.1.5]{Kr}, segue facilmente dalla Definizione \ref{psh} che vale anche per funzioni plurisubarmoniche) applicato a $\Phi$, e per le definizioni di $\chi_1$,$\chi_2$ e $\rho$, abbiamo che $u<0$ su $\Omega$;
        \item dalle disuguaglianze sulla forma di Levi che seguono dalle condizioni imposte su $\phi$, e dalle definizioni di $\chi_1$ e $\chi_2$, segue per \cite[Theorem 2.6.2]{H} che $u$ è plurisubarmonica su $\Omega$.
    \end{itemize}

    Per la simmetria di rotazione della punta $\Omega\cap V_{j^*}$, per ogni $w\in\Omega\cap V_{j^*}$ con $\mathfrak{Re}w_n$ sufficientemente piccolo si ha che $\delta_\Omega(w)=\text{dist}\big(\mathfrak{Re}w_n+iS(w),\text{graph}(\psi)\big)$, dove con $\text{graph}(\psi)$ s'intende il grafico di $\psi$ e $S(w)=\sqrt{(\mathfrak{Im}w_n)^2+\|w'\|^2}$. Segue dunque, da semplici stime geometriche e dal Lemma \ref{analisibase}, ponendo $\xi^w_n=\pi_n(\xi^w)$, che
    \begin{equation}\label{limite_brutto}
        \frac{\delta_\Omega(w)}{\psi(\mathfrak{Re}w_n)-S(w)}=\frac{\left|(\mathfrak{Re}\xi^w_n-\mathfrak{Re}w_n)+i\big(\psi(\mathfrak{Re}\xi^w_n)-S(w)\big)\right|}{\psi(\mathfrak{Re}w_n)-S(w)}\longrightarrow 1
    \end{equation}
    per $\mathfrak{Re}w_n\longrightarrow 0$. Allora esiste una costante $A''>0$ tale che:
    \begin{gather}
        \{z\in\Omega\cap V_{j^*}\mid0<\mathfrak{Re}z_n<A''\}\subseteq\Omega\cap U^{(1)};\label{7punto4}\\
        \psi(x)\in(0,1)\text{ per ogni }x\in(0,A'');\label{7punto4due}\\
        \frac{\delta_\Omega(w)}{\psi(\mathfrak{Re}w_n)-S(w)}>1/2\label{7punto5}
    \end{gather}
    per ogni $w\in\Omega\cap V_{j^*}$ tale che $\mathfrak{Re}w_n\in(0,A'')$.

    Fissiamo ora un punto $w\in\Omega\cap V_{j^*}$ tale che $0<\mathfrak{Re}w_n<A''$; allora esiste una costante $b>0$ tale che
    \begin{equation}\label{7punto6}
        \begin{aligned}
            K_\Omega(w;v)&\ge b\frac{\|v\|}{|u(w)|^{1/2}}\\
            &=b\frac{\|v\|}{\big(\psi(\mathfrak{Re}w_n)-S(w)\big)^{1/2}\big(\psi(\mathfrak{Re}w_n)+S(w)\big)^{1/2}}\\
            &\ge\frac{b}{\sqrt{2}}\cdot\frac{\|v\|}{\big(\psi(\mathfrak{Re}w_n)-S(w)\big)^{1/2}} \ge \frac{b}{2}\cdot\frac{\|v\|}{\delta_\Omega(w)^{1/2}},
        \end{aligned}
    \end{equation}
    dove la prima disuguaglianza segue dal Lemma \ref{pshestimate}, l'uguaglianza segue da \eqref{7punto4} e dalle definizioni di $\chi_1$ e $\chi_2$, e la penultima e ultima disuguaglianza seguono, rispettivamente, da \eqref{7punto4due} e da \eqref{7punto5}.\\

    Passo 2: una stima dall'alto per $M_\Omega$.

    Dato che le stime fatte al passo 1 valgono per un qualsiasi $q_{j^*}$, che sono in numero finito, ne deduciamo che esistono delle costanti $\beta,A_1'',\dots,A_N''>0$ tali che
    \begin{equation}\label{7punto7}
        K_\Omega(w;v)\ge\beta\frac{\|v\|}{\delta_\Omega(w)^{1/2}}
    \end{equation}
    per ogni $w\in\Omega\cap\mathbb{U}_j^{-1}(\{z\mid\in\mathbb{C}^n\mid \mathfrak{Re}z_n<A_j''\})$, per ogni $v\in\mathbb{C}^n$ e per $j=1,\dots,N$. Adesso poniamo
    \begin{gather*}
        \mathcal{M}_0:=\partial\Omega\cap\bigcap_{j=1}^N \mathbb{U}_j^{-1}(\{z\in\mathbb{C}^n\mid \mathfrak{Re}z_n>A_j''/2\}),\\
        \mathcal{M}_1:=\partial\Omega\setminus\bigcup_{j=1}^N \mathbb{U}_j^{-1}(\{z\in\mathbb{C}^n\mid \mathfrak{Re}z_n<A_j''\}).
    \end{gather*}
    
    Dalla Definizione \ref{defcaltrop}, in ogni punto di $\partial\Omega\setminus\{q_1,\dots,q_N\}$ esiste una funzione di definizione locale. Poiché $\mathcal{M}_0$ è relativamente compatto in $\partial\Omega$, usando delle partizioni dell'unità possiamo incollare tra loro un numero finito di tali funzioni per ottenere una funzione di definzione per $\mathcal{M}_0$; siccome possiamo sempre farlo in modo che vicino a un punto specifico sia uguale alla funzione di definizione locale per $\partial\Omega\setminus\{q_1,\dots,q_N\}$, e dato che la stretta pseudoconvessità in un punto è indipendente dalla funzione di definizione scelta (\cite[Section 3.2]{Kr}), usando anche \cite[Proposition 3.2.1]{Kr} si trova che $\mathcal{M}_0$ soddisfa le ipotesi del Lemma \ref{psdcvxcpt}.

    Dunque esistono un intorno $\mathcal{V}$ di $\mathcal{M}_1$ in $\overline{\Omega}$ e una costante $\beta'>0$ tali che
    \begin{equation}\label{7punto8}
        K_\Omega(w;v)\ge \beta'\frac{\|v\|}{\delta_\Omega(w)^{1/2}}
    \end{equation}
    per ogni $w\in\mathcal{V}\cap\Omega$ e $v\in\mathbb{C}^n$. Consideriamo adesso l'insieme
    $$\Omega\setminus\left(\mathcal{V}\cup\bigcup_{j=1}^N \mathbb{U}_j^{-1}(\{z\in\mathbb{C}^n\mid\mathfrak{Re}z_n<A_j''\})\right),$$
    che è compatto per definizione; allora la pseudometrica di Kobayashi ha un minimo positivo su tale insieme. Per la \eqref{7punto7} e la \eqref{7punto8}, segue che
    $$\frac{1}{K_\Omega(w;v)}\lesssim\delta_\Omega(w)^{1/2}$$
    per ogni $w\in\Omega$ e per ogni $v\in\mathbb{C}^n$ con $\|v\|=1$. In particolare, $M_\Omega(r)\lesssim r^{1/2}$.\\

    Passo 3: il comportamento di $k_\Omega$.

    Fissiamo inizialmente un punto $q_j$, e siano $a_j$, $h_j$ e $B_j$ le costanti date dal Lemma \ref{6punto4} con $\psi=\psi_j$. Per semplicità di notazione poniamo $a=a_j$ e $h=h_j$. Consideriamo un punto $w\in\Omega\cap\mathbb{U}_j^{-1}(\{z\in\mathbb{C}^n\mid\mathfrak{Re}z_n<B_j/2\})$ e scriviamo $\mathbb{U}_j(w)=(\omega',\omega_n)$. Sia $\Psi_{j,w}:\mathcal{Q}^{\alpha,a,h}\longrightarrow\mathbb{C}^n$ la funzione olomorfa data da
    $$\psi_{j,w}(\zeta):=\mathbb{U}_j^{-1}\Big(\omega',\psi_j^{-1}\big(S(\omega)\big)+i\mathfrak{Im}\omega_n+\zeta\Big),$$
    con $\alpha=1/(p_j-1)$ e $S(\omega)=\sqrt{(\mathfrak{Im}\omega_n)^2+\|\omega'\|^2}$. Notiamo che questa mappa è un'immersione affine di $\mathcal{Q}^{\alpha,a,h}$ in $\mathbb{C}^n$. Vogliamo vedere che è un'immersione a valori in $\Omega$, per potervi ottenere la stime per $k_\Omega$.

    Notiamo che $\mathfrak{Re}\omega_n<B_j/2$; allora, per come è stato preso $w$ e per la Definizione \ref{defcaltrop}, si ha $\|\omega'\|<\psi_j(B_j/2)$. Segue quindi dai punti (1) e (2) del Lemma \ref{6punto4} che:
    \begin{nlist}
        \item si ha
        \begin{align*}
            \{\omega'\}&\times\Big(\psi_j^{-1}\big(S(\omega)\big)+i\mathfrak{Im}\omega_n+\mathcal{Q}^{\alpha,a,h}\Big)\\
            &\subseteq \{(z',z_n)\in\mathbb{C}^n\mid\mathfrak{Re}z_n\in(0,A_j)\text{ e }(\mathfrak{Im}z_n)^2+\|z'\|^2<\psi_j(\mathfrak{Re}z_n)^2\};
        \end{align*}
        \item si ha $\mathfrak{Re}\omega_n\in\Big(\psi_j^{-1}\big(S(\omega)\big),\psi_j^{-1}\big(S(\omega)\big)+o\Big)$.
    \end{nlist}

    Adesso poniamo $K_j:=\mathbb{U}_j^{-1}(K)$ e $z_w:=\Psi_{j,w}(o)$, dove $K$ è il compatto dato dal Lemma \ref{6punto4}. Allora segue, dalla parte (3) del lemma, che
    \begin{equation}\label{7punto9}
        z_w\in K_j
    \end{equation}
    per ogni $w\in\Omega\cap V_j$ tale che $0<\mathfrak{Re}\omega_n<B_j/2$.

    Dal punto (i) sopra segue che $\Psi_{j,w}(\mathcal{Q}^{\alpha,a,h})\subseteq\Omega$. Per definizione abbiamo che $\Psi_{j,w}\Big(-\psi_j^{-1}\big(S(\omega)\big)+\mathfrak{Re}\omega_n\Big)=w$. Allora, per la Proposizione \ref{semicontr}, si ha $k_\Omega(z_w,w)\le k_{\mathcal{Q}^{\alpha,a,h}}\Big(o,-\psi_j^{-1}\big(S(\omega)\big)+\mathfrak{Re}\omega_n\Big)$. Dal punto (ii) e dalla Proposizione \ref{qaah_biolo} con $x_0=o$ troviamo che esiste una costante $C^{(j)}>0$ tale che $k_\Omega(z_w,w)\le C^{(j)}+\dfrac{\pi}{4h}\left|\mathfrak{Re}\omega_n-\psi_j^{-1}\big(S(\omega)\big)\right|^{-(p_j-1)}$. Dato che $\mathbb{U}_j$ preserva le distanze euclidee, quest'ultima disuguaglianza insieme al punto (4) del Lemma \ref{6punto4} ci dicono che
    \begin{equation}\label{7punto10}
        k_\Omega(z_w,w) \le C^{(j)}+\frac{\pi}{4h}\delta_\Omega(w)^{-(p_j-1)}
    \end{equation}
    per ogni $w\in\Omega\cap V_j$ tale che $0<\mathfrak{Re}\omega_n<B_j/2$. Siccome il punto $q_j$ è stato scelto arbitrariamente, la \eqref{7punto9} e la \eqref{7punto10} valgono per ogni $j=1,\dots,N$.

    Dato che $\partial\Omega$ è di classe $C^2$ all'infuori dei punti $q_1,\dots,q_N$ e $\Omega$ è limitato, esistono un compatto $K_0\subseteq\Omega$ e una costante $R>0$ tali che per ogni punto
    \begin{equation}\label{7punto11}
        w\in\Omega\setminus\left(K_0\cup\bigcup_{j=1}^N\mathbb{U}_j^{-1}(\{z\in\mathbb{C}^n\mid\mathfrak{Re}z_n<B_j/2\})\right)
    \end{equation}
    esiste un punto
    $$\xi^w\in\partial\Omega\setminus\bigcup_{j=1}^N\mathbb{U}_j^{-1}(\{z\in\mathbb{C}^n\mid\mathfrak{Re}z_n<B_j/4\})$$
    tale che, detto $\eta^w$ il vettore unitario normale a $\partial\Omega$ entrante in $\xi^w$, allora:
    \begin{itemize}
        \item si ha $\xi^w+D(R;R)\eta^w\subseteq \Omega$;
        \item si ha che $w$ appartiene al segmento congiungente $\xi^w$ a $\xi^w+R\eta^w=:z^w$;
        \item si ha $z^w\in K_0$.
    \end{itemize}
    Per trovare $K_0$ e $R$, basta considerare un intorno tubolare dato \cite[Chapter 9, Theorem 20]{Sp}. Chiediamo anche che $\delta_\Omega(w)<1$ per ogni $w\not\in K_0$. Per ogni tale $w$ esiste un unico numero $t(w)\in(0,R)$ tale che $\xi^w+t(w)\eta^w=w$. Da ciò, e dalla Proposizione \ref{semicontr}, segue che
    \begin{equation}\label{7punto12}
        \begin{aligned}
            k_\Omega(z^w,w)&\le k_{D(R;R)}\big(0,1-t(w)\big)=\frac{1}{2}\log\left(\frac{2-t(w)/R}{t(w)/R}\right)\\
            &\le\log{\sqrt{2R}}+\frac{1}{2}\log\left(\frac{1}{\|\xi^w-w\|}\right)\le\log{\sqrt{2R}}+\frac{1}{2}\log\left(\frac{1}{\delta_\Omega(w)}\right)
        \end{aligned}
    \end{equation}
    per ogni $w$ che soddisfa la \eqref{7punto11}. Fissiamo ora un punto $z_0\in\Omega$ e poniamo
    \begin{gather*}
        K^*:=K_0\cup K_1\cup\dots\cup K_N\\
        \text{e}\\
        C_0:=\displaystyle\sup_{x\in K^*}k_\Omega(z_0,x)+\max\{\log{\sqrt{2R}},C^{(1)},\dots,C^{(N)}\};
    \end{gather*}
    allora, dalla disuguaglianza triangolare per $k_\Omega$, dalla \eqref{7punto9} e dalle disuguaglianze \eqref{7punto10} e \eqref{7punto12}, troviamo che esiste una costante $C_1>0$ tale che
    $$k_\Omega(z_0,z)\le C_0+C_1\delta_\Omega(z)^{-\max_{j=1,\dots,N}p_j+1}$$
    per ogni $z\in\Omega$.\\

    Passo 4: $\Omega$ è $(\lambda,\kappa)$-visibile per ogni $\lambda\ge 1$ e $\kappa>0$.

    Poniamo $p_0:=\displaystyle\max_{j=1,\dots,N} p_j$; abbiamo $p_0\in(1,3/2)$ per ipotesi. Per mostrare la tesi di questo passaggio, ci basterà verificare le ipotesi del Teorema \ref{extvis}. Prendendo $S=\emptyset$, $U=\mathbb{C}^n$, $z_0$ il punto scelto alla fine del passo 3 e $f(r)=C_0+C_1r^{p_0-1}$, e usando i passaggi 2 e 3 dimostrati sopra, la verifica è immediata. In questo passaggio abbiamo usato la condizione $p_0<3/2$, cioè $1/2-p_0>-1$, per ottenere l'ipotesi (3) del Teorema \ref{extvis}.\\

    Passo 5: $\Omega$ non è un dominio Goldilocks.

    Vogliamo mostrare che la condizione (2) nella Definizione \ref{gold} non è soddisfatta da $\Omega$. Per farlo, fissiamo un punto $q_{j^*}$; come fatto al passo 1, ci restringiamo a $\Omega\cap V_{j^*}$. Riprendendo la notazione di quel passaggio, prendiamo $A=A_{j^*}$, $A''=A_{j^*}''$, $\psi=\psi_{j^*}$, $q_{j^*}=0$ e $p=p_{j^*}$. Sia $z_0=(0,\dots,0,A''/2)$; mostriamo che per ogni $z\in\Omega\cap V_{j^*}$ tale che $0<\mathfrak{Re}z_n<A''/2$ si ha
    \begin{equation}\label{7punto14}
        k_\Omega(z_0,z)\gtrsim(\mathfrak{Re}z_n)^{-(p-1)}-(A''/2)^{-(p-1)}.
    \end{equation}

    Fissiamo uno $z$ che soddisfa le suddette condizioni. Sia $\gamma:[0,1]\longrightarrow\Omega$ una curva $C^1$ a tratti tale che $\gamma(0)=z_0$, $\gamma(1)=z$ e $\gamma([0,1])\subseteq\Omega\cap V_{j^*}$. Detta $\gamma_n$ l'ultima coordinata di $\gamma$, per continuità di $\mathfrak{Re}\gamma_n$ troviamo che esistono $\alpha,\beta\in[0,1]$ tali che $(\mathfrak{Re}\gamma_n)([\alpha,\beta])=[\mathfrak{Re}z_n,A''/2]$. Quindi
    $$\int_0^1 K_\Omega\big(\gamma(t);\gamma'(t)\big)\diff t\ge\int_\alpha^\beta K_\Omega\big(\gamma(t);\gamma'(t)\big)\diff t\ge \int_\alpha^\beta \frac{b\|\gamma'(t)\|}{\left|u\big(\gamma(t)\big)\right|^{1/2}}\diff t,$$
    dove la seconda disuguaglianza segue dalla \eqref{7punto6}. Per ogni $w\in\Omega\cap V_{j^*}$ con $0<\mathfrak{Re}w_n<A''$ si ha
    $$|u(w)|=\big(\psi(\mathfrak{Re}w_n)\big)^2-\|w'\|^2-(\mathfrak{Im}w_n)^2 \le \big(\psi(\mathfrak{Re}w_n)\big)^2 \le C^2(\mathfrak{Re}w_n)^{2p},$$
    dove abbiamo preso $C=C_{j^*}$. Combinando con la disuguaglianza precedente, troviamo
    \begin{align*}
        \int_0^1 K_\Omega\big(\gamma(t);\gamma'(t)\big)\diff t&\ge \frac{b}{C}\int_\alpha^\beta \frac{|(\mathfrak{Re}\gamma_n)'(t)|}{\big((\mathfrak{Re}\gamma_n)(t)\big)^p}\diff t\\
        &\ge \left|\frac{b}{C}\int_\alpha^\beta \frac{(\mathfrak{Re}\gamma_n)'(t)}{\big((\mathfrak{Re}\gamma_n)(t)\big)^p}\diff t\right|=\frac{b}{C}\int_{\mathfrak{Re}z_n}^{A''/2}\frac{1}{t^p}\diff t,
    \end{align*}
    dove il cambio di variabile nell'ultimo passaggio non è propriamente consentito ($\mathfrak{Re}\gamma_n$ potrebbe non essere monotona su $[\alpha,\beta]$), ma possiamo farlo su un numero finito di intervalli dove è monotona e la cui immagine partiziona $[\mathfrak{Re}z_n,A''/2]$, mentre i pezzi rimanenti possiamo assorbirli nella precedente disuguaglianza. La disuguaglianza \eqref{7punto14} segue applicando il Teorema \ref{lung_int}.

    Poniamo adesso $\mu^x:=(0,\dots,0,x)$ per $0<x<A''/2$. Per la \eqref{7punto14} si ha che
    $$k_\Omega(z_0,\mu^x) \gtrsim x^{-(p-1)}-(A''/2)^{-(p-1)}.$$
    
    Dalla \eqref{limite_brutto} segue che $\delta_\Omega(\mu^x) \approx x^p$ per $x\in(0,A''/2)$. Dunque, combinando con la disuguaglianza precedente, troviamo
    $$k_\Omega(z_0,\mu^x) \gtrsim \delta_\Omega(\mu^x)^{-1+1/p}-B$$
    per una costante $B$ opportuna. Dato che abbiamo $p=p_{j^*}>1$, otteniamo che ${\dfrac{\delta_\Omega(\mu^x)^{-1+1/p}}{\log\left(\frac{1}{\delta_\Omega(\mu^x)}\right)}\longrightarrow +\infty}$ per $\mu^x\longrightarrow 0=q_{j^*}$. Quindi $k_\Omega(z_0,\mu^x)$ non può soddisfare la condizione (2) nella Definizione \ref{gold}, come voluto. Questo conclude la dimostrazione.
\end{proof}

Vediamo ora che i domini Caltrop sono taut. Ci servirà il seguente risultato.

\begin{lm} \label{ainftyecompl}
    Sia $\Omega\subseteq\mathbb{C}^n$ un dominio limitato tale che esiste (e quindi per ogni) uno $z_0\in\Omega$ tale che per ogni $\xi\in\partial\Omega$ si ha $\displaystyle\lim_{w\longrightarrow\xi}k_\Omega(z_0,w)=+\infty$; allora $(\Omega,k_\Omega)$ è completo.
\end{lm}

\begin{proof}
    Sia $\{w_\nu\}_{\nu\in\mathbb{N}}\subseteq\Omega$ una successione di Cauchy; è sufficiente mostrare che ammette una sottosuccessione convergente. Per limitatezza di $\Omega$, ammette una sottosuccessione $\{w_{\nu_j}\}_{j\in\mathbb{N}}$ convergente a $\xi\in\overline{\Omega}$. Supponiamo per assurdo che $\xi\in\partial\Omega$; allora $\displaystyle\lim_{j\longrightarrow+\infty}k_\Omega(z_0,w_{\nu_j})=+\infty$, ma questo, per la disuguaglianza triangolare, è in contraddizione con il fatto che la successione sia di Cauchy.
\end{proof}

\begin{thm}
    (\cite[Theorem 8.2]{BM}) Sia $\Omega\subseteq\mathbb{C}^n$ un dominio Caltrop. Allora $(\Omega,k_\Omega)$ è completo; in particolare, per la Proposizione \ref{comp_is_taut}, è taut.
\end{thm}

\begin{proof}
    Adottiamo la notazione della dimostrazione del Teorema \ref{ilprecteo}, della quale possiamo usare tutti i risultati intermedi, visto che siamo nelle stesse ipotesi. Fissiamo $z_0\in\Omega$ e $\xi\in\partial\Omega$.
    
    Consideriamo prima il caso $\xi\in\partial\Omega\setminus\{q_1,\dots,q_N\}$. Prendiamo un punto $\eta\in\partial\Omega\setminus(\{q_1,\dots,q_N,\xi\})$. Per ipotesi, $\Omega$ è strettamente pseudoconvesso in $\xi$ e $\eta$. Prendiamo due aperti $V_\xi$ e $V_\eta$ di $\mathbb{C}^n$, intorni rispettivamente di $\xi$ e $\eta$, e una costante $C>0$ dati da \cite[Corollary 2.4]{FR}. Sia $b_\eta\in\Omega\cap V_\eta$. Consideriamo una successione $\{w_\nu\}_{\nu\in\mathbb{N}}\subseteq\Omega$ tale che $w_\nu\longrightarrow\xi$ per $\nu\longrightarrow+\infty$; possiamo assumere $w_\nu\in\Omega\cap V_\xi$ per ogni $\nu\in\mathbb{N}$. Allora \cite[Corollary 2.4]{FR} ci dice che
    \begin{align*}
        k_\Omega(z_0,w_\nu)&\ge k_\Omega(w_\nu,b_\eta)-k_\Omega(b_\eta,z_0)\\
        &\ge \frac{1}{2}\log{\frac{1}{\delta_\Omega(w_\nu)}}+\frac{1}{2}\log{\frac{1}{\delta_\Omega(b_\eta)}}-k_\Omega(b_\eta,z_0)-C,
    \end{align*}
    e quest'ultima quantità tende a $+\infty$ per $\nu$ che tende a $+\infty$.

    Adesso vogliamo mostrare la stessa cosa per $\xi=q_{j^*}$, dove $j^*\in\{1,\dots,N\}$. Consideriamo una successione $\{w_\nu\}_{\nu\in\mathbb{N}}\subseteq\Omega$ tale che $w_\nu\longrightarrow q_{j^*}$ per $\nu\longrightarrow+\infty$; possiamo assumere $w_\nu\in\Omega\cap V_{j^*}$ per ogni $\nu\in\mathbb{N}$. Ricordiamo che possiamo assumere senza perdita di generalità $q_{j^*}=0$ e $\mathbb{U}_{j^*}=\id$. Poniamo inoltre $w_{\nu,n}:=\pi_n(w_\nu)$. Allora $\mathfrak{Re}w_{\nu,n}\longrightarrow0$ per $\nu\longrightarrow+\infty$. Scrivendo $z_0'$ al posto di $z_0$ nella disuguaglianza \eqref{7punto14}, si ha che
    $$k_\Omega(z_0',w_\nu)\gtrsim (\mathfrak{Re}w_{\nu,n})^{-(p-1)}-(A''/2)^{-(p-1)},$$
    e quest'ultima quantità va a $+\infty$ per $\nu\longrightarrow+\infty$; per disuguaglianza triangolare, vale anche con lo $z_0$ scelto nella prima parte della dimostrazione. Si conclude applicando il Lemma \ref{ainftyecompl}.
\end{proof}

Da quanto dimostrato, discende un teorema di tipo ``Wolff-Denjoy''.

\begin{cor}
    Siano $\Omega$ un dominio Caltrop e $F:\Omega \longrightarrow \Omega$ una funzione olomorfa. Allora vale esattamente una delle seguenti affermazioni:
    \begin{itemize}
        \item le orbite dei punti di $\Omega$ tramite $F$ sono relativamente compatte in $\Omega$; oppure,
        \item esiste un unico punto di $\partial \Omega$ tale che la successione delle iterate di $F$ converge, uniformemente sui compatti, a quel punto.
    \end{itemize}
\end{cor}

\begin{proof}
    Da quanto visto segue che $\Omega$ soddisfa le ipotesi del Corollario \ref{wd_type}.
\end{proof}
\subsection{Un altro esempio}
L'ultimo esempio è quello mostrato in \cite[Section 5.2]{CMS}; si tratta di un altro esempio di dominio non di tipo Goldilocks che soddisfa la condizione di visibilità.

Iniziamo considerando la funzione $\Phi_0:\mathbb{C}^2\longrightarrow\mathbb{R}$ definita da
$$\Phi_0(z):=\begin{cases}
    \exp(-1/\|z\|^2)-\mathfrak{Im}(z_2) &\mbox{se }z=(z_1,z_2)\not=(0,0)\\
    0 &\mbox{se }z=0.
\end{cases}$$

Poiché la matrice hessiana di $\Phi_0$ (vista come funzione da $\mathbb{R}^4$ in $\mathbb{R}$) è la stessa della funzione $\exp(-1/\|z\|^2)$ estesa a $0$ nell'origine, che è convessa vicino all'origine, esiste $0<\epsilon<1$ tale che $\Phi_0$ è convessa in $\mathbb{B}^2_{2\epsilon}$. Scegliamo inoltre una funzione liscia $\psi:\mathbb{C}^2\longrightarrow[0,1]$ tale che $\psi\equiv 1$ in $\mathbb{B}^2_{2\epsilon}$ e $\text{supp}\,{\psi}\subseteq \mathbb{B}^2_{3\epsilon}$. Poniamo $\Phi:=\Phi_0\cdot\psi$ e $c_0:=\displaystyle\sup_{z\in\mathbb{C}^2}\big(-\Phi(z)\big)>0$.

Scegliamo adesso una funzione liscia $\chi:[0,+\infty)\longrightarrow[0,+\infty)$ che sia identicamente nulla in $[0,\epsilon^2]$, strettamente crescente in $[\epsilon^2,+\infty)$ e strettamente convessa in $\big(\epsilon^2,(\epsilon+\delta)^2\big)$ per $0<\delta<\epsilon$; per esempio, possiamo prendere $\chi(t)=\exp\big(-1/(t-\epsilon^2)\big)$ per $t>\epsilon^2$ e $0$ altrove. Poniamo $c_1:=\chi\big((\epsilon+\delta/2)^2\big)$ e $C:=c_0/c_1$. Definiamo
$$\Psi(z):=C\chi(\|z\|^2)$$
per ogni $z\in\mathbb{C}^2$.

Osserviamo che:
\begin{itemize}
    \item la funzione $\Psi$ è liscia e non negativa su tutto $\mathbb{C}^2$, nulla in $\overline{\mathbb{B}^2_\epsilon}$, e strettamente convessa e strettamente positiva in $\mathbb{B}^2_{\epsilon+\delta}\setminus\overline{\mathbb{B}^2_\epsilon}$;
    \item si ha $\Psi(z)\ge c_0$ per ogni $z\in\mathbb{C}^2\setminus\mathbb{B}^2_{\epsilon+\delta/2}$, da cui $\Psi(z)+\Phi(z)\ge 0$ per ogni $z\in\mathbb{C}^2\setminus\mathbb{B}^2_{\epsilon+\delta/2}$;
    \item si ha $\Psi(z)+\Phi(z)=\Phi(z)=\Phi_0(z)$ per ogni $z\in\mathbb{B}^2_\epsilon$.
\end{itemize}

Consideriamo il dominio
$$\Omega:=\{z=(z_1,z_2)\in\mathbb{C}^2\mid \rho(z):=\Psi(z)+\Phi(z)<0\}.$$

Notiamo che $\Omega\subseteq\mathbb{B}^2_{\epsilon+\delta/2}$, dove $\rho=\Psi+\Phi_0$, che è una funzione convessa; per cui $\Omega$ è un dominio convesso limitato. Calcolando il gradiente di $\rho$, vediamo che esiste al più un punto $p_0\in\partial\Omega$ dove il gradiente si annulla, che è della forma $p_0=(0,ic)$; inoltre, $p_0\in\overline{\mathbb{B}^2_{\epsilon+\delta/2}}\setminus\overline{\mathbb{B}^2_\epsilon}$. Dunque $\Omega$ è un dominio limitato e convesso tale che $\partial\Omega\setminus\{p_0\}$ è liscio. Si ha anche che ogni punto di $(\partial\Omega\setminus\{p_0\})\cap(\mathbb{B}^2_{\epsilon+\delta}\setminus\overline{\mathbb{B}^2_\epsilon})$ è un punto del bordo di $\Omega$ strettamente convesso (perché in $\mathbb{B}^2_{\epsilon+\delta}\setminus\overline{\mathbb{B}^2_\epsilon}$ la funzione $\Psi$ è strettamente convessa e la funzione $\Phi_0$ è convessa); quindi per \cite[Proposition 3.1.9]{Kr} è pseudoconvesso, e per \cite[Corollary 5.6]{D'A} è un punto di tipo finito. Poniamo $A:=\partial\Omega\cap\overline{\mathbb{B}^2_\epsilon}$ e osserviamo che
$$A=\overline{\mathbb{B}^2_\epsilon}\cap\{z\in\mathbb{C}^2\mid \Phi_0(z)=0\};$$
si ha anche che ogni punto di $A$ diverso da $(0,0)$ è un punto del bordo di $\Omega$ di tipo finito (perché $\Phi_0$ è strettamente convessa in $\overline{\mathbb{B}^2_\epsilon}\setminus\{(0,0)\}$, per cui ogni punto di $A$ è strettamente convesso). \\

Possiamo ora procedere a dimostrare che $\Omega$ è $(\lambda,\kappa)$-visibile per ogni $\lambda \ge 1$ e $\kappa>0$.

\begin{prop} \label{safinisvis}
    (\cite[Corollary 1.10]{CMS}) Sia $\Omega$ un dominio limitato di $\mathbb{C}^d$. Supponiamo che esista un compatto $S\subseteq\partial\Omega$ tale che $S_a$, l'insieme dei punti di accumulazione di $S$, sia finito, e inoltre che ogni punto $p\in\partial\Omega\setminus S$ sia un punto liscio di bordo pseudoconvesso e di tipo finito. Allora $\Omega$ è $(\lambda,\kappa)$-visibile per ogni $\lambda \ge 1$ e $\kappa>0$.
\end{prop}

\begin{proof}
    Mostriamo che, dati $p,q\in\partial\Omega$ con $p\not=q$, sono soddisfatte le ipotesi (i) e (ii) del Teorema \ref{extvis}. Per farlo, consideriamo $S_0:=S_a\cup\{p,q\}$. Allora, per finitezza di $S_0$, esiste $\epsilon_0>0$ tale che $\overline{B(x,\epsilon_0)}\cap\overline{B(x',\epsilon_0)}=\emptyset$ per ogni $x,x'\in S_0$. Adesso poniamo
    $$S_1:=(S\cup\{p,q\})\setminus\left(\bigcup_{x\in S_a}\overline{B(x,\epsilon_0)}\right);$$
    notiamo che $S_1$ è un insieme finito disgiunto dal compatto $K:=\displaystyle\bigcup_{x\in S_a}\overline{B(x,\epsilon_0)}$. Dunque esiste $\epsilon_1>0$ tale che:
    \begin{itemize}
        \item si ha $\overline{B(y,\epsilon_1)}\cap K=\emptyset$ per ogni $y\in S_1$;
        \item $\overline{B(y,\epsilon_1)}\cap\overline{B(y',\epsilon_1)}=\emptyset$ per ogni $y,y'\in S_1$ con $y\not=y'$.
    \end{itemize}
    
    Distinguiamo ora due casi. \\

    Caso 1: $p\not\in K$.

    Basta prendere $p'=p$ e $r=\epsilon_1$. \\

    Caso 2: $p\in K$.

    In questo caso esiste un $x_0\in S_a$ tale che $p\in\overline{B(x_0,\epsilon_0)}$. Consideriamo la seguente famiglia di insiemi con chiusure mutualmente disgiunte:
    $$\mathcal{B}:=\{B(x,\epsilon_0)\mid x\in S_a\}\cup\{B(y,\epsilon_1)\mid y\in S_1\};$$
    allora esiste $\epsilon_2>0$ tale che $\epsilon_2<\text{dist}(B_1,B_2)/4$ per ogni $B_1,B_2\in\mathcal{B}$. Segue che $\mathcal{C}:=\{B(x,\epsilon_0+\epsilon_2)\mid x\in S_a\}\cup\{B(y,\epsilon_1+\epsilon_2)\mid y\in S_1\}$ è una famiglia di insiemi con chiusure mutualmente disgiunte. Allora basta prendere $p'=x_0$ e $r=\epsilon_1+\epsilon_2$. \\

    Per concludere mostriamo adesso che, per ogni $\xi\in\partial\Omega\setminus S$, esistono un intorno $U$ e una funzione $f$ che soddisfano le ipotesi (1), (2) e (3) del Teorema \ref{extvis}. Fissiamo un tale $\xi$; allora sono soddisfatte le ipotesi del Teorema \ref{cho} e di \cite[Proposition 2.5]{FR}, per cui esistono un intorno $U$ di $\xi$, due costanti $c,\epsilon>0$, un punto $z_0\in\Omega$ e una costante $A$ tali che, ponendo $f(x):=A+\dfrac{1}{2}\log{x}$ per ogni $x\in(0,+\infty)$, si ha
    \begin{gather*}
        k_\Omega(z,z_0) \le f\big(1/\delta_\Omega(z)\big)\\
        \text{e}\\
        K_\Omega(z;v) \ge c\frac{\|v\|}{\delta_\Omega(z)^{\epsilon}}
    \end{gather*}
    per ogni $z\in\Omega\cap U$ e $v\in\mathbb{C}^d$. Ne consegue facilmente che le ipotesi (1), (2) e (3) del Teorema \ref{extvis} sono soddisfatte, come voluto.
\end{proof}

Basta allora prendere $S=\{p_0,(0,0)\}$ per ottenere che $\Omega$ soddisfa le ipotesi della Proposizione \ref{safinisvis}, dunque è $(\lambda,\kappa)$-visibile per ogni $\lambda\ge 1$ e $\kappa>0$. \\

Mostriamo adesso che $\Omega$ non soddisfa la condizione (1) nella Definizione \ref{gold}. Iniziamo notando che
$$\Omega\cap\mathbb{B}^2_{\epsilon/2}=\{(z_1,z_2)\in\mathbb{B}^2_{\epsilon/2}\mid \mathfrak{Im}(z_2)>\exp(-1/\|z\|^2)\};$$
dunque, per $r>0$ sufficientemente piccolo, si ha che $p_r:=(0,ir)\in\Omega$. Poniamo $v:=(1,0)$ e $s:=\sqrt{\dfrac{1}{\log(1/r)}-r^2}$; allora la funzione $\varphi:\mathbb{D}\longrightarrow\Omega$ data da $\varphi(\zeta)=p_r+\zeta sv$ è ben definita (cioè l'immagine è effettivamente contenuta in $\Omega$) e olomorfa, per cui
$$K_\Omega(p_r;v) \le \frac{1}{s}.$$

Adesso, poiché $(0,0)\in\partial\Omega$, si ha $\delta_\Omega(p_r) \le r$, per cui
$$M_\Omega(r) \ge \frac{1}{K_\Omega(p_r;v)} \ge s=\sqrt{\frac{1}{\log(1/r)}-r^2};$$
per cui ci basta mostrare che, per $r_0>0$ sufficientemente piccolo affinché l'integranda sia definita, si ha
$$\int_0^{r_0}\frac{1}{r}\sqrt{\frac{1}{\log(1/r)}-r^2}\diff r=+\infty.$$

Ciò segue facilmente confrontando con la funzione $r \longmapsto \dfrac{1}{r}\cdot\dfrac{1}{\sqrt{\log(1/r)}}$.

\newpage

\section{Ulteriori risultati} \label{Ulteriori risultati}
\subsection{Sottovarietà non relativamente compatte}
Vediamo adesso i risultati del preprint \cite{BZ2} sui domini illimitati; continuando quanto fatto finora, punteremo a generalizzare questi risultati,\marginpar{Forse dovrei chiamare la sezione ``Sottovarietà non relativamente compatte''?} arrivando a un teorema di tipo ``Wolff-Denjoy'' per sottovarietà taut e con visibilità, ma non necessariamente relativamente compatte.

Come si può facilmente vedere pensando all'esempio della mappa $z\longmapsto z+1$ nel semipiano superiore $\mathbb{H}=\{z\in\mathbb{C}\mid\mathfrak{Im}z>0\}$, restringerci ai punti del bordo non sarà sufficiente. In questo caso, grazie a un biolomorfismo con $\mathbb{D}$ ci possiamo ricondurre al Teorema di Wolff-Denjoy originale; troviamo così che il limite è un punto di $\partial\mathbb{D}$ nel quale non possiamo estendere il biolomorfismo. È dunque chiaro che i risultati che vogliamo andare a studiare dipendono da come il dominio o la sottovarietà si immergano nella varietà ambiente, e in generale non possiamo aspettarci di avere sempre un bordo sufficientemente ricco per descrivere la dinamica delle iterate di funzioni olomorfe.

Dobbiamo dunque estendere la nostra varietà come spazio topologico. Il modo giusto di farlo per ritrovare un teorema di tipo ``Wolff-Denjoy'' è la end compactification. Il concetto fondamentale per definire la end compactification è quello di end, definito da Freudenthal in \cite{F}. Diamo la definizione data in \cite[Chapter 1, Problem 19]{Sp}.

\begin{defn} \label{end}
    Sia $X$ uno spazio topologico non compatto. Una \textit{end} di $X$ è una funzione $e$ con dominio $\{K\subseteq X\mid K\text{ è comatto}\}$ tale che:
    \begin{nlist}
        \item a ogni compatto $K\subseteq X$ associa una componente connessa non vuota di $X\setminus K$;
        \item per ogni coppia di compatti $K_1\subseteq K_2\subseteq X$ si ha $e(K_2)\subseteq e(K_1)$.
    \end{nlist}
    Indichiamo con $\mathcal{E}(X)$ l'insieme di tutte le end di $X$.
\end{defn}

\begin{oss} \label{endnonrelcpt}
    Dato un compatto $K\subseteq X$, se una componente connessa $C$ di $X\setminus K$ è relativamente compatta in $X$, non si potrà mai avere $e(K)=C$; altrimenti, non potrebbe essere soddisfatta la condizione (ii) nella Definizione \ref{end} con $K$ e $K\cup\overline{C}$.
\end{oss}

Intuitivamente, una end è un modo di scegliere, andando all'infinito, ``da che parte andare''. Un esempio semplice per capire quest'interpretazione è l'albero binario infinito (lo si può pensare come un sottospazio di $\mathbb{R}^2$), dove a ogni livello ci troviamo in un nodo e abbiamo due possibili strade tra cui scegliere; poiché i livelli sono numerabili, è facile vedere che la cardinalità delle end è quella del continuo.

\begin{oss}
    Supponiamo che $X$ ammetta un'esaustione in compatti, cioè che esista una successione $\{K_n\}_{n\in\mathbb{N}}$ tale che $K_n\subseteq{\mathop K\limits^ \circ}_{n+1}$ per ogni $n$ e che $\displaystyle\bigcup_{n=1}^{+\infty} K_n=X$; nei casi che studieremo $X$ sarà sempre una varietà connessa, per cui ammetterà un'esaustione in compatti. Allora una end $e$ è univocamente detereminata dalle sue immagini sui $K_n$ (segue facilmente dalle proprietà della end e da quelle dell'esaustione). Quindi, nel seguito, ci basterà fissare un'esaustione in compatti e lavorare con quella.
\end{oss}

Vogliamo ora mettere una topologia su $X^\mathcal{E}:=X\cup\mathcal{E}(X)$ che lo renda uno spazio compatto. Sebbene le end sono state definite con questo scopo in mente, sono comunque necessarie delle ipotesi. Si tratta di ipotesi che in generale sono soddisfatte da varietà astratte, ma poiché ci interesserà compattificare la chiusura di una sottovarietà dovremo prestare attenzione a un'ipotesi in particolare; si vedano l'Osservazione \ref{servelocconn} e l'Esempio \ref{servelocconnex}.

\begin{prop} \label{endiscpt}
    (\cite[Chapter 1, Problem 19]{Sp})\marginpar{Di nuovo cito un esercizio... quando avevo cercato la dim., l'unica che avevo trovato era online; può darsi che si trovi in \cite{F}, ma non so il tedesco} Sia $X$ uno spazio topologico connesso, localmente connesso, compatto e di Hausdorff. Mettiamo su $X^\mathcal{E}$ la topologia generata dalla topologia di $X$ e dai seguenti intorni per $e\in\mathcal{E}(X)$ al variare di $K\subseteq X$ compatto:
    $$N_K(e)=e(K)\cup\{f\in\mathcal{E}(X)\mid f(K)=e(K)\};$$
    allora $X\cup\mathcal{E}(X)$ è uno spazio compatto e di Hausdorff.
\end{prop}

\begin{oss} \label{endiscptsucc}
    Se $X$ ammette un'esaustione in compatti $\{K_n\}_{n\in\mathbb{N}}$, ponendo $U_n^e=N_{K_n}(e)$ si ha che $\{U_n^e\}_{n\in\mathbb{N}}$ è un sistema fondamentale di intorni per $e$. Allora, se è anche primo numerabile e soddisfa le ipotesi della Proposizione \ref{endiscpt}, si ha che $X^\mathcal{E}$ è compatto e primo numerabile, per cui è compatto per successioni.
\end{oss}

\begin{oss} \label{servelocconn}
    Come già accennato, data $X$ sottovarietà connessa di una varietà $Y$, andremo a studiare $\overline{X}^\mathcal{E}$. In tal caso, è facile verificare tutte le ipotesi della Proposizione \ref{endiscpt} tranne una: la locale connessione. Il motivo è perché in generale non è vera, come vedremo nell'esempio seguente.
\end{oss}

\begin{ex} \label{servelocconnex}
    Si consideri l'embedding di $(0,1)$ in $\mathbb{R}^2$ dato dalla funzione $x\longmapsto \sin(1/x)/x$. A meno di considerarne un ``ispessimento'' che diventa sempre più piccolo al tendere di $x$ a $0$, possiamo anche renderlo un dominio proprio e semplicemente connesso $\Omega$ di $\mathbb{C}$ (dunque biolomorfo a $\mathbb{D}$) avente bordo liscio al di fuori dell'asse immaginario.

    Il sottospazio $\overline{\Omega}$ non è localmente connesso. Ragionando come nell'Osservazione \ref{endiscptsucc}, $\overline{\Omega}^\mathcal{E}$ è primo numerabile. Per vedere che non vale la Proposizione \ref{endiscpt}, ci basta dunque vedere che non è compatto per successioni. Consideriamo allora una successione di punti contenuti nelle ``gobbe''. Questa chiaramente non ammette sottosuccessioni convergenti in $\overline{\Omega}$. Tuttavia, per l'Osservazione \ref{endnonrelcpt} non ammette nemmeno sottosuccessioni convergenti a un punto di $\mathcal{E}(\overline{\Omega})$; infatti, poiché ogni esaustione in compatti prima o poi dovrà coprire il compatto $\overline{\Omega}\cap[0,1]$, le uniche componenti connesse non relativamente compatte dei complementari dei compatti saranno, definitivamente, due semirette dell'asse immaginario. È chiaro però che nessun punto della successione vi può appartenere, ma dato che queste semirette, unite alle opportune end, formano un sistema fondamentale di intorni per le end stesse, segue che non ci sono nemmeno sottosuccessioni convergenti a una end.
\end{ex}

Possiamo ora estendere il concetto di visibilità a sottovarietà connesse non relativamente compatte.

\begin{defn} \label{visibility}
    Sia $X$ una sottovarietà complessa e connessa di una varietà complessa $Y$. Supponiamo che $\overline{X}$ sia localmente connessa e definiamo il \textit{bordo end} come ${\partial^\mathcal{E}X:=\partial_YX\cup\mathcal{E}(X)}$. Fissiamo $\lambda \ge 1$ e $\kappa \ge 0$; diciamo che $X$ è \textit{$(\lambda,\kappa)$-ultravisibile} se:
    \begin{enumerate}
        \item ogni due punti distinti di $X$ possono essere collegati da una $(\lambda,\kappa)$-simil-geodetica;
        \item per ogni coppia di punti $p,q\in\partial^\mathcal{E}X$ con $p\not=q$, esistono in $\overline{X}^\mathcal{E}$ due intorni $V$ e $W$, di $p$ e $q$ rispettivamente, con chiusura disgiunta, e un compatto $K$ di $X$ tali che  ogni $(\lambda,\kappa)$-simil-geodetica in $X$ che collega un punto di $V$ a un punto di $W$ interseca $K$.
    \end{enumerate}
\end{defn}

\begin{oss}
    Non escludiamo la possibilità che l'ipotesi che $\overline{X}$ sia localmente connessa sia ridondante, perché potrebbe seguire in qualche modo dalla condizione di ultravisibilità (per esempio, quest'ultima potrebbe implicare una qualche regolarità per $\partial_YX$). Non sappiamo se sia effettivamente vero, e in caso lo sia crediamo che la dimostrazione sia tutt'altro che semplice.
    
    Facciamo però notare che per l'Esempio \ref{servelocconnex}, sfruttando un biolomorfismo\marginpar{Esiste?} con $\mathbb{D}$ esteso ai punti regolari del bordo, si può mostrare che non vale la condizione di ultravisibilità. Esistono\marginpar{Serve una reference} anche esempi, sempre ispirati al seno del topologo, di domini limitati di $\mathbb{C}$ biolomorfi a $\mathbb{D}$ per i quali non vale l'analogo del teorema di Wolff-Denjoy; dal Corollario \ref{wd_type} segue che non soddisfano la condizione di visibilità, in quanto è facile verificare che soddisfano tutte le altre ipotesi.
\end{oss}

Adesso vogliamo dimostrare un teorema di tipo ``Wolff-Denjoy'' per sottovarietà Kobayashi-iperboliche, taut e $(1,\kappa_0)$-ultravisibili. Per farlo, in alcuni punti del nostro ragionamento riadatteremo le dimostrazioni degli enunciati visti nella sezione \ref{Un teorema di tipo ``Wolff-Denjoy'' per varietà taut con visibilità}, ma in altri dovremo ricavarci nuovi risultati.

\begin{lm} \label{limpt}
    Sia $X$ una sottovarietà complessa e connessa di una varietà complessa $Y$. Supponiamo che $\overline{X}$ sia localmente connessa e che esista $\kappa_0>0$ tale che $X$ sia $(1,\kappa_0)$-ultravisibile. Siano $Z$ una varietà Kobayashi-iperbolica e $\{f_n\}_{n\in\mathbb{N}}\subseteq\textnormal{Hol}(Z,X)$ una successione compattamente divergente. Allora esistono $\xi\in\partial^\mathcal{E}X$ e una sottosuccessione $\{f_{n_j}\}_{j\in\mathbb{N}}$ tali che $f_{n_j}(z)\longrightarrow\xi$ per ogni $z\in Z$.
\end{lm}

\begin{proof}
    Sia $z_0\in Z$. Per l'Osservazione \ref{endiscptsucc}, a meno di sottosuccessioni possiamo supporre che esista $\xi_0\in\partial^\mathcal{E}X$ tale che $f_n(z_0)\longrightarrow\xi_0$. Supponiamo per assurdo che esista $z_1\in Z$ tale che la successione $\{f_n(z_1)\}_{n\in\mathbb{N}}$ non converge a $\xi_0$. Considerando una curva continua che collega $z_0$ a $z_1$, si trova facilmente che possiamo eventualmente sostituirli con due punti tali che $k_Z(z_0,z_1)<\kappa_0/2$. Di nuovo a meno di sottosuccessioni, possiamo supporre che esista $\xi_1\in\partial^\mathcal{E}X$ tale che $\xi_1\not=\xi_0$ e $f_n(z_1)\longrightarrow\xi_1$.

    Per il Teorema \ref{similgeo_conn} esiste una $(1,\kappa_0/2)$-simil-geodetica $\sigma:[0,T]\longrightarrow Z$ tale che $\sigma(0)=z_0$ e $\sigma(T)=z_1$. Poniamo $\sigma_n:=f_n\circ\sigma$, vogliamo dimostrare che $\sigma_n$ è una $(1,\kappa_0)$-simil-geodetica per ogni $n$. Per la Proposizione \ref{semicontr} e per la definizione di $(1,\kappa_0/2)$-simil-geodetica, abbiamo che
    $$k_X\big(\sigma_n(s),\sigma_n(t)\big) \le k_Z\big(\sigma(s),\sigma(t)\big) \le |t-s|+\kappa_0/2\le |t-s|+\kappa_0$$
    per ogni $s,t\in[0,T]$. Inoltre, sempre dalla definizione di di $(1,\kappa_0/2)$-simil-geodetica si ha che
    \begin{align*}
        |t-s|-\kappa_0 &\le |0-T|-\kappa_0/2-\kappa_0/2 \le k_Z\big(\sigma(0),\sigma(T)\big)-\kappa_0/2\\
        &=k_Z(z_0,z_1)-\kappa_0/2<0\le k_X\big(\sigma_n(s),\sigma_n(t)\big)
    \end{align*}
    per ogni $s,t\in[0,T]$. Infine, dalla Proposizione \ref{metrdecr} e dalla definizione di $(1,\kappa_0/2)$-simil-geodetica segue che
    $$K_X\big(\sigma_n(t);\sigma_n'(t)\big)\le K_Z\big(\sigma(t);\sigma'(t)\big) \le 1$$
    per ogni $t\in[0,T]$. Dunque $\sigma_n$ è una $(1,\kappa_0)$-simil-geodetica per ogni $n$, come volevamo.

    Adesso, poiché $\sigma_n(0)=f_n(z_0)\longrightarrow\xi_0$ e $\sigma_n(T)=f_n(z_1)\longrightarrow\xi_1$ e $X$ è $(1,\kappa_0)$-ultravisibile, esiste un compatto $K\subseteq X$ tale che
    $$\emptyset\not=K\cap\sigma_n([0,T])=K\cap f_n\big(\sigma([0,T])\big)$$
    per ogni $n$, in contraddizione con l'ipotesi che la successione $\{f_n\}_{n\in\mathbb{N}}$ sia compattamente divergente.
\end{proof}

\begin{prop}
    Sia $X$ una sottovarietà Kobayashi-iperbolica di una varietà complessa $Y$. Supponiamo che $\overline{X}$ sia localmente connessa e che esista $\kappa_0>0$ tale che $X$ sia $(1,\kappa_0)$-ultravisibile. Sia $F\in\textnormal{Hol}(X,X)$ tale che la successione $\{F^n\}_{n\in\mathbb{N}}$ sia compattamente divergente. Allora esiste $\xi\in\partial^\mathcal{E}X$ tale che per ogni funzione $\mu:\mathbb{N}\longrightarrow\mathbb{N}$ strettamente crescente per cui esiste $y_0 \in X$ tale che
    \begin{equation}
        \lim_{j\longrightarrow+\infty} k_X\big(F^{\mu(j)}(y_0),y_0\big)=+\infty
    \end{equation}
    si ha
    \begin{equation}
        \lim_{j\longrightarrow+\infty} F^{\mu(j)}(z)=\xi
    \end{equation}
    per ogni $z \in X$.
\end{prop}

\begin{proof}
    Si ripete la dimostrazione della Proposizione \ref{inf_impl_ugu} con le seguenti modifiche: $\overline{X}^\mathcal{E}$ e $\partial^\mathcal{E}X$ al posto di $\overline{X}$ e $\partial_YX$; ``per compattezza per successioni di $\overline{X}^\mathcal{E}$'' al posto di ``per relativa compattezza di $X$''; ultravisibilità al posto di visibilità; si utilizza il Lemma \ref{limpt} al posto del Lemma \ref{lemmasottosotto1}.
\end{proof}

\begin{lm} \label{nonazeroplease}
    Sia $X$ una sottovarietà Kobayashi-iperbolica di una varietà complessa $Y$. Supponiamo che esista $\kappa_0>0$ tale che $X$ sia $(1,\kappa_0)$-ultravisibile. Siano $\{x_n\}_{n\in\mathbb{N}}$ e $\{y_n\}_{n\in\mathbb{N}}$ due successioni di punti di $X$ tali che $x_n\longrightarrow\xi$ e $y_n\longrightarrow\xi'$, con $\xi,\xi'\in\partial^\mathcal{E}X$ e $\xi\not=\xi'$.
    
    Allora non può essere che $\displaystyle\lim_{n\longrightarrow+\infty}k_X(x_n,y_n)=0$.
\end{lm}

\begin{proof}
    Supponiamo per assurdo che la tesi sia falsa, e prendiamo due successioni che la contraddicono. Per il Teorema \ref{similgeo_conn} possiamo prendere, per ogni $n\in\mathbb{N}$, una $(1,1/n)$-simil-geodetica $\sigma_n:[0,T_n]\longrightarrow X$ tale che $\sigma_n(0)=x_n$ e ${\sigma_n(T_n)=y_n}$. Per $n$ sufficientemente grande sono tutte $(1,\kappa_0)$-simil-geodetiche, per cui esistono un compatto $K\subseteq X$ e $t_n\in[0,T_n]$ tali che $\sigma_n(t_n)\in K$ per ogni $n$. Dato che $X$ è una varietà, possiamo prendere un compatto ${H\subseteq X}$ tale che $K\subseteq{\mathop H\limits^ \circ}$. Allora $\partial^\text{top}H$ e $K$ sono due compatti disgiunti, dove $\partial^\text{top}$ è il bordo topologico in $X$ (chiusura meno parte interna); dunque ${\displaystyle\inf_{\substack{x\in\partial^\text{top}H,\\y\in K}} k_X(x,y)=\epsilon>0}$.

    Poiché $\sigma_n(0)$ e $\sigma_n(T_n)$ convergono a punti del bordo end, per $n$ sufficientemente grande appartengono a $X\setminus H$. Ma $\sigma_n(t_n)\in K\subseteq{\mathop H\limits^ \circ}$; quindi deve esistere $t_n'$ tale che $\sigma_n(t_n')\in\partial H$. Segue che
    \begin{align*}
        0&<\epsilon\le k_X\big(\sigma_n(t_n),\sigma_n(t_n')\big) \le |t_n-t_n'|+1/n\le |0-T_n|+1/n \\
        &\le k_X\big(\sigma_n(0),\sigma_n(T_n)\big)+2/n=k_X(x_n,y_n)+2/n\longrightarrow0,
    \end{align*}
    contraddizione.
\end{proof}

\begin{lm} \label{unaltrosottosotto}
    Sia $X$ una sottovarietà Kobayashi-iperbolica di una varietà complessa $Y$. Supponiamo che $\overline{X}$ sia localmente connessa e che esista $\kappa_0>0$ tale che $X$ sia $(1,\kappa_0)$-ultravisibile. Sia $F\in\textnormal{Hol}(X,X)$ tale che la successione $\{F^n\}_{n\in\mathbb{N}}$ sia compattamente divergente.
    
    Supponiamo che esistano un compatto $K\subseteq X$, una funzione strettamente crescente $\mu:\mathbb{N}\longrightarrow\mathbb{N}$ e $\xi\in\partial^\mathcal{E}X$ tali che per ogni intorno $U$ di $\xi$ in $\overline{X}^\mathcal{E}$ esiste $j_0$ tale che per ogni $j\ge j_0$ si ha $F^{\mu(j)}(K)\subseteq U$. Allora la successione $\{F^{\mu(j)}\}_{j\in\mathbb{N}}$ converge alla costante $\xi$ in $C^0(X,\overline{X}^\mathcal{E})$.
\end{lm}

\begin{proof}
    Supponiamo per assurdo che la tesi non sia vera; esistono dunque un compatto $H\subseteq X$ e un intorno $U$ di $\xi$ in $\overline{X}^\mathcal{E}$ tali che ${F^{\mu(j)}(H)\not\subseteq U}$ frequentemente. Allora esistono una sottosuccessione $\{j_n\}_{n\in\mathbb{N}}\subseteq\mathbb{N}$ e una successione $\{z_n\}_{n\in\mathbb{N}}\subseteq H$ tali che $F^{\mu(j_n)}(z_n)\not\in U$ per ogni $n\in\mathbb{N}$.
    
    A meno di sottosuccessioni, usando il Lemma \ref{limpt} e la compattezza per successioni di $\overline{X}^\mathcal{E}$, possiamo supporre che $z_n\longrightarrow\tilde{z}\in H$, che ${F^{\mu(j_n)}(z)\longrightarrow\tilde{\xi}\in\overline{X}^\mathcal{E}}$ per ogni $z\in X$ e che $F^{\mu(j_n)}(z_n)\longrightarrow\xi'\in\overline{X}^\mathcal{E}$. Dato che $\{F^n\}_{n\in\mathbb{N}}$ è compattamente divergente, dev'essere $\xi'\in\partial^\mathcal{E}X$. Siccome $F^{\mu(j_n)}(z)\longrightarrow\xi$ per ogni $z\in K$, dev'essere $\tilde{\xi}=\xi$. Visto che $F^{\mu(j_n)}(z_n)\not\in U$ per ogni $n\in\mathbb{N}$, dev'essere $\xi\not=\xi'$.

    Adesso notiamo che per la Proposizione \ref{semicontr} si ha
    $$k_X\big(F^{\mu(j_n)}(z_n),F^{\mu(j_n)}(\tilde{z})\big)\le k_X(z_n,\tilde{z})\longrightarrow0$$
    per $n\longrightarrow+\infty$. Basta allora applicare il Lemma \ref{nonazeroplease}, con $x_n=F^{\mu(j_n)}(z_n)$ e $y_n=F^{\mu(j_n)}(\tilde{z})$, per ottenere una contraddizione.
\end{proof}

\begin{lm} \label{estraiunasotto}
    Sia $X$ una sottovarietà Kobayashi-iperbolica di una varietà complessa $Y$. Supponiamo che $\overline{X}$ sia localmente connessa e che esista $\kappa_0>0$ tale che $X$ sia $(1,\kappa_0)$-ultravisibile. Sia $F\in\textnormal{Hol}(X,X)$ tale che la successione $\{F^n\}_{n\in\mathbb{N}}$ sia compattamente divergente.

    Per ogni funzione strettamente crescente $\mu:\mathbb{N}\longrightarrow\mathbb{N}$ esistono $\xi\in\partial^\mathcal{E}X$ e una sottosuccessione $\{j_n\}_{n\in\mathbb{N}}\subseteq\mathbb{N}$ tale che la successione $\{F^{\mu(j_n)}\}_{n\in\mathbb{N}}$ converge alla costante $\xi$ in $C^0(X,\overline{X}^\mathcal{E})$.
\end{lm}

\begin{proof}
    Fissiamo $z_0\in X$. Per la compattezza per successioni di $\overline{X}^\mathcal{E}$ e la divergenza dai compatti di $\{F^n\}_{n\in\mathbb{N}}$, esistono $\xi\in\partial^\mathcal{E}X$ e una sottosuccessione $\{j_n\}_{n\in\mathbb{N}}\subseteq\mathbb{N}$ tali che $F^{\mu(j_n)}(z_0)\longrightarrow\xi$. Allora la successione $\{F^{\mu(j_n)}\}_{n\in\mathbb{N}}$ converge alla costante $\xi$ uniformemente sul compatto $\{z_0\}$. Si conclude applicando il Lemma \ref{unaltrosottosotto}.
\end{proof}

Siamo ora pronti a dimostrare il teorema di tipo ``Wolff-Denjoy'' nel caso non relativamente compatto.
\subsection{Estensioni al bordo}
Elenchiamo ora, senza dimostrarli, alcuni teoremi che legano l'ipotesi di visibilità alle estensioni al bordo di funzioni. Risultati di estensione erano noti da tempo per il caso di mappe tra domini regolari, per esempio strettamente pseudoconvessi o con bordo liscio. Gli enunciati che vedremo hanno ipotesi che permettono una minore regolarità del bordo (abbiamo visto, tra gli altri esempi, domini con visibilità che hanno delle cuspidi), ma tornerà in gioco la Gromov-iperbolicità, che nei teoremi di tipo ``Wolff-Denjoy'' non era presente tra le ipotesi.

\begin{thm}
    (\cite[Theorem 1.5]{BZ1}) Siano $D$ e $\Omega$ due domini limitati di $\mathbb{C}^n$. Supponiamo che $D$ sia pseudoconvesso con bordo $C^2$, e che $\Omega$ sia un dominio Goldilocks che soddisfa una condizione di cono interno. Allora ogni funzione olomorfa propria $F:D\longrightarrow\Omega$ si estende a una funzione continua su $\overline{D}$.
\end{thm}

I teoremi di estensione si hanno non solo per le funzioni olomorfe proprie, ma anche per i quasi-embedding isometrici.

\begin{defn}
    Siano $(X,d_1)$ e $(Y,d_2)$ due spazi metrici, e fissiamo $\lambda\ge 1$ e $\kappa>0$. Una funzione $F:X\longrightarrow Y$ è detta \textit{$(\lambda,\kappa)$-quasi-embedding isometrico rispetto a $d_1$ e $d_2$} se si ha
    $$\frac{1}{\lambda}d_2\big(F(x_1),F(x_2)\big)-\kappa \le d_1(x_1,x_2) \le \lambda d_2\big(F(x_1),F(x_2)\big)+\kappa$$
    per ogni $x_1,x_2\in X$.
    
    La funzione $F$ è detta \textit{quasi-embedding isometrico rispetto a $d_1$ e $d_2$} se è un $(\lambda,\kappa)$-quasi-embedding isometrico rispetto a $d_1$ e $d_2$ per qualche $\lambda\ge 1$ e $\kappa>0$.
\end{defn}
\vspace*{-\baselineskip}

Risultati di estensione sono ben noti per i quasi-embedding isometrici tra spazi Gromov-iperbolici, si veda \cite[Part III, Chapter H, Theorem 3.9]{BH}.

\begin{thm}
    (\cite[Theorem 1.7]{BZ1}) Siano $D$ un dominio limitato di $\mathbb{C}^m$ e $\Omega$ un dominio Goldilocks di $\mathbb{C}^n$. Supponiamo che $(D,k_D)$ sia uno spazio metrico proprio e Gromov-iperbolico. Sia $F:D\longrightarrow\Omega$ un quasi-embedding isometrico rispetto alle distanze di Kobayashi e continuo; allora esiste un'estensione continua $\tilde{F}:D\cup\partial^GD\longrightarrow\overline{\Omega}$.
\end{thm}

Bharali e Zimmer hanno dimostrato teoremi di estensione anche per domini con visibilità non necessariamente limitati.

\begin{defn}
    Sia $\Omega$ un dominio di $\mathbb{C}^n$. Diciamo che \textit{$\Omega$ ha delle buone geodetiche} se:
    \begin{itemize}
        \item è Kobayashi-iperbolico e completo rispetto a $k_\Omega$;
        \item per ogni coppia di successioni $\{z_j\}_{j\in\mathbb{N}},\{w_j\}_{j\in\mathbb{N}}\subseteq\Omega$ tali per cui si abbia $\displaystyle\lim_{j\longrightarrow+\infty}z_j=\displaystyle\lim_{j\longrightarrow+\infty}w_j=\xi\in\partial^\mathcal{E}\Omega$, e date $\sigma_j$ delle geodetiche congiungenti $z_j$ a $w_j$ per ogni $j\in\mathbb{N}$, si ha che esiste (e quindi per ogni) $o\in\Omega$ tale che $\displaystyle\lim_{j\longrightarrow+\infty}k_\Omega(o,\sigma_j)=+\infty$.
    \end{itemize}
\end{defn}

\begin{thm}
    (\cite[Theorem 1.6]{BZ2}) Siano $\Omega_1\subseteq\mathbb{C}^{n_1}$ e $\Omega_2\subseteq\mathbb{C}^{n_2}$ due domini tali che:
    \begin{enumerate}[label={(\arabic*)}]
        \item il dominio $\Omega_1$ ha delle buone geodetiche;
        \item il dominio $\Omega_2$ è $(\lambda,\kappa)$-visibile per ogni $\lambda\ge 1$ e $\kappa\ge 0$.
    \end{enumerate}

    Sia $f:\Omega_1\longrightarrow\Omega_2$ un quasi-embedding isometrico rispetto alle distanze di Kobayashi; allora $f$ si estende a una funzione continua $\tilde{f}:\overline{\Omega}_1^\mathcal{E}\longrightarrow\overline{\Omega}_2^\mathcal{E}$.
\end{thm}

\begin{thm} \label{citchemiserveprima}
    (\cite[Theorem 1.9]{BZ2}) Siano $\Omega_1,\Omega_2\subsetneq\mathbb{C}$ due domini lipschitziani. Sia $f:\Omega_1\longrightarrow\Omega_2$ un biolomorfismo; allora $f$ si estende a un omeomorfismo $\tilde{f}:\overline{\Omega}_1^\mathcal{E}\longrightarrow\overline{\Omega}_2^\mathcal{E}$.
\end{thm}

Il prossimo risultato permette di andare in senso opposto: data un estensione al bordo, dedurne la visibilità del dominio.

\begin{thm}
    (\cite[Theorem 1.10]{BZ2}) Sia $\Omega\subseteq\mathbb{C}^n$ un dominio Kobayashi-iperbolico, e supponiamo che $(\Omega,k_\Omega)$ sia Gromov-iperbolico. Se l'identità $\id_\Omega$ si estende a un omeomorfismo da $\Omega\cup\partial^G\Omega$ in $\overline{\Omega}^\mathcal{E}$ allora:
    \begin{enumerate}[label={(\arabic*)}]
        \item si ha che $\Omega$ è completo rispetto alla distanza di Kobayashi;
        \item si ha che $\Omega$ è $(1,\kappa)$-visibile per ogni $\kappa\ge0$.
    \end{enumerate}
\end{thm}

Passiamo ora ai teoremi di \cite{CMS}, che trattano il caso di sottovarietà qualsiasi.

\begin{defn}
    Sia $X$ una sottovarietà complessa, connessa e limitata di $\mathbb{C}^m$. Un sottoinsieme $S$ di $X$ si dice \textit{sottospazio geodetico} se:
    \begin{itemize}
        \item lo spazio metrico $(S,k_X\restrict{S\times S})$ è completo;
        \item per ogni coppia di punti distinti di $S$, esiste ua geodetica di $(X,k_X)$ che li congiunge e che sia tutta contenuta in $S$.
    \end{itemize}
\end{defn}

\begin{defn}
    Sia $X$ una sottovarietà complessa, connessa e limitata di $\mathbb{C}^n$. Un sottospazio geodetico $S$ si dice \textit{sottospazio di visibilità} se per ogni $p,q\in\overline{S}\setminus S$ (la chiusura è in $\mathbb{C}^n$) con $p\not=q$ esistono in $\mathbb{C}^n$ due intorni $U$ e $V$, rispettivamente di $p$ e di $q$, e un compatto $K\subseteq S$ tali che $\overline{U}\cap\overline{V}=\emptyset$ e, per ogni geodetica $\sigma:[a,b]\longrightarrow S$ di $(X,k_X)$ che collega un punto di $U\cap S$ a un punto di $V\cap S$, si ha $\sigma([a,b])\cap K\not=\emptyset$.
\end{defn}

\begin{defn}
    Sia $(X,d)$ uno spazio metrico e $(\iota,\tilde{X})$ una sua compattificazione. Un \textit{loop geodetico di $X$ in $\tilde{X}$} è una geodetica $\sigma:\mathbb{R}\longrightarrow X$ (con la distanza euclidea in partenza e $d$ in arrivo) tale che l'insieme dei suoi punti limite a $+\infty$ è uguale a quello dei punti limite a $-\infty$.
\end{defn}

\begin{thm}
    (\cite[Theorem 1.4]{CMS}) Sia $X$ una sottovarietà complessa, connessa e limitata di $\mathbb{C}^n$. Sia $S\subseteq X$ un sottospazio geodetico di $M$ tale che $(S,k_X\restrict{S\times S})$ sia Gromov-iperbolico. Allora $S$ è sottospazio di visibilità se e solo se l'identità $\id_S$ si estende a una funzione $\tilde{id}_S:S\cup\partial^GS\longrightarrow\overline{S}$ continua e suriettiva.

    Inoltre, tale estensione è un omeomorfismo se e solo se $S$ non ha loop geodetici in $\overline{S}$.
\end{thm}

\begin{thm}
    (\cite[Theorem 1.5]{CMS}) Siano $X\subseteq\mathbb{C}^m$ e $Y\subseteq\mathbb{C}^n$ due sottovarietà complesse, connesse e limitate, e supponiamo che $(X,k_X)$ sia completo e Gromov-iperbolico. Sia $f:X\longrightarrow Y$ un'isometria rispetto alle distanze di Kobayashi, e supponiamo che $S:=f(X)$ sia un sottospazio di visibilità di $Y$. Allora $f$ si estende a una funzione continua $\tilde{f}:X\cup\partial^GX\longrightarrow\overline{Y}$.

    Inoltre, se $S$ non ha loop geodetici in $\overline{S}$ allora $\tilde{f}$ è un omeomorfismo tra $X\cup\partial^GX$ e $\overline{S}$.
\end{thm}

\newpage

\begin{thebibliography}{widest entry}
  \bibitem[A1]{A1} M. Abate: \textbf{Iteration theory of holomorphic maps on taut manifolds}. Mediterranean Press, Cosenza, 1989 [\url{http://www.dm.unipi.it/˜abate/libri/libriric/libriric.html}]
  \bibitem[A2]{A2} M. Abate: Iteration theory, compactly divergent sequences and commuting holomorphic maps. \textit{Annali della Scuola Normale Superiore di Pisa. Classe di Scienze. Serie V}, \textbf{18} (1991), no. 2, 167--191
  \bibitem[A3]{A3} M. Abate: A characterization of hyperbolic manifolds. \textit{Proceedings of the American Mathematical Society}, \textbf{117} (1993), no. 3, 789--793
  \bibitem[A4]{A4} M. Abate: Dynamics in several complex variables. In Metrical and dynamical aspects in complex analysis, Ed. L. Blanc-Centi, Lecture Notes in Mathematics 2195, Springer, Berlin, 2017, pp. 25--54
  \bibitem[B]{B} T. J. Barth: The Kobayashi distance induces the standard topology. \textit{Proceedings of the American Mathematical Society}, \textbf{35} (1972), 439--441
  \bibitem[BB]{BB} Z. M. Balogh, M. Bonk: Gromov hyperbolicity and the Kobayashi metric on strictly pseudoconvex domains. \textit{Commentarii Mathematici Helvetici}, \textbf{75} (2000), no. 3, 504--533
  \bibitem[BH]{BH} M. R. Bridson, A. Haefliger: \textbf{Metric-Spaces of Non-Positive Curvature}. Springer, New York, 1999
  \bibitem[BZ1]{BZ1} A. Zimmer, G. Barali: Goldilocks domains, a weak notion of visibility, and applications. \textit{Advances in Mathematics}, \textbf{310} (2017), 377--425
  \bibitem[CMS]{CMS} V. S. Chandel, A. Maitra, A. D. Sarkar: Notions of Visibility with respect to the Kobayashi distance: Comparison and Applications. Preprint, arXiv:2111.00549v1 (2021)
  \bibitem[N]{N} R. Narasimhan: \textbf{Several Complex Variables}. University of Chicago Press, Chicago, 1971
  \bibitem[NTT]{NTT} N. Nikolov, P. J. Thomas, M. Trybuła: Gromov (non)hyperbolicity of certain domains in $\mathbb{C}^2$. Preprint, arXiv:1403.7673v2 (2015)
  \bibitem[Ka]{Ka} A. Karlsson: Non-expanding maps and Busemann functions. \textit{Ergodic Theory and Dynamical Systems}, \textbf{21} (2001), no. 5, 1447--1457
  \bibitem[Kr]{Kr} S. G. Krantz: \textbf{Function Theeory of Several Complex Variables: Second Edition}. AMS Chelsea Publishing, Providence, 2001
  \bibitem[K1]{K1} S. Kobayashi: Invariant distances on complex manifolds and holomorphic mappings. \textit{Journal of the Mathematical Society of Japan}, \textbf{19} (1967), 460--480
  \bibitem[K2]{K2} S. Kobayashi: \textbf{Hyperbolic Manifolds and Holomorphic Mappings: An Introduction (Second Edition)}. World Scientific Publishing Co., Singapore, 2005
  \bibitem[R]{R} H. L. Royden: Remarks on the Kobayashi metric. In Several Complex Variables II, Proceedings of the International Mathematical Conference, Lecture Notes in Mathematics \textbf{185}, Springer, Berlin, 1971, pp. 125--137
  \bibitem[W]{W} H. Wu: Normal families of holomorphic mappings. \textit{Acta Mathematica}. \textbf{119} (1967), 193--233.
\end{thebibliography}

\addcontentsline{toc}{section}{Riferimenti bibliografici}

\newpage

\section*{Ringraziamenti}
\addcontentsline{toc}{section}{Ringraziamenti}
Volendo, si possono aggiungere dei ringraziamenti.


\end{document}
