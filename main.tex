\documentclass{article}
\usepackage{mstyle}
\usepackage{pgfplots}
\usetikzlibrary{intersections, pgfplots.fillbetween}

\title{Teoremi di tipo ``Wolff-Denjoy'' in più variabili complesse (titolo da rivedere, tanto la copertina va fatta tutta nel dettaglio)}
\author{Marco Vergamini Relatore: Marco Abate}
\date{Data: si vedrà}

\begin{document}
\maketitle
\newpage
\tableofcontents
\newpage


\section*{Introduzione}
\addcontentsline{toc}{section}{Introduzione}
L'obiettivo di questa tesi è dimostrare alcune possibili generalizzazioni, in più variabili complesse o anche per varietà complesse astratte che soddisfano opportune ipotesi, del teorema di Wolff-Denjoy, dimostrato indipendentemente nel 1926 da Denjoy in \cite{D} e da Wolff in \cite{Wo}. Riportiamo l'enunciato di tale teorema.

\begin{thm}
    (Wolff-Denjoy) Sia $f$ una funzione olomorfa nel disco unitario in $\mathbb{C}$ a valori nel disco stesso. Allora vale esattamente una delle seguenti affermazioni:
    \begin{itemize}
        \item la funzione $f$ ha un punto fisso nel disco; oppure,
        \item esiste un unico punto del bordo del disco tale che la successione delle iterate di $f$ converge, uniformemente sui compatti, a quel punto.
    \end{itemize}
\end{thm}

Era già nota da tempo la generalizzazione, dovuta ad Abate (\cite[Theorem 0.5]{A2}), per domini limitati e strettamente pseudoconvessi in più variabili.\\
SCALETTA PER DOMANI (POSSIBILE INIZIO: "Più recentemente, ..."):\\
- PARLARE DI BZ1;\\
- PARLARE DI BM;\\
- PARLARE DI CMS;\\
- RIASSUMERE COSA VIENE FATTO NELLA TESI (CERCARE DI CAPIRE SE VERRANNO FATTI I DOMINI ILLIMITATI E LE ESTENSIONI AL BORDO).

\newpage

\section{Preliminari}
\subsection{Notazione e definizioni di base}
Introduciamo la notazione che useremo:
\begin{itemize}
    \item scriviamo $\Omega$ per indicare un \textit{dominio} di $\mathbb{C}^n$, vale a dire un aperto connesso;
    \item con \textit{varietà complessa} s'intende una varietà differenziabile reale di dimensione pari con i cambi di carta olomorfi se visti come fra aperti di $\mathbb{C}^n$;
    \item data $X$ varietà complessa e $x \in X$, indichiamo con $T_xX$ lo spazio tangente a $X$ in $x$, che nel caso dei domini è canonicamente identificato con $\mathbb{C}^n$;
    \item dati $X,Y$ spazi topologici, quando parliamo di convergenza nell'insieme $C^0(X,Y)$ delle funzioni continue da $X$ a $Y$ sottintendiamo sempre che si parla della topologia compatta-aperta, che nel caso in cui $Y$ sia uno spazio metrico coincide con la topologia della convergenza uniforme sui compatti (potrebbe capitare che commetteremo abusi di notazione in merito);
    \item date $X,Y$ varietà complesse, indichiamo con $\text{Hol}(X,Y)$ l'insieme delle funzioni olomorfe da $X$ a $Y$, con $\mathcal{O}(X)$ l'insieme delle funzioni olomorfe da $X$ in $\mathbb{C}$ e con $\text{Aut}(X)$ l'insieme delle funzioni biolomorfe da $X$ in sé;
    \item data $f \in \text{Hol}(X,Y)$, indichiamo con $Df(x)$ il differenziale di $f$ in $x \in X$;
    \item il disco unitario è $\mathbb{D}=\{z \in \mathbb{C} \mid |z|<1\}$, mentre $\mathbb{D}^n$ è il polidisco in $\mathbb{C}^n$ e $\mathbb{D}_r=\{z \in \mathbb{C} \mid |z|<r\}$ è il disco di raggio $r>0$;
    \item la palla unitaria (euclidea) in $\mathbb{C}^n$ è $\mathbb{B}^n=\{z \in \mathbb{C}^n \mid \|z\|<1\}$, dove $\|\cdot\|$ indica la norma euclidea, mentre $\mathbb{B}_r^n=\{z \in \mathbb{C}^n \mid \|z\|<r\}$ è la palla (euclidea) di raggio $r>0$;
    \item dato un dominio $\Omega\subseteq\mathbb{C}^n$ e $x \in \mathbb{C}^n$, scriviamo $\delta(x)=\displaystyle\inf_{w \in \partial\Omega}\|x-w\|$ per indicare la distanza euclidea di $x$ dal bordo di $\Omega$;
    \item ALTRO?
\end{itemize}

Ricordiamo cosa sono la metrica e la distanza di Poincaré in $\mathbb{D}$.

\begin{defn}
    La \textit{metrica di Poincaré} (o \textit{iperbolica}) su $\mathbb{D}$ è data da
    \begin{equation}
        K_{\mathbb{D}}(z;v)=\frac{1}{1-|z|^2}|v|
    \end{equation}
    per ogni $z \in \mathbb{D}$ e $v \in \mathbb{C}\cong T_z^{\mathbb{C}}\mathbb{D}$. La metrica $K_{\mathbb{D}}$ è hermitiana completa con curvatura gaussiana costante uguale a $-4$.
\end{defn}

\begin{defn}
    La \textit{distanza di Poincaré} (o \textit{iperbolica}) $k_{\mathbb{D}}$ su $\mathbb{D}$ è la forma integrata della metrica di Poincaré. Per fatti noti di geometria iperbolica, è una distanza completa la cui espressione è data da
    \begin{equation}
        k_{\mathbb{D}}(z_1,z_2)=\frac{1}{2}\log{\frac{1+\left|\frac{z_1-z_2}{1-\bar{z}_1z_2}\right|}{1-\left|\frac{z_1-z_2}{1-\bar{z}_1z_2}\right|}}
    \end{equation}
    per ogni $z_1,z_2 \in \mathbb{D}$.
\end{defn}

Oltre alla curvatura negativa costante, la metrica e la distanza di Poincaré sono tali che le funzioni olomorfe dal disco unitario in sé sono semicontrazioni rispetto ad esse (lemma di Schwarz-Pick, si veda ad esempio \cite[Chapter I, paragraph 1, Theorem 1.1]{K1}).

Quello che vogliamo fare ora è generalizzare la metrica e la distanza di Poincaré ad una qualsiasi varietà complessa mantenendo queste proprietà, in particolare quella di rendere le funzioni olomorfe delle semicontrazioni. Ci sono vari modi per farlo, noi nello specifico vedremo la (pseudo)metrica e la (pseudo)distanza di Kobayashi, introdotte nel 1967 in \cite{K2}.

\begin{defn}
    Sia $X$ una varietà; la \textit{pseudometrica di Kobayashi} su $X$ è
    \begin{equation}\begin{split}
        K_X(x;Z)=&\inf\{|v| \mid v \in \mathbb{C}, \text{ esiste }f \in \text{Hol}(\mathbb{D},X) \\
        &\text{ tale che } f(0)=x, Df(0)v=Z\}
    \end{split}\end{equation}
    per ogni $x \in X$ e $Z \in T_xX$.
\end{defn}

\begin{oss}
    Non possiamo parlare di metrica perché, ad esempio, $K_{\mathbb{C}^n}\equiv 0$; vedremo però tra poco che per i domini limitati è effettivamente una metrica. Notiamo anche che, data $f \in \text{Hol}(X,Y)$, allora dalla definizione segue che $K_Y\big(f(x);Df(x)Z\big) \le K_X(x;Z)$ per ogni $x \in X$ e $Z \in T_xX$.
\end{oss}

Definiamo adesso la (pseudo)distanza di Kobayashi; più avanti vedremo (SOLO CIT DEL RISULTATO ORIGINALE O ANCHE DIM?) com'è collegata alla pseudometrica di Kobayashi.

\begin{defn}
    Sia $X$ una varietà complessa connessa; la \textit{pseudodistanza di Kobayashi} su $X$ è data da
    \begin{equation}\begin{split}
        k_X(z,w)=&\inf\Bigg\{\sum_{j=1}^m k_{\mathbb{D}}(z_{j-1},z_j) \mid m \in \mathbb{N}, z_j \in \mathbb{D}\text{ per }j=0,\dots,m\text{ tali che}\\
        &\text{esistono }\varphi_j\in \text{Hol}(\mathbb{D},X) \text{ con } \varphi_1(z_0)=z,\varphi_m(z_m)=w\Bigg\}
    \end{split}\end{equation}
    per $z,w \in X$, dove $k_{\mathbb{D}}$ è la distanza di Poincaré.
\end{defn}

\begin{oss} \label{k_lip}
    È facile vedere che $k_X$ è una pseudodistanza, ma in generale non è una distanza, ad esempio perché, come prima, $k_{\mathbb{C}^n}\equiv 0$. Inoltre, data $f \in \text{Hol}(X,Y)$, dalla definizione segue che $k_Y\big(f(x),f(y)\big) \le k_X(x,y)$ per ogni $x,y \in X$.
\end{oss}

\begin{defn}
    Sia $X$ una varietà complessa connessa. Se $k_X$ è una distanza, diremo che $X$ è \textit{Kobayashi-iperbolica}.
\end{defn}

Il seguente risultato per le varietà Kobayashi-iperboliche verrà spesso usato implicitamente.

\begin{prop}
    (INSERIRE CIT) Una varietà complessa connessa $X$ è Kobayashi-iperbolica se e solo se $k_X$ induce su $X$ la topologia di varietà.
\end{prop}

Diamo ora delle definizioni che ci serviranno per enunciare i risultati già noti nel caso dei domini regolari.

\begin{defn}
    Una funzione continua $\mu:\mathbb{C}^n \longrightarrow [0,+\infty)$ è detta \textit{funzionale di Minkowski} se
    \begin{nlist}
        \item $\mu(Z)=0$ se e solo se $Z=0$;
        \item $\mu(\zeta Z)=|\zeta|\mu(Z)$ per ogni $Z \in \mathbb{C}^n$ e $\zeta\in\mathbb{C}$.
    \end{nlist}
    Dato $\Omega\subseteq\mathbb{C}^n$ un dominio, poniamo $\mu_{\Omega}(Z)=\inf_{w\in\mathbb{C}^n\setminus\Omega}\mu(Z-w)$.
\end{defn}

\begin{defn}
    Sia $A\subseteq\mathbb{C}$ un aperto. Una funzione $u:A \longrightarrow \mathbb{R}\cup\{-\infty\}$ semicontinua superiormente è detta \textit{subarmonica} se per ogni $a \in A$, per ogni $r>0$ tale che $\overline{D(a,r)} \subset A$ e per ogni $h$ continua in $\overline{D(a,r)}$ e armonica in $D(a,r)$, se $h\restrict{\partial D(a,r)} \ge u\restrict{\partial D(a,r)}$, allora anche $h\restrict{D(a,r)} \ge u \restrict{D(a,r)}$.

    Sia $A\subseteq\mathbb{C}^n$ un aperto. Una funzione $u:A \longrightarrow \mathbb{R}\cup\{-\infty\}$ semicontinua superiormente è detta \textit{plurisubarmonica} se per ogni $a \in A$ e per ogni $Z \in \mathbb{C}^n$ l'applicazione $\zeta \longmapsto u(a+\zeta Z)$ è subarmonica dove definita.
\end{defn}

\begin{defn}
    Un dominio $\Omega \subseteq \mathbb{C}^n$ si dice \textit{(Hartogs) pseudoconvesso} se esiste un funzionale di Minkowski $\mu$ tale che $-\log\mu_{\Omega}$ è plurisubarmonica in $\Omega$.
\end{defn}

Nel caso di domini regolari, si può dare una definizione di pseudoconvessità più operativa equivalente.

\begin{defn}
    Sia $\Omega \subseteq \mathbb{C}^n$ un dominio con bordo $C^2$, cioè esiste $\rho \in C^2(\mathbb{C}^n)$ tale che $\Omega=\{\rho(z)<0\}$ e $\diff\rho\not=0$ in ogni punto di $\partial\Omega$.
    
    Dato $p \in \partial\Omega$, lo \textit{spazio tangente complesso} a $\partial\Omega$ in $p$ è
    \begin{equation}
        H_p\partial\Omega=\{Z \in \mathbb{C}^n \mid \langle \bar{\partial}\rho(p),Z\rangle=0\}.
    \end{equation}
    Diciamo che $\Omega$ è \textit{Levi pseudoconvesso} se la \textit{forma di Levi}
    \begin{equation}
        L_{\rho}(p;Z)=\sum_{\nu,\mu=1}^n \frac{\partial^2\rho}{\partial z_\nu\partial\bar{z}_\mu}(p)Z_\nu\bar{Z}_\mu, \quad Z=(Z_1,\dots,Z_n) \in \mathbb{C}^n
    \end{equation}
      è semidefinita positiva in $H_p\partial\Omega$ per ogni $p \in \partial\Omega$. Diciamo che è \textit{strettamente pseudoconvesso} se la forma di Levi è definita positiva.
\end{defn}

Vale il seguente risultato. (INSERIRE CIT, POI: MEGLIO CITARLO COME FATTO O COME TEOREMA?)
\begin{ftt}
    Sia $\Omega \subseteq \mathbb{C}^n$ un dominio limitato con bordo $C^2$. Allora $\Omega$ è Levi pseudoconvesso se e solo se è Hartogs pseudoconvesso.
\end{ftt}

Nella prossima sottosezione citeremo alcuni risultati sulla geometria dei domini limitati strettamente pseudoconvessi dotati della distanza di Kobayashi. In particolare, vedremo che sono Gromov-iperbolici, il che permette di derivare un teorema di tipo Wolff-Denjoy per questi domini.

\begin{defn}
    Sia $(X,d)$ uno spazio metrico proprio, cioè tale che ogni sottoinsieme chiuso e limitato è compatto. Dati $x,y,w \in X$ il \textit{prodotto di Gromov} tra $x$ e $y$ con punto base $w$ è $(x,y)_w=\frac{1}{2}\big(d(x,w)+d(y,w)-d(x,y)\big)$. Dato $\delta \ge 0$, diciamo che $X$ è \textit{$\delta$-iperbolico} se
    $$(x,y)_w \ge \min\{(x,z)_w,(y,z)_w\}-\delta\text{ per ogni }x,y,z,w \in X.$$
    Se $(X,d)$ è $\delta$-iperbolico per qualche $\delta \ge 0$, diremo che è \textit{Gromov-iperbolico}.

    (SERVE DAVVERO LA DEF DEL BORDO?) Fissato $w \in X$, il \textit{bordo iperbolico} $\partial_GX$ è costruito come classi di equivalenza delle successioni $(x_i)$ che convergono a infinito, cioè tali che $\displaystyle\lim_{i,j\rightarrow \infty}(x_i,x_j)_w=\infty$; due tali successioni $(x_i), (y_i)$ sono equivalenti se $\displaystyle\lim_{i\rightarrow \infty}(x_i,y_i)_w=\infty$. Questa costruzione è indipendente dalla scelta di $w$.
\end{defn}

Prima di passare a vedere i risultati noti della teoria sulla pseudometrica e la pseudodistanza di Kobayashi e sui domini strettamente pseudoconvessi, introduciamo il concetto di varietà taut, che sarà per noi un'ipotesi importante per i teoremi che andremo a dimostrare: infatti, quest'ipotesi ci darà la dicotomia nella tesi del teorema. Vedremo anche con un esempio l'importanza di tale ipotesi. Prima di dare la definizione, ci servirà un risultato sul comportamento delle funzioni olomorfe a valori in una varietà Kobayashi-iperbolica.

\begin{prop}
    (INSERIRE CIT) Sia $X$ una varietà complessa connessa. Allora $X$ è Kobayashi-iperbolica se e solo se $\text{\normalfont{Hol}}(\mathbb{D},X)$ è relativamente compatto in $C^0(\mathbb{D},X^*)$, dove $X^*$ è la compattificazione di Alexandroff di $X$. In tal caso, $\text{\normalfont{Hol}}(Y,X)$ è relativamente compatto in $C^0(Y,X^*)$ per ogni varietà complessa $Y$.
\end{prop}

\begin{defn}
    Una varietà complessa $X$ si dice \textit{taut} se è Kobayashi-iperbolica e ogni mappa nella chiusura di $\text{Hol}(\mathbb{D},X)$ in $C^0(\mathbb{D},X^*)$ è in $\text{Hol}(\mathbb{D},X)$ oppure è la mappa costante a $\infty$.
\end{defn}

Per finire, diamo delle definizioni che ci serviranno per parlare del comportamente delle iterate di funzioni olomorfe.

\begin{defn}
    Siano $X$ e $Y$ due spazi topologici. Diciamo che una successione $\{f_{\nu}\}_{\nu \in \mathbb{N}} \subset C^0(X,Y)$ è \textit{compattamente divergente} se per ogni coppia di compatti $H\subseteq X$ e $K\subseteq Y$ esiste $\nu_0 \in \mathbb{N}$ tale che $f(H)\cap K=\emptyset$ per ogni $\nu \ge \nu_0$.

    Una famiglia $\mathcal{F} \subseteq C^0(X,Y)$ è detta \textit{normale} se ogni successione in $\mathcal{F}$ ammette una sottosuccessione che converge uniformemente sui compatti oppure è compattamente divergente.
\end{defn}
\subsection{Risultati noti della teoria}
Vediamo ora alcuni risultati noti della teoria che si saranno utili nelle nostre dimostrazioni. \\

DA FARE QUI:\\
UN SACCO DI RISULTATI, DA DIMOSTRARE E NON, SU PSEUDOMETRICA E PSEUDODISTANZA\\
UN SACCO DI RISULTATI, DA DIMOSTRARE E NON, SULLE VARIETÀ TAUT\\
ALTRE COSE CHE TI VENGONO IN MENTE VIA VIA CHE SCRIVI E CHE SONO TROPPO DI BASE PER TROVARE POSTO ALTROVE\\
NON TI SCORDARE IL TEOREMA DI MONTEL (E FORSE SERVIRÀ ANCHE WEIERSTRASS); SERVIRÀ ANCHE ASCOLI ARZELÀ? SPERIAMO DI NO

\newpage

%\section{Un teorema di tipo ``Wolff-Denjoy'' per varietà taut con (???)-visibilità}
%\subsection{Il concetto di visibilità}
%Nella sezione precedente abbiamo visto come l'ipotesi di varietà taut ci permette di dire, se le orbite di una certa funzione non sono relativamente compatte, che la successione delle iterate è compattamente divergente.

Per ottenere un risultato di tipo Wolff-Denjoy, nel caso in cui le iterate siano compattamente divergenti dobbiamo dire due cose: che le iterate convergono uniformemente sui compatti a una funzione a valori nel bordo euclideo, e che in realtà tale funzione è una costante.

Per ottenere la convergenza uniforme al bordo ci basterà supporre che la varietà sia embeddata in un qualche $\mathbb{C}^d$ e limitata, dopodiché si applica il teorema di Montel. Per dire che la funzione è costante, invece, ci serviranno delle ipotesi aggiuntive di tipo geometrico: la condizione di visibilità per le simil-geodetiche. \\

Nel caso di un dominio con bordo regolare, l'ipotesi di essere strettamente pseudoconvesso permetteva di concludere la condizione geometrica di Gromov-iperbolicità; \marginpar{Capire se torna}sotto l'ipotesi di Gromov-iperbolicità, si può dimsotrare che il dominio soddisfa una condizione di visibilità per le geodetiche (si veda la discussione che precede il Theorem 1.4 in \cite{BZ1} per delle reference più approfondite).

Tuttavia, nella prossima sezione vedremo esempi di domini che soddisfano la condizione di visibilità per le simil-geodetiche ma che non sono Gromov-iperbolici.
\marginpar{Capire se è davvero questo il tipo di controesempio che ci serve}
Inoltre, non vale neanche l'altra implicazione: esistono domini Gromov-iperbolici per i quali non vale la condizione di visibilità per le circa geodetiche.

Dunque non c'è implicazione né in un senso né nell'altro, e il risultato che otteniamo è indipendente da quello per domini Gromov-iperbolici. Prima di enunciarlo, ci serviranno alcune definizioni, partendo da quella di simil-geodetica.

\begin{defn}
    Sia $X$ una varietà connessa complessa embeddata in $\mathbb{C}^d$ e limitata, e fissiamo $\lambda \ge 1$ e $\kappa \ge 0$. Sia $I\subseteq \mathbb{R}$ un intervallo; una curva $\sigma:I \longrightarrow X$ è detta una \textit{$(\lambda,\kappa)$-simil-geodetica} se
    \begin{enumerate}
        \item per ogni $s,t \in I$ si ha
        $$\frac{1}{\lambda}|t-s|-\kappa \le k_X\big(\sigma(s),\sigma(t)\big)\le\lambda|t-s|+\kappa;$$
        \item $\sigma$ è assolutamente continua (quindi $\sigma'(t)$ esiste per quasi ogni $t \in I$) e per quasi ogni $t \in I$ si ha
        $$K_X\big(\sigma(t);\sigma'(t)\big) \le \lambda.$$
    \end{enumerate}
\end{defn}

Diamo adesso la definizione di condizione di visibilità.

\begin{defn} \label{visibility}
    Sia $X$ una varietà connessa complessa embeddata in $\mathbb{C}^d$ e limitata, e fissiamo $\lambda \ge 1$ e $\kappa \ge 0$. Diciamo che $X$ ha la \textit{condizione di visibilità rispetto alle $(\lambda,\kappa)$-simil-geodetiche} se
    \begin{enumerate}
        \item ogni due punti distinti di $X$ possono essere collegati da una $(\lambda,\kappa)$-simil-geodetica;
        \item per ogni coppia di punti $p,q\in\partial X$ con $p\not=q$, esistono in $\mathbb{C}^d$ due intorni $V$ e $W$ di $p$ e $q$ rispettivamente e un compatto $K$ di $X$ tali che: $\overline{V}\cap\overline{W}=\emptyset$; ogni $(\lambda,\kappa)$-simil-geodetica in $X$ che collega un punto di $V$ a un punto di $W$ interseca $K$.
    \end{enumerate}
\end{defn}

Le simil-geodetiche sono delle curve che, a meno di costanti moltiplicative e additive, si comportano come le geodetiche, cioè le curve che minimizzano la lunghezza.

\marginpar{Qua ci va un bel disegnino in inkscape, sia del caso vietato che del caso previsto dalla definizione}
Quello che chiediamo nella Definizione \ref{visibility} è che, per andare da un punto a un altro del bordo con tali curve, allora non possiamo stare arbitrariamente vicini al bordo, ma siamo costretti a ``piegarci'' verso l'interno; in pratica, stiamo chiedendo che ci sia una sorta di curvatura negativa.

\begin{ftt}
    Il dominio $\Omega$ definito nell'Esempio \ref{servetaut} soddisfa la condizione di visibilità per le simil-geodetiche. Per vederlo, consideriamo due casi:
    \begin{nlist}
        \item uno dei due punti è l'origine. Allora basta prendere come compatto un qualsiasi insieme della forma $\{r \le |z| \le R\}$ con $0<r<R<1$ e i due intorni aperti sufficientemente piccoli;
        \item i due punti sono entrambi sulla sfera unitaria. Per \cite[Proposition 6]{NTT}, è facile vedere che, se la palla unitaria soddisfa la condizione di visibilità per simil-geodetiche, allora $\Omega$ la soddisfa in questo caso. Adesso, la palla unitaria è Gromov-iperbolica (è facile verificare, ad esempio, che è strettamente pseudoconvessa), dunque soddisfa la condizione di visibilità rispetto alle geodetiche. Usando \cite[Part III, Chapter H, Paragraph 1, Theorem 1.7]{BH} e il fatto che è uno spazio metrico proprio, possiamo facilmente concludere che soddisfa anche la condizione di visibilità per le simil-geodetiche.
    \end{nlist}
\end{ftt}
%\subsection{Risultati utili}
%Prima di andare a vedere il teorema di tipo ``Wolff-Denjoy'', dobbiamo mostrare diversi risultati preliminari. Visto che andremo a dimostrare la versione del teorema che si trova in \cite{CMS}, tali risultati sono per la maggior parte dimostrati, e il resto citati, nel suddetto articolo. \\

Cominciamo con delle stime dal basso e dall'alto per la metrica di Kobayashi, che permettono anche di ottenere la lipschitzianità delle simil-geodetiche.

\begin{lm} \label{polycpt}
    Sia $X$ una varietà complessa. Se un sottoinsieme compatto $K\subseteq X$ è contenuto nel polidisco di una carta di $X$, allora esiste una costante $C=C(K)>0$ tale che $K_X(z;Z) \le C\|Z\|_X$ per ogni $z \in K$ e $Z \in T_zX$.
\end{lm}

\begin{proof}
    Siano $n=\dim{X}$ e $D=\mathbb{D}_{r_1}\times\dots\times\mathbb{D}_{r_n}$ il polidisco che contiene $K$. Applicando la Proposizione \ref{metrdecr} all'inclusione e passando in coordinate, per ogni $z \in K$ e $Z \in T_zX$ si ha che
    $$K_X(z;Z) \le K_D(z;Z) \le \max_{j=1,\dots,n} \frac{r_j|Z_j|}{r_j^2-|z_j|^2},$$
    dove $|\cdot|$ è il modulo, cioè la norma euclidea delle coordinate della carta. Per ottenere la seconda disuguaglianza, ragioniamo al seguente modo: a meno di riscalare tutto per una costante, possiamo supporre che il membro destro sia uguale a $1$ (se fosse $0$, avremmo $Z=0$ e la tesi sarebbe immediata). Consideriamo adesso la funzione $f\in\text{Hol}(\mathbb{D},D)$ che manda $\zeta\in\mathbb{D}$ nell'elemento di $D$ che ha come $j$-esima coordinata $\dfrac{r_j\alpha_j\zeta+z_j}{1+\bar{z}_j\alpha_j\zeta/r_j}$, dove $\alpha_j=\dfrac{r_jZ_j}{r_j^2-|z_j|^2}$; allora $f(0)=z$ e si ha che $Df(0)\cdot 1=Z$ e la disuguaglianza discende dalla definizione di $K_D$. Poiché, per compattezza di $K$, la quantità $r_j^2-|z_j|^2$ è limitata dal basso da una costante positiva per $j=1,\dots,n$, esiste una costante $C_0>0$ tale che
    $$K_X(z;Z) \le C_0\max_{j=1,\dots,n}\{|Z_j|\}.$$
    
    
    Consideriamo la norma hermitiana come una funzione
    $$\|\cdot\|_X:K\times\big\{v\in\mathbb{C}^n\mid\max_{j=1,\dots,n}\{|v_j|\}=1\big\}\longrightarrow (0,+\infty),$$
    dove il secondo fattore nel dominio della funzione è visto come sottoinsieme del tangente al variare dei punti nel primo fattore. Per compattezza tale funzione ammette un minimo $c>0$; a meno di riscalare abbiamo che
    $$\|Z\|_X \ge c\max_{j=1,\dots,n}\{|Z_j|\}$$
    per ogni $z\in K$ e $Z\in T_zX$. Basta allora prendere $C=C_0/c$.
\end{proof}

\begin{prop} \label{metrica_bilip}
    Sia $X$ una varietà complessa. Allora:
    \begin{enumerate}[label={(\arabic*)}]
        \item se $X$ è connessa, è Kobayashi-iperbolica se e solo se per ogni compatto $K\subseteq X$ esiste una costante $c=c(K)>0$ tale che $c\|Z\|_X \le K_X(z;Z)$ per ogni $z \in X$ e $Z \in T_zX$;
        \item per ogni compatto $K\subseteq X$ esiste una costante $C=C(K)>0$ tale che $K_X(z;Z) \le C\|Z\|_X$ per ogni $z \in X$ e $Z \in T_zX$;
        \item se $X$ è una sottovarietà limitata di $\mathbb{C}^d$ esiste una costante $c>0$ tale che $c\|Z\| \le K_X(z;Z)$ per ogni $z \in X$ e $Z \in T_zX$.
    \end{enumerate}
\end{prop}

\begin{proof}
    (1) Mostriamo che se vale quella condizione allora $X$ è Kobayashi-iperbolica. Prendiamo due punti distinti $z_0,w_0\in X$, per cui esiste un intorno compatto $K$ di $z_0$ con $w_0\not\in K$; in particolare, ogni curva da $z_0$ a $w_0$ deve uscire da $K$. Quindi per il Teorema \ref{lung_int} si ha
    $$k_X(z_0,w_0) \ge c\cdot d_X\big(z_0,\partial_X K)>0,$$
    da cui $k_X$ è una distanza e $X$ è Kobayashi-iperbolica.

    Viceversa, supponiamo che $X$ sia Kobayashi-iperbolica. Prendiamo un punto $z_0\in X$ e fissiamo un intorno $U$ di $X$ contenuto in una carta e che sia biolomorfo a $\mathbb{B}^n$, dove $n=\dim{X}$. Consideriamo l'aperto $V\subseteq U$ corrispondente a $\mathbb{B}_{1/2}^n$ tramite il biolomorfismo, che è ancora un intorno di $z_0$. Scegliamo inoltre $\epsilon>0$ tale che $B_X(z_0,2\epsilon)\subset\subset V$ (ricordiamo che con $B_X$ si intendono le palle rispetto a $k_X$). In particolare, per la Definizione \ref{poidist}, il punto (i) della Proposizione \ref{k_polidisco} e la Proposizione \ref{semicontr}, per ogni $\varphi\in\text{Hol}(\mathbb{D},X)$ tale che $\varphi(0)\in B_X(z_0,\epsilon)$ si ha $\varphi(\mathbb{D}_{\tanh{\epsilon}})\subseteq B_X(z_0,2\epsilon)$.

    Dati $z\in B_X(z_0,\epsilon)$ e $Z\in T_zX$, siano $\varphi\in\text{Hol}(\mathbb{D},X)$ e $v\in\mathbb{C}$ tali che $\varphi(0)=z$ e $\diff_0\varphi(v)=Z$. Allora, ponendo $\psi(\zeta)=\varphi\big((\tanh{\epsilon})\zeta\big)$, abbiamo che $\psi\in\text{Hol}(\mathbb{D},V)$, $\psi(0)=z$ e $\diff_0\psi(v)=(\tanh{\epsilon})Z$. Dalla definizione della pseudometrica di Kobayashi segue che
    $$(\tanh{\epsilon})K_V(z;Z) \le k_X(z;Z)$$
    per ogni $z\in B_X(z_0,\epsilon)$ e $Z\in T_zX$. Dal punto (3), che dimostreremo indipendentemente tra poco, segue che la condizione voluta è vera in $\mathbb{B}_{1/2}^n$, e dunque in $V$, senza che la costante dipenda da un compatto, ma rispetto alla metrica euclidea. Tuttavia, considerano il compatto $\overline{\mathbb{B}_{1/2}^n}$ (motivo per cui abbiamo dovuto prendere la palla più piccola) e chiamando $E_j$ i vettori delle base canonica di $T_wX$ identificato, al variare di $w\in\overline{\mathbb{B}_{1/2}^n}$, con $\mathbb{C}^n$, si ha
    \begin{align*}
        \|Z\|_X \le \sum_{j=1}^n |a_j|\cdot\|E_j\|_X &\le \max_{\substack{w\in\overline{\mathbb{B}_{1/2}^n},\\j=1,\dots,n}}\|E_j\|_X \sum_{j=1}^n |a_j| \\
        &\le C_1\sqrt{\sum_{j=1}^n |a_j|^2}=C_1\|Z\|
    \end{align*}
    per ogni $w\in\overline{\mathbb{B}_{1/2}^n}$ e $Z\in T_wX$ della forma $Z=\displaystyle\sum_{j=1}^n a_jE_j$, e per una qualche costante $C_1>0$, per cui la condizione vale anche con la metrica $\|\cdot\|_X$. Dato allora un compatto $K$, basta ricoprirlo con un numero finito di intorni della forma $B_X(z_0,\epsilon)$ e prendere la costante più piccola al variare di tali intorni.

    (2) Per ogni $z \in K$, scegliamo un polidisco $U_z$ centrato in $z$ e contenuno in una carta di $X$; sia $U_z'\subseteq U_z$ un altro polidisco, nella stessa carta, centrato in $z$ e relativamente compatto in $U_z$ per ogni $z \in K$. Dato che $K$ è compatto, esistono $z_1,\dots,z_l$ tali che $K\subseteq\displaystyle\bigcup_{j=1}^l U_{z_j}'$. Allora, poiché $\overline{U}_{z_j}'$ è un sottoinsieme compatto di $U_{z_j}$ per $j=1,\dots,l$, per il Lemma \ref{polycpt} abbiamo
    $$K_X(z,Z) \le C_j\|Z\|_X$$
    per ogni $z \in U_{z_j}'$ e $Z\in T_zX$, dove $C_j>0$ è una costante che dipende dal compatto $\overline{U}_{z_j}'$. Basta allora porre $C(K)=\displaystyle\max_{j=1,\dots,l}\{C_j\}$.

    (1) Supponiamo per assurdo che esistano $z_j \in X$ e $Z_j \in T_{z_j}X$, con $Z_j\not=0$, tali che $\displaystyle \lim_{j \longrightarrow +\infty} K_X(z_j;Z_j)/\|Z_j\| \longrightarrow 0$. Senza perdita di generalità possiamo supporre $\|Z_j\|=1$ per ogni $j$. Per definizione di $K_X$, esistono delle funzioni $f_j\in\text{Hol}(\mathbb{D},X)$ e dei $v_j \in \mathbb{C}$ tali che $f_j(0)=z_j$ e
    $$|v_j| \le K_X(z_j;Z_j)+1/j\quad\text{ e }\quad \diff_0 f_j(v_j)=Z_j.$$

    Segue che $\displaystyle \lim_{j \longrightarrow +\infty}\|\diff_0f_j(1)\| \ge \lim_{j \longrightarrow +\infty}1/|v_j|=+\infty$. A meno di sottosuccessioni e di riordinare le coordinate, possiamo supporre che siano le prime componenti dei vettori $\diff_0f_j(1)$ a tendere a $+\infty$. Chiamiamo $g_j$ la prima componente di $f_j$, cosicché $g_j'$ è la prima componente di $\diff f_j(1)$. Le $g_j$ sono le composizioni delle $f_j$ con un embedding e una proiezione, dunque sono olomorfe; inoltre, poiché $X$ è limitata, sono equilimitate. Esiste quindi un $r>0$ tale che $g_j \in \text{Hol}(\mathbb{D},\mathbb{D}_r)$ per ogni $j$. Adesso, noi sappiamo che $\displaystyle\lim_{j\longrightarrow +\infty} |g_j'(0)|=+\infty$; basta allora applicare il lemma di Schwarz a $\dfrac{g_j-g_j(0)}{2r}$ con $j$ sufficientemente grande per ottenere una contraddizione.
\end{proof}

\begin{oss} \label{basta_herm}
    Per il punto (3) della Proposizione precedente non è necessario che la metrica sia quella euclidea, basta una metrica hermitiana qualsiasi. Per vederlo, basta usare il punto (1) e la Proposizione \ref{metrdecr}.
\end{oss}

\begin{cor} \label{dist_da_sotto}
    Siano $Y$ una varietà Kobayashi-iperbolica e $X$ una sottovarietà complessa, connessa e relativamente compatta di $Y$. Allora esiste $c>0$ tale che $c\cdot d_X(z,w) \le k_X(z,w)$ per ogni $z,w \in X$.
\end{cor}

\begin{proof}
    Per il Teorema \ref{lung_int} ci basta mostrare che $c\cdot d_X(z,w) \le l_X(\gamma)$ per ogni curva $C^1$ a tratti $\gamma:[a,b] \longrightarrow X$ tale che $\gamma(a)=z$ e $\gamma(b)=w$. Prendendo $c=c(\overline{X})>0$ dato dal punto (1) della Proposizione \ref{metrica_bilip}, ricordando che $\|\cdot\|_X=\|\cdot\|_Y\restrict{X}$ e usando la Proposizione \ref{metrdecr}, abbiamo che
    \begin{align*}
        l_X(\gamma)&=\int_a^b K_X\big(\gamma(t);\gamma'(t)\big)\diff t \ge \int_a^b K_Y\big(\gamma(t);\gamma'(t)\big)\diff t\\
        &\ge \int_a^b c\|\gamma'(t)\|_Y\diff t=\int_a^b c\|\gamma'(t)\|_X\diff t\ge c\cdot d_X(z,w),
    \end{align*}
    come voluto.
\end{proof}

\begin{prop} \label{simillip}
    Siano $Y$ una varietà Kobayashi-iperbolica e $X$ una sottovarietà complessa, connessa e relativamente compatta di $Y$. Allora per ogni $\lambda \ge 1$ esiste una costante $C=C(\lambda)>0$ tale che ogni $(\lambda,\kappa)$-simil-geodetica è $C$-lipschitziana rispetto a $d_X$.
\end{prop}

\begin{proof}
    Ogni $(\lambda,\kappa)$-simil-geodetica $\sigma:I\longrightarrow X$ è, per definizione, assolutamente continua rispetto a $d_X$. Allora, per il teorema fondamentale del calcolo integrale, per ogni $s,t\in I$ abbiamo che
    $$\sigma(t)=\sigma(s)+\int_s^t\sigma'(r)\diff r.$$
    
    Per il punto (1) della Proposizione \ref{metrica_bilip}, esiste una costante $c=c(\overline{X})>0$ tale che $c\|Z\|_Y \le K_Y(z;Z)$ per ogni $z\in X$ e $Z\in T_zX$, e per definizione di $(\lambda,\kappa)$-simil-geodetica $K_X\big(\sigma(t);\sigma'(t)\big) \le \lambda$ per quasi ogni $t\in I$. Dunque, usando anche la Proposizione \ref{metrdecr}, si ha che $\|\sigma'(t)\|_Y \le \lambda/c$ per quasi ogni $t\in I$, da cui
    $$d_X\big(\sigma(t),\sigma(s)\big) \le \int_s^t\|\sigma'(r)\|_X\diff r=\int_s^t\|\sigma'(r)\|_Y\diff r \le \frac{\lambda}{c}|t-s|,$$
    cioè $\sigma$ è $\lambda/c$-lipschitziana rispetto alla distanza euclidea.
\end{proof}

Il seguente Lemma è un fatto tecnico che ci servirà tra poco.

\begin{lm} \label{restrdis}
    Siano $X$ una varietà complessa e connessa e $\sigma:[a,b] \longrightarrow X$ una curva assolutamente continua rispetto a $d_X$. Se
    $$l_X(\sigma) \le k_X\big(\sigma(a),\sigma(b)\big)+\kappa,$$
    allora, per ogni $a \le s \le t \le b$, si ha
    $$l_X(\sigma\restrict{[s,t]}) \le k_X\big(\sigma(s),\sigma(t)\big)+\kappa.$$
\end{lm}

\begin{proof}
    Siano $s$ e $t$ come sopra. Allora
    $$l_X(\sigma\restrict{[s,t]})=l_X(\sigma)-l_X(\sigma\restrict{[a,s]})-l_X(\sigma\restrict{[t,b]}).$$

    Usando la nostra ipotesi e il punto (ii) del Teorema \ref{lung_int}, troviamo
    $$l_X(\sigma\restrict{[s,t]}) \le k_X\big(\sigma(a),\sigma(b)\big)+\kappa-k_X\big(\sigma(a),\sigma(s)\big)-k_X\big(\sigma(t),\sigma(b)\big);$$
    applicando la disuguaglianza triangolare, si ottiene la tesi.
\end{proof}

 Adesso vogliamo mostrare che le varietà Kobayashi-iperboliche sono connesse per archi simil-geodetici.

\begin{thm} \label{similgeo_conn}
    Sia $X$ una varietà Kobayashi-iperbolica. Per ogni $z,w \in X$ e ogni $\kappa>0$ esiste una $(1,\kappa)$-simil-geodetica $\sigma:[a,b]\longrightarrow X$ tale che $\sigma(a)=z$ e $\sigma(b)=w$.
\end{thm}

\begin{proof}
    Per il punto (i) del Teorema \ref{lung_int}, a meno di riparametrizzare esiste una curva $C^1$ a tratti $\gamma:[0,1] \longrightarrow X$ tale che $\gamma(0)=z,\gamma(1)=w$ e
    $$l_X(\gamma)<k_X(z,w)+\kappa;$$
    inoltre, a meno di perturbare di poco la curva, possiamo assumere che sia $C^1$ e che $\gamma'(t)\not=0$ per ogni $t\in[0,1]$. Consideriamo la funzione $f:[0,1] \longrightarrow [0,+\infty)$ data da
    $$f(t)=\int_0^t K_X\big(\gamma(r);\gamma'(r)\big)\diff r.$$

    Poiché $\gamma([0,1])$ è compatto in $X$, per i punti (1) e (2) della Proposizione \ref{metrica_bilip} esiste $C>0$ tale che
    $$\frac{1}{C}\|\gamma'(t)\|_X \le K_X\big(\gamma(t);\gamma'(t)\big) \le C\|\gamma'(t)\|_X\text{ per ogni }t\in[0,1].$$

    Dato che $\|\gamma'(t)\|_X>0$ per ogni $t\in[0,1]$ e $\gamma'$ è continua, esistono $A,B>0$ tali che $A \le \|\gamma'(t)\|_X\le B$ per ogni $t\in[0,1]$. Dunque $f$ è una funzione bilipschitziana e, di conseguenza, strettamente crescente. Sia ora $g:[0,l_X(\gamma)] \longrightarrow [0,1]$ l'inversa di $f$. Vogliamo dire che la curva $\sigma=\gamma\circ g:[0,l_X(\gamma)] \longrightarrow X$ è una $(1,\kappa)$-simil-geodetica; sostanzialmente, $\sigma$ è la riparametrizzazione per lunghezza d'arco di $\gamma$.

    Poiché $g$ è bilipschitziana (perché lo è la sua inversa) e $\gamma$ è $C^1$, abbiamo che $\sigma$ è lipschitziana, per cui anche assolutamente continua, rispetto a $d_X$; allora, per i $t$ per i quali $g'(t)$ esiste, si ha $\sigma'(t)=\gamma'\big(g(t)\big)g'(t)$. Inoltre, per tali $t$ anche $f'\big(g(t)\big)$ esiste ed è non-nullo, e $g'(t)=1/f'\big(g(t)\big)>0$. Per il teorema fondamentale del calcolo integrale, si ha che $f'$ esiste per quasi ogni $s\in[0,1]$ e $f'(s)=K_X\big(\gamma(s);\gamma'(s)\big)$. Siccome $g$ è bilipschitziana, la preimmagine degli $s\in[0,1]$ per cui $f'(s)$ esiste è un sottoinsieme di $[0,l_X(\gamma)]$ di misura piena. Visto che $\gamma'(s)\not=0$ per ogni $s\in[0,1]$, otteniamo che
    $$g'(t)=\frac{1}{K_X\Big(\gamma\big(g(t)\big);\gamma'\big(g(t)\big)\Big)}$$
    per quasi ogni $t\in[0,l_X(\gamma)]$. Per tali $t$ si ha che
    $$K_X\big(\sigma(t);\sigma'(t)\big)=K_X\Big(\gamma\big(g(t)\big);\gamma'\big(g(t)\big)g'(t)\Big)=1;$$
    quindi $l_X(\sigma)=l_X(\gamma) \le k_X(z,w)+\kappa$. Per il Lemma \ref{restrdis} si ha, per ogni $0 \le s \le t \le l_X(\gamma)$, che
    $$|t-s|=l_X(\sigma\restrict{[s,t]}) \le k_X\big(\sigma(s),\sigma(t)\big)+\kappa.$$

    Dato che $\sigma$ è assolutamente continua, per il punto (ii) del Teorema \ref{lung_int} abbiamo anche che
    $$k_X\big(\sigma(s),\sigma(t)\big) \le l_X(\sigma\restrict{[s,t]})=|s-t|$$
    per ogni $0 \le s \le t \le l_X(\gamma)$. Segue dunque che $\sigma$ è una $(1,\kappa)$-simil-geodetica.
\end{proof}

Adesso ci servirà un Lemma quasi ovvio.

\begin{lm} \label{quasiovvio}
    Sia $X$ una varietà complessa e connessa. Se $\sigma:[a,b] \longrightarrow X$ è una $(1,\kappa)$-simil-geodetica per qualche $\kappa>0$, allora per ogni $t \in [a,b]$ si ha
    $$k_X\big(\sigma(a),\sigma(t)\big)+k_X\big(\sigma(t),\sigma(b)\big) \le k_X\big(\sigma(a),\sigma(b)\big)+3\kappa.$$
\end{lm}

\begin{proof}
    È un'immediata conseguenza della definizione di $(1,\kappa)$-simil-geodetica.
\end{proof}

Il seguente Lemma, invece, ci servirà per la prossima dimostrazione. È un risultato sulla convergenza puntuale, mentre più avanti ne vedremo uno sulla convergenza uniforme sui compatti per varietà $(1,\kappa_0)$-visibili per qualche $\kappa_0>0$.

\begin{lm} \label{lemmasottosotto1}
    Sia $X$ una sottovarietà complessa e relativamente compatta di una varietà complessa e connessa $Y$, e sia $F\in\textnormal{Hol}(X,X)$. Dati una funzione strettamente crescente $\mu:\mathbb{N}\longrightarrow\mathbb{N}$ e $z\in X$, se l'insieme dei punti limite della successione $\{F^{\mu(j)}(z)\}_{j\in\mathbb{N}}$ consiste in un solo punto $\xi$ allora abbiamo che $\displaystyle\lim_{j\longrightarrow+\infty}F^{\mu(j)}(z)\longrightarrow\xi$.
\end{lm}

\begin{proof}
    Supponiamo per assurdo che non valga la tesi. Allora esistono un $\epsilon>0$ e una sottosuccessione $\{j_n\}_{n\in\mathbb{N}}$ tali che $d_Y\big(F^{\mu(j_n)}(z),\xi\big) \ge \epsilon$ per ogni $n\in\mathbb{N}$. Per relativa compattezza di $X$, a meno di prendere un'ulteriore sottosuccessione possiamo suppore che $\displaystyle\lim_{n\longrightarrow+\infty}F^{\mu(j_n)}(z)\longrightarrow\xi'$. Per ipotesi dev'essere $\xi'=\xi$, in contraddizione con il fatto che $d_Y\big(F^{\mu(j_n)}(z),\xi\big) \ge \epsilon$ per ogni $n\in\mathbb{N}$.
\end{proof}

Quello che andremo ora a dimostrare è uno dei fatti cruciali per ottenere il teorema di tipo ``Wolff-Denjoy''. Esso afferma che, sotto condizioni di visibilità per le simil-geodetiche, le sottosuccessioni di iterate di una funzione olomorfa che ``tendono a infinito'' convergono tutte, puntualmente, a un unico punto del bordo.

\begin{prop} \label{inf_impl_ugu}
    Sia $X$ una sottovarietà complessa, connessa e relativamente compatta di una varietà Kobayashi-iperbolica $Y$. Supponiamo che esista un $\kappa_0>0$ tale che $X$ sia $(1,\kappa_0)$-visibile. Data una funzione $F\in\textnormal{Hol}(X,X)$, esiste $\xi\in\partial_YX$ tale che per ogni funzione $\mu:\mathbb{N}\longrightarrow\mathbb{N}$ strettamente crescente per cui esiste $y_0 \in X$ tale che
    \begin{equation}
        \lim_{j\longrightarrow+\infty} k_X\big(F^{\mu(j)}(y_0),y_0\big)=+\infty
    \end{equation}
    si ha
    \begin{equation}
        \lim_{j\longrightarrow+\infty} F^{\mu(j)}(z)=\xi
    \end{equation}
    per ogni $z \in X$.
\end{prop}

\begin{proof}
    Se $\displaystyle\limsup_{n\longrightarrow+\infty}k_X\big(F^n(x),x\big)<+\infty$ per ogni $x\in X$ l'affermazione è vera a vuoto per ogni $\xi\in\partial_YX$. Altrimenti, esiste $x_0\in X$ tale per cui possiamo prendere una funzione strettamente crescente $\nu:\mathbb{N}\longrightarrow\mathbb{N}$ tale che:
    \begin{itemize}
        \item si ha $k_X\big(F^{\nu(j)}(x_0),x_0\big) \ge k_X\big(F^k(x_0),x_0\big)$ per ogni $j \in \mathbb{N}$ e per ogni $k \le \nu(j)$;
        \item si ha $\displaystyle\lim_{j\longrightarrow+\infty}k_X\big(F^{\nu(j)}(x_0),x_0\big)=+\infty$;
        \item la successione $\{F^{\nu(j)}(x_0)\}_{j\in\mathbb{N}}$ converge a un certo $\xi\in\partial_YX$.
    \end{itemize}
    
    Infatti, definendo induttivamente $l_0=0$ e $l_n$ come il minimo numero naturale $h>l_{n-1}$ tale che $k_X\big(F^h(x_0),x_0\big) \ge \max\left\{n,k_X\big(F^{l_{n-1}}(x_0),x_0\big)\right\}$, abbiamo $k_X\big(F^{l_n}(x_0),x_0\big) \ge k_X\big(F^k(x_0),x_0\big)$ per ogni $n$ e per ogni $k \le l_n$; inoltre, abbiamo $\displaystyle\lim_{n\longrightarrow+\infty}k_X\big(F^{l_n}(x_0),x_0\big)=+\infty$. Per limitatezza di $X$, esiste un'ulteriore sottosuccessione $l_{n_j}$ tale che $F^{l_{n_j}}(x_0)$ converge a un certo $\xi\in\overline{X}$, e in realtà $\xi\in\partial_YX$ visto che la distanza di Kobayashi da $x_0$ tende a $+\infty$; basta allora porre $\nu(j)=l_{n_j}$.\\

    Vogliamo ora mostrare la seguente asserzione.

    Siano $\{m_j\}_{j\in\mathbb{N}}$ e $\{m'_j\}_{j\in\mathbb{N}}$ due successioni strettamente crescenti di numeri naturali e $z_0,z'_0\in X$ tali che:
    \begin{enumerate}[label={(\arabic*)}]
        \item per ogni $j\in\mathbb{N}$ si ha $m_j \ge m'_j$;
        \item per ogni $j\in\mathbb{N}$ e $k \le m_j$ si ha $k_X\big(F^{m_j}(z_0),z_0\big) \ge k_X\big(F^k(z_0),z_0\big)$;
        \item si ha $\displaystyle\lim_{j\longrightarrow+\infty}k_X\big(F^{m_j}(z_0),z_0\big)=\lim_{j\longrightarrow+\infty}k_X\big(F^{m'_j}(z'_0),z_0\big)=+\infty$;
        \item le successioni $\{F^{m_j}(z_0)\}_{j\in\mathbb{N}}$ e $\{F^{m'_j}(z'_0)\}_{j\in\mathbb{N}}$ convergono, rispettivamente, a $\zeta$ e $\zeta'$ in $\partial_YX$;
    \end{enumerate}
    allora $\zeta=\zeta'$.

    Supponiamo per assurdo che $\zeta\not=\zeta'$. Per l'Osservazione \ref{kobisdist} anche $X$ è Kobayashi-iperbolica; quindi grazie al Teorema \ref{similgeo_conn} possiamo scegliere, per ogni $j\in\mathbb{N}$, una $(1,\kappa_0)$-simil-geodetica $\sigma_j:[0,T_j] \longrightarrow X$ tale che $\sigma_j(0)=F^{m_j}(z_0)$ e $\sigma_j(T_j)=F^{m'_j}(z'_0)$. Adesso, dato che abbiamo assunto che $\{F^{m_j}(z_0)\}_{j\in\mathbb{N}}$ e $\{F^{m'_j}(z'_0)\}_{j\in\mathbb{N}}$ convergano a due punti di $\partial_YX$ distinti e $X$ ha la visibilità rispetto alle $(1,\kappa_0)$-simil-geodetiche, esistono una costante $0<R<+\infty$ e, per ogni $j\in\mathbb{N}$, un $t_j\in[0,T_j]$ tali che $k_X\big(z_0,\sigma_j(t_j)\big)<R$. Per il Lemma \ref{quasiovvio} si ha dunque che
    \begin{equation} \label{3k_02R}
        \begin{aligned}
            k_X\big(F^{m_j}(z_0),F^{m'_j}(z'_0)\big) &\ge k_X\big(F^{m_j}(z_0),\sigma_j(t_j)\big)+k_X\big(\sigma_j(t_j),F^{m'_j}(z'_0)\big)-3\kappa_0\\
            &\ge k_X\big(F^{m_j}(z_0),z_0\big)-k_X\big(z_0,\sigma_j(t_j)\big)\\
            &\quad+k_X\big(F^{m'_j}(z'_0),z_0\big)-k_X\big(z_0,\sigma_j(t_j)\big)-3\kappa_0\\
            &\ge k_X\big(F^{m_j}(z_0),z_0\big)+k_X\big(F^{m'_j}(z'_0),z_0\big)-3\kappa_0-2R;
        \end{aligned}
    \end{equation}
    d'altra parte, abbiamo anche che
    \begin{equation} \label{-3k_02R}
        \begin{aligned}
            k_X\big(F^{m_j}(z_0),F^{m'_j}(z'_0)\big) &\le k_X\big(F^{m_j-m'_j}(z_0),z'_0\big) \\
            &\le k_X\big(F^{m_j-m'_j}(z_0),z_0\big)+k_X(z_0,z'_0) \\
            &\le k_X\big(F^{m_j}(z_0),z_0\big)+k_X(z_0,z'_0),
        \end{aligned}
    \end{equation}
    dove per la prima e la terza disuguaglianza abbiamo usato, rispettivamente, le condizioni (1) e (2) sulle successioni $\{m_j\}_{j\in\mathbb{N}}$ e $\{m'_j\}_{j\in\mathbb{N}}$; nella prima, abbiamo anche usato che le funzioni olomorfe sono semicontrazioni rispetto a $k_X$. Concatenando la \eqref{3k_02R} e la \eqref{-3k_02R} e riarrangiando i termini, otteniamo
    $$k_X\big(F^{m'_j}(z'_0),z_0\big) \le k_X(z_0,z'_0)+3\kappa_0+2R,$$
    che è in contraddizione con la condizione (3).\\

    Adesso che l'asserzione è stata dimostrata, possiamo concludere la dimostrazione. Usando la disuguaglianza triangolare e il fatto che le funzioni olomorfe sono semicontrazioni rispetto a $k_X$, troviamo che
    \begin{gather*}
        k_X\big(F^{\mu(j)}(z),z'\big) \ge k_X\big(F^{\mu(j)}(y_0),y_0\big)-k_X(y_0,z)-k_X(z',y_0);
    \end{gather*}
    segue che $\displaystyle\lim_{j\longrightarrow+\infty}k_X\big(F^{\mu(j)}(z),z'\big)=+\infty$ per ogni $z,z' \in X$.

     Fissiamo ora uno $z\in X$, e prendiamo $\xi'$ punto limite di $\{F^{\mu(j)}(z)\}_{j\in\mathbb{N}}$. Allora deve esistere una funzione strettamente crescente $\tau:\mathbb{N}\longrightarrow\mathbb{N}$ tale che $\displaystyle \lim_{j\longrightarrow+\infty} F^{(\mu\circ\tau)(j)}(z)=\xi'$; inoltre, poiché $\displaystyle\lim_{j\longrightarrow+\infty}k_X\big(F^{\mu(j)}(z),z\big)=+\infty$, dev'essere $\xi'\in\partial_YX$. Scegliamo una funzione strettamente crescente $\tau':\mathbb{N}\longrightarrow\mathbb{N}$ tale che $\nu\circ\tau' \ge \mu\circ\tau$ e applichiamo l'asserzione dimostrata sopra alle successioni $m_j=(\nu\circ\tau')(j), m'_j=(\mu\circ\tau)(j)$ e ai punti $z_0=x_0,z'_0=z$, per i quali si verificano facilmente le condizioni (1), (2), (3) e (4). Troviamo così $\xi'=\xi$ e si conclude grazie al Lemma \ref{lemmasottosotto1}.
\end{proof}

Anche il seguente teorema ci aiuterà nella nostra dimostrazione. Esso afferma che, sotto condizioni di visibilità per le simil-geodetiche, le successioni di funzioni olomorfe che convergono uniformemente sui compatti di $X$ devono necessariamente convergere a una costante.

\begin{thm} \label{lim_is_const}
    Sia $X$ una sottovarietà complessa, connessa e relativamente compatta di una varietà Kobayashi-iperbolica $Y$. Supponiamo che esista una costante $\kappa_0>0$ tale che $X$ sia $(1,\kappa_0)$-visibile. Consideriamo una successione $\{F_\nu\}_{\nu\in\mathbb{N}} \subseteq \textnormal{Hol}(X,X)$ che converge uniformemente sui compatti di $X$ a una $\psi\in\textnormal{Hol}(X,Y)$ con $\psi(X)\subseteq\partial_YX$. Allora $\psi$ è costante.
\end{thm}

\begin{proof}
    Supponiamo, per assurdo, che $\psi$ non sia costante. Allora, dati $x \in X$ e $\epsilon>0$, la restrizione di $\psi$ a $B_X(x,\epsilon)$, la palla aperta rispetto alla distanza di Kobayashi di centro $x$ e raggio $\epsilon$, non è costante. Infatti, tale palla è un aperto di $X$, e se $\psi$ fosse costante su di essa lo sarebbe su tutta $X$ per il principio di identità delle funzioni olomorfe, poiché $X$ è connessa. Questa, però, sarebbe una contraddizione alla nostra assunzione.

    Fissiamo $x_0 \in X$ e poniamo $\epsilon_0=\kappa_0/3$. Per quanto detto, deve esistere un $x_1 \in B_X(x_0,\epsilon_0)$ tale che $\{F_\nu(x_0)\}_{\nu\in\mathbb{N}}$ e $\{F_\nu(x_1)\}_{\nu\in\mathbb{N}}$ convergono, rispettivamente, a $\xi$ e $\eta$, con $\xi,\eta \in \partial_YX$ e $\xi\not=\eta$. Consideriamo adesso una $(1,\kappa_0/3)$-simil-geodetica $\gamma:[a,b]\longrightarrow X$ tale che $\gamma(a)=x_0$ e $\gamma(b)=x_1$, che esiste per il Teorema \ref{similgeo_conn}; da \eqref{simil-geo1} nella definizione di simil-geodetica si ha che
    $$|a-b|-\kappa_0/3 \le k_X(x_0,x_1) \implies |a-b| \le k_X(x_0,x_1)+\kappa_0/3 \le 2\kappa_0/3.$$

    Per ogni $\nu\in\mathbb{N}$, definiamo $\sigma_\nu:[a,b]\longrightarrow X$ come $\sigma_\nu=F_\nu\circ\gamma$. Mostriamo che $\sigma_\nu$ è una $(1,\kappa_0)$-simil-geodetica per ogni $\nu\in\mathbb{N}$. Per ogni $s,t \in [a,b]$ si ha
    \begin{align*}k_X\big(\sigma_\nu(s),\sigma_\nu(t)\big)&=k_X\Big(F_\nu\big(\gamma(s)\big),F_\nu\big(\gamma(t)\big)\Big)\\
        &\le k_X\big(\gamma(s),\gamma(t)\big) \le |s-t|+\kappa_0/3 \le |s-t|+\kappa_0,
    \end{align*}
    dove abbiamo usato che le funzioni olomorfe sono delle semicontrazioni rispetto alla distanza di Kobayashi e, dato che $\gamma$ è una $(1,\kappa_0/3)$-simil-geodetica, la \eqref{simil-geo1}. Inoltre, si ha che $|s-t|-\kappa_0 \le |a-b|-\kappa_0 \le -\kappa_0/3<0$ per ogni $s,t \in [a,b]$; dunque
    \begin{equation} \label{snu1}
        |s-t|-\kappa_0 \le k_X\big(\sigma_\nu(s),\sigma_\nu(t)\big) \le |s-t|+\kappa_0.
    \end{equation}

    Infine, per ogni $t$ si ha
    \begin{equation} \label{snu2}
        \begin{aligned}
            K_X\big(\sigma_\nu(t);\sigma_\nu'(t)\big)&=K_X\Big(F_\nu\big(\gamma(s)\big);\diff_{\gamma(t)}F_\nu\big(\gamma'(t)\big)\Big) \\
            &\le K_X\big(\gamma(t);\gamma'(t)\big) \le 1,
        \end{aligned}
    \end{equation}
    dove abbiamo usato che le funzioni olomorfe non aumentano la metrica di Kobayashi e, visto che $\gamma$ è una $(1,\kappa_0/3)$-simil-geodetica, la \eqref{simil-geo2}. Si ha anche che la curva $\sigma_\nu$ è assolutamente continua rispetto a $d_X$ per ogni $\nu$, poiché composizione di $\gamma$, che è assolutamente continua per definizione di $(1,\kappa_0/3)$-simil-geodetica, e $F_\nu$, che è olomorfa e dunque lipschitziana (sempre rispetto a $d_X$) sul compatto $\gamma([a,b])$. Grazie alla \eqref{snu1} e alla \eqref{snu2}, possiamo concludere che $\sigma_\nu$ è una $(1,\kappa_0)$-simil-geodetica per ogni $\nu$.
    
    Adesso, poiché $\{F_\nu\}_{\nu\in\mathbb{N}}$ converge uniformemente sui compatti di $X$ a una funzione olomorfa $\psi$ a valori in $\partial_YX$, ne consegue che per ogni compatto $K\subseteq X$ esiste un $\nu(K)\in\mathbb{N}$ tale che, per ogni $\nu\ge\nu(K)$, si ha $\sigma_\nu([a,b])\cap K=\emptyset$. Ma così troviamo una contraddizione, perché $X$ è $(1,\kappa_0)$-visibile.
\end{proof}

Concludiamo la questa sottosezione con l'analogo del Lemma \ref{lemmasottosotto1} per la convergenza uniforme sui compatti.

\begin{lm} \label{lemmasottosotto2}
    Sia $X$ una sottovarietà complessa, connessa e tautly embdedded di una varietà Kobayashi-iperbolica $Y$. Supponiamo che esista un $\kappa_0>0$ tale che $X$ sia $(1,\kappa_0)$-visibile. Consideriamo una $F\in\textnormal{Hol}(X,X)$ tale che la successione delle iterate di $F$ è compattamente divergente, e supponiamo che esistano un compatto $K$ di $X$, una funzione strettamente crescente $\mu:\mathbb{N}\longrightarrow\mathbb{N}$ e $\xi\in\partial_YX$ tali che la successione $\{F^{\mu(j)}\}_{j\in\mathbb{N}}$ converge alla costante $\xi$ uniformemente su $K$; allora converge alla costante $\xi$ uniformemente su tutti i compatti di $X$.
\end{lm}

\begin{proof}
    Supponiamo per assurdo che non valga la tesi; allora esistono un compatto $H\subseteq X$, un $\epsilon>0$, una successione di numeri naturali $\{j_n\}_{n\in\mathbb{N}}$ e una successione $\{z_n\}_{n\in\mathbb{N}}$ con $z_n\in H$ per ogni $n\in\mathbb{N}$ tali che $d_Y\big(F^{\mu(j_n)}(z_n),\xi\big)\ge\epsilon$ per ogni $n\in\mathbb{N}$. Per la Proposizione \ref{montel} $\text{Hol}(X,X)$ è relativamente compatto in $\text{Hol}(X,Y)$; dalla dimostrazione della suddetta Proposizione segue anche che $C^0(X,Y)$ è metrizzabile, per cui $\text{Hol}(X,X)$ è relativamente compatto per successioni in $\text{Hol}(X,Y)$. Possiamo allora trovare una sottosuccessione $\{j_{n_m}\}_{m\in\mathbb{N}}$ tale che la successione $\{F^{\mu(j_{n_m})}\}_{m\in\mathbb{N}}$ converge uniformemente sui compatti di $X$ a una funzione olomorfa $\tilde{F}:X \rightarrow Y$ con $\tilde{F}(X)\subseteq\overline{X}$. Poiché la successione delle iterate di $F$ è compattamente divergente, dev'essere $\tilde{F}(X)\subseteq\partial_YX$; allora per il Teorema \ref{lim_is_const} $\tilde{F}$ è costante. Siccome su $K$ la successione $\{F^{\mu(j)}\}_{j\in\mathbb{N}}$ converge alla costante $\xi$, la funzione $\tilde{F}$ dev'essere proprio tale costante; ma per costruzione la convergenza non può essere uniforme sul compatto $H$, assurdo.
\end{proof}
%\subsection{Teorema di tipo ``Wolff-Denjoy'' e sua dimostrazione (cambia i titoli che fanno schifo...)}
%Andiamo adesso ad enunciare e dimostrare la versione più generale di un teorema di tipo Wolff-Denjoy per varietà Kobayashi-iperboliche. Riportiamo la dimostrazione data in \cite{CMS}, ma notiamo che la strategia e le tecniche impiegate sono sostanzialmente riprese da \cite{BZ1} e \cite{BM}. Ognuno di questi articoli ha generalizzato il risultato ottenuto nel precedente.

\begin{thm}
    (\cite[Theorem 1.15]{CMS}) Sia $X$ una varietà complessa, connessa, embeddata in $\mathbb{C}^d$ e limitata. Supponiamo che $X$ sia taut e che esista un $\kappa_0>0$ tale che $X$ soddisfi la condizione di visibilità rispetto alle $(1,\kappa_0)$-simil-geodetiche.
    
    Sia $F:X \longrightarrow X$ una funzione olomorfa. Allora vale esattamente una delle seguenti affermazioni:
    \begin{itemize}
        \item le orbite di $F$ sono relativamente compatte in $X$;
        \item esiste un unico punto di $\partial X$ tale che le iterate di $F$ convergono tutte, uniformemente sui compatti, a quel punto.
    \end{itemize}
\end{thm}

\begin{proof}
    Poiché $X$ è taut, per il Teorema \ref{dicotomia} o l'orbita di $z$ tramite $F$ è relativamente compatta per ogni $z \in X$, oppure la successione delle iterate $\{F^\nu\}_{\nu\in\mathbb{N}}$ è compattamente divergente. Supponiamo che le orbite di $F$ non siano relativamente compatte in $X$; allora la successione delle iterate di $F$ è compattamente divergente.

    Consideriamo una sottosuccessione qualsiasi di $\{F^\nu\}_{\nu\in\mathbb{N}}$. Usando il teorema di Montel, possiamo trovare una sottosottosuccessione che converge uniformemente sui compatti a una funzione olomorfa $\tilde{F}:X \rightarrow \overline{X}$. Infatti, poiché $X$ è limitata allora la sottosuccessione è uniformemente limitata sui compatti; possiamo dunque fissare un ricoprimento numerabile fatto di aperti relativamente compatti contenuti in carte di $X$ ed estrarre, con un procedimento diagonale nel quale si applica il teorema di Montel ad ogni tale carta, una sottosottosuccessione che converga uniformemente sui compatti di ogni carta, in particolare sugli aperti del ricoprimento. Prendendo un'esaustione in compatti di $X$ (ad esempio, le palle chiuse rispetto a $k_X$) si può concludere facilmente che la convergenza è uniforme su tutti i compatti di $X$.

    Poiché le iterate di $F$ sono compattamente divergenti, si ha che $\tilde{F}\in\text{Hol}(X,\partial X)$. Allora, per il Teorema \ref{lim_is_const}, troviamo che $\tilde{F}$ è costante. Identifichiamo quindi
    $$\Gamma:=\overline{\{F^\nu\}}_{\nu\in\mathbb{N}}\setminus\{F^\nu\}_{\nu\in\mathbb{N}}$$
    come un insieme di punti di $\partial X$, dove la chiusura è intesa rispetto alla topologia compatta-aperta. Supponiamo, per assurdo, che $\Gamma$ contenga almeno due punti. \\

    Caso 1: esiste (e quindi per ogni) $o \in X$ tale che
    $$\limsup_{\nu\longrightarrow+\infty} k_X\big(F^\nu(o),o\big)=+\infty.$$
    Possiamo dunque scegliere una sottosuccessione $\{\nu_j\}_{j\in\mathbb{N}}$ tale che
    \begin{enumerate}[label={(\arabic*)}]
        \item per ogni $j\in\mathbb{N}$ e $k\le\nu_j$ si ha $k_X\big(F^k(o),o\big) \le k_X\big(F^{\nu_j}(o),o\big)$;
        \item $\{F^{\nu_j}\}_{j\in\mathbb{N}}$ converge, uniformemente sui compatti di $X$, a $\xi\in\partial X$.
    \end{enumerate}

    Adesso, poiché abbiamo assunto che $\Gamma$ contenga almeno due elementi, esiste una sottosuccessione $\{\mu_j\}_{j\in\mathbb{N}}$ tale che $\{F^{\mu_j}\}_{j\in\mathbb{N}}$ converge, uniformemente sui compatti di $X$, a $\eta\in\partial X$ con $\eta\not=\xi$. Segue immediatamente dalla Proposizione \ref{inf_impl_ugu} che non possiamo avere $\displaystyle\limsup_{j\longrightarrow+\infty} k_X\big(F^{\mu_j}(o),o\big)=+\infty$. Perciò dev'essere $\displaystyle\limsup_{j\longrightarrow+\infty} k_X\big(F^{\mu_j}(o),o\big)<+\infty$, quindi
    \begin{align*}
        \limsup_{h\longrightarrow+\infty}\limsup_{j\longrightarrow+\infty}\, &k_X\big(F^{\nu_h}(o),F^{\mu_j}(o)\big) \\
        &\ge \limsup_{h\longrightarrow+\infty}\limsup_{j\longrightarrow+\infty} \Big(k_X\big(F^{\nu_h}(o),o\big)-k_X\big(F^{\mu_j}(o),o\big)\Big)=+\infty.
    \end{align*}

    Fissiamo ora un $l\in\mathbb{N}$. Se applichiamo il Teorema \ref{lim_is_const} ad una sottosuccessione arbitraria di $\{F^{\mu_j-l}\}_{j\in\mathbb{N}}$ che converge uniformemente sui compatti di $X$, dal momento che sul compatto $\{F^l(o)\}$ converge a $\eta$ otteniamo (usando anche il teorema di Montel) che tutta la sottosuccessione converge, uniformemente sui compatti di $X$, a $\eta$.

    Poniamo $M_l=\displaystyle\limsup_{j\longrightarrow+\infty}k_X\big(F^{\mu_j-l}(o),o\big)$. Affermiamo che $\displaystyle\limsup_{l\longrightarrow+\infty}M_l<+\infty$. Supponiamo per assurdo che non sia così, allora esiste una sottosuccessione $\{l_m\}_{m\in\mathbb{N}}$ tale che $M_{l_m}>m$ per ogni $m$. Per definizione di $M_l$ e per quanto appena trovato sulla successione $\{F^{\mu_j-l}\}_{j\in\mathbb{N}}$, abbiamo quindi che esiste una sottosottosuccessione $\{j_m\}_{m\in\mathbb{N}}$ tale che
    \begin{enumerate}[label={(\arabic*)}]
        \item $\|F^{\mu_{j_m}-l_m}(o)-\eta\|<1/k$;
        \item $k_X\big(F^{\mu_{j_m}-l_m}(o),o\big)>k$.
    \end{enumerate}
    Per la Proposizione \ref{inf_impl_ugu} deve dunque essere $\eta=\xi$, contraddizione. Perciò segue che $\displaystyle\limsup_{l\longrightarrow+\infty}M_l<+\infty$. Allora
    \begin{align*}
        \limsup_{h\longrightarrow+\infty}\limsup_{j\longrightarrow+\infty}\, &k_X\big(F^{\nu_h}(o),F^{\mu_j}(o)\big) \\
        &\le \limsup_{h\longrightarrow+\infty}\limsup_{j\longrightarrow+\infty} k_X\big(o,F^{\mu_j-\nu_h}(o)\big)=\limsup_{h\longrightarrow+\infty}M_{\nu_h}<+\infty,
    \end{align*}
    in contraddizione con quanto trovato prima; questo conclude il Caso 1.\\

    Caso 2: esiste (e quindi per ogni) $o \in X$ tale che
    $$\limsup_{\nu\longrightarrow+\infty} k_X\big(F^\nu(o),o\big)<+\infty.$$
    Ricordiamo che abbiamo assunto che esistano due punti distinti $\xi,\eta\in\Gamma$. Poiché $X$ soddisfa la condizione di visibilità rispetto alle $(1,\kappa_0)$-simil-geodetiche, esistono $V_\xi, V_\eta$ intorni in $\mathbb{C}^d$ rispettivamente di $\xi$ e di $\eta$, con $\overline{V}_\xi\cap\overline{V}_\eta=\emptyset$, e un compatto $K$ di $X$ tali che ogni $(1,\kappa_0)$-simil-geodetica in $X$ che collega un punto di $V_\xi$ a un punto di $V_\eta$ interseca $K$.
    Adesso definiamo, per $\delta>0$ arbitrario, la funzione $G_\delta:K\times K\longrightarrow [0,+\infty)$ data da
    $$G_\delta(x_1,x_2)=\inf\{k_X\big(F^m(x_1),x_2\big)\mid m\in\mathbb{N}, \|F^m(x_1)-\xi\|<\delta\}.$$
\end{proof}

%\newpage

%\section{Esempi (di domini tesi con (???)-visibilità ???) (forse una sezione per esempio?)}
%\subsection{Domini Goldilock}
%\input{secs/goldilock.tex}
%\subsection{Domini Caltrop}
%\input{secs/caltrop.tex}
%\subsection{Altri esempi (forse due sottosezioni?)}
%L'ultimo esempio è quello mostrato in \cite[Section 5.2]{CMS}; si tratta di un altro esempio di dominio non di tipo Goldilocks che soddisfa la condizione di visibilità.

Iniziamo considerando la funzione $\Phi_0:\mathbb{C}^2\longrightarrow\mathbb{R}$ definita da
$$\Phi_0(z):=\begin{cases}
    \exp(-1/\|z\|^2)-\mathfrak{Im}(z_2) &\mbox{se }z=(z_1,z_2)\not=(0,0)\\
    0 &\mbox{se }z=0.
\end{cases}$$

Poiché la matrice hessiana di $\Phi_0$ (vista come funzione da $\mathbb{R}^4$ in $\mathbb{R}$) è la stessa della funzione $\exp(-1/\|z\|^2)$ estesa a $0$ nell'origine, che è convessa vicino all'origine, esiste $0<\epsilon<1$ tale che $\Phi_0$ è convessa in $\mathbb{B}^2_{2\epsilon}$. Scegliamo inoltre una funzione liscia $\psi:\mathbb{C}^2\longrightarrow[0,1]$ tale che $\psi\equiv 1$ in $\mathbb{B}^2_{2\epsilon}$ e $\text{supp}\,{\psi}\subseteq \mathbb{B}^2_{3\epsilon}$. Poniamo $\Phi:=\Phi_0\cdot\psi$ e $c_0:=\displaystyle\sup_{z\in\mathbb{C}^2}\big(-\Phi(z)\big)>0$.

Scegliamo adesso una funzione liscia $\chi:[0,+\infty)\longrightarrow[0,+\infty)$ che sia identicamente nulla in $[0,\epsilon^2]$, strettamente crescente in $[\epsilon^2,+\infty)$ e strettamente convessa in $\big(\epsilon^2,(\epsilon+\delta)^2\big)$ per $0<\delta<\epsilon$; per esempio, possiamo prendere $\chi(t)=\exp\big(-1/(t-\epsilon^2)\big)$ per $t>\epsilon^2$ e $0$ altrove. Poniamo $c_1:=\chi\big((\epsilon+\delta/2)^2\big)$ e $C:=c_0/c_1$. Definiamo
$$\Psi(z):=C\chi(\|z\|^2)$$
per ogni $z\in\mathbb{C}^2$.

Osserviamo che:
\begin{itemize}
    \item la funzione $\Psi$ è liscia e non negativa su tutto $\mathbb{C}^2$, nulla in $\overline{\mathbb{B}^2_\epsilon}$, e strettamente convessa e strettamente positiva in $\mathbb{B}^2_{\epsilon+\delta}\setminus\overline{\mathbb{B}^2_\epsilon}$;
    \item si ha $\Psi(z)\ge c_0$ per ogni $z\in\mathbb{C}^2\setminus\mathbb{B}^2_{\epsilon+\delta/2}$, da cui $\Psi(z)+\Phi(z)\ge 0$ per ogni $z\in\mathbb{C}^2\setminus\mathbb{B}^2_{\epsilon+\delta/2}$;
    \item si ha $\Psi(z)+\Phi(z)=\Phi(z)=\Phi_0(z)$ per ogni $z\in\mathbb{B}^2_\epsilon$.
\end{itemize}

Consideriamo il dominio
$$\Omega:=\{z=(z_1,z_2)\in\mathbb{C}^2\mid \rho(z):=\Psi(z)+\Phi(z)<0\}.$$

Notiamo che $\Omega\subseteq\mathbb{B}^2_{\epsilon+\delta/2}$, dove $\rho=\Psi+\Phi_0$, che è una funzione convessa; per cui $\Omega$ è un dominio convesso limitato. Calcolando il gradiente di $\rho$, vediamo che esiste al più un punto $p_0\in\partial\Omega$ dove il gradiente si annulla, che è della forma $p_0=(0,ic)$; inoltre, $p_0\in\overline{\mathbb{B}^2_{\epsilon+\delta/2}}\setminus\overline{\mathbb{B}^2_\epsilon}$. Dunque $\Omega$ è un dominio limitato e convesso tale che $\partial\Omega\setminus\{p_0\}$ è liscio. Si ha anche che ogni punto di $(\partial\Omega\setminus\{p_0\})\cap(\mathbb{B}^2_{\epsilon+\delta}\setminus\overline{\mathbb{B}^2_\epsilon})$ è un punto del bordo di $\Omega$ strettamente convesso (perché in $\mathbb{B}^2_{\epsilon+\delta}\setminus\overline{\mathbb{B}^2_\epsilon}$ la funzione $\Psi$ è strettamente convessa e la funzione $\Phi_0$ è convessa); quindi per \cite[Proposition 3.1.9]{Kr} è pseudoconvesso, e per \cite[Corollary 5.6]{D'A} è un punto di tipo finito. Poniamo $A:=\partial\Omega\cap\overline{\mathbb{B}^2_\epsilon}$ e osserviamo che
$$A=\overline{\mathbb{B}^2_\epsilon}\cap\{z\in\mathbb{C}^2\mid \Phi_0(z)=0\};$$
si ha anche che ogni punto di $A$ diverso da $(0,0)$ è un punto del bordo di $\Omega$ di tipo finito (perché $\Phi_0$ è strettamente convessa in $\overline{\mathbb{B}^2_\epsilon}\setminus\{(0,0)\}$, per cui ogni punto di $A$ è strettamente convesso). \\

Possiamo ora procedere a dimostrare che $\Omega$ è $(\lambda,\kappa)$-visibile per ogni $\lambda \ge 1$ e $\kappa>0$.

\begin{prop} \label{safinisvis}
    (\cite[Corollary 1.10]{CMS}) Sia $\Omega$ un dominio limitato di $\mathbb{C}^d$. Supponiamo che esista un compatto $S\subseteq\partial\Omega$ tale che $S_a$, l'insieme dei punti di accumulazione di $S$, sia finito, e inoltre che ogni punto $p\in\partial\Omega\setminus S$ sia un punto liscio di bordo pseudoconvesso e di tipo finito. Allora $\Omega$ è $(\lambda,\kappa)$-visibile per ogni $\lambda \ge 1$ e $\kappa>0$.
\end{prop}

\begin{proof}
    Mostriamo che, dati $p,q\in\partial\Omega$ con $p\not=q$, sono soddisfatte le ipotesi (i) e (ii) del Teorema \ref{extvis}. Per farlo, consideriamo $S_0:=S_a\cup\{p,q\}$. Allora, per finitezza di $S_0$, esiste $\epsilon_0>0$ tale che $\overline{B(x,\epsilon_0)}\cap\overline{B(x',\epsilon_0)}=\emptyset$ per ogni $x,x'\in S_0$. Adesso poniamo
    $$S_1:=(S\cup\{p,q\})\setminus\left(\bigcup_{x\in S_a}\overline{B(x,\epsilon_0)}\right);$$
    notiamo che $S_1$ è un insieme finito disgiunto dal compatto $K:=\displaystyle\bigcup_{x\in S_a}\overline{B(x,\epsilon_0)}$. Dunque esiste $\epsilon_1>0$ tale che:
    \begin{itemize}
        \item si ha $\overline{B(y,\epsilon_1)}\cap K=\emptyset$ per ogni $y\in S_1$;
        \item $\overline{B(y,\epsilon_1)}\cap\overline{B(y',\epsilon_1)}=\emptyset$ per ogni $y,y'\in S_1$ con $y\not=y'$.
    \end{itemize}
    
    Distinguiamo ora due casi. \\

    Caso 1: $p\not\in K$.

    Basta prendere $p'=p$ e $r=\epsilon_1$. \\

    Caso 2: $p\in K$.

    In questo caso esiste un $x_0\in S_a$ tale che $p\in\overline{B(x_0,\epsilon_0)}$. Consideriamo la seguente famiglia di insiemi con chiusure mutualmente disgiunte:
    $$\mathcal{B}:=\{B(x,\epsilon_0)\mid x\in S_a\}\cup\{B(y,\epsilon_1)\mid y\in S_1\};$$
    allora esiste $\epsilon_2>0$ tale che $\epsilon_2<\text{dist}(B_1,B_2)/4$ per ogni $B_1,B_2\in\mathcal{B}$. Segue che $\mathcal{C}:=\{B(x,\epsilon_0+\epsilon_2)\mid x\in S_a\}\cup\{B(y,\epsilon_1+\epsilon_2)\mid y\in S_1\}$ è una famiglia di insiemi con chiusure mutualmente disgiunte. Allora basta prendere $p'=x_0$ e $r=\epsilon_1+\epsilon_2$. \\

    Per concludere mostriamo adesso che, per ogni $\xi\in\partial\Omega\setminus S$, esistono un intorno $U$ e una funzione $f$ che soddisfano le ipotesi (1), (2) e (3) del Teorema \ref{extvis}. Fissiamo un tale $\xi$; allora sono soddisfatte le ipotesi del Teorema \ref{cho} e di \cite[Proposition 2.5]{FR}, per cui esistono un intorno $U$ di $\xi$, due costanti $c,\epsilon>0$, un punto $z_0\in\Omega$ e una costante $A$ tali che, ponendo $f(x):=A+\dfrac{1}{2}\log{x}$ per ogni $x\in(0,+\infty)$, si ha
    \begin{gather*}
        k_\Omega(z,z_0) \le f\big(1/\delta_\Omega(z)\big)\\
        \text{e}\\
        K_\Omega(z;v) \ge c\frac{\|v\|}{\delta_\Omega(z)^{\epsilon}}
    \end{gather*}
    per ogni $z\in\Omega\cap U$ e $v\in\mathbb{C}^d$. Ne consegue facilmente che le ipotesi (1), (2) e (3) del Teorema \ref{extvis} sono soddisfatte, come voluto.
\end{proof}

Basta allora prendere $S=\{p_0,(0,0)\}$ per ottenere che $\Omega$ soddisfa le ipotesi della Proposizione \ref{safinisvis}, dunque è $(\lambda,\kappa)$-visibile per ogni $\lambda\ge 1$ e $\kappa>0$. \\

Mostriamo adesso che $\Omega$ non soddisfa la condizione (1) nella Definizione \ref{gold}. Iniziamo notando che
$$\Omega\cap\mathbb{B}^2_{\epsilon/2}=\{(z_1,z_2)\in\mathbb{B}^2_{\epsilon/2}\mid \mathfrak{Im}(z_2)>\exp(-1/\|z\|^2)\};$$
dunque, per $r>0$ sufficientemente piccolo, si ha che $p_r:=(0,ir)\in\Omega$. Poniamo $v:=(1,0)$ e $s:=\sqrt{\dfrac{1}{\log(1/r)}-r^2}$; allora la funzione $\varphi:\mathbb{D}\longrightarrow\Omega$ data da $\varphi(\zeta)=p_r+\zeta sv$ è ben definita (cioè l'immagine è effettivamente contenuta in $\Omega$) e olomorfa, per cui
$$K_\Omega(p_r;v) \le \frac{1}{s}.$$

Adesso, poiché $(0,0)\in\partial\Omega$, si ha $\delta_\Omega(p_r) \le r$, per cui
$$M_\Omega(r) \ge \frac{1}{K_\Omega(p_r;v)} \ge s=\sqrt{\frac{1}{\log(1/r)}-r^2};$$
per cui ci basta mostrare che, per $r_0>0$ sufficientemente piccolo affinché l'integranda sia definita, si ha
$$\int_0^{r_0}\frac{1}{r}\sqrt{\frac{1}{\log(1/r)}-r^2}\diff r=+\infty.$$

Ciò segue facilmente confrontando con la funzione $r \longmapsto \dfrac{1}{r}\cdot\dfrac{1}{\sqrt{\log(1/r)}}$.

%\newpage

\begin{thebibliography}{widest entry}
  \bibitem[A1]{A1} M. Abate: \textbf{Iteration theory of holomorphic maps on taut manifolds}. Mediterranean Press, Cosenza, 1989 [\url{http://www.dm.unipi.it/˜abate/libri/libriric/libriric.html}]
  \bibitem[A2]{A2} M. Abate: Iteration theory, compactly divergent sequences and commuting holomorphic maps. \textit{Annali della Scuola Normale Superiore di Pisa. Classe di Scienze. Serie V}, \textbf{18} (1991), no. 2, 167--191
  \bibitem[A3]{A3} M. Abate: A characterization of hyperbolic manifolds. \textit{Proceedings of the American Mathematical Society}, \textbf{117} (1993), no. 3, 789--793
  \bibitem[A4]{A4} M. Abate: Dynamics in several complex variables. In Metrical and dynamical aspects in complex analysis, Ed. L. Blanc-Centi, Lecture Notes in Mathematics 2195, Springer, Berlin, 2017, pp. 25--54
  \bibitem[B]{B} T. J. Barth: The Kobayashi distance induces the standard topology. \textit{Proceedings of the American Mathematical Society}, \textbf{35} (1972), 439--441
  \bibitem[BB]{BB} Z. M. Balogh, M. Bonk: Gromov hyperbolicity and the Kobayashi metric on strictly pseudoconvex domains. \textit{Commentarii Mathematici Helvetici}, \textbf{75} (2000), no. 3, 504--533
  \bibitem[BH]{BH} M. R. Bridson, A. Haefliger: \textbf{Metric-Spaces of Non-Positive Curvature}. Springer, New York, 1999
  \bibitem[BZ1]{BZ1} A. Zimmer, G. Barali: Goldilocks domains, a weak notion of visibility, and applications. \textit{Advances in Mathematics}, \textbf{310} (2017), 377--425
  \bibitem[CMS]{CMS} V. S. Chandel, A. Maitra, A. D. Sarkar: Notions of Visibility with respect to the Kobayashi distance: Comparison and Applications. Preprint, arXiv:2111.00549v1 (2021)
  \bibitem[N]{N} R. Narasimhan: \textbf{Several Complex Variables}. University of Chicago Press, Chicago, 1971
  \bibitem[NTT]{NTT} N. Nikolov, P. J. Thomas, M. Trybuła: Gromov (non)hyperbolicity of certain domains in $\mathbb{C}^2$. Preprint, arXiv:1403.7673v2 (2015)
  \bibitem[Ka]{Ka} A. Karlsson: Non-expanding maps and Busemann functions. \textit{Ergodic Theory and Dynamical Systems}, \textbf{21} (2001), no. 5, 1447--1457
  \bibitem[Kr]{Kr} S. G. Krantz: \textbf{Function Theeory of Several Complex Variables: Second Edition}. AMS Chelsea Publishing, Providence, 2001
  \bibitem[K1]{K1} S. Kobayashi: Invariant distances on complex manifolds and holomorphic mappings. \textit{Journal of the Mathematical Society of Japan}, \textbf{19} (1967), 460--480
  \bibitem[K2]{K2} S. Kobayashi: \textbf{Hyperbolic Manifolds and Holomorphic Mappings: An Introduction (Second Edition)}. World Scientific Publishing Co., Singapore, 2005
  \bibitem[R]{R} H. L. Royden: Remarks on the Kobayashi metric. In Several Complex Variables II, Proceedings of the International Mathematical Conference, Lecture Notes in Mathematics \textbf{185}, Springer, Berlin, 1971, pp. 125--137
  \bibitem[W]{W} H. Wu: Normal families of holomorphic mappings. \textit{Acta Mathematica}. \textbf{119} (1967), 193--233.
\end{thebibliography}

\addcontentsline{toc}{section}{Riferimenti bibliografici}

\newpage

\section*{Ringraziamenti}
\addcontentsline{toc}{section}{Ringraziamenti}
Volendo, si possono aggiungere dei ringraziamenti.


\end{document}
