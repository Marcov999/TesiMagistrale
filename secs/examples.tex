Dopo aver dimostrato il Teorema \ref{wd_type}, viene naturale chiedersi: esistono sottovarietà limitate di $\mathbb{C}^d$ che possiamo dimostrare essere taut e visibili per le simil-geodetiche, anche senza ipotesi di regolarità (l'esempio che già conosciamo, i domini strettamente pseudoconvessi, hanno regolarità $C^2$)?

In questa sezione andremo a vedere tre esempi di domini\marginpar{Capire come dire che non sono Gromov, aggiungere all'intro della sezione} limitati in $\mathbb{C}^n$ che soddisfano la condizione di visibilità: le prime sono due classi di domini introdotte in \cite{BZ1} e \cite{BM}, e ne vedremo esempi espliciti; l'ultimo esempio è stato presentato in \cite{CMS}. Inoltre, i domini della classe introdotta in \cite{BM} sono anche taut, perciò per essi vale automaticamente il Teorema \ref{wd_type}.