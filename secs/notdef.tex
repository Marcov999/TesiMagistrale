Introduciamo le notazioni e definizioni che useremo:
\begin{itemize}
    \item  un \textit{dominio} in $\mathbb{C}^n$ è un aperto connesso;
    \item una \textit{varietà complessa} di dimensione $n$ è una varietà differenziabile reale, di dimensione $2n$ e tale che i cambi di carta siano olomorfi se considerati fra aperti di $\mathbb{C}^n$;
    \item dati una varietà complessa $X$ e $x \in X$, indichiamo con $T_xX$ lo spazio tangente complesso a $X$ in $x$, che nel caso dei domini è canonicamente identificato con $\mathbb{C}^n$;
    \item dati $X,Y$ spazi topologici, $C^0(X,Y)$ è lo spazio delle funzioni continue da $X$ a $Y$ considerato con la topologia compatta-aperta. Nel caso in cui $Y$ sia uno spazio metrico, tale topologia coincide con la topologia della convergenza uniforme sui compatti;
    \item dati un insieme $X$ e una funzione $f:X\longrightarrow X$, poniamo induttivamente $f^0=\id_X$ e $f^{k+1}=f\circ f^k$. Chiamiamo $f^k$ l'\textit{iterata} $k$-esima di $f$ e, per ogni $x \in X$, l'\textit{orbita} di $x$ tramite $f$ è l'insieme $\{f^k(x)\mid k \in \mathbb{N}\}$;
    \item date $X,Y$ varietà complesse, indichiamo con $\text{Hol}(X,Y)$ l'insieme delle funzioni olomorfe da $X$ a $Y$, con $\mathcal{O}(X)$ l'insieme delle funzioni olomorfe da $X$ in $\mathbb{C}$ e con $\text{Aut}(X)$ l'insieme dei biolomorfismi di $X$ in sé;
    \item data $f \in \text{Hol}(X,Y)$, indichiamo con $\diff f_x$ il differenziale di $f$ in $x \in X$;
    \item il disco unitario è $\mathbb{D}=\{z \in \mathbb{C} \mid |z|<1\}$; il disco di centro $0$ e raggio $r>0$ è $\mathbb{D}_r=\{z \in \mathbb{C} \mid |z|<r\}$; il disco di centro $a\in\mathbb{C}$ e raggio $r>0$ è $D(a,r)=\{z\in\mathbb{C}\mid |z-a|<r\}$; il polidisco in $\mathbb{C}^n$ è $\mathbb{D}^n=\mathbb{D}\times\dots\times\mathbb{D}$;
    \item la palla unitaria (euclidea) in $\mathbb{C}^n$ è $\mathbb{B}^n=\{z \in \mathbb{C}^n \mid \|z\|<1\}$, dove $\|\cdot\|$ indica la norma euclidea, mentre $\mathbb{B}_r^n=\{z \in \mathbb{C}^n \mid \|z\|<r\}$ è la palla (euclidea) di centro l'origine $O\in\mathbb{C}^n$ e raggio $r>0$;
    \item dato un sottoinsieme $\Omega\subseteq \mathbb{C}^n$, indichiamo il \textit{bordo euclideo} con $\partial\Omega:=\overline{\Omega}\setminus\Omega$;
    \item dato un dominio $\Omega\subseteq\mathbb{C}^n$ e $z \in \mathbb{C}^n$, scriviamo $\delta(z)=\displaystyle\inf_{p \in \partial\Omega}\|z-p\|$ per indicare la distanza euclidea di $x$ dal bordo euclideo di $\Omega$.
\end{itemize}

Ricordiamo cosa sono la metrica e la distanza di Poincaré in $\mathbb{D}$.

\begin{defn}
    La \textit{metrica di Poincaré} (o \textit{iperbolica}) su $\mathbb{D}$ è data da
    \begin{equation}
        \lambda_{\mathbb{D}}(z;v)=\frac{1}{1-|z|^2}|v|
    \end{equation}
    per ogni $z \in \mathbb{D}$ e $v \in \mathbb{C}\cong T_z\mathbb{D}$. La metrica $\lambda_{\mathbb{D}}$ è hermitiana di curvatura gaussiana costante uguale a $-4$ (si veda \cite[Definition 1.10.1 and Example 1.10.2]{A5}).
\end{defn}

\begin{defn}
    La \textit{distanza di Poincaré} (o \textit{iperbolica}) $\omega$ su $\mathbb{D}$ è la forma integrata della metrica di Poincaré. È una distanza completa la cui espressione è data da
    \begin{equation}
        \omega(z_1,z_2)=\frac{1}{2}\log{\frac{1+\left|\frac{z_1-z_2}{1-\bar{z}_1z_2}\right|}{1-\left|\frac{z_1-z_2}{1-\bar{z}_1z_2}\right|}}
    \end{equation}
    per ogni $z_1,z_2 \in \mathbb{D}$ (si veda \cite[point (iii) of Proposition 1.2.7 and Proposition 1.9.14]{A5}).
\end{defn}

Oltre ad avere curvatura negativa costante, la metrica e la distanza di Poincaré sono tali che le funzioni olomorfe dal disco unitario in sé sono semicontrazioni rispetto ad esse.

\begin{lm}
    (lemma di Schwarz-Pick) Sia $f \in \text{\normalfont{Hol}}(\mathbb{D},\mathbb{D})$.
    Allora per ogni $z, w \in \mathbb{D}$ si ha
    $$\left|\frac{f(z)-f(w)}{1-\overline{f(w)}f(z)}\right| \le \left|\frac{z-w}{1-\bar{w}z}\right| \text{ e } \frac{|f'(z)|}{1-|f(z)|^2} \le \frac{1}{1-|z|^2};$$
    inoltre, se vale l'uguaglianza nella prima per $z_0,w_0$ con $z_0\not=w_0$ o nella seconda per $z_0$ allora $f\in\textnormal{Aut}(\mathbb{D})$ e vale sempre l'uguaglianza.
\end{lm}

Per la dimostrazione si rimanda a \cite[Chapter I, Theorem 1.1]{Ko2}.

\begin{cor}
    Sia $f\in\textnormal{Hol}(\mathbb{D},\mathbb{D})$. Allora si ha
    \begin{equation}
        \omega\big(f(z),f(w)\big) \le \omega(z,w)
    \end{equation}
    per ogni $z,w\in\mathbb{D}$.
\end{cor}
  
\begin{proof}
    Discende dal lemma di Schwarz-Pick e dal fatto che la funzione $\text{arctanh}$ è strettamente crescente.
\end{proof}

Ci tornerà utile anche la versione semplificata del lemma di Schwarz-Pick.

\begin{lm}
    (lemma di Schwarz) Sia $f\in\textnormal{Hol}(\mathbb{D},\mathbb{D})$ tale che $f(0)=0$. Allora per ogni $z\in\mathbb{D}$ si ha $|f(z)| \le |z|$ e $|f'(0)| \le 1$; inoltre, se vale l'uguaglianza nella prima per $z_0\not=0$ oppure nella seconda allora $f(z)=e^{i\theta}z$ per qualche $\theta\in\mathbb{R}$.
\end{lm}

Enunciamo adesso dei fatti noti sulle geodetiche del disco unitario con la metrica di Poincaré. Visto che ne parleremo più in generale, diamo la definizione per spazi metrici.

\begin{defn}
    Siano $(X,d)$ uno spazio metrico e $I\subseteq\mathbb{R}$ un intervallo. Una curva $\sigma:I\longrightarrow X$ è detta \textit{geodetica} se
    $$d\big(\sigma(t_1),\sigma(t_2)\big)=|t_1-t_2|$$
    per ogni $t_1,t_2\in I$.
\end{defn}

\begin{oss}\marginpar{Sono in dubbio se definire una lunghezza in generale per spazi metrici: potrebbe confondersi con la def. \ref{lung_X}}
    Dati una geodetica $\sigma:I\longrightarrow X$ e $a,b\in I$, si ha
    $$d\big(\sigma(a),\sigma(b)\big)=\inf_{a=t_0,t_1,\dots,t_n=b} \sum_{j=1}^n d\big(\sigma(t_{j-1}),\sigma(t_j)\big);$$
    cioè, le geodetiche sono le curve che ``minimizzano la lunghezza''.
\end{oss}

\begin{prop}
    (\cite[point (iv) of Proposition 1.2.7]{A5}) Le geodetiche di $(\mathbb{D},\omega)$ sono i diametri di $\mathbb{D}$ e le intersezioni con $\mathbb{D}$ dei cerchi euclidei ortogonali a $\partial\mathbb{D}$. In particolare, ogni coppia di punti distinti è connessa da un'unica geodetica.
\end{prop}

Quello che vogliamo fare ora è generalizzare la metrica e la distanza di Poincaré a una qualsiasi varietà complessa mantenendo la proprietà di rendere le funzioni olomorfe delle semicontrazioni. Ci sono vari modi per farlo, noi nello specifico vedremo la (pseudo)metrica e la (pseudo)distanza di Kobayashi, introdotte nel 1967 in \cite{Ko1}.

\begin{defn}
    Sia $X$ una varietà complessa; la \textit{pseudometrica di Kobayashi} su $X$ è
    \begin{equation}\begin{split}
        K_X(x;Z)=&\inf\{|v| \mid v \in \mathbb{C}, \text{ esiste }f \in \text{Hol}(\mathbb{D},X) \\
        &\text{ tale che } f(0)=x, \diff f_0(v)=Z\}
    \end{split}\end{equation}
    per ogni $x \in X$ e $Z \in T_xX$.
\end{defn}

\begin{oss} \label{metr_noncr}
    Non possiamo sempre parlare di metrica perché, per esempio, $K_{\mathbb{C}^n}\equiv 0$. Infatti, dati $z\in\mathbb{C}^n$ e $Z\in T_z\mathbb{C}^n=\mathbb{C}^n$, abbiamo che la funzione $f_\epsilon\in\text{Hol}(\mathbb{D},\mathbb{C}^n)$ data da $f_\epsilon(\zeta)=z+\zeta Z/\epsilon$ è tale che $\diff (f_\epsilon)_0(\epsilon)=Z$ per ogni $\epsilon>0$; di conseguenza, $K_{\mathbb{C}^n}(z;Z)=0$.
\end{oss}

Vediamo adesso che le funzioni olomorfe sono semicontrazioni rispetto alla pseudometrica di Kobayashi.

\begin{prop} \label{metrdecr}
    Siano $X$ e $Y$ varietà complesse, e sia $f \in \textnormal{Hol}(X,Y)$. Allora si ha
    \begin{equation}
        K_Y\big(f(x);\diff f_x(Z)\big) \le K_X(x;Z)
    \end{equation}
    per ogni $x \in X$ e $Z \in T_xX$.
\end{prop}

\begin{proof}
    Dati $x\in X$ e $Z\in T_xX$, poniamo
    \begin{align*}
        A_X(x;Z)=&\{v \in \mathbb{C} \mid \text{esiste }g \in \text{Hol}(\mathbb{D},X) \\
        &\text{ tale che } g(0)=x, \diff g_0(v)=Z\},
    \end{align*}
    e sia $A_Y\big(f(x);\diff f_x(Z)\big)$ definito analogamente. Per definizione,
    \begin{gather*}
        K_X(x;Z)=\inf\{|v|\mid v\in A_X(x;Z)\}\\
        \text{ e }\\
        K_Y\big(f(x);\diff f_x(Z)\big)=\inf\{|v|\mid v\in A_Y\big(f(x);\diff f_x(Z)\big)\};
    \end{gather*}
    quindi ci basta mostrare che $A_X(x;Z)\subseteq A_Y\big(f(x);\diff f_x(Z)\big)$. Dato $v\in A_X(x;Z)$, prendiamo $g\in\text{Hol}(\mathbb{D},X)$ tale che $g(0)=x$ e $\diff g_0(v)=Z$. Allora abbiamo $(f\circ g)(0)=f(x)$ e $\diff(f\circ g)_0(v)=(\diff f_x\circ\diff g_0)(v)=\diff f_x(Z)$; dunque $v \in A_Y\big(f(x);\diff f_x(Z)\big)$. Perciò $A_X(x;Z)\subseteq A_Y\big(f(x);\diff f_x(Z)\big)$, come voluto.
\end{proof}

Nei casi a cui siamo interessati, la pseudometrica di Kobayashi è effettivamente una metrica.

\begin{prop}
    Sia $X$ una sottovarietà complessa e limitata di $\mathbb{C}^d$. Allora $K_X$ è una metrica, cioè $K_X(z;Z)>0$ per ogni $z\in X$ e $0\not=Z\in T_zX$.
\end{prop}

\begin{proof}
    È un'immediata conseguenza di un risultato più forte che dimostreremo nella prossima sezione, il punto (1) della Proposizione \ref{metrica_bilip}.
\end{proof}

Definiamo adesso la (pseudo)distanza di Kobayashi; più avanti vedremo com'è collegata alla pseudometrica di Kobayashi.

\begin{defn}
    Sia $X$ una varietà complessa e connessa; la \textit{pseudodistanza di Kobayashi} su $X$ è data da
    \begin{equation}\begin{split}
        k_X(z,w)=&\inf\Bigg\{\sum_{j=1}^m \omega(\zeta_{j-1},\zeta_j) \bigg\vert \text{esistono }m\in\mathbb{N},\text{ punti }\zeta_0,\dots,\zeta_m \in \mathbb{D}\text{ e}\\
        &\text{funzioni }\varphi_1,\dots,\varphi_m\in\text{Hol}(\mathbb{D},X) \text{ tali che } \varphi_1(\zeta_0)=z,\varphi_m(\zeta_m)=w\\
        &\text{e }\varphi_j(\zeta_j)=\varphi_{j+1}(\zeta_j)\text{ per }j=1,\dots,m-1\Bigg\}
    \end{split}\end{equation}
    per $z,w \in X$.
\end{defn}

\begin{oss} \label{k_lip}
    È facile vedere che $k_X$ è una pseudodistanza, ma in generale non è una distanza, ad esempio perché, come prima, $k_{\mathbb{C}^n}\equiv 0$. Infatti, dati $z,w\in X$, possiamo considerare i punti $\zeta_0=0$ e $1>\zeta_1=\epsilon>0$ e la funzione $\varphi_1\in\text{Hol}(\mathbb{D},\mathbb{C}^n)$ tale che $\varphi_1(\zeta)=z+\zeta(w-z)/\epsilon$. Si ha $\varphi_1(\zeta_0)=z$ e $\varphi_1(\zeta_1)=w$; perciò, per definizione, $k_{\mathbb{C}^n}(z,w) \le \omega(0,\epsilon)$ per ogni $1>\epsilon>0$, da cui $k_{\mathbb{C}^n}(z,w)=0$.
    
    Vedremo però più avanti (Osservazione \ref{kobisdist}) che se $X$ è una sottovarietà complessa, connessa e limitata di $\mathbb{C}^d$ allora $k_X$ è effettivamente una distanza.
\end{oss}

Vediamo adesso che le funzioni olomorfe sono delle semicontrazioni rispetto alla pseudodistanza di Kobayashi.

\begin{prop} \label{semicontr}
    Siano $X$ e $Y$ varietà complesse e connesse, e consideriamo $f \in \textnormal{Hol}(X,Y)$. Allora
    \begin{equation}
        k_Y\big(f(x),f(y)\big) \le k_X(x,y)
    \end{equation}
    per ogni $x,y \in X$.
\end{prop}

\begin{proof}
    Dati $x,y\in X$, poniamo
    \begin{align*}
        A_X(x,y)=&\Bigg\{\sum_{j=1}^m \omega(\zeta_{j-1},\zeta_j) \bigg\vert \text{esistono }m\in\mathbb{N},\text{ punti }\zeta_0,\dots,\zeta_m \in \mathbb{D}\text{ e}\\
        &\text{funzioni }\varphi_1,\dots,\varphi_m\in\text{Hol}(\mathbb{D},X) \text{ tali che } \varphi_1(\zeta_0)=x,\varphi_m(\zeta_m)=y\\
        &\text{e }\varphi_j(\zeta_j)=\varphi_{j+1}(\zeta_j)\text{ per }j=1,\dots,m-1\Bigg\},
    \end{align*}
    e sia $A_Y\big(f(x),f(y)\big)$ definito analogamente. Per definizione,
    $$k_X(x,y)=\inf A_X(x,y)\quad \text{e}\quad k_Y\big(f(x),f(y)\big)=\inf A_Y\big(f(x),f(y)\big);$$
    quindi ci basta mostrare che $A_X(x,y)\subseteq A_Y\big(f(x),f(y)\big)$. Dati $\zeta_0,\dots,\zeta_m \in \mathbb{D}$ e $\varphi_1,\dots,\varphi_m\in\text{Hol}(\mathbb{D},X)$ che realizzano $\displaystyle\sum_{j=1}^m \omega(\zeta_{j-1},\zeta_j)$ in $A_X(x,y)$, si verifica immediatamente che $\zeta_0,\dots,\zeta_m \in \mathbb{D}$ e $f\circ\varphi_1,\dots,f\circ\varphi_m\in\text{Hol}(\mathbb{D},Y)$ realizzano lo stesso numero in $A_Y\big(f(x),f(y)\big)$. Perciò $A_X(x,y)\subseteq A_Y\big(f(x),f(y)\big)$, come voluto.
\end{proof}

Segue immediatamente l'invarianza per biolomorfismi.

\begin{cor} \label{bioloiso}
    Siano $X$ e $Y$ varietà complesse e connesse, e consideriamo un biolomorfismo $f:X \longrightarrow Y$. Allora
    \begin{equation}
        k_Y\big(f(x),f(y)\big)=k_X(x,y)
    \end{equation}
    per ogni $x,y \in X$.
\end{cor}

\begin{defn}
    Una varietà complessa e connessa $X$ è \textit{Kobayashi-iperbolica} se $k_X$ è una distanza.
\end{defn}

\begin{oss} \label{lengthspace}
    Dalla definizione di $k_X$ segue che ogni varietà Kobayashi-iperbolica è uno \textit{spazio di lunghezze} nel senso di \cite[Part I, Definition 3.1]{BH}.
\end{oss}

Il seguente risultato per le varietà Kobayashi-iperboliche verrà spesso usato implicitamente.

\begin{prop}
    (Barth, \cite{B}) Sia $X$ una varietà complessa e connessa. Allora $X$ è Kobayashi-iperbolica se e solo se $k_X$ vi induce la topologia di varietà.
\end{prop}

Diamo ora delle definizioni che ci serviranno per enunciare i risultati già noti nel caso dei domini regolari.

\begin{defn}
    Una funzione continua $\mu:\mathbb{C}^n \longrightarrow [0,+\infty)$ è detta \textit{funzionale di Minkowski} se
    \begin{nlist}
        \item $\mu(Z)=0$ se e solo se $Z=0$;
        \item $\mu(\zeta Z)=|\zeta|\mu(Z)$ per ogni $Z \in \mathbb{C}^n$ e $\zeta\in\mathbb{C}$.
    \end{nlist}
    Dato $\Omega\subseteq\mathbb{C}^n$ un dominio, poniamo $\mu_{\Omega}(Z)=\displaystyle\inf_{w\in\mathbb{C}^n\setminus\Omega}\mu(Z-w)$.
\end{defn}

\begin{defn}
    Sia $A\subseteq\mathbb{C}$ un aperto. Una funzione $u:A \longrightarrow \mathbb{R}\cup\{-\infty\}$ semicontinua superiormente è detta \textit{subarmonica} se per ogni $a \in A$, per ogni $r>0$ tale che $\overline{D(a,r)} \subset A$ e per ogni $h$ continua in $\overline{D(a,r)}$ e armonica in $D(a,r)$, se $h\restrict{\partial D(a,r)} \ge u\restrict{\partial D(a,r)}$ allora anche $h\restrict{D(a,r)} \ge u \restrict{D(a,r)}$.

    Sia $A\subseteq\mathbb{C}^n$ un aperto. Una funzione $u:A \longrightarrow \mathbb{R}\cup\{-\infty\}$ semicontinua superiormente è detta \textit{plurisubarmonica} se per ogni $a \in A$ e per ogni $Z \in \mathbb{C}^n$ l'applicazione $\zeta \longmapsto u(a+\zeta Z)$ è subarmonica dove definita.
\end{defn}

\begin{defn}
    Un dominio $\Omega \subseteq \mathbb{C}^n$ si dice \textit{(Hartogs) pseudoconvesso} se esiste un funzionale di Minkowski $\mu$ tale che $-\log\mu_{\Omega}$ è plurisubarmonica in $\Omega$.
\end{defn}

 Il seguente risultato ci servirà per un controesempio.

 \begin{thm}
    (\cite[Theorem 5.1.2]{Kr}) Ogni dominio pseudoconvesso è un dominio di olomorfia.
 \end{thm}

Nel caso di domini regolari, si può dare una definizione di pseudoconvessità più operativa equivalente.

\begin{defn}
    Sia $\Omega \subseteq \mathbb{C}^n$ un dominio con bordo $C^2$, cioè tale che esista una funzione $\rho \in C^2(\mathbb{C}^n)$ tale che $\Omega=\{\rho(z)<0\}$ e $\diff\rho\not=0$ in ogni punto di $\partial\Omega$.
    
    Dato $p \in \partial\Omega$, lo \textit{spazio tangente complesso} a $\partial\Omega$ in $p$ è
    $$H_p\partial\Omega=\{Z \in \mathbb{C}^n \mid \langle \bar{\partial}\rho(p),Z\rangle=0\},$$
    dove $\bar{\partial}\rho(p)=\left(\dfrac{\partial\rho}{\partial\bar{z}_1}(p),\dots,\dfrac{\partial\rho}{\partial\bar{z}_n}(p)\right)$.

    Diciamo che $\Omega$ è \textit{Levi pseudoconvesso} se la \textit{forma di Levi}
    $$L_{\rho}(p;Z)=\sum_{\nu,\mu=1}^n \frac{\partial^2\rho}{\partial z_\nu\partial\bar{z}_\mu}(p)Z_\nu\bar{Z}_\mu, \quad Z=(Z_1,\dots,Z_n) \in \mathbb{C}^n$$
      è semidefinita positiva in $H_p\partial\Omega$ per ogni $p \in \partial\Omega$. Diciamo che è \textit{strettamente pseudoconvesso} se la forma di Levi è definita positiva in $H_p\partial\Omega$ per ogni $p \in \partial\Omega$.
\end{defn}

Vale il seguente risultato.
\begin{thm}\marginpar{Nel risultato citato non c'è l'ipotesi di limitatezza; teniamolo a mente, potrebbe tornare utile}
    (\cite[Theorem 3.3.5]{Kr}) Sia $\Omega \subseteq \mathbb{C}^n$ un dominio limitato con bordo $C^2$. Allora $\Omega$ è Levi pseudoconvesso se e solo se è Hartogs pseudoconvesso.
\end{thm}

Nella sottosezione \ref{risnoti} citeremo alcuni risultati sulla geometria dei domini limitati strettamente pseudoconvessi dotati della distanza di Kobayashi. In particolare, vedremo che sono Gromov-iperbolici, il che permette di derivare un teorema di tipo ``Wolff-Denjoy'' per questi domini.

\begin{defn}
    Sia $(X,d)$ uno spazio metrico. Dati $x,y,w \in X$ il \textit{prodotto di Gromov} tra $x$ e $y$ con punto base $w$ è $(x,y)_w=\frac{1}{2}\big(d(x,w)+d(y,w)-d(x,y)\big)$. Dato $\delta \ge 0$, diciamo che $X$ è \textit{$\delta$-iperbolico} se
    $$(x,y)_w \ge \min\{(x,z)_w,(y,z)_w\}-\delta\text{ per ogni }x,y,z,w \in X.$$

    Se $(X,d)$ è $\delta$-iperbolico per qualche $\delta \ge 0$, diremo che è \textit{Gromov-iperbolico}.
\end{defn}

\begin{defn}
    Sia\marginpar{Per definirlo non c'è molto altro da dire, l'Osservazione dopo serviva ad aggiungere i fatti interessanti} $(X,d)$ uno spazio metrico Gromov-iperbolico. Fissato $w \in X$, il \textit{bordo iperbolico} $\partial_GX$ è dato dalle classi di equivalenza delle successioni $(x_i)$ che convergono a infinito, cioè tali che $\displaystyle\lim_{i,j\rightarrow+\infty}(x_i,x_j)_w=+\infty$; due tali successioni $(x_i), (y_i)$ sono equivalenti se $\displaystyle\lim_{i\rightarrow+\infty}(x_i,y_i)_w=+\infty$. Questa costruzione è indipendente dalla scelta di $w$.

    Inoltre, è possibile estendere il prodotto di Gromov a tutto $X\cup\partial_GX$ in modo che valga ancora la disuguaglianza per la Gromov-iperbolicità, eventualmente cambiando $\delta$. Il bordo iperbolico possiede anche una \textit{classe canonica di distanze} correlate al prodotto di Gromov, che inducono tutte la stessa topologia.
\end{defn}

\begin{oss}
    È possibile\marginpar{In \cite{BH} usa l'ipotesi proprio e geodetico per dire che la topologia è la stessa; non vorrei fosse una cosa diversa da quella a cui mi sto riferendo} mettere una topologia su $X\cup\partial_GX$ che lo rende uno spazio compatto, e tale che ristretta a $\partial_GX$ coincida con la topologia indotta dalla classe canonica di distanze. Si veda \cite[Part III, Chapter H, Paragraph 3]{BH} per i dettagli.
\end{oss}

Prima di passare a vedere i risultati noti della teoria sulla pseudometrica e la pseudodistanza di Kobayashi e sui domini strettamente pseudoconvessi, introduciamo il concetto di varietà taut, che sarà per noi un'ipotesi importante per ciò che andremo a dimostrare: infatti, quest'ipotesi ci darà la dicotomia nella tesi dei teoremi di tipo ``Wolff-Denjoy''. Vedremo anche con un esempio l'importanza di tale ipotesi. Prima di dare la definizione, ci servirà un risultato sul comportamento delle funzioni olomorfe a valori in una varietà Kobayashi-iperbolica; non lo dimostreremo tutto, ma per la parte che andremo a mostrare avremo bisogno del ben noto teorema di Ascoli-Arzelà.

\begin{thm}
    (Ascoli-Arzelà, \cite[Chapter 7, Theorem 17]{Ke}) Siano $X$ uno spazio metrico e $Y$ uno spazio metrico localmente compatto. Allora un famiglia $\mathcal{F}\subseteq C^0(Y,X)$ è relativamente compatta in $C^0(Y,X)$ se e solo se le seguenti due condizioni sono soddisfatte:
    \begin{nlist}
        \item $\mathcal{F}$ è equicontinua;
        \item l'insieme $\{f(y)\mid f\in\mathcal{F}\}$ è relativamente compatto in $X$ per ogni $y\in Y$.
    \end{nlist}
\end{thm}

\begin{prop} \label{alex}
    (\cite[Theorem 1.3]{A3}) Sia $X$ una varietà complessa e connessa. Allora $X$ è Kobayashi-iperbolica se e solo se $\textnormal{Hol}(\mathbb{D},X)$ è relativamente compatto in $C^0(\mathbb{D},X^*)$, dove $X^*$ è la compattificazione di Alexandroff di $X$.
    
    Inoltre, se $X$ è Kobayashi-iperbolica allora $\textnormal{Hol}(Y,X)$ è relativamente compatto in $C^0(Y,X^*)$ per ogni varietà complessa $Y$.
\end{prop}

\begin{proof}
    Dimostriamo solamente che se $X$ è Kobayashi-iperbolica allora $\textnormal{Hol}(Y,X)$ è relativamente compatto in $C^0(Y,X^*)$ per ogni varietà complessa $Y$, dando per buona la prima parte. Poiché $X$ è una varietà,\marginpar{Se devo giustificarlo, direi \cite[4.16]{Ke}} possiamo fissare una distanza $d$ su $X^*$ che induca la topologia della compattificazione. Dato che $\text{Hol}(\mathbb{D},X)$ è relativamente compatto in $C^0(\mathbb{D},X^*)$, per il teorema di Ascoli-Arzelà è equicontinuo.
    
    Sia $Y$ una varietà complessa, mostriamo che $\text{Hol}(Y,X)$ è equicontinuo. Se per assurdo non lo fosse, esisterebbero $\epsilon>0$, un punto $z\in Y$ e due successioni $\{z_\nu\}_{\nu\in\mathbb{N}}\subseteq Y$ e $\{f_\nu\}_{\nu\in\mathbb{N}}\subseteq\text{Hol}(Y,X)$ tali che $d\big(f_\nu(z_\nu),f_\nu(z)\big)\ge\epsilon$ per ogni $\nu\in\mathbb{N}$. Scegliendo delle coordinate opportune, possiamo assumere che $Y$ sia la palla euclidea di un qualche $\mathbb{C}^n$ e che $z=0$. Definiamo $g_\nu\in\text{Hol}(\mathbb{D},X)$ nel seguente modo: $g_\nu(\zeta)=f_\nu(\zeta z_\nu/\|z_\nu\|)$; allora $\|z_\nu\| \longrightarrow 0$ per $\nu\longrightarrow 0$ e $d\big(g_\nu(\|z_\nu\|),g_\nu(0)\big)=d\big(f_\nu(z_\nu),f_\nu(z)\big)\ge\epsilon$ per ogni $\nu\in\mathbb{N}$, contraddizione con il fatto che $\text{Hol}(\mathbb{D},X)$ è equicontinuo.

    Dunque $\text{Hol}(Y,X)$ è equicontinuo; per compattezza di $X^*$ valgono le ipotesi del teorema di Ascoli-Arzelà, per cui abbiamo che è relativamente compatto in $C^0(Y,X^*)$, come voluto.
\end{proof}

\begin{defn}
    Una varietà complessa $X$ si dice \textit{taut} se è Kobayashi-iperbolica e ogni funzione nella chiusura di $\text{Hol}(\mathbb{D},X)$ in $C^0(\mathbb{D},X^*)$ è in $\text{Hol}(\mathbb{D},X)$ oppure è la funzione costante $\infty$.
\end{defn}

Per finire, diamo delle definizioni che ci serviranno per parlare del comportamente delle iterate di funzioni olomorfe.

\begin{defn}
    Siano $X$ e $Y$ due spazi topologici. Diciamo che una successione $\{f_{\nu}\}_{\nu \in \mathbb{N}} \subseteq C^0(X,Y)$ è \textit{compattamente divergente} se, per ogni coppia di compatti $H\subseteq X$ e $K\subseteq Y$, esiste $\nu_0 \in \mathbb{N}$ tale che $f_\nu(H)\cap K=\emptyset$ per ogni $\nu \ge \nu_0$.

    Una famiglia $\mathcal{F} \subseteq C^0(X,Y)$ è detta \textit{normale} se ogni successione in $\mathcal{F}$ ammette una sottosuccessione che converge a una funzione in $C^0(X,Y)$ oppure è compattamente divergente.
\end{defn}

\begin{defn}
    Sia $f\in C^0(X,X)$. Diciamo che $g\in C^0(X,X)$ è una \textit{funzione limite} di $f$ se esiste una sottosuccessione delle iterate di $f$ che converge a $g$ in $C^0(X,X)$. Denotiamo con $\Gamma(f)$ l'insieme di tutte le funzioni limite di $f$.
\end{defn}

\begin{defn}
    Una \textit{retrazione olomorfa} di una varietà complessa $X$ è una funzione $\rho\in\text{Hol}(X,X)$ tale che $\rho^2=\rho$. L'immagine di una retrazione olomorfa è detta \textit{retratto olomorfo}.
\end{defn}