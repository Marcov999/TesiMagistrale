Introduciamo la notazione che useremo:
\begin{itemize}
    \item scriviamo $\Omega$ per indicare un \textit{dominio} di $\mathbb{C}^n$, vale a dire un aperto connesso;
    \item con \textit{varietà complessa} s'intende una varietà differenziabile reale di dimensione pari con i cambi di carta olomorfi se visti come fra aperti di $\mathbb{C}^n$;
    \item data $X$ varietà complessa e $x \in X$, indichiamo con $T_xX$ lo spazio tangente a $X$ in $x$, che nel caso dei domini è canonicamente identificato con $\mathbb{C}^n$;
    \item dati $X,Y$ spazi topologici, quando parliamo di convergenza nell'insieme $C^0(X,Y)$ delle funzioni continue da $X$ a $Y$ sottintendiamo sempre che si parla della topologia compatta-aperta, che nel caso in cui $Y$ sia uno spazio metrico coincide con la topologia della convergenza uniforme sui compatti (potrebbe capitare che commetteremo abusi di notazione in merito);
    \item date $X,Y$ varietà complesse, indichiamo con $\text{Hol}(X,Y)$ l'insieme delle funzioni olomorfe da $X$ a $Y$, con $\mathcal{O}(X)$ l'insieme delle funzioni olomorfe da $X$ in $\mathbb{C}$ e con $\text{Aut}(X)$ l'insieme delle funzioni biolomorfe da $X$ in sé;
    \item data $f \in \text{Hol}(X,Y)$, indichiamo con $Df(x)$ il differenziale di $f$ in $x \in X$;
    \item il disco unitario è $\mathbb{D}=\{z \in \mathbb{C} \mid |z|<1\}$, mentre $\mathbb{D}^n$ è il polidisco in $\mathbb{C}^n$ e $\mathbb{D}_r=\{z \in \mathbb{C} \mid |z|<r\}$ è il disco di raggio $r>0$;
    \item la palla unitaria (euclidea) in $\mathbb{C}^n$ è $\mathbb{B}^n=\{z \in \mathbb{C}^n \mid \|z\|<1\}$, dove $\|\cdot\|$ indica la norma euclidea, mentre $\mathbb{B}_r^n=\{z \in \mathbb{C}^n \mid \|z\|<r\}$ è la palla (euclidea) di raggio $r>0$;
    \item dato un dominio $\Omega\subseteq\mathbb{C}^n$ e $x \in \mathbb{C}^n$, scriviamo $\delta(x)=\displaystyle\inf_{w \in \partial\Omega}\|x-w\|$ per indicare la distanza euclidea di $x$ dal bordo di $\Omega$;
    \item ALTRO?
\end{itemize}

Ricordiamo cosa sono la metrica e la distanza di Poincaré in $\mathbb{D}$.

\begin{defn}
    La \textit{metrica di Poincaré} (o \textit{iperbolica}) su $\mathbb{D}$ è data da
    \begin{equation}
        K_{\mathbb{D}}(z;v)=\frac{1}{1-|z|^2}|v|
    \end{equation}
    per ogni $z \in \mathbb{D}$ e $v \in \mathbb{C}\cong T_z^{\mathbb{C}}\mathbb{D}$. La metrica $K_{\mathbb{D}}$ è hermitiana completa con curvatura gaussiana costante uguale a $-4$.
\end{defn}

\begin{defn}
    La \textit{distanza di Poincaré} (o \textit{iperbolica}) $k_{\mathbb{D}}$ su $\mathbb{D}$ è la forma integrata della metrica di Poincaré. Per fatti noti di geometria iperbolica, è una distanza completa la cui espressione è data da
    \begin{equation}
        k_{\mathbb{D}}(z_1,z_2)=\frac{1}{2}\log{\frac{1+\left|\frac{z_1-z_2}{1-\bar{z}_1z_2}\right|}{1-\left|\frac{z_1-z_2}{1-\bar{z}_1z_2}\right|}}
    \end{equation}
    per ogni $z_1,z_2 \in \mathbb{D}$.
\end{defn}

Oltre alla curvatura negativa costante, la metrica e la distanza di Poincaré sono tali che le funzioni olomorfe dal disco unitario in sé sono semicontrazioni rispetto ad esse (lemma di Schwarz-Pick, si veda ad esempio \cite[Chapter I, paragraph 1, Theorem 1.1]{K1}).

Quello che vogliamo fare ora è generalizzare la metrica e la distanza di Poincaré ad una qualsiasi varietà complessa mantenendo queste proprietà, in particolare quella di rendere le funzioni olomorfe delle semicontrazioni. Ci sono vari modi per farlo, noi nello specifico vedremo la (pseudo)metrica e la (pseudo)distanza di Kobayashi, introdotte nel 1967 in \cite{K2}.

\begin{defn}
    Sia $X$ una varietà; la \textit{pseudometrica di Kobayashi} su $X$ è
    \begin{equation}\begin{split}
        K_X(x;Z)=&\inf\{|v| \mid v \in \mathbb{C}, \text{ esiste }f \in \text{Hol}(\mathbb{D},X) \\
        &\text{ tale che } f(0)=x, Df(0)v=Z\}
    \end{split}\end{equation}
    per ogni $x \in X$ e $Z \in T_xX$.
\end{defn}

\begin{oss}
    Non possiamo parlare di metrica perché, ad esempio, $K_{\mathbb{C}^n}\equiv 0$; vedremo però tra poco che per i domini limitati è effettivamente una metrica. Notiamo anche che, data $f \in \text{Hol}(X,Y)$, allora dalla definizione segue che $K_Y\big(f(x);Df(x)Z\big) \le K_X(x;Z)$ per ogni $x \in X$ e $Z \in T_xX$.
\end{oss}

Definiamo adesso la (pseudo)distanza di Kobayashi; più avanti vedremo (SOLO CIT DEL RISULTATO ORIGINALE O ANCHE DIM?) com'è collegata alla pseudometrica di Kobayashi.

\begin{defn}
    Sia $X$ una varietà complessa connessa; la \textit{pseudodistanza di Kobayashi} su $X$ è data da
    \begin{equation}\begin{split}
        k_X(z,w)=&\inf\Bigg\{\sum_{j=1}^m k_{\mathbb{D}}(z_{j-1},z_j) \mid m \in \mathbb{N}, z_j \in \mathbb{D}\text{ per }j=0,\dots,m\text{ tali che}\\
        &\text{esistono }\varphi_j\in \text{Hol}(\mathbb{D},X) \text{ con } \varphi_1(z_0)=z,\varphi_m(z_m)=w\Bigg\}
    \end{split}\end{equation}
    per $z,w \in X$, dove $k_{\mathbb{D}}$ è la distanza di Poincaré.
\end{defn}

\begin{oss} \label{k_lip}
    È facile vedere che $k_X$ è una pseudodistanza, ma in generale non è una distanza, ad esempio perché, come prima, $k_{\mathbb{C}^n}\equiv 0$. Inoltre, data $f \in \text{Hol}(X,Y)$, dalla definizione segue che $k_Y\big(f(x),f(y)\big) \le k_X(x,y)$ per ogni $x,y \in X$.
\end{oss}

\begin{defn}
    Sia $X$ una varietà complessa connessa. Se $k_X$ è una distanza, diremo che $X$ è \textit{Kobayashi-iperbolica}.
\end{defn}

Il seguente risultato per le varietà Kobayashi-iperboliche verrà spesso usato implicitamente.

\begin{prop}
    (INSERIRE CIT) Una varietà complessa connessa $X$ è Kobayashi-iperbolica se e solo se $k_X$ induce su $X$ la topologia di varietà.
\end{prop}

Diamo ora delle definizioni che ci serviranno per enunciare i risultati già noti nel caso dei domini regolari.

\begin{defn}
    Una funzione continua $\mu:\mathbb{C}^n \longrightarrow [0,+\infty)$ è detta \textit{funzionale di Minkowski} se
    \begin{nlist}
        \item $\mu(Z)=0$ se e solo se $Z=0$;
        \item $\mu(\zeta Z)=|\zeta|\mu(Z)$ per ogni $Z \in \mathbb{C}^n$ e $\zeta\in\mathbb{C}$.
    \end{nlist}
    Dato $\Omega\subseteq\mathbb{C}^n$ un dominio, poniamo $\mu_{\Omega}(Z)=\inf_{w\in\mathbb{C}^n\setminus\Omega}\mu(Z-w)$.
\end{defn}

\begin{defn}
    Sia $A\subseteq\mathbb{C}$ un aperto. Una funzione $u:A \longrightarrow \mathbb{R}\cup\{-\infty\}$ semicontinua superiormente è detta \textit{subarmonica} se per ogni $a \in A$, per ogni $r>0$ tale che $\overline{D(a,r)} \subset A$ e per ogni $h$ continua in $\overline{D(a,r)}$ e armonica in $D(a,r)$, se $h\restrict{\partial D(a,r)} \ge u\restrict{\partial D(a,r)}$, allora anche $h\restrict{D(a,r)} \ge u \restrict{D(a,r)}$.

    Sia $A\subseteq\mathbb{C}^n$ un aperto. Una funzione $u:A \longrightarrow \mathbb{R}\cup\{-\infty\}$ semicontinua superiormente è detta \textit{plurisubarmonica} se per ogni $a \in A$ e per ogni $Z \in \mathbb{C}^n$ l'applicazione $\zeta \longmapsto u(a+\zeta Z)$ è subarmonica dove definita.
\end{defn}

\begin{defn}
    Un dominio $\Omega \subseteq \mathbb{C}^n$ si dice \textit{(Hartogs) pseudoconvesso} se esiste un funzionale di Minkowski $\mu$ tale che $-\log\mu_{\Omega}$ è plurisubarmonica in $\Omega$.
\end{defn}

Nel caso di domini regolari, si può dare una definizione di pseudoconvessità più operativa equivalente.

\begin{defn}
    Sia $\Omega \subseteq \mathbb{C}^n$ un dominio con bordo $C^2$, cioè esiste $\rho \in C^2(\mathbb{C}^n)$ tale che $\Omega=\{\rho(z)<0\}$ e $\diff\rho\not=0$ in ogni punto di $\partial\Omega$.
    
    Dato $p \in \partial\Omega$, lo \textit{spazio tangente complesso} a $\partial\Omega$ in $p$ è
    \begin{equation}
        H_p\partial\Omega=\{Z \in \mathbb{C}^n \mid \langle \bar{\partial}\rho(p),Z\rangle=0\}.
    \end{equation}
    Diciamo che $\Omega$ è \textit{Levi pseudoconvesso} se la \textit{forma di Levi}
    \begin{equation}
        L_{\rho}(p;Z)=\sum_{\nu,\mu=1}^n \frac{\partial^2\rho}{\partial z_\nu\partial\bar{z}_\mu}(p)Z_\nu\bar{Z}_\mu, \quad Z=(Z_1,\dots,Z_n) \in \mathbb{C}^n
    \end{equation}
      è semidefinita positiva in $H_p\partial\Omega$ per ogni $p \in \partial\Omega$. Diciamo che è \textit{strettamente pseudoconvesso} se la forma di Levi è definita positiva.
\end{defn}

Vale il seguente risultato. (INSERIRE CIT, POI: MEGLIO CITARLO COME FATTO O COME TEOREMA?)
\begin{ftt}
    Sia $\Omega \subseteq \mathbb{C}^n$ un dominio limitato con bordo $C^2$. Allora $\Omega$ è Levi pseudoconvesso se e solo se è Hartogs pseudoconvesso.
\end{ftt}

Nella prossima sottosezione citeremo alcuni risultati sulla geometria dei domini limitati strettamente pseudoconvessi dotati della distanza di Kobayashi. In particolare, vedremo che sono Gromov-iperbolici, il che permette di derivare un teorema di tipo Wolff-Denjoy per questi domini.

\begin{defn}
    Sia $(X,d)$ uno spazio metrico proprio, cioè tale che ogni sottoinsieme chiuso e limitato è compatto. Dati $x,y,w \in X$ il \textit{prodotto di Gromov} tra $x$ e $y$ con punto base $w$ è $(x,y)_w=\frac{1}{2}\big(d(x,w)+d(y,w)-d(x,y)\big)$. Dato $\delta \ge 0$, diciamo che $X$ è \textit{$\delta$-iperbolico} se
    $$(x,y)_w \ge \min\{(x,z)_w,(y,z)_w\}-\delta\text{ per ogni }x,y,z,w \in X.$$
    Se $(X,d)$ è $\delta$-iperbolico per qualche $\delta \ge 0$, diremo che è \textit{Gromov-iperbolico}.

    (SERVE DAVVERO LA DEF DEL BORDO?) Fissato $w \in X$, il \textit{bordo iperbolico} $\partial_GX$ è costruito come classi di equivalenza delle successioni $(x_i)$ che convergono a infinito, cioè tali che $\displaystyle\lim_{i,j\rightarrow \infty}(x_i,x_j)_w=\infty$; due tali successioni $(x_i), (y_i)$ sono equivalenti se $\displaystyle\lim_{i\rightarrow \infty}(x_i,y_i)_w=\infty$. Questa costruzione è indipendente dalla scelta di $w$.
\end{defn}

Prima di passare a vedere i risultati noti della teoria sulla pseudometrica e la pseudodistanza di Kobayashi e sui domini strettamente pseudoconvessi, introduciamo il concetto di varietà taut, che sarà per noi un'ipotesi importante per i teoremi che andremo a dimostrare: infatti, quest'ipotesi ci darà la dicotomia nella tesi del teorema. Vedremo anche con un esempio l'importanza di tale ipotesi. Prima di dare la definizione, ci servirà un risultato sul comportamento delle funzioni olomorfe a valori in una varietà Kobayashi-iperbolica.

\begin{prop}
    (INSERIRE CIT) Sia $X$ una varietà complessa connessa. Allora $X$ è Kobayashi-iperbolica se e solo se $\text{\normalfont{Hol}}(\mathbb{D},X)$ è relativamente compatto in $C^0(\mathbb{D},X^*)$, dove $X^*$ è la compattificazione di Alexandroff di $X$. In tal caso, $\text{\normalfont{Hol}}(Y,X)$ è relativamente compatto in $C^0(Y,X^*)$ per ogni varietà complessa $Y$.
\end{prop}

\begin{defn}
    Una varietà complessa $X$ si dice \textit{taut} se è Kobayashi-iperbolica e ogni mappa nella chiusura di $\text{Hol}(\mathbb{D},X)$ in $C^0(\mathbb{D},X^*)$ è in $\text{Hol}(\mathbb{D},X)$ oppure è la mappa costante a $\infty$.
\end{defn}

Per finire, diamo delle definizioni che ci serviranno per parlare del comportamente delle iterate di funzioni olomorfe.

\begin{defn}
    Siano $X$ e $Y$ due spazi topologici. Diciamo che una successione $\{f_{\nu}\}_{\nu \in \mathbb{N}} \subset C^0(X,Y)$ è \textit{compattamente divergente} se per ogni coppia di compatti $H\subseteq X$ e $K\subseteq Y$ esiste $\nu_0 \in \mathbb{N}$ tale che $f(H)\cap K=\emptyset$ per ogni $\nu \ge \nu_0$.

    Una famiglia $\mathcal{F} \subseteq C^0(X,Y)$ è detta \textit{normale} se ogni successione in $\mathcal{F}$ ammette una sottosuccessione che converge uniformemente sui compatti oppure è compattamente divergente.
\end{defn}