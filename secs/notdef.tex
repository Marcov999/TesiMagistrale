Introduciamo la notazione che useremo:
\begin{itemize}
    \item scriviamo $\Omega$ per indicare un \textit{dominio} di $\mathbb{C}^n$, vale a dire un aperto connesso;
    \item con \textit{varietà complessa} s'intende una varietà differenziabile reale di dimensione pari con i cambi di carta olomorfi se visti come fra aperti di $\mathbb{C}^n$;
    \item data $X$ varietà complessa e $x \in X$, indichiamo con $T_xX$ lo spazio tangente a $X$ in $x$, che nel caso dei domini è canonicamente identificato con $\mathbb{C}^n$;
    \item date $X,Y$ varietà complesse, indichiamo con $\text{Hol}(X,Y)$ l'insieme delle funzioni olomorfe da $X$ a $Y$;
    \item data $f \in \text{Hol}(X,Y)$, indichiamo con $Df(x)$ il differenziale di $f$ in $x \in X$;
    \item il disco unitario è $\mathbb{D}=\{z \in \mathbb{C} \mid |z|<1\}$, mentre $\mathbb{D}^n$ è il polidisco in $\mathbb{C}^n$ e $\mathbb{D}_r=\{z \in \mathbb{C} \mid |z|<r\}$ è il disco di raggio $r>0$;
    \item la palla unitaria (euclidea) in $\mathbb{C}^n$ è $\mathbb{B}^n=\{z \in \mathbb{C}^n \mid \|z\|<1\}$, dove $\|\cdot\|$ indica la norma euclidea, mentre $\mathbb{B}_r^n=\{z \in \mathbb{C}^n \mid \|z\|<r\}$ è la palla (euclidea) di raggio $r>0$;
    \item ALTRO?
\end{itemize}

Ricordiamo cosa sono la metrica e la distanza di Poincaré in $\mathbb{D}$.

\begin{defn}
    La \textit{metrica di Poincaré} (o \textit{iperbolica}) su $\mathbb{D}$ è data da
    \begin{equation}
        K_{\mathbb{D}}(z;v)=\frac{1}{1-|z|^2}|v|
    \end{equation}
    per ogni $z \in \mathbb{D}$ e $v \in \mathbb{C}\cong T_z^{\mathbb{C}}\mathbb{D}$. La metrica $K_{\mathbb{D}}$ è hermitiana completa con curvatura gaussiana costante uguale a $-4$.
\end{defn}

\begin{defn}
    La \textit{distanza di Poincaré} (o \textit{iperbolica}) $k_{\mathbb{D}}$ su $\mathbb{D}$ è la forma integrata della metrica di Poincaré. Per fatti noti di geometria iperbolica, è una distanza completa la cui espressione è data da
    \begin{equation}
        k_{\mathbb{D}}(z_1,z_2)=\frac{1}{2}\log{\frac{1+\left|\frac{z_1-z_2}{1-\bar{z}_1z_2}\right|}{1-\left|\frac{z_1-z_2}{1-\bar{z}_1z_2}\right|}}
    \end{equation}
    per ogni $z_1,z_2 \in \mathbb{D}$.
\end{defn}

Oltre alla curvatura negativa costante, la metrica e la distanza di Poincaré sono tali che le funzioni olomorfe dal disco unitario in sé sono semicontrazioni rispetto ad esse (Lemma di Schwarz-Pick, INSERIRE UNA CITAZIONE). Quello che vogliamo fare ora è generalizzare la metrica e la distanza di Poincaré ad una qualsiasi varietà complessa mantenendo queste proprietà, in particolare quella di rendere le funzioni olomorfe delle semicontrazioni. Ci sono vari modi per farlo, noi nello specifico vedremo la (pseudo)metrica e la (pseudo)distanza di Kobayashi.

\begin{defn}
    Sia $X$ una varietà; la \textit{(pseudo)metrica di Kobayashi} su $X$ è
    \begin{equation}\begin{split}
        K_X(x;Z)=&\inf\{|v| \mid v \in \mathbb{C}, \text{ esiste }f \in \text{Hol}(\mathbb{D},X) \\
        &\text{ tale che } f(0)=x, Df(0)v=Z\}
    \end{split}\end{equation}
    per ogni $x \in X$ e $Z \in T_xX$.
\end{defn}

\begin{oss}
    Non possiamo parlare di metrica perché, ad esempio, $K_{\mathbb{C}^n}\equiv 0$; vedremo però tra poco che per i domini limitati è effettivamente una metrica. Notiamo anche che, data $f \in \text{Hol}(X,Y)$, allora dalla definizione segue che $K_Y\big(f(x);Df(x)Z\big) \le K_X(x;Z)$ per ogni $x \in X$ e $Z \in T_xX$.
\end{oss}

Definiamo adesso la (pseudo)distanza di Kobayashi; più avanti vedremo (SOLO CIT DEL RISULTATO ORIGINALE O ANCHE DIM?) com'è collegata alla pseudometrica di Kobayashi.

\begin{defn}
    Sia $X$ una varietà complessa connessa; la \textit{(pseudo)distanza di Kobayashi} su $X$ è data da
    \begin{equation}\begin{split}
        k_X(z,w)=&\inf\Bigg\{\sum_{j=1}^m k_{\mathbb{D}}(z_{j-1},z_j) \mid m \in \mathbb{N}, z_j \in \mathbb{D}\text{ per }j=0,\dots,m\text{ tali che}\\
        &\text{esistono }\varphi_j\in \text{Hol}(\mathbb{D},X) \text{ con } \varphi_1(z_0)=z,\varphi_m(z_m)=w\Bigg\}
    \end{split}\end{equation}
    per $z,w \in X$, dove $k_{\mathbb{D}}$ è la distanza di Poincaré.
\end{defn}

\begin{oss}
    È facile vedere che $k_X$ è una pseudodistanza, ma in generale non è una distanza, ad esempio perché, come prima, $k_{\mathbb{C}^n}\equiv 0$. Inoltre, data $f \in \text{Hol}(X,Y)$, dalla definizione segue che $k_Y\big(f(x),f(y)\big) \le k_X(x,y)$ per ogni $x,y \in X$.
\end{oss}

\begin{defn}
    Sia $X$ una varietà complessa connessa. Se $k_X$ è una distanza, diremo che $X$ è \textit{Kobayashi-iperbolica}.
\end{defn}

DA QUALCHE PARTE POTREBBE SERVIRTI METTERE LA PROP CHE PER LE VARIETÀ K-IPERBOLICHE LA TOPOLOGIA INDOTTA DA K È LA STESSA CHE QUELLA DI VARIETÀ

Prima di passare a vedere alcuni risultati noti della teoria sulla pseudometrica e la pseudodistanza di Kobayashi, introduciamo il concetto di varietà taut, che sarà per noi un'ipotesi importante per i teoremi che andremo a dimostrare: sarà quest'ipotesi a darci la dicotomia nella tesi del teorema. Vedremo anche con un esempio l'importanza di tale ipotesi. \\

FORSE QUI CI VA THEOREM 1.3.1 DI LILLECAP1, CON ANCHE LA DEFINIZIONE DI X* COME COMPATTIFICAZIONE DI ALEXANDROFF

\begin{defn}
    Una varietà complessa $X$ si dice \text{taut} se è Kobayashi-iperbolica e ogni mappa nella chiusura di $\text{Hol}(\mathbb{D},X)$ in $C^0(\mathbb{D},X^*)$ è in $\text{Hol}(\mathbb{D},X)$ oppure è la mappa costante a $\infty$.
\end{defn}

Per finire, diamo delle definizioni che ci serviranno per parlare del comportamente delle iterate di funzioni olomorfe.

\begin{defn}
    Siano $X$ e $Y$ due spazi topologici. Diciamo che una successione $\{f_{\nu}\}_{\nu \in \mathbb{N}} \subset C^0(X,Y)$ è \textit{compattamente divergente} se per ogni coppia di compatti $H\subseteq X$ e $K\subseteq Y$ esiste $\nu_0 \in \mathbb{N}$ tale che $f(H)\cap K=\emptyset$ per ogni $\nu \ge \nu_0$.

    Una famiglia $\mathcal{F} \subseteq C^0(X,Y)$ è detta \textit{normale} se ogni successione in $\mathcal{F}$ ammette una sottosuccessione che converge uniformemente sui compatti oppure è compattamente divergente.
\end{defn}

DA METTERE: COR 1.3.4 di LILLECAP1, MA QUESTO FORSE MEGLIO NEI RISULTATI