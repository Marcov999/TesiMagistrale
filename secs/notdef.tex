Introduciamo le notazioni e definizioni che useremo:
\begin{itemize}
    \item  un \textit{dominio} in $\mathbb{C}^n$ è un aperto connesso;
    \item una \textit{varietà complessa} di dimensione $n$ è una varietà differenziabile reale, di dimensione $2n$ e tale che i cambi di carta siano olomorfi se considerati fra aperti di $\mathbb{C}^n$;
    \item dati una varietà complessa $X$ e $x \in X$, indichiamo con $T_xX$ lo spazio tangente complesso a $X$ in $x$, che nel caso dei domini è canonicamente identificato con $\mathbb{C}^n$;
    \item nei vari enunciati, quando è data una varietà complessa $X$ sottintendiamo anche di aver fissato una metrica hermitiana $\|\cdot\|_X$ su $X$, che induce la distanza $d_X$ se $X$ è connessa; quando $X$ è sottovarietà complessa di una varietà complessa $Y$, fissiamo la metrica hermitiana su $Y$ e consideriamo su $X$ la metrica indotta; se la varietà ambiente è $Y=\mathbb{C}^n$ useremo sempre $\|\cdot\|_{\mathbb{C}^n}=\|\cdot\|$, la metrica euclidea su $\mathbb{C}^n$;
    \item dati $X,Y$ spazi topologici, $C^0(X,Y)$ è lo spazio delle funzioni continue da $X$ a $Y$ considerato con la topologia compatta-aperta. Nel caso in cui $Y$ sia uno spazio metrico, tale topologia coincide con la topologia della convergenza uniforme sui compatti;
    \item date $X,Y$ varietà complesse, indichiamo con $\text{Hol}(X,Y)$ l'insieme delle funzioni olomorfe da $X$ a $Y$, con $\mathcal{O}(X)$ l'insieme delle funzioni olomorfe da $X$ in $\mathbb{C}$ e con $\text{Aut}(X)$ l'insieme dei biolomorfismi di $X$ in sé;
    \item data $f \in \text{Hol}(X,Y)$, indichiamo con $\diff_x f$ il differenziale di $f$ in $x \in X$;
    \item il disco unitario è $\mathbb{D}=\{z \in \mathbb{C} \mid |z|<1\}$; il disco di centro $0$ e raggio $r>0$ è $\mathbb{D}_r=\{z \in \mathbb{C} \mid |z|<r\}$; il disco di centro $a\in\mathbb{C}$ e raggio $r>0$ è $D(a,r)=\{z\in\mathbb{C}\mid |z-a|<r\}$; il polidisco unitario in $\mathbb{C}^n$ è $\mathbb{D}^n=\mathbb{D}\times\dots\times\mathbb{D}$;
    \item la palla unitaria (euclidea) in $\mathbb{C}^n$ è $\mathbb{B}^n=\{z \in \mathbb{C}^n \mid \|z\|<1\}$, dove $\|\cdot\|$ indica la norma euclidea, mentre $\mathbb{B}_r^n=\{z \in \mathbb{C}^n \mid \|z\|<r\}$ è la palla (euclidea) di centro l'origine $O\in\mathbb{C}^n$ e raggio $r>0$;
    \item dati uno spazio topologico $Y$ e un suo sottoinsieme $A\subseteq Y$, il bordo relativo di $A$ in $Y$ è $\partial_Y A:=\overline{A}\setminus A$, dove la chiusura è fatta in $Y$; se $Y=\mathbb{C}^n$ allora $\partial_{\mathbb{C}^n}A=\partial A$ è il bordo euclideo;
    \item dato un dominio $\Omega\subseteq\mathbb{C}^n$ e $z \in \mathbb{C}^n$, scriviamo $\delta_\Omega(z)=\displaystyle\inf_{p \in \partial\Omega}\|z-p\|$ per indicare la distanza euclidea di $x$ dal bordo euclideo di $\Omega$.
\end{itemize}

Ricordiamo cosa sono la metrica e la distanza di Poincaré in $\mathbb{D}$.

\begin{defn}
    La \textit{metrica di Poincaré} (o \textit{iperbolica}) su $\mathbb{D}$ è data da
    \begin{equation}
        \lambda_{\mathbb{D}}(z;v)=\frac{1}{1-|z|^2}|v|
    \end{equation}
    per ogni $z \in \mathbb{D}$ e $v \in \mathbb{C}\cong T_z\mathbb{D}$. La metrica $\lambda_{\mathbb{D}}$ è hermitiana di curvatura gaussiana costante uguale a $-4$ (si veda \cite[Section 1-5]{Ah}).
\end{defn}

\begin{defn} \label{poidist}
    La \textit{distanza di Poincaré} (o \textit{iperbolica}) $\omega$ su $\mathbb{D}$ è la forma integrata della metrica di Poincaré. È una distanza completa la cui espressione è data da
    \begin{equation}
        \omega(z_1,z_2)=\frac{1}{2}\log{\frac{1+\left|\frac{z_1-z_2}{1-\bar{z}_1z_2}\right|}{1-\left|\frac{z_1-z_2}{1-\bar{z}_1z_2}\right|}}=\text{arctanh\,}{\left|\frac{z_1-z_2}{1-\bar{z}_1z_2}\right|}
    \end{equation}
    per ogni $z_1,z_2 \in \mathbb{D}$ (si veda \cite[Section 1-1]{Ah}).
\end{defn}

Oltre ad avere curvatura negativa costante, la metrica e la distanza di Poincaré sono tali che le funzioni olomorfe dal disco unitario in sé sono semicontrazioni rispetto ad esse.

\begin{lm}
    (lemma di Schwarz-Pick) Sia $f \in \text{\normalfont{Hol}}(\mathbb{D},\mathbb{D})$.
    Allora per ogni $z, w \in \mathbb{D}$ si ha
    $$\left|\frac{f(z)-f(w)}{1-\overline{f(w)}f(z)}\right| \le \left|\frac{z-w}{1-\bar{w}z}\right| \text{ e } \frac{|f'(z)|}{1-|f(z)|^2} \le \frac{1}{1-|z|^2};$$
    inoltre, se vale l'uguaglianza nella prima per $z_0,w_0$ con $z_0\not=w_0$ o nella seconda per $z_0$ allora $f\in\textnormal{Aut}(\mathbb{D})$ e vale sempre l'uguaglianza.
\end{lm}

Per la dimostrazione si rimanda a \cite[Chapter I, Theorem 1.1]{Ko2}.

\begin{cor}
    Sia $f\in\textnormal{Hol}(\mathbb{D},\mathbb{D})$. Allora si ha
    \begin{equation}
        \omega\big(f(z),f(w)\big) \le \omega(z,w)
    \end{equation}
    per ogni $z,w\in\mathbb{D}$.
\end{cor}
  
\begin{proof}
    Discende dal lemma di Schwarz-Pick e dal fatto che la funzione $\text{arctanh}$ è strettamente crescente.
\end{proof}

Ci tornerà utile anche la versione semplificata del lemma di Schwarz-Pick.

\begin{lm}
    (lemma di Schwarz) Sia $f\in\textnormal{Hol}(\mathbb{D},\mathbb{D})$ tale che $f(0)=0$. Allora per ogni $z\in\mathbb{D}$ si ha $|f(z)| \le |z|$ e $|f'(0)| \le 1$; inoltre, se vale l'uguaglianza nella prima per $z_0\not=0$ oppure nella seconda allora $f(z)=e^{i\theta}z$ per qualche $\theta\in\mathbb{R}$.
\end{lm}

Enunciamo adesso dei fatti noti sulle geodetiche del disco unitario con la metrica di Poincaré. Visto che ne parleremo più in generale, diamo la definizione per spazi metrici.

\begin{defn}
    Siano $(X,d)$ uno spazio metrico e $I\subseteq\mathbb{R}$ un intervallo. Una curva $\sigma:I\longrightarrow X$ è detta \textit{geodetica} se
    $$d\big(\sigma(t_1),\sigma(t_2)\big)=|t_1-t_2|$$
    per ogni $t_1,t_2\in I$.
\end{defn}

\begin{oss}
    Dati una geodetica $\sigma:I\longrightarrow X$ e $a,b\in I$, si ha
    $$d\big(\sigma(a),\sigma(b)\big)=\inf_{a=t_0<t_1<\dots<t_n=b} \sum_{j=1}^n d\big(\sigma(t_{j-1}),\sigma(t_j)\big);$$
    cioè, le geodetiche sono le curve che ``minimizzano la lunghezza''.
\end{oss}

\begin{prop}
    Le geodetiche di $(\mathbb{D},\omega)$ sono i diametri di $\mathbb{D}$ e le intersezioni con $\mathbb{D}$ dei cerchi euclidei ortogonali a $\partial\mathbb{D}$. In particolare, ogni coppia di punti distinti è connessa da un'unica geodetica.
\end{prop}

Per una dimostrazione si veda \cite[point (iv) of Proposition 1.2.7]{A6}.\\

Quello che vogliamo fare ora è generalizzare la metrica e la distanza di Poincaré a una qualsiasi varietà complessa mantenendo la proprietà di rendere le funzioni olomorfe delle semicontrazioni. Ci sono vari modi per farlo, noi nello specifico vedremo la (pseudo)metrica e la (pseudo)distanza di Kobayashi, introdotte nel 1967 in \cite{Ko1}.

\begin{defn}
    Sia $X$ una varietà complessa; la \textit{pseudometrica di Kobayashi} su $X$ è
    \begin{equation}\begin{split}
        K_X(x;Z)=&\inf\{|v| \mid v \in \mathbb{C}, \text{ esiste }f \in \text{Hol}(\mathbb{D},X) \\
        &\text{ tale che } f(0)=x, \diff_0 f(v)=Z\}
    \end{split}\end{equation}
    per ogni $x \in X$ e $Z \in T_xX$.
\end{defn}

\begin{oss} \label{metr_noncr}
    Non possiamo sempre parlare di metrica perché, per esempio, $K_{\mathbb{C}^n}\equiv 0$. Infatti, dati $z\in\mathbb{C}^n$ e $Z\in T_z\mathbb{C}^n=\mathbb{C}^n$, abbiamo che la funzione $f_\epsilon\in\text{Hol}(\mathbb{D},\mathbb{C}^n)$ data da $f_\epsilon(\zeta)=z+\zeta Z/\epsilon$ è tale che $\diff_0 f_\epsilon(\epsilon)=Z$ per ogni $\epsilon>0$; di conseguenza, $K_{\mathbb{C}^n}(z;Z)=0$.
\end{oss}

Vediamo adesso che le funzioni olomorfe sono semicontrazioni rispetto alla pseudometrica di Kobayashi.

\begin{prop} \label{metrdecr}
    Siano $X$ e $Y$ varietà complesse, e sia $f \in \textnormal{Hol}(X,Y)$. Allora si ha
    \begin{equation}
        K_Y\big(f(x);\diff_x f(Z)\big) \le K_X(x;Z)
    \end{equation}
    per ogni $x \in X$ e $Z \in T_xX$.
\end{prop}

\begin{proof}
    Dati $x\in X$ e $Z\in T_xX$, poniamo
    \begin{align*}
        A_X(x;Z)=&\{v \in \mathbb{C} \mid \text{esiste }g \in \text{Hol}(\mathbb{D},X) \\
        &\text{ tale che } g(0)=x, \diff_0 g(v)=Z\},
    \end{align*}
    e sia $A_Y\big(f(x);\diff_x f(Z)\big)$ definito analogamente. Per definizione,
    \begin{gather*}
        K_X(x;Z)=\inf\{|v|\mid v\in A_X(x;Z)\}\\
        \text{ e }\\
        K_Y\big(f(x);\diff_x f(Z)\big)=\inf\{|v|\mid v\in A_Y\big(f(x);\diff_x f(Z)\big)\};
    \end{gather*}
    quindi ci basta mostrare che $A_X(x;Z)\subseteq A_Y\big(f(x);\diff_x f(Z)\big)$. Dato $v\in A_X(x;Z)$, prendiamo $g\in\text{Hol}(\mathbb{D},X)$ tale che $g(0)=x$ e $\diff_0 g(v)=Z$. Allora abbiamo $(f\circ g)(0)=f(x)$ e $\diff_0(f\circ g)(v)=(\diff_x f\circ\diff_0 g)(v)=\diff_x f(Z)$; dunque $v \in A_Y\big(f(x);\diff_x f(Z)\big)$. Perciò $A_X(x;Z)\subseteq A_Y\big(f(x);\diff_x f(Z)\big)$, come voluto.
\end{proof}

Nei casi a cui saremo interessati, la pseudometrica di Kobayashi è effettivamente una metrica.

\begin{prop}
    Sia $X$ una sottovarietà complessa e limitata di $\mathbb{C}^n$. Allora $K_X$ è una metrica, cioè $K_X(z;Z)>0$ per ogni $z\in X$ e $0\not=Z\in T_zX$.
\end{prop}

\begin{proof}
    È un'immediata conseguenza di un risultato più forte che dimostreremo nella prossima sezione, il punto (3) della Proposizione \ref{metrica_bilip}.
\end{proof}

Definiamo adesso la (pseudo)distanza di Kobayashi; più avanti vedremo com'è collegata alla pseudometrica di Kobayashi.

\begin{defn}
    Sia $X$ una varietà complessa e connessa; la \textit{pseudodistanza di Kobayashi} su $X$ è data da
    \begin{equation}\begin{split}
        k_X(z,w)=&\inf\Bigg\{\sum_{j=1}^m \omega(\zeta_{j-1},\zeta_j) \bigg\vert \text{esistono }m\in\mathbb{N},\text{ punti }\zeta_0,\dots,\zeta_m \in \mathbb{D}\text{ e}\\
        &\text{funzioni }\varphi_1,\dots,\varphi_m\in\text{Hol}(\mathbb{D},X) \text{ tali che } \varphi_1(\zeta_0)=z,\varphi_m(\zeta_m)=w\\
        &\text{e }\varphi_j(\zeta_j)=\varphi_{j+1}(\zeta_j)\text{ per }j=1,\dots,m-1\Bigg\}
    \end{split}\end{equation}
    per $z,w \in X$.
\end{defn}

\begin{oss} \label{k_lip}
    È facile vedere che $k_X$ è una pseudodistanza, ma in generale non è una distanza, ad esempio perché, come prima, $k_{\mathbb{C}^n}\equiv 0$. Infatti, dati $z,w\in X$, possiamo considerare i punti $\zeta_0=0$ e $1>\zeta_1=\epsilon>0$ e la funzione $\varphi_1\in\text{Hol}(\mathbb{D},\mathbb{C}^n)$ tale che $\varphi_1(\zeta)=z+\zeta(w-z)/\epsilon$. Si ha $\varphi_1(\zeta_0)=z$ e $\varphi_1(\zeta_1)=w$; perciò, per definizione, $k_{\mathbb{C}^n}(z,w) \le \omega(0,\epsilon)$ per ogni $1>\epsilon>0$, da cui $k_{\mathbb{C}^n}(z,w)=0$.
    
    Vedremo però più avanti (Osservazione \ref{kobisdist}) che se $X$ è una sottovarietà complessa, connessa e limitata di $\mathbb{C}^n$ allora $k_X$ è effettivamente una distanza.
\end{oss}

Vediamo adesso che le funzioni olomorfe sono delle semicontrazioni rispetto alla pseudodistanza di Kobayashi.

\begin{prop} \label{semicontr}
    Siano $X$ e $Y$ varietà complesse e connesse, e consideriamo $f \in \textnormal{Hol}(X,Y)$. Allora
    \begin{equation}
        k_Y\big(f(x),f(y)\big) \le k_X(x,y)
    \end{equation}
    per ogni $x,y \in X$.
\end{prop}

\begin{proof}
    Dati $x,y\in X$, poniamo
    \begin{align*}
        A_X(x,y)=&\Bigg\{\sum_{j=1}^m \omega(\zeta_{j-1},\zeta_j) \bigg\vert \text{esistono }m\in\mathbb{N},\text{ punti }\zeta_0,\dots,\zeta_m \in \mathbb{D}\text{ e}\\
        &\text{funzioni }\varphi_1,\dots,\varphi_m\in\text{Hol}(\mathbb{D},X) \text{ tali che } \varphi_1(\zeta_0)=x,\varphi_m(\zeta_m)=y\\
        &\text{e }\varphi_j(\zeta_j)=\varphi_{j+1}(\zeta_j)\text{ per }j=1,\dots,m-1\Bigg\},
    \end{align*}
    e sia $A_Y\big(f(x),f(y)\big)$ definito analogamente. Per definizione,
    $$k_X(x,y)=\inf A_X(x,y)\quad \text{e}\quad k_Y\big(f(x),f(y)\big)=\inf A_Y\big(f(x),f(y)\big);$$
    quindi ci basta mostrare che $A_X(x,y)\subseteq A_Y\big(f(x),f(y)\big)$. Dati $\zeta_0,\dots,\zeta_m \in \mathbb{D}$ e $\varphi_1,\dots,\varphi_m\in\text{Hol}(\mathbb{D},X)$ che realizzano $\displaystyle\sum_{j=1}^m \omega(\zeta_{j-1},\zeta_j)$ in $A_X(x,y)$, si verifica immediatamente che $\zeta_0,\dots,\zeta_m \in \mathbb{D}$ e $f\circ\varphi_1,\dots,f\circ\varphi_m\in\text{Hol}(\mathbb{D},Y)$ realizzano lo stesso numero in $A_Y\big(f(x),f(y)\big)$. Perciò $A_X(x,y)\subseteq A_Y\big(f(x),f(y)\big)$, come voluto.
\end{proof}

Segue immediatamente l'invarianza per biolomorfismi.

\begin{cor} \label{bioloiso}
    Siano $X$ e $Y$ varietà complesse e connesse, e consideriamo un biolomorfismo $f:X \longrightarrow Y$. Allora
    \begin{equation}
        k_Y\big(f(x),f(y)\big)=k_X(x,y)
    \end{equation}
    per ogni $x,y \in X$.
\end{cor}

\begin{defn}
    Una varietà complessa e connessa $X$ è \textit{Kobayashi-iperbolica} se $k_X$ è una distanza.
\end{defn}

\begin{oss} \label{lengthspace}
    Dalla definizione di $k_X$ segue che ogni varietà Kobayashi-iperbolica è uno \textit{spazio di lunghezze} nel senso di \cite[Part I, Definition 3.1]{BH}.
\end{oss}

Il seguente risultato per le varietà Kobayashi-iperboliche verrà spesso usato implicitamente.

\begin{prop}
    (Barth, \cite{B2}) Sia $X$ una varietà complessa e connessa. Allora $X$ è Kobayashi-iperbolica se e solo se $k_X$ vi induce la topologia di varietà.
\end{prop}