In questa sezione dimostreremo alcuni risultati di convergenza di successioni di funzioni olomorfe. Cominciamo con un lemma sulla convergenza puntuale.

\begin{lm} \label{limpt-bis}
    Sia $X$ una sottovarietà complessa, connessa e relativamente compatta di una varietà complessa $Y$. Supponiamo che esista $\kappa_0>0$ tale che $X$ sia $(1,\kappa_0)$-visibile.
    
    Siano $Z$ una varietà Kobayashi-iperbolica e $\{f_n\}_{n\in\mathbb{N}}\subseteq\textnormal{Hol}(Z,X)$ una successione compattamente divergente. Allora esistono $\xi\in\partial_YX$ e una sottosuccessione $\{f_{n_j}\}_{j\in\mathbb{N}}$ tali che $f_{n_j}(z)\longrightarrow\xi$ per ogni $z\in Z$.
\end{lm}

\begin{proof}
    Sia $z_0\in Z$. Per relativa compattezza, a meno di sottosuccessioni possiamo supporre che esista $\xi_0\in\partial_YX$ tale che $f_n(z_0)\longrightarrow\xi_0$. Supponiamo per assurdo che esista $z_1\in Z$ tale che la successione $\{f_n(z_1)\}_{n\in\mathbb{N}}$ non converge a $\xi_0$. Considerando una curva continua che collega $z_0$ a $z_1$, si trova facilmente che possiamo eventualmente sostituirli con due punti tali che $k_Z(z_0,z_1)<\kappa_0/2$. Di nuovo a meno di sottosuccessioni, possiamo supporre che esista $\xi_1\in\partial_YX$ tale che $\xi_1\not=\xi_0$ e $f_n(z_1)\longrightarrow\xi_1$.

    Per il Teorema \ref{similgeo_conn} esiste una $(1,\kappa_0/2)$-simil-geodetica $\sigma:[0,T]\longrightarrow Z$ tale che $\sigma(0)=z_0$ e $\sigma(T)=z_1$. Poniamo $\sigma_n:=f_n\circ\sigma$, vogliamo dimostrare che $\sigma_n$ è una $(1,\kappa_0)$-simil-geodetica per ogni $n$. Per la Proposizione \ref{semicontr} e per la definizione di $(1,\kappa_0/2)$-simil-geodetica, abbiamo che
    $$k_X\big(\sigma_n(s),\sigma_n(t)\big) \le k_Z\big(\sigma(s),\sigma(t)\big) \le |t-s|+\kappa_0/2\le |t-s|+\kappa_0$$
    per ogni $s,t\in[0,T]$. Inoltre, sempre dalla definizione di di $(1,\kappa_0/2)$-simil-geodetica si ha che
    \begin{align*}
        |t-s|-\kappa_0 &\le |0-T|-\kappa_0/2-\kappa_0/2 \le k_Z\big(\sigma(0),\sigma(T)\big)-\kappa_0/2\\
        &=k_Z(z_0,z_1)-\kappa_0/2<0\le k_X\big(\sigma_n(s),\sigma_n(t)\big)
    \end{align*}
    per ogni $s,t\in[0,T]$. Infine, dalla Proposizione \ref{metrdecr} e dalla definizione di $(1,\kappa_0/2)$-simil-geodetica segue che
    $$K_X\big(\sigma_n(t);\sigma_n'(t)\big)\le K_Z\big(\sigma(t);\sigma'(t)\big) \le 1$$
    per ogni $t\in[0,T]$. Dunque $\sigma_n$ è una $(1,\kappa_0)$-simil-geodetica per ogni $n$, come volevamo.

    Adesso, poiché $\sigma_n(0)=f_n(z_0)\longrightarrow\xi_0$ e $\sigma_n(T)=f_n(z_1)\longrightarrow\xi_1$ e $X$ è $(1,\kappa_0)$-visibile, esiste un compatto $K\subseteq X$ tale che
    $$\emptyset\not=K\cap\sigma_n([0,T])=K\cap f_n\big(\sigma([0,T])\big)$$
    per ogni $n$, in contraddizione con l'ipotesi che la successione $\{f_n\}_{n\in\mathbb{N}}$ sia compattamente divergente.
\end{proof}

Quello che andremo a dimostrare adesso è uno dei fatti cruciali per ottenere il teorema di tipo ``Wolff-Denjoy''. Esso afferma che, sotto condizioni di visibilità per le simil-geodetiche, le sottosuccessioni di iterate di una funzione olomorfa che ``tendono a infinito'' convergono tutte, puntualmente, a un unico punto del bordo.

\begin{prop} \label{inf_impl_ugu}
    Sia $X$ una sottovarietà complessa, connessa e relativamente compatta di una varietà Kobayashi-iperbolica $Y$. Supponiamo che esista un $\kappa_0>0$ tale che $X$ sia $(1,\kappa_0)$-visibile. Data una funzione $F\in\textnormal{Hol}(X,X)$, esiste $\xi\in\partial_YX$ tale che per ogni funzione $\mu:\mathbb{N}\longrightarrow\mathbb{N}$ strettamente crescente per cui esiste $y_0 \in X$ tale che
    \begin{equation}
        \lim_{j\longrightarrow+\infty} k_X\big(F^{\mu(j)}(y_0),y_0\big)=+\infty
    \end{equation}
    si ha
    \begin{equation}
        \lim_{j\longrightarrow+\infty} F^{\mu(j)}(z)=\xi
    \end{equation}
    per ogni $z \in X$.
\end{prop}

\begin{proof}
    Se $\displaystyle\limsup_{n\longrightarrow+\infty}k_X\big(F^n(x),x\big)<+\infty$ per ogni $x\in X$ l'affermazione è vera a vuoto per ogni $\xi\in\partial_YX$. Altrimenti, scelto $x_0\in X$ tale che $\displaystyle\limsup_{n\longrightarrow+\infty}k_X\big(F^n(x_0),x_0\big)=+\infty$ possiamo costruire una funzione strettamente crescente $\nu:\mathbb{N}\longrightarrow\mathbb{N}$ tale che:
    \begin{itemize}
        \item si ha $k_X\big(F^{\nu(j)}(x_0),x_0\big) \ge k_X\big(F^k(x_0),x_0\big)$ per ogni $j \in \mathbb{N}$ e per ogni $k \le \nu(j)$;
        \item si ha $\displaystyle\lim_{j\longrightarrow+\infty}k_X\big(F^{\nu(j)}(x_0),x_0\big)=+\infty$;
        \item la successione $\{F^{\nu(j)}(x_0)\}_{j\in\mathbb{N}}$ converge a un certo $\xi\in\partial_YX$.
    \end{itemize}
    
    Infatti, definendo induttivamente $l_0=0$ e $l_n$ come il minimo numero naturale $h>l_{n-1}$ tale che $k_X\big(F^h(x_0),x_0\big) \ge \max\left\{n,k_X\big(F^{l_{n-1}}(x_0),x_0\big)\right\}$, abbiamo $k_X\big(F^{l_n}(x_0),x_0\big) \ge k_X\big(F^k(x_0),x_0\big)$ per ogni $n$ e per ogni $k \le l_n$; inoltre, abbiamo $\displaystyle\lim_{n\longrightarrow+\infty}k_X\big(F^{l_n}(x_0),x_0\big)=+\infty$. Per relativa compattezza di $X$, esiste un'ulteriore sottosuccessione $l_{n_j}$ tale che $F^{l_{n_j}}(x_0)$ converge a un certo $\xi\in\overline{X}$, e in realtà $\xi\in\partial_YX$ visto che la distanza di Kobayashi da $x_0$ tende a $+\infty$; basta allora porre $\nu(j)=l_{n_j}$.\\

    Vogliamo ora mostrare la seguente asserzione.

    Siano $\{m_j\}_{j\in\mathbb{N}}$ e $\{m'_j\}_{j\in\mathbb{N}}$ due successioni strettamente crescenti di numeri naturali e $z_0,z'_0\in X$ tali che:
    \begin{enumerate}[label={(\arabic*)}]
        \item per ogni $j\in\mathbb{N}$ si ha $m_j \ge m'_j$;
        \item per ogni $j\in\mathbb{N}$ e $k \le m_j$ si ha $k_X\big(F^{m_j}(z_0),z_0\big) \ge k_X\big(F^k(z_0),z_0\big)$;
        \item si ha $\displaystyle\lim_{j\longrightarrow+\infty}k_X\big(F^{m_j}(z_0),z_0\big)=\lim_{j\longrightarrow+\infty}k_X\big(F^{m'_j}(z'_0),z_0\big)=+\infty$;
        \item le successioni $\{F^{m_j}(z_0)\}_{j\in\mathbb{N}}$ e $\{F^{m'_j}(z'_0)\}_{j\in\mathbb{N}}$ convergono, rispettivamente, a $\zeta$ e $\zeta'$ in $\partial_YX$;
    \end{enumerate}
    allora $\zeta=\zeta'$.

    Supponiamo per assurdo che $\zeta\not=\zeta'$. Per il Corollario \ref{kobisdist} anche $X$ è Kobayashi-iperbolica; quindi grazie al Teorema \ref{similgeo_conn} possiamo scegliere, per ogni $j\in\mathbb{N}$, una $(1,\kappa_0)$-simil-geodetica $\sigma_j:[0,T_j] \longrightarrow X$ tale che $\sigma_j(0)=F^{m_j}(z_0)$ e $\sigma_j(T_j)=F^{m'_j}(z'_0)$. Adesso, dato che abbiamo assunto che $\{F^{m_j}(z_0)\}_{j\in\mathbb{N}}$ e $\{F^{m'_j}(z'_0)\}_{j\in\mathbb{N}}$ convergano a due punti di $\partial_YX$ distinti e $X$ ha la visibilità rispetto alle $(1,\kappa_0)$-simil-geodetiche, esistono una costante $0<R<+\infty$ e, per ogni $j\in\mathbb{N}$, un $t_j\in[0,T_j]$ tali che $k_X\big(z_0,\sigma_j(t_j)\big)<R$. Per il Lemma \ref{quasiovvio} si ha dunque che
    \begin{equation} \label{3k_02R}
        \begin{aligned}
            k_X\big(F^{m_j}(z_0),F^{m'_j}(z'_0)\big) &\ge k_X\big(F^{m_j}(z_0),\sigma_j(t_j)\big)+k_X\big(\sigma_j(t_j),F^{m'_j}(z'_0)\big)-3\kappa_0\\
            &\ge k_X\big(F^{m_j}(z_0),z_0\big)-k_X\big(z_0,\sigma_j(t_j)\big)\\
            &\quad+k_X\big(F^{m'_j}(z'_0),z_0\big)-k_X\big(z_0,\sigma_j(t_j)\big)-3\kappa_0\\
            &\ge k_X\big(F^{m_j}(z_0),z_0\big)+k_X\big(F^{m'_j}(z'_0),z_0\big)-3\kappa_0-2R;
        \end{aligned}
    \end{equation}
    d'altra parte, abbiamo anche che
    \begin{equation} \label{-3k_02R}
        \begin{aligned}
            k_X\big(F^{m_j}(z_0),F^{m'_j}(z'_0)\big) &\le k_X\big(F^{m_j-m'_j}(z_0),z'_0\big) \\
            &\le k_X\big(F^{m_j-m'_j}(z_0),z_0\big)+k_X(z_0,z'_0) \\
            &\le k_X\big(F^{m_j}(z_0),z_0\big)+k_X(z_0,z'_0),
        \end{aligned}
    \end{equation}
    dove per la prima e la terza disuguaglianza abbiamo usato, rispettivamente, le condizioni (1) e (2) sulle successioni $\{m_j\}_{j\in\mathbb{N}}$ e $\{m'_j\}_{j\in\mathbb{N}}$; nella prima, abbiamo anche usato che le funzioni olomorfe sono semicontrazioni rispetto a $k_X$. Concatenando la \eqref{3k_02R} e la \eqref{-3k_02R} e riarrangiando i termini, otteniamo
    $$k_X\big(F^{m'_j}(z'_0),z_0\big) \le k_X(z_0,z'_0)+3\kappa_0+2R,$$
    che è in contraddizione con la condizione (3).\\

    Adesso che l'asserzione è stata dimostrata, possiamo concludere la dimostrazione. Usando la disuguaglianza triangolare e il fatto che le funzioni olomorfe sono semicontrazioni rispetto a $k_X$, troviamo che
    \begin{gather*}
        k_X\big(F^{\mu(j)}(z),z'\big) \ge k_X\big(F^{\mu(j)}(y_0),y_0\big)-k_X(y_0,z)-k_X(z',y_0);
    \end{gather*}
    segue che $\displaystyle\lim_{j\longrightarrow+\infty}k_X\big(F^{\mu(j)}(z),z'\big)=+\infty$ per ogni $z,z' \in X$.

     Fissiamo ora uno $z\in X$, e prendiamo $\xi'$ punto limite di $\{F^{\mu(j)}(z)\}_{j\in\mathbb{N}}$. Allora deve esistere una funzione strettamente crescente $\tau:\mathbb{N}\longrightarrow\mathbb{N}$ tale che $\displaystyle \lim_{j\longrightarrow+\infty} F^{(\mu\circ\tau)(j)}(z)=\xi'$; inoltre, poiché $\displaystyle\lim_{j\longrightarrow+\infty}k_X\big(F^{\mu(j)}(z),z\big)=+\infty$, dev'essere $\xi'\in\partial_YX$. Scegliamo una funzione strettamente crescente $\tau':\mathbb{N}\longrightarrow\mathbb{N}$ tale che $\nu\circ\tau' \ge \mu\circ\tau$ e applichiamo l'asserzione dimostrata sopra alle successioni $m_j=(\nu\circ\tau')(j), m'_j=(\mu\circ\tau)(j)$ e ai punti $z_0=x_0,z'_0=z$, per i quali si verificano facilmente le condizioni (1), (2), (3) e (4). Troviamo così $\xi'=\xi$ e si conclude grazie al Lemma \ref{lemmasottosotto1}.
\end{proof}

Anche la prossima proposizione ci aiuterà nella nostra dimostrazione. Essa afferma che, sotto condizioni di visibilità per le simil-geodetiche, vale una sorta di lemma ``sotto-sotto'', cioè da ogni sottosuccessione si può estrarre una sottosottosuccessione convergente (a una costante). Prima ci serviranno un paio di lemmi.

\begin{lm} \label{nonazeroplease-bis}
    Sia $X$ una sottovarietà Kobayashi-iperbolica di una varietà complessa $Y$. Supponiamo che esista $\kappa_0>0$ tale che $X$ sia $(1,\kappa_0)$-visibile. Siano $\{x_n\}_{n\in\mathbb{N}}$ e $\{y_n\}_{n\in\mathbb{N}}$ due successioni di punti di $X$ tali che $x_n\longrightarrow\xi$ e $y_n\longrightarrow\xi'$, con $\xi,\xi'\in\partial_YX$ e $\xi\not=\xi'$.
    
    Allora non può essere che $\displaystyle\lim_{n\longrightarrow+\infty}k_X(x_n,y_n)=0$.
\end{lm}

\begin{proof}
    Supponiamo per assurdo che la tesi sia falsa, e prendiamo due successioni che la contraddicono. Per il Teorema \ref{similgeo_conn} possiamo prendere, per ogni $n\in\mathbb{N}$, una $(1,1/n)$-simil-geodetica $\sigma_n:[0,T_n]\longrightarrow X$ tale che $\sigma_n(0)=x_n$ e ${\sigma_n(T_n)=y_n}$. Per $n$ sufficientemente grande sono tutte $(1,\kappa_0)$-simil-geodetiche, per cui esistono un compatto $K\subseteq X$ e $t_n\in[0,T_n]$ tali che $\sigma_n(t_n)\in K$ per ogni $n$. Dato che $X$ è una varietà, possiamo prendere un compatto ${H\subseteq X}$ tale che $K\subseteq{\mathop H\limits^ \circ}$. Allora $\partial^\text{top}H$ e $K$ sono due compatti disgiunti, dove $\partial^\text{top}$ è il bordo topologico in $X$ (chiusura meno parte interna); dunque ${\displaystyle\inf_{\substack{x\in\partial^\text{top}H,\\y\in K}} k_X(x,y)=\epsilon>0}$.

    Poiché $\sigma_n(0)$ e $\sigma_n(T_n)$ convergono a punti del bordo, per $n$ sufficientemente grande appartengono a $X\setminus H$. Ma $\sigma_n(t_n)\in K\subseteq{\mathop H\limits^ \circ}$; quindi deve esistere $t_n'$ tale che $\sigma_n(t_n')\in\partial^\text{top}H$. Segue che
    \begin{align*}
        0&<\epsilon\le k_X\big(\sigma_n(t_n),\sigma_n(t_n')\big) \le |t_n-t_n'|+1/n\le |0-T_n|+1/n \\
        &\le k_X\big(\sigma_n(0),\sigma_n(T_n)\big)+2/n=k_X(x_n,y_n)+2/n\longrightarrow0,
    \end{align*}
    contraddizione.
\end{proof}

\begin{lm} \label{unaltrosottosotto-bis}
    Sia $X$ una sottovarietà Kobayashi-iperbolica di una varietà complessa $Y$. Supponiamo che $X$ sia relativamente compatta e che esista ${\kappa_0>0}$ tale che $X$ sia $(1,\kappa_0)$-visibile. Sia $F\in\textnormal{Hol}(X,X)$ tale che la successione $\{F^n\}_{n\in\mathbb{N}}$ sia compattamente divergente.
    
    Supponiamo che esistano un compatto $K\subseteq X$, una funzione strettamente crescente $\mu:\mathbb{N}\longrightarrow\mathbb{N}$ e $\xi\in\partial_YX$ tali che la successione $F^{\mu(j)}$ converge alla costante $\xi$ uniformemente su $K$. Allora la successione $\{F^{\mu(j)}\}_{j\in\mathbb{N}}$ converge alla costante $\xi$ uniformemente su tutti i compatti di $X$.
\end{lm}

\begin{proof}
    Supponiamo per assurdo che la tesi non sia vera; esistono dunque un compatto $H\subseteq X$ e un intorno $U$ di $\xi$ in $\overline{X}$ tali che ${F^{\mu(j)}(H)\not\subseteq U}$ frequentemente. Allora esistono una sottosuccessione $\{j_n\}_{n\in\mathbb{N}}\subseteq\mathbb{N}$ e una successione $\{z_n\}_{n\in\mathbb{N}}\subseteq H$ tali che $F^{\mu(j_n)}(z_n)\not\in U$ per ogni $n\in\mathbb{N}$.
    
    A meno di sottosuccessioni, usando il Lemma \ref{limpt-bis} e la relativa compattezza di $X$, possiamo supporre che $z_n\longrightarrow\tilde{z}\in H$, che ${F^{\mu(j_n)}(z)\longrightarrow\tilde{\xi}\in\overline{X}}$ per ogni $z\in X$ e che $F^{\mu(j_n)}(z_n)\longrightarrow\xi'\in\overline{X}$. Dato che $\{F^n\}_{n\in\mathbb{N}}$ è compattamente divergente, dev'essere $\xi'\in\partial_YX$. Siccome $F^{\mu(j_n)}(z)\longrightarrow\xi$ per ogni $z\in K$, dev'essere $\tilde{\xi}=\xi$. Visto che $F^{\mu(j_n)}(z_n)\not\in U$ per ogni $n\in\mathbb{N}$, dev'essere $\xi\not=\xi'$.

    Adesso notiamo che per la Proposizione \ref{semicontr} si ha
    $$k_X\big(F^{\mu(j_n)}(z_n),F^{\mu(j_n)}(\tilde{z})\big)\le k_X(z_n,\tilde{z})\longrightarrow0$$
    per $n\longrightarrow+\infty$. Basta allora applicare il Lemma \ref{nonazeroplease-bis} con $x_n=F^{\mu(j_n)}(z_n)$ e $y_n=F^{\mu(j_n)}(\tilde{z})$ per ottenere una contraddizione.
\end{proof}

\begin{prop} \label{estraiunasotto-bis}
    Sia $X$ una sottovarietà Kobayashi-iperbolica di una varietà complessa $Y$. Supponiamo che $X$ sia relativamente compatta e che esista $\kappa_0>0$ tale che $X$ sia $(1,\kappa_0)$-visibile. Sia $F\in\textnormal{Hol}(X,X)$ tale che la successione $\{F^n\}_{n\in\mathbb{N}}$ sia compattamente divergente.

    Per ogni funzione strettamente crescente $\mu:\mathbb{N}\longrightarrow\mathbb{N}$ esistono $\xi\in\partial_YX$ e una sottosuccessione $\{j_n\}_{n\in\mathbb{N}}\subseteq\mathbb{N}$ tale che la successione $\{F^{\mu(j_n)}\}_{n\in\mathbb{N}}$ converge alla costante $\xi$ uniformemente su tutti i compatti di $X$.
\end{prop}

\begin{proof}
    Fissiamo $z_0\in X$. Per la relativa compattezza di $X$ e la divergenza dai compatti di $\{F^n\}_{n\in\mathbb{N}}$, esistono $\xi\in\partial_YX$ e una sottosuccessione $\{j_n\}_{n\in\mathbb{N}}\subseteq\mathbb{N}$ tali che $F^{\mu(j_n)}(z_0)\longrightarrow\xi$. Allora la successione $\{F^{\mu(j_n)}\}_{n\in\mathbb{N}}$ converge alla costante $\xi$ uniformemente sul compatto $\{z_0\}$. Si conclude applicando il Lemma \ref{unaltrosottosotto-bis}.
\end{proof}