Quello che andremo a dimostrare adesso è uno dei fatti cruciali per ottenere il teorema di tipo ``Wolff-Denjoy''. Esso afferma che, sotto condizioni di visibilità per le simil-geodetiche, le sottosuccessioni di iterate di una funzione olomorfa che ``tendono a infinito'' convergono tutte, puntualmente, a un unico punto del bordo.

\begin{prop} \label{inf_impl_ugu}
    Sia $X$ una sottovarietà complessa, connessa e relativamente compatta di una varietà Kobayashi-iperbolica $Y$. Supponiamo che esista un $\kappa_0>0$ tale che $X$ sia $(1,\kappa_0)$-visibile. Data una funzione $F\in\textnormal{Hol}(X,X)$, esiste $\xi\in\partial_YX$ tale che per ogni funzione $\mu:\mathbb{N}\longrightarrow\mathbb{N}$ strettamente crescente per cui esiste $y_0 \in X$ tale che
    \begin{equation}
        \lim_{j\longrightarrow+\infty} k_X\big(F^{\mu(j)}(y_0),y_0\big)=+\infty
    \end{equation}
    si ha
    \begin{equation}
        \lim_{j\longrightarrow+\infty} F^{\mu(j)}(z)=\xi
    \end{equation}
    per ogni $z \in X$.
\end{prop}

\begin{proof}
    Se $\displaystyle\limsup_{n\longrightarrow+\infty}k_X\big(F^n(x),x\big)<+\infty$ per ogni $x\in X$ l'affermazione è vera a vuoto per ogni $\xi\in\partial_YX$. Altrimenti, scelto $x_0\in X$ tale che $\displaystyle\limsup_{n\longrightarrow+\infty}k_X\big(F^n(x_0),x_0\big)=+\infty$ possiamo costruire una funzione strettamente crescente $\nu:\mathbb{N}\longrightarrow\mathbb{N}$ tale che:
    \begin{itemize}
        \item si ha $k_X\big(F^{\nu(j)}(x_0),x_0\big) \ge k_X\big(F^k(x_0),x_0\big)$ per ogni $j \in \mathbb{N}$ e per ogni $k \le \nu(j)$;
        \item si ha $\displaystyle\lim_{j\longrightarrow+\infty}k_X\big(F^{\nu(j)}(x_0),x_0\big)=+\infty$;
        \item la successione $\{F^{\nu(j)}(x_0)\}_{j\in\mathbb{N}}$ converge a un certo $\xi\in\partial_YX$.
    \end{itemize}
    
    Infatti, definendo induttivamente $l_0=0$ e $l_n$ come il minimo numero naturale $h>l_{n-1}$ tale che $k_X\big(F^h(x_0),x_0\big) \ge \max\left\{n,k_X\big(F^{l_{n-1}}(x_0),x_0\big)\right\}$, abbiamo $k_X\big(F^{l_n}(x_0),x_0\big) \ge k_X\big(F^k(x_0),x_0\big)$ per ogni $n$ e per ogni $k \le l_n$; inoltre, abbiamo $\displaystyle\lim_{n\longrightarrow+\infty}k_X\big(F^{l_n}(x_0),x_0\big)=+\infty$. Per relativa compattezza di $X$, esiste un'ulteriore sottosuccessione $l_{n_j}$ tale che $F^{l_{n_j}}(x_0)$ converge a un certo $\xi\in\overline{X}$, e in realtà $\xi\in\partial_YX$ visto che la distanza di Kobayashi da $x_0$ tende a $+\infty$; basta allora porre $\nu(j)=l_{n_j}$.\\

    Vogliamo ora mostrare la seguente asserzione.

    Siano $\{m_j\}_{j\in\mathbb{N}}$ e $\{m'_j\}_{j\in\mathbb{N}}$ due successioni strettamente crescenti di numeri naturali e $z_0,z'_0\in X$ tali che:
    \begin{enumerate}[label={(\arabic*)}]
        \item per ogni $j\in\mathbb{N}$ si ha $m_j \ge m'_j$;
        \item per ogni $j\in\mathbb{N}$ e $k \le m_j$ si ha $k_X\big(F^{m_j}(z_0),z_0\big) \ge k_X\big(F^k(z_0),z_0\big)$;
        \item si ha $\displaystyle\lim_{j\longrightarrow+\infty}k_X\big(F^{m_j}(z_0),z_0\big)=\lim_{j\longrightarrow+\infty}k_X\big(F^{m'_j}(z'_0),z_0\big)=+\infty$;
        \item le successioni $\{F^{m_j}(z_0)\}_{j\in\mathbb{N}}$ e $\{F^{m'_j}(z'_0)\}_{j\in\mathbb{N}}$ convergono, rispettivamente, a $\zeta$ e $\zeta'$ in $\partial_YX$;
    \end{enumerate}
    allora $\zeta=\zeta'$.

    Supponiamo per assurdo che $\zeta\not=\zeta'$. Per il Corollario \ref{kobisdist} anche $X$ è Kobayashi-iperbolica; quindi grazie al Teorema \ref{similgeo_conn} possiamo scegliere, per ogni $j\in\mathbb{N}$, una $(1,\kappa_0)$-simil-geodetica $\sigma_j:[0,T_j] \longrightarrow X$ tale che $\sigma_j(0)=F^{m_j}(z_0)$ e $\sigma_j(T_j)=F^{m'_j}(z'_0)$. Adesso, dato che abbiamo assunto che $\{F^{m_j}(z_0)\}_{j\in\mathbb{N}}$ e $\{F^{m'_j}(z'_0)\}_{j\in\mathbb{N}}$ convergano a due punti di $\partial_YX$ distinti e $X$ ha la visibilità rispetto alle $(1,\kappa_0)$-simil-geodetiche, esistono una costante $0<R<+\infty$ e, per ogni $j\in\mathbb{N}$, un $t_j\in[0,T_j]$ tali che $k_X\big(z_0,\sigma_j(t_j)\big)<R$. Per il Lemma \ref{quasiovvio} si ha dunque che
    \begin{equation} \label{3k_02R}
        \begin{aligned}
            k_X\big(F^{m_j}(z_0),F^{m'_j}(z'_0)\big) &\ge k_X\big(F^{m_j}(z_0),\sigma_j(t_j)\big)+k_X\big(\sigma_j(t_j),F^{m'_j}(z'_0)\big)-3\kappa_0\\
            &\ge k_X\big(F^{m_j}(z_0),z_0\big)-k_X\big(z_0,\sigma_j(t_j)\big)\\
            &\quad+k_X\big(F^{m'_j}(z'_0),z_0\big)-k_X\big(z_0,\sigma_j(t_j)\big)-3\kappa_0\\
            &\ge k_X\big(F^{m_j}(z_0),z_0\big)+k_X\big(F^{m'_j}(z'_0),z_0\big)-3\kappa_0-2R;
        \end{aligned}
    \end{equation}
    d'altra parte, abbiamo anche che
    \begin{equation} \label{-3k_02R}
        \begin{aligned}
            k_X\big(F^{m_j}(z_0),F^{m'_j}(z'_0)\big) &\le k_X\big(F^{m_j-m'_j}(z_0),z'_0\big) \\
            &\le k_X\big(F^{m_j-m'_j}(z_0),z_0\big)+k_X(z_0,z'_0) \\
            &\le k_X\big(F^{m_j}(z_0),z_0\big)+k_X(z_0,z'_0),
        \end{aligned}
    \end{equation}
    dove per la prima e la terza disuguaglianza abbiamo usato, rispettivamente, le condizioni (1) e (2) sulle successioni $\{m_j\}_{j\in\mathbb{N}}$ e $\{m'_j\}_{j\in\mathbb{N}}$; nella prima, abbiamo anche usato che le funzioni olomorfe sono semicontrazioni rispetto a $k_X$. Concatenando la \eqref{3k_02R} e la \eqref{-3k_02R} e riarrangiando i termini, otteniamo
    $$k_X\big(F^{m'_j}(z'_0),z_0\big) \le k_X(z_0,z'_0)+3\kappa_0+2R,$$
    che è in contraddizione con la condizione (3).\\

    Adesso che l'asserzione è stata dimostrata, possiamo concludere la dimostrazione. Usando la disuguaglianza triangolare e il fatto che le funzioni olomorfe sono semicontrazioni rispetto a $k_X$, troviamo che
    \begin{gather*}
        k_X\big(F^{\mu(j)}(z),z'\big) \ge k_X\big(F^{\mu(j)}(y_0),y_0\big)-k_X(y_0,z)-k_X(z',y_0);
    \end{gather*}
    segue che $\displaystyle\lim_{j\longrightarrow+\infty}k_X\big(F^{\mu(j)}(z),z'\big)=+\infty$ per ogni $z,z' \in X$.

     Fissiamo ora uno $z\in X$, e prendiamo $\xi'$ punto limite di $\{F^{\mu(j)}(z)\}_{j\in\mathbb{N}}$. Allora deve esistere una funzione strettamente crescente $\tau:\mathbb{N}\longrightarrow\mathbb{N}$ tale che $\displaystyle \lim_{j\longrightarrow+\infty} F^{(\mu\circ\tau)(j)}(z)=\xi'$; inoltre, poiché $\displaystyle\lim_{j\longrightarrow+\infty}k_X\big(F^{\mu(j)}(z),z\big)=+\infty$, dev'essere $\xi'\in\partial_YX$. Scegliamo una funzione strettamente crescente $\tau':\mathbb{N}\longrightarrow\mathbb{N}$ tale che $\nu\circ\tau' \ge \mu\circ\tau$ e applichiamo l'asserzione dimostrata sopra alle successioni $m_j=(\nu\circ\tau')(j), m'_j=(\mu\circ\tau)(j)$ e ai punti $z_0=x_0,z'_0=z$, per i quali si verificano facilmente le condizioni (1), (2), (3) e (4). Troviamo così $\xi'=\xi$ e si conclude grazie al Lemma \ref{lemmasottosotto1}.
\end{proof}

Anche il seguente teorema ci aiuterà nella nostra dimostrazione. Esso afferma che, sotto condizioni di visibilità per le simil-geodetiche, le successioni di funzioni olomorfe che convergono uniformemente sui compatti di $X$ devono necessariamente convergere a una costante.

\begin{thm} \label{lim_is_const}
    Sia $X$ una sottovarietà complessa e connessa di una varietà Kobayashi-iperbolica $Y$. Supponiamo che esista un $\kappa_0>0$ tale che $X$ sia $(1,\kappa_0)$-visibile. Consideriamo una successione $\{F_\nu\}_{\nu\in\mathbb{N}} \subseteq \textnormal{Hol}(X,X)$ che converge uniformemente sui compatti di $X$ a una $\psi\in\textnormal{Hol}(X,Y)$ con $\psi(X)\subseteq\partial_YX$. Allora $\psi$ è costante.
\end{thm}

\begin{proof}
    Supponiamo, per assurdo, che $\psi$ non sia costante. Allora, dati $x \in X$ e $\epsilon>0$, la restrizione di $\psi$ a $B_X(x,\epsilon)$, la palla aperta rispetto alla distanza di Kobayashi di centro $x$ e raggio $\epsilon$, non è costante. Infatti, tale palla è un aperto di $X$, e se $\psi$ fosse costante su di essa lo sarebbe su tutta $X$ per il principio di identità delle funzioni olomorfe, poiché $X$ è connessa. Questa, però, sarebbe una contraddizione alla nostra assunzione.

    Fissiamo $x_0 \in X$ e poniamo $\epsilon_0=\kappa_0/3$. Per quanto detto, deve esistere un $x_1 \in B_X(x_0,\epsilon_0)$ tale che $\{F_\nu(x_0)\}_{\nu\in\mathbb{N}}$ e $\{F_\nu(x_1)\}_{\nu\in\mathbb{N}}$ convergono, rispettivamente, a $\xi$ e $\eta$, con $\xi,\eta \in \partial_YX$ e $\xi\not=\eta$. Consideriamo adesso una $(1,\kappa_0/3)$-simil-geodetica $\gamma:[a,b]\longrightarrow X$ tale che $\gamma(a)=x_0$ e $\gamma(b)=x_1$, che esiste per il Teorema \ref{similgeo_conn}; da \eqref{simil-geo1} nella definizione di simil-geodetica si ha che
    $$|a-b|-\kappa_0/3 \le k_X(x_0,x_1) \implies |a-b| \le k_X(x_0,x_1)+\kappa_0/3 \le 2\kappa_0/3.$$

    Per ogni $\nu\in\mathbb{N}$, definiamo $\sigma_\nu:[a,b]\longrightarrow X$ come $\sigma_\nu=F_\nu\circ\gamma$. Mostriamo che $\sigma_\nu$ è una $(1,\kappa_0)$-simil-geodetica per ogni $\nu\in\mathbb{N}$. Per ogni $s,t \in [a,b]$ si ha
    \begin{align*}k_X\big(\sigma_\nu(s),\sigma_\nu(t)\big)&=k_X\Big(F_\nu\big(\gamma(s)\big),F_\nu\big(\gamma(t)\big)\Big)\\
        &\le k_X\big(\gamma(s),\gamma(t)\big) \le |s-t|+\kappa_0/3 \le |s-t|+\kappa_0,
    \end{align*}
    dove abbiamo usato che le funzioni olomorfe sono delle semicontrazioni rispetto alla distanza di Kobayashi e, dato che $\gamma$ è una $(1,\kappa_0/3)$-simil-geodetica, la \eqref{simil-geo1}. Inoltre, si ha che $|s-t|-\kappa_0 \le |a-b|-\kappa_0 \le -\kappa_0/3<0$ per ogni $s,t \in [a,b]$; dunque
    \begin{equation} \label{snu1}
        |s-t|-\kappa_0 \le k_X\big(\sigma_\nu(s),\sigma_\nu(t)\big) \le |s-t|+\kappa_0.
    \end{equation}

    Infine, per ogni $t$ si ha
    \begin{equation} \label{snu2}
        \begin{aligned}
            K_X\big(\sigma_\nu(t);\sigma_\nu'(t)\big)&=K_X\Big(F_\nu\big(\gamma(s)\big);\diff_{\gamma(t)}F_\nu\big(\gamma'(t)\big)\Big) \\
            &\le K_X\big(\gamma(t);\gamma'(t)\big) \le 1,
        \end{aligned}
    \end{equation}
    dove abbiamo usato la Proposizione \ref{metrdecr} e, visto che $\gamma$ è una $(1,\kappa_0/3)$-simil-geodetica, la \eqref{simil-geo2}. Si ha anche che la curva $\sigma_\nu$ è assolutamente continua rispetto a $d_X$ per ogni $\nu$, poiché composizione di $\gamma$, che è assolutamente continua per definizione di $(1,\kappa_0/3)$-simil-geodetica, e $F_\nu$, che è olomorfa e dunque lipschitziana (sempre rispetto a $d_X$) sul compatto $\gamma([a,b])$. Grazie alla \eqref{snu1} e alla \eqref{snu2}, possiamo concludere che $\sigma_\nu$ è una $(1,\kappa_0)$-simil-geodetica per ogni $\nu$.
    
    Adesso, poiché $\{F_\nu\}_{\nu\in\mathbb{N}}$ converge uniformemente sui compatti di $X$ a una funzione olomorfa $\psi$ a valori in $\partial_YX$, ne consegue che per ogni compatto $K\subseteq X$ esiste un $\nu(K)\in\mathbb{N}$ tale che, per ogni $\nu\ge\nu(K)$, si ha $\sigma_\nu([a,b])\cap K=\emptyset$. Ma così troviamo una contraddizione, perché $X$ è $(1,\kappa_0)$-visibile.
\end{proof}

Concludiamo questa sottosezione con un risultato sulla convergenza uniforme sui compatti.

\begin{lm} \label{lemmasottosotto2}
    Sia $X$ una sottovarietà complessa, connessa e tautly embedded di una varietà Kobayashi-iperbolica $Y$. Supponiamo che esista un $\kappa_0>0$ tale che $X$ sia $(1,\kappa_0)$-visibile. Consideriamo una $F\in\textnormal{Hol}(X,X)$ tale che la successione delle iterate di $F$ è compattamente divergente, e supponiamo che esistano un compatto $K$ di $X$, una funzione strettamente crescente $\mu:\mathbb{N}\longrightarrow\mathbb{N}$ e $\xi\in\partial_YX$ tali che la successione $\{F^{\mu(j)}\}_{j\in\mathbb{N}}$ converga alla costante $\xi$ uniformemente su $K$; allora converge alla costante $\xi$ uniformemente su tutti i compatti di $X$.
\end{lm}

\begin{proof}
    Supponiamo per assurdo che non valga la tesi; allora esistono un compatto $H\subseteq X$, un $\epsilon>0$, una successione di numeri naturali $\{j_n\}_{n\in\mathbb{N}}$ e una successione $\{z_n\}_{n\in\mathbb{N}}$ con $z_n\in H$ per ogni $n\in\mathbb{N}$ tali che $d_Y\big(F^{\mu(j_n)}(z_n),\xi\big)\ge\epsilon$ per ogni $n\in\mathbb{N}$. Per la Proposizione \ref{montel} $\text{Hol}(X,X)$ è relativamente compatto in $\text{Hol}(X,Y)$; dalla dimostrazione della suddetta Proposizione segue anche che $C^0(X,Y)$ è metrizzabile, per cui $\text{Hol}(X,X)$ è relativamente compatto per successioni in $\text{Hol}(X,Y)$. Possiamo allora trovare una sottosuccessione $\{j_{n_m}\}_{m\in\mathbb{N}}$ tale che la successione $\{F^{\mu(j_{n_m})}\}_{m\in\mathbb{N}}$ converge uniformemente sui compatti di $X$ a una funzione olomorfa $\tilde{F}:X \rightarrow Y$ con $\tilde{F}(X)\subseteq\overline{X}$. Poiché la successione delle iterate di $F$ è compattamente divergente, dev'essere $\tilde{F}(X)\subseteq\partial_YX$; allora per il Teorema \ref{lim_is_const} $\tilde{F}$ è costante. Siccome su $K$ la successione $\{F^{\mu(j)}\}_{j\in\mathbb{N}}$ converge alla costante $\xi$, la funzione $\tilde{F}$ dev'essere proprio tale costante; ma per costruzione la convergenza non può essere uniforme sul compatto $H$, assurdo.
\end{proof}