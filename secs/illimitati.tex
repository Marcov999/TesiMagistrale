Vediamo adesso i risultati del preprint \cite{BZ2} sui domini illimitati; continuando quanto fatto finora, punteremo a generalizzare questi risultati,\marginpar{Forse dovrei chiamare la sezione ``Sottovarietà non relativamente compatte''?} arrivando a un teorema di tipo ``Wolff-Denjoy'' per sottovarietà taut e con visibilità, ma non necessariamente relativamente compatte.

Come si può facilmente vedere pensando all'esempio della mappa $z\longmapsto z+1$ nel semipiano superiore $\mathbb{H}=\{z\in\mathbb{C}\mid\mathfrak{Im}z>0\}$, restringerci ai punti del bordo non sarà sufficiente. In questo caso, grazie a un biolomorfismo con $\mathbb{D}$ ci possiamo ricondurre al Teorema di Wolff-Denjoy originale; troviamo così che il limite è un punto di $\partial\mathbb{D}$ nel quale non possiamo estendere il biolomorfismo. È dunque chiaro che i risultati che vogliamo andare a studiare dipendono da come il dominio o la sottovarietà si immergano nella varietà ambiente, e in generale non possiamo aspettarci di avere sempre un bordo sufficientemente ricco per descrivere la dinamica delle iterate di funzioni olomorfe.

Dobbiamo dunque estendere la nostra varietà come spazio topologico. Il modo giusto di farlo per ritrovare un teorema di tipo ``Wolff-Denjoy'' è la end compactification. Il concetto fondamentale per definire la end compactification è quello di end, definito da Freudenthal in \cite{F}. Diamo la definizione data in \cite[Chapter 1, Problem 19]{Sp}.

\begin{defn} \label{end}
    Sia $X$ uno spazio topologico non compatto. Una \textit{end} di $X$ è una funzione $e$ con dominio $\{K\subseteq X\mid K\text{ è comatto}\}$ tale che:
    \begin{nlist}
        \item a ogni compatto $K\subseteq X$ associa una componente connessa non vuota di $X\setminus K$;
        \item per ogni coppia di compatti $K_1\subseteq K_2\subseteq X$ si ha $e(K_2)\subseteq e(K_1)$.
    \end{nlist}
    Indichiamo con $\mathcal{E}(X)$ l'insieme di tutte le end di $X$.
\end{defn}

\begin{oss} \label{endnonrelcpt}
    Dato un compatto $K\subseteq X$, se una componente connessa $C$ di $X\setminus K$ è relativamente compatta in $X$, non si potrà mai avere $e(K)=C$; altrimenti, non potrebbe essere soddisfatta la condizione (ii) nella Definizione \ref{end} con $K$ e $K\cup\overline{C}$.
\end{oss}

Intuitivamente, una end è un modo di scegliere, andando all'infinito, ``da che parte andare''. Un esempio semplice per capire quest'interpretazione è l'albero binario infinito (lo si può pensare come un sottospazio di $\mathbb{R}^2$), dove a ogni livello ci troviamo in un nodo e abbiamo due possibili strade tra cui scegliere; poiché i livelli sono numerabili, è facile vedere che la cardinalità delle end è quella del continuo.

\begin{oss}
    Supponiamo che $X$ ammetta un'esaustione in compatti, cioè che esista una successione $\{K_n\}_{n\in\mathbb{N}}$ tale che $K_n\subseteq{\mathop K\limits^ \circ}_{n+1}$ per ogni $n$ e che $\displaystyle\bigcup_{n=1}^{+\infty} K_n=X$; nei casi che studieremo $X$ sarà sempre una varietà connessa, per cui ammetterà un'esaustione in compatti. Allora una end $e$ è univocamente detereminata dalle sue immagini sui $K_n$ (segue facilmente dalle proprietà della end e da quelle dell'esaustione). Quindi, nel seguito, ci basterà fissare un'esaustione in compatti e lavorare con quella.
\end{oss}

Vogliamo ora mettere una topologia su $X^\mathcal{E}:=X\cup\mathcal{E}(X)$ che lo renda uno spazio compatto. Sebbene le end sono state definite con questo scopo in mente, sono comunque necessarie delle ipotesi. Si tratta di ipotesi che in generale sono soddisfatte da varietà astratte, ma poiché ci interesserà compattificare la chiusura di una sottovarietà dovremo prestare attenzione a un'ipotesi in particolare; si vedano l'Osservazione \ref{servelocconn} e l'Esempio \ref{servelocconnex}.

\begin{prop} \label{endiscpt}
    (\cite[Chapter 1, Problem 19]{Sp})\marginpar{Di nuovo cito un esercizio... quando avevo cercato la dim., l'unica che avevo trovato era online; può darsi che si trovi in \cite{F}, ma non so il tedesco} Sia $X$ uno spazio topologico connesso, localmente connesso, compatto e di Hausdorff. Mettiamo su $X^\mathcal{E}$ la topologia generata dalla topologia di $X$ e dai seguenti intorni per $e\in\mathcal{E}(X)$ al variare di $K\subseteq X$ compatto:
    $$N_K(e)=e(K)\cup\{f\in\mathcal{E}(X)\mid f(K)=e(K)\};$$
    allora $X\cup\mathcal{E}(X)$ è uno spazio compatto e di Hausdorff.
\end{prop}

\begin{oss} \label{endiscptsucc}
    Se $X$ ammette un'esaustione in compatti $\{K_n\}_{n\in\mathbb{N}}$, ponendo $U_n^e=N_{K_n}(e)$ si ha che $\{U_n^e\}_{n\in\mathbb{N}}$ è un sistema fondamentale di intorni per $e$. Allora, se è anche primo numerabile e soddisfa le ipotesi della Proposizione \ref{endiscpt}, si ha che $X^\mathcal{E}$ è compatto e primo numerabile, per cui è compatto per successioni.
\end{oss}

\begin{oss} \label{servelocconn}
    Come già accennato, data $X$ sottovarietà connessa di una varietà $Y$, andremo a studiare $\overline{X}^\mathcal{E}$. In tal caso, è facile verificare tutte le ipotesi della Proposizione \ref{endiscpt} tranne una: la locale connessione. Il motivo è perché in generale non è vera, come vedremo nell'esempio seguente.
\end{oss}

\begin{ex} \label{servelocconnex}
    Si consideri l'embedding di $(0,1)$ in $\mathbb{R}^2$ dato dalla funzione $x\longmapsto \sin(1/x)/x$. A meno di considerarne un ``ispessimento'' che diventa sempre più piccolo al tendere di $x$ a $0$, possiamo anche renderlo un dominio proprio e semplicemente connesso $\Omega$ di $\mathbb{C}$ (dunque biolomorfo a $\mathbb{D}$) avente bordo liscio al di fuori dell'asse immaginario.

    Il sottospazio $\overline{\Omega}$ non è localmente connesso. Ragionando come nell'Osservazione \ref{endiscptsucc}, $\overline{\Omega}^\mathcal{E}$ è primo numerabile. Per vedere che non vale la Proposizione \ref{endiscpt}, ci basta dunque vedere che non è compatto per successioni. Consideriamo allora una successione di punti contenuti nelle ``gobbe''. Questa chiaramente non ammette sottosuccessioni convergenti in $\overline{\Omega}$. Tuttavia, per l'Osservazione \ref{endnonrelcpt} non ammette nemmeno sottosuccessioni convergenti a un punto di $\mathcal{E}(\overline{\Omega})$; infatti, poiché ogni esaustione in compatti prima o poi dovrà coprire il compatto $\overline{\Omega}\cap[0,1]$, le uniche componenti connesse non relativamente compatte dei complementari dei compatti saranno, definitivamente, due semirette dell'asse immaginario. È chiaro però che nessun punto della successione vi può appartenere, ma dato che queste semirette, unite alle opportune end, formano un sistema fondamentale di intorni per le end stesse, segue che non ci sono nemmeno sottosuccessioni convergenti a una end.
\end{ex}

Possiamo ora estendere il concetto di visibilità a sottovarietà connesse non relativamente compatte.

\begin{defn} \label{visibility}
    Sia $X$ una sottovarietà complessa e connessa di una varietà complessa $Y$. Supponiamo che $\overline{X}$ sia localmente connessa e definiamo il \textit{bordo end} come ${\partial^\mathcal{E}X:=\partial_YX\cup\mathcal{E}(X)}$. Fissiamo $\lambda \ge 1$ e $\kappa \ge 0$; diciamo che $X$ è \textit{$(\lambda,\kappa)$-ultravisibile} se:
    \begin{enumerate}
        \item ogni due punti distinti di $X$ possono essere collegati da una $(\lambda,\kappa)$-simil-geodetica;
        \item per ogni coppia di punti $p,q\in\partial^\mathcal{E}X$ con $p\not=q$, esistono in $\overline{X}^\mathcal{E}$ due intorni $V$ e $W$, di $p$ e $q$ rispettivamente, con chiusura disgiunta, e un compatto $K$ di $X$ tali che  ogni $(\lambda,\kappa)$-simil-geodetica in $X$ che collega un punto di $V$ a un punto di $W$ interseca $K$.
    \end{enumerate}
\end{defn}

\begin{oss}
    Non escludiamo la possibilità che l'ipotesi che $\overline{X}$ sia localmente connessa sia ridondante, perché potrebbe seguire in qualche modo dalla condizione di ultravisibilità (per esempio, quest'ultima potrebbe implicare una qualche regolarità per $\partial_YX$). Non sappiamo se sia effettivamente vero, e in caso lo sia crediamo che la dimostrazione sia tutt'altro che semplice.
    
    Facciamo però notare che per l'Esempio \ref{servelocconnex}, sfruttando un biolomorfismo\marginpar{Esiste?} con $\mathbb{D}$ esteso ai punti regolari del bordo, si può mostrare che non vale la condizione di ultravisibilità. Esistono\marginpar{Serve una reference} anche esempi, sempre ispirati al seno del topologo, di domini limitati di $\mathbb{C}$ biolomorfi a $\mathbb{D}$ per i quali non vale l'analogo del teorema di Wolff-Denjoy; dal Corollario \ref{wd_type} segue che non soddisfano la condizione di visibilità, in quanto è facile verificare che soddisfano tutte le altre ipotesi.
\end{oss}

Adesso vogliamo dimostrare un teorema di tipo ``Wolff-Denjoy'' per sottovarietà Kobayashi-iperboliche, taut e $(1,\kappa_0)$-ultravisibili. Per farlo, in alcuni punti del nostro ragionamento riadatteremo le dimostrazioni degli enunciati visti nella sezione \ref{Un teorema di tipo ``Wolff-Denjoy'' per varietà taut con visibilità}, ma in altri dovremo ricavarci nuovi risultati.

\begin{lm} \label{limpt}
    Sia $X$ una sottovarietà complessa e connessa di una varietà complessa $Y$. Supponiamo che $\overline{X}$ sia localmente connessa e che esista $\kappa_0>0$ tale che $X$ sia $(1,\kappa_0)$-ultravisibile. Siano $Z$ una varietà Kobayashi-iperbolica e $\{f_n\}_{n\in\mathbb{N}}\subseteq\textnormal{Hol}(Z,X)$ una successione compattamente divergente. Allora esistono $\xi\in\partial^\mathcal{E}X$ e una sottosuccessione $\{f_{n_j}\}_{j\in\mathbb{N}}$ tali che $f_{n_j}(z)\longrightarrow\xi$ per ogni $z\in Z$.
\end{lm}

\begin{proof}
    Sia $z_0\in Z$. Per l'Osservazione \ref{endiscptsucc}, a meno di sottosuccessioni possiamo supporre che esista $\xi_0\in\partial^\mathcal{E}X$ tale che $f_n(z_0)\longrightarrow\xi_0$. Supponiamo per assurdo che esista $z_1\in Z$ tale che la successione $\{f_n(z_1)\}_{n\in\mathbb{N}}$ non converge a $\xi_0$. Considerando una curva continua che collega $z_0$ a $z_1$, si trova facilmente che possiamo eventualmente sostituirli con due punti tali che $k_Z(z_0,z_1)<\kappa_0/2$. Di nuovo a meno di sottosuccessioni, possiamo supporre che esista $\xi_1\in\partial^\mathcal{E}X$ tale che $\xi_1\not=\xi_0$ e $f_n(z_1)\longrightarrow\xi_1$.

    Per il Teorema \ref{similgeo_conn} esiste una $(1,\kappa_0/2)$-simil-geodetica $\sigma:[0,T]\longrightarrow Z$ tale che $\sigma(0)=z_0$ e $\sigma(T)=z_1$. Poniamo $\sigma_n:=f_n\circ\sigma$, vogliamo dimostrare che $\sigma_n$ è una $(1,\kappa_0)$-simil-geodetica per ogni $n$. Per la Proposizione \ref{semicontr} e per la definizione di $(1,\kappa_0/2)$-simil-geodetica, abbiamo che
    $$k_X\big(\sigma_n(s),\sigma_n(t)\big) \le k_Z\big(\sigma(s),\sigma(t)\big) \le |t-s|+\kappa_0/2\le |t-s|+\kappa_0$$
    per ogni $s,t\in[0,T]$. Inoltre, sempre dalla definizione di di $(1,\kappa_0/2)$-simil-geodetica si ha che
    \begin{align*}
        |t-s|-\kappa_0 &\le |0-T|-\kappa_0/2-\kappa_0/2 \le k_Z\big(\sigma(0),\sigma(T)\big)-\kappa_0/2\\
        &=k_Z(z_0,z_1)-\kappa_0/2<0\le k_X\big(\sigma_n(s),\sigma_n(t)\big)
    \end{align*}
    per ogni $s,t\in[0,T]$. Infine, dalla Proposizione \ref{metrdecr} e dalla definizione di $(1,\kappa_0/2)$-simil-geodetica segue che
    $$K_X\big(\sigma_n(t);\sigma_n'(t)\big)\le K_Z\big(\sigma(t);\sigma'(t)\big) \le 1$$
    per ogni $t\in[0,T]$. Dunque $\sigma_n$ è una $(1,\kappa_0)$-simil-geodetica per ogni $n$, come volevamo.

    Adesso, poiché $\sigma_n(0)=f_n(z_0)\longrightarrow\xi_0$ e $\sigma_n(T)=f_n(z_1)\longrightarrow\xi_1$ e $X$ è $(1,\kappa_0)$-ultravisibile, esiste un compatto $K\subseteq X$ tale che
    $$\emptyset\not=K\cap\sigma_n([0,T])=K\cap f_n\big(\sigma([0,T])\big)$$
    per ogni $n$, in contraddizione con l'ipotesi che la successione $\{f_n\}_{n\in\mathbb{N}}$ sia compattamente divergente.
\end{proof}

\begin{prop}
    Sia $X$ una sottovarietà Kobayashi-iperbolica di una varietà complessa $Y$. Supponiamo che $\overline{X}$ sia localmente connessa e che esista $\kappa_0>0$ tale che $X$ sia $(1,\kappa_0)$-ultravisibile. Sia $F\in\textnormal{Hol}(X,X)$ tale che la successione $\{F^n\}_{n\in\mathbb{N}}$ sia compattamente divergente. Allora esiste $\xi\in\partial^\mathcal{E}X$ tale che per ogni funzione $\mu:\mathbb{N}\longrightarrow\mathbb{N}$ strettamente crescente per cui esiste $y_0 \in X$ tale che
    \begin{equation}
        \lim_{j\longrightarrow+\infty} k_X\big(F^{\mu(j)}(y_0),y_0\big)=+\infty
    \end{equation}
    si ha
    \begin{equation}
        \lim_{j\longrightarrow+\infty} F^{\mu(j)}(z)=\xi
    \end{equation}
    per ogni $z \in X$.
\end{prop}

\begin{proof}
    Si ripete la dimostrazione della Proposizione \ref{inf_impl_ugu} con le seguenti modifiche: $\overline{X}^\mathcal{E}$ e $\partial^\mathcal{E}X$ al posto di $\overline{X}$ e $\partial_YX$; ``per compattezza per successioni di $\overline{X}^\mathcal{E}$'' al posto di ``per relativa compattezza di $X$''; ultravisibilità al posto di visibilità; si utilizza il Lemma \ref{limpt} al posto del Lemma \ref{lemmasottosotto1}.
\end{proof}

\begin{lm} \label{nonazeroplease}
    Sia $X$ una sottovarietà Kobayashi-iperbolica di una varietà complessa $Y$. Supponiamo che esista $\kappa_0>0$ tale che $X$ sia $(1,\kappa_0)$-ultravisibile. Siano $\{x_n\}_{n\in\mathbb{N}}$ e $\{y_n\}_{n\in\mathbb{N}}$ due successioni di punti di $X$ tali che $x_n\longrightarrow\xi$ e $y_n\longrightarrow\xi'$, con $\xi,\xi'\in\partial^\mathcal{E}X$ e $\xi\not=\xi'$.
    
    Allora non può essere che $\displaystyle\lim_{n\longrightarrow+\infty}k_X(x_n,y_n)=0$.
\end{lm}

\begin{proof}
    Supponiamo per assurdo che la tesi sia falsa, e prendiamo due successioni che la contraddicono. Per il Teorema \ref{similgeo_conn} possiamo prendere, per ogni $n\in\mathbb{N}$, una $(1,1/n)$-simil-geodetica $\sigma_n:[0,T_n]\longrightarrow X$ tale che $\sigma_n(0)=x_n$ e ${\sigma_n(T_n)=y_n}$. Per $n$ sufficientemente grande sono tutte $(1,\kappa_0)$-simil-geodetiche, per cui esistono un compatto $K\subseteq X$ e $t_n\in[0,T_n]$ tali che $\sigma_n(t_n)\in K$ per ogni $n$. Dato che $X$ è una varietà, possiamo prendere un compatto ${H\subseteq X}$ tale che $K\subseteq{\mathop H\limits^ \circ}$. Allora $\partial^\text{top}H$ e $K$ sono due compatti disgiunti, dove $\partial^\text{top}$ è il bordo topologico in $X$ (chiusura meno parte interna); dunque ${\displaystyle\inf_{\substack{x\in\partial^\text{top}H,\\y\in K}} k_X(x,y)=\epsilon>0}$.

    Poiché $\sigma_n(0)$ e $\sigma_n(T_n)$ convergono a punti del bordo end, per $n$ sufficientemente grande appartengono a $X\setminus H$. Ma $\sigma_n(t_n)\in K\subseteq{\mathop H\limits^ \circ}$; quindi deve esistere $t_n'$ tale che $\sigma_n(t_n')\in\partial H$. Segue che
    \begin{align*}
        0&<\epsilon\le k_X\big(\sigma_n(t_n),\sigma_n(t_n')\big) \le |t_n-t_n'|+1/n\le |0-T_n|+1/n \\
        &\le k_X\big(\sigma_n(0),\sigma_n(T_n)\big)+2/n=k_X(x_n,y_n)+2/n\longrightarrow0,
    \end{align*}
    contraddizione.
\end{proof}

\begin{lm} \label{unaltrosottosotto}
    Sia $X$ una sottovarietà Kobayashi-iperbolica di una varietà complessa $Y$. Supponiamo che $\overline{X}$ sia localmente connessa e che esista $\kappa_0>0$ tale che $X$ sia $(1,\kappa_0)$-ultravisibile. Sia $F\in\textnormal{Hol}(X,X)$ tale che la successione $\{F^n\}_{n\in\mathbb{N}}$ sia compattamente divergente.
    
    Supponiamo che esistano un compatto $K\subseteq X$, una funzione strettamente crescente $\mu:\mathbb{N}\longrightarrow\mathbb{N}$ e $\xi\in\partial^\mathcal{E}X$ tali che per ogni intorno $U$ di $\xi$ in $\overline{X}^\mathcal{E}$ esiste $j_0$ tale che per ogni $j\ge j_0$ si ha $F^{\mu(j)}(K)\subseteq U$. Allora la successione $\{F^{\mu(j)}\}_{j\in\mathbb{N}}$ converge alla costante $\xi$ in $C^0(X,\overline{X}^\mathcal{E})$.
\end{lm}

\begin{proof}
    Supponiamo per assurdo che la tesi non sia vera; esistono dunque un compatto $H\subseteq X$ e un intorno $U$ di $\xi$ in $\overline{X}^\mathcal{E}$ tali che ${F^{\mu(j)}(H)\not\subseteq U}$ frequentemente. Allora esistono una sottosuccessione $\{j_n\}_{n\in\mathbb{N}}\subseteq\mathbb{N}$ e una successione $\{z_n\}_{n\in\mathbb{N}}\subseteq H$ tali che $F^{\mu(j_n)}(z_n)\not\in U$ per ogni $n\in\mathbb{N}$.
    
    A meno di sottosuccessioni, usando il Lemma \ref{limpt} e la compattezza per successioni di $\overline{X}^\mathcal{E}$, possiamo supporre che $z_n\longrightarrow\tilde{z}\in H$, che ${F^{\mu(j_n)}(z)\longrightarrow\tilde{\xi}\in\overline{X}^\mathcal{E}}$ per ogni $z\in X$ e che $F^{\mu(j_n)}(z_n)\longrightarrow\xi'\in\overline{X}^\mathcal{E}$. Dato che $\{F^n\}_{n\in\mathbb{N}}$ è compattamente divergente, dev'essere $\xi'\in\partial^\mathcal{E}X$. Siccome $F^{\mu(j_n)}(z)\longrightarrow\xi$ per ogni $z\in K$, dev'essere $\tilde{\xi}=\xi$. Visto che $F^{\mu(j_n)}(z_n)\not\in U$ per ogni $n\in\mathbb{N}$, dev'essere $\xi\not=\xi'$.

    Adesso notiamo che per la Proposizione \ref{semicontr} si ha
    $$k_X\big(F^{\mu(j_n)}(z_n),F^{\mu(j_n)}(\tilde{z})\big)\le k_X(z_n,\tilde{z})\longrightarrow0$$
    per $n\longrightarrow+\infty$. Basta allora applicare il Lemma \ref{nonazeroplease}, con $x_n=F^{\mu(j_n)}(z_n)$ e $y_n=F^{\mu(j_n)}(\tilde{z})$, per ottenere una contraddizione.
\end{proof}

\begin{lm} \label{estraiunasotto}
    Sia $X$ una sottovarietà Kobayashi-iperbolica di una varietà complessa $Y$. Supponiamo che $\overline{X}$ sia localmente connessa e che esista $\kappa_0>0$ tale che $X$ sia $(1,\kappa_0)$-ultravisibile. Sia $F\in\textnormal{Hol}(X,X)$ tale che la successione $\{F^n\}_{n\in\mathbb{N}}$ sia compattamente divergente.

    Per ogni funzione strettamente crescente $\mu:\mathbb{N}\longrightarrow\mathbb{N}$ esistono $\xi\in\partial^\mathcal{E}X$ e una sottosuccessione $\{j_n\}_{n\in\mathbb{N}}\subseteq\mathbb{N}$ tale che la successione $\{F^{\mu(j_n)}\}_{n\in\mathbb{N}}$ converge alla costante $\xi$ in $C^0(X,\overline{X}^\mathcal{E})$.
\end{lm}

\begin{proof}
    Fissiamo $z_0\in X$. Per la compattezza per successioni di $\overline{X}^\mathcal{E}$ e la divergenza dai compatti di $\{F^n\}_{n\in\mathbb{N}}$, esistono $\xi\in\partial^\mathcal{E}X$ e una sottosuccessione $\{j_n\}_{n\in\mathbb{N}}\subseteq\mathbb{N}$ tali che $F^{\mu(j_n)}(z_0)\longrightarrow\xi$. Allora la successione $\{F^{\mu(j_n)}\}_{n\in\mathbb{N}}$ converge alla costante $\xi$ uniformemente sul compatto $\{z_0\}$. Si conclude applicando il Lemma \ref{unaltrosottosotto}.
\end{proof}

Siamo ora pronti a dimostrare il teorema di tipo ``Wolff-Denjoy'' nel caso non relativamente compatto.