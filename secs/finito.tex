Vogliamo ora discutere dei domini di tipo finito, che sono un esempio di domini Goldilocks. Iniziamo con la definizione di dominio di tipo finito nel senso di D'Angelo, introdotta in \cite{D'A}.

\begin{defn}
    Dati un aperto $A\subseteq\mathbb{C}^n$, una funzione $g\in C^\infty(A)$ e $p\in A$, \textit{l'ordine (o la molteplicità) di $g$ in $p$} è $v_p(g)$, il grado del primo termine non nullo dello sviluppo di Taylor in $p$ di $g-g(p)$. Se la funzione è a valori in più variabili, si considera il minimo degli ordini delle componenti.
\end{defn}

\begin{defn}
    Un dominio $\Omega\subseteq\mathbb{C}^n$ ha bordo $C^\infty$ se esiste una funzione $\rho\in C^\infty(\mathbb{C}^n)$ tale che $\Omega=\{\rho(z)<0\}$ e $\diff\rho\not=0$ in ogni punto di $\partial\Omega$.
\end{defn}
    
\begin{defn} \label{discanal}
    Siano $\Omega\subseteq\mathbb{C}^n$ un dominio con bordo $C^\infty$ e $\xi\in\partial\Omega$. L'insieme dei \textit{dischi analitici} che toccano il bordo di $\Omega$ in $\xi$ e che sono lisci nell'origine è dato da
    $$\mathcal{D}_{\xi}=\{f\in\text{Hol}(\mathbb{D},\mathbb{C}^n)\mid f(0)=\xi\}.$$
\end{defn}

\begin{defn}
    Siano $\Omega\subseteq\mathbb{C}^n$ un dominio con bordo $C^\infty$ e funzione di definzione $\rho$ e $\xi\in\partial\Omega$. Il \textit{tipo} di $\xi$ è dato da
    $$\Delta_1(\xi):=\sup_{f\in\mathcal{D}_{\xi}} \frac{v_0(\rho\circ f)}{v_0(f)}.$$
\end{defn}

\begin{defn}
    Un dominio $\Omega\subseteq\mathbb{C}^n$ con bordo $C^\infty$ si dice \textit{di tipo finito nel senso di D'Angelo} (o più brevemente \textit{di tipo finito}) se $\Delta_1(\xi)<+\infty$ per ogni $\xi\in\partial\Omega$.
\end{defn}

\begin{oss}
    \begin{enumerate}
        \item Esistono diverse definizioni di dominio di tipo finito, che sono sostanzialmente equivalenti (o quasi) in due variabili o per domini convessi in qualsiasi numero di variabili; negli altri casi, invece, la situazione è più complicata. Noi ci limiteremo al caso di domini di tipo finito nel senso di D'Angelo, poiché è ciò che ci serve per ottenere la stima dal basso sulla metrica di Kobayashi.
        \item È naturale supporre, come faremo a breve, la pseudoconvessità del dominio; infatti, si può mostrare (\cite[Corollary 5.6]{D'A} con $p=p_0$) che in tal caso il tipo del dominio è almeno $2$. Inoltre, i punti strettamente pseudoconvessi hanno tipo esattamente uguale a $2$ (\cite[Corollary 5.8]{D'A}).
    \end{enumerate}
\end{oss}

Ci servirà il seguente fatto.

\begin{thm} \label{cho}
    (\cite[Theorem 1]{Ch}) Sia $\Omega\subseteq\mathbb{C}^n$ un dominio limitato. Siano inoltre $\xi\in\partial\Omega$ di tipo finito e $U$ un intorno di $\xi$ tale che $\partial\Omega\cap U$ è liscio e pseudoconvesso. Allora esistono un intorno $V\subseteq U$ di $\xi$ e due costanti $c,\epsilon>0$ tali che si ha
    $$K_\Omega(z;Z) \ge c\frac{\|Z\|}{\delta_\Omega(z)^\epsilon}$$
    per ogni $z\in\Omega\cap V$ e $Z\in T_z\Omega$.
\end{thm}

Possiamo allora dimostrare che per i domini limitati, pseudoconvessi e di tipo finito vale la condizione (1) nella definizione di dominio Goldilocks.

\begin{cor} \label{finito_is_1}
    (\cite[Lemma 2.6]{BZ1}) Sia $\Omega\subseteq\mathbb{C}^n$ un dominio limitato, pseudoconvesso e di tipo finito. Allora $\Omega$ soddisfa la condizione (1) nella Definizione \ref{gold}.
\end{cor}

\begin{proof}
    Poiché $\Omega$ è limitato, pseudoconvesso e di tipo finito, $\partial\Omega$ è compatto e ogni intorno di ogni suo punto soddisfa le ipotesi del Teorema \ref{cho}. Possiamo allora trovare un numero finito di aperti $V_1,\dots, V_N$ che ricoprono $\partial\Omega$ e delle costanti $c,\epsilon_1,\dots,\epsilon_N>0$ tali che
    $$K_\Omega(x;v) \ge c\cdot\delta_\Omega(z)^{-\epsilon_j}$$
    per ogni $z\in\Omega\cap V_j$ e $v\in T_z\Omega$ con $\|v\|=1$.

    Basta dunque prendere $s=\min\{\epsilon_1,\dots,\epsilon_N\}$ e $r>0$ piccolo abbastanza affinché $r<1$ e $\{z\in\Omega\mid\delta_\Omega(z) \le r\}\subseteq V_1\cup\dots\cup V_N$. Segue che
    $$M_\Omega(r) \le r^s/c$$
    con $s>0$ per $r$ sufficientemente piccolo, per cui la condizione (1) nella Definizione \ref{gold} è soddisfatta.
\end{proof}

Andiamo ora a dimostrare che soddisfano anche la condizione (2). Per questa servono ipotesi meno stringenti.

\begin{prop} \label{finito_is_2}
    (\cite[Proposition 1.2]{A1}) Sia $\Omega\subseteq\mathbb{C}^n$ un dominio limitato con bordo $C^2$. Allora $\Omega$ soddisfa la condizione (2) nella Definizione \ref{gold}.
\end{prop}

\begin{proof}
    Per \cite[Chapter 9, Theorem 20]{Sp}, $\partial\Omega$ ammette un intorno tubolare $U_\epsilon$, con $\epsilon>0$, tale che:
    \begin{nlist}
        \item si ha $U_\epsilon=\{z\in\mathbb{C}^n\mid\delta_\Omega(z)<\epsilon\}$;
        \item per ogni $z\in\Omega\cap U_\epsilon$ esiste un unico punto $\pi(z)\in\partial\Omega$ con $\|\pi(z)-z\|=\delta_\Omega(z)$;
        \item per ogni $z\in\Omega$, la fibra $\pi^{-1}\big(\pi(z)\big)$ è un sottoinsieme della normale a $\partial\Omega$ in $\pi(z)$;
        \item la mappa $z\longmapsto\big(\pi(z),\delta_\Omega(z)\big)$ è un omeomorfismo tra $U_\epsilon$ e $\partial\Omega\times(-\epsilon,\epsilon)$.
    \end{nlist}
    È facile osservare che, per ogni $z\in\Omega$ con $\delta_\Omega(z)=\epsilon$, si ha che la palla euclidea di centro $z$ e raggio $\epsilon$ è tutta contenuta in $\Omega$. Presi allora $p,q\in\Omega$ con $\pi(p)=\pi(q)$ e $\epsilon\ge\delta_\Omega(p)\ge\delta_\Omega(q)$, poniamo $z_0$ il punto (interno a $\Omega$) della normale a $\partial\Omega$ in $\pi(p)$ tale che $\delta_\Omega(z_0)=\epsilon$. Usando la funzione olomorfa $\varphi:\mathbb{D}\longrightarrow\Omega$ data da $\varphi(\zeta)=z_0+\zeta\big(\pi(p)-z_0\big)$, troviamo
    $$k_\Omega(p,q) \le \dfrac{1}{2}\log{\dfrac{\delta_\Omega(p)}{\delta_\Omega(q)}}.$$

    Adesso, notiamo che l'insieme $K=\{z\in\Omega\mid\delta_\Omega(z)\ge\epsilon\}$ è compatto. Dato $x_0\in\Omega$, basta allora prendere $\alpha=1/2$ e $C=\frac{1}{2}\log{\epsilon}+D_{k_{\Omega}}(x_0,K)$, dove abbiamo posto $D_{k_\Omega}(x_0,K)=\max\{k_\Omega(x_0,z)\mid z\in K\}$. Segue dunque che $\Omega$ soddisfa la condizione (2) nella Definizione \ref{gold}.
\end{proof}

Usando il Corollario \ref{finito_is_1}, la Proposizione \ref{finito_is_2} e la Proposizione \ref{psdcvx_is_taut}, otteniamo il seguente risultato.

\begin{cor}
    Sia $\Omega\subseteq\mathbb{C}^n$ un dominio limitato, pseudoconvesso e di tipo finito. Allora $\Omega$ è un dominio Goldilocks e una varietà taut.
    
    In particolare, per il Corollario \ref{gold_is_vis} si ha che $\Omega$ soddisfa le ipotesi del Corollario \ref{wd_type}.
\end{cor}