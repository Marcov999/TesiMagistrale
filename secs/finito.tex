Iniziamo con la definizione di dominio finito nel senso di D'Angelo, introdotta in \cite{D'A}.

\begin{defn}
    \marginpar{Controllare se questa definizione va bene}
    Data una funzione $g\in C^\infty(\mathbb{C}^n)$ e $p\in\mathbb{C}^n$, \textit{l'ordine di contatto di $g$ in $p$} è $v_p(g)$, il grado del primo polinomio non nullo dell'espansione in Taylor in $p$ di $g-g(p)$. Se la funzione è a valori in più variabili, si considera il minimo degli ordini delle componenti.

    Sia $\Omega\subseteq\mathbb{C}^n$ un dominio con bordo $C^\infty$, cioè esiste una funzione $\rho\in C^\infty(\mathbb{C}^n)$ tale che $\Omega=\{\rho(z)<0\}$ e $\diff\rho\not=0$ in ogni punto di $\partial\Omega$. Sia $\textbf{f}$ il germe di una funzione olomorfa a valori in una $1$-varietà algebrica tale che $f(0)=z_0\in\partial\Omega$. Poniamo $\Delta_1(z_0)=\displaystyle\sup_{\textbf{f}} \frac{v_0(\rho\circ f)}{v_0(f)}$.

    La quantità $\Delta_1(z_0)$ è detta il \textit{tipo} di $z_0\in\partial\Omega$. Il dominio $\Omega$ si dice \textit{di tipo finito nel senso di D'Angelo} (o più brevemente \textit{di tipo finito}) se $\Delta_1(z_0)<+\infty$ per ogni $z_0\in\partial\Omega$.
\end{defn}

Ci servirà anche il seguente fatto.

\begin{thm} \label{cho}
    (\cite[Theorem 1]{C}) Sia $\Omega\subseteq\mathbb{C}^n$ un dominio limitato. Sia $z_0\in\partial\Omega$ di tipo finito e $U$ un intorno di $z_0$ tale che $\partial\Omega\cap U$ è liscio e pseudoconvesso. Allora esiste un intorno $V\subseteq U$ di $z_0$ e due costanti $c,\epsilon>0$ tali che si ha
    $$K_\Omega(z;Z) \ge c\frac{\|Z\|}{\delta_\Omega(z)^\epsilon}$$
    per ogni $z\in\Omega$ e $Z\in T_z\Omega$.
\end{thm}

Possiamo allora dimostrare che per i domini limitati, pseudoconvessi e di tipo finito vale la condizione (1) nella definizione di dominio Goldilocks.

\begin{cor}
    Sia $\Omega$ un dominio limitato, pseudoconvesso e di tipo finito. Allora $\Omega$ soddisfa la condizione (1) nella Definizione \ref{gold}.
\end{cor}

\begin{proof}
    Poiché $\Omega$ è limitato, pseudoconvesso e di tipo finito, $\partial\Omega$ è compatto e ogni intorno di ogni suo punto soddisfa le ipotesi del Teorema \ref{cho}. Possiamo allora trovare un numero finito di aperti $V_1,\dots, V_n$ che ricoprono $\partial\Omega$ e delle costanti $c,\epsilon_1,\dots,\epsilon_N>0$ tali che $K_\Omega(x;v) \ge c\cdot\delta_\Omega(z)^{-\epsilon_j}$ per ogni $z\in\Omega\cap V_j$ e $v\in T_z\Omega$ con $\|v\|=1$.

    Basta dunque prendere $s=\min\{\epsilon_1,\dots,\epsilon_N\}$ e $r>0$ piccolo abbastanza affinché $r<1$ e $\{z\in\Omega\mid\delta_\Omega(z) \le r\}\subseteq V_1\cup\dots\cup V_N$. Segue che $M_\Omega(r) \le r^s/c$ con $s>0$ per $r$ sufficientemente piccolo, dunque la condizione (1) nella Definizione \ref{gold} è soddisfatta.
\end{proof}

Andiamo ora a dimostrare che soddisfano anche la condizione (2). Per questa, servono ipotesi meno stringenti.

\begin{prop}
    Sia $\Omega\subseteq\mathbb{C}^n$ limitato con bordo $C^2$. Allora $\Omega$ soddisfa la condizione (2) nella Definizione \ref{gold}.
\end{prop}

\begin{proof}
    Per \cite[Lemma 2.1]{BB}, $\partial\Omega$ ammette un intorno tubolare $U_\epsilon$, con $\epsilon>0$, tale che:
    \begin{nlist}
        \item si ha $U_\epsilon=\{z\in\mathbb{C}^n\mid\delta_\Omega(z)<\epsilon\}$;
        \item per ogni $z\in\Omega\cap U_\epsilon$ esiste un unico punto $\pi(z)\in\partial\Omega$ con $\|\pi(z)-z\|=\delta_\Omega(z)$;
        \item per ogni $z\in\Omega$, la fibra $\pi^{-1}\big(\pi(z)\big)$ è un sottoinsieme della normale a $\partial\Omega$ in $\pi(z)$;
        \item la mappa $z\longmapsto\big(\pi(z),\delta_\Omega(z)\big)$ è un omeomorfismo tra $U_\epsilon$ e $\partial\Omega\times(0,\epsilon)$.
    \end{nlist}
    A meno di prendere $\epsilon$ sufficientemente piccolo, per ogni $z\in\Omega$ con $\delta_\Omega(z)=\epsilon$ si ha che la palla euclidea di centro $z$ e raggio $\epsilon$ è tutta contenuta in $\Omega$. Presi allora $p,q\in\Omega$ con $\pi(p)=\pi(q)$ e $\epsilon\ge\delta_\Omega(p)\ge\delta_\Omega(q)$, poniamo $z_0$ il punto (interno a $\Omega$) della normale a $\partial\Omega$ in $\pi(p)$ tale che $\delta_\Omega(z_0)=\epsilon$. Usando la funzione $\varphi:\mathbb{D}\longrightarrow\Omega$ tale che $\varphi(\zeta)=z_0+\zeta\big(\pi(p)-z_0\big)$, troviamo $k_\Omega(p,q) \le \dfrac{1}{2}\log{\dfrac{\delta_\Omega(p)}{\delta_\Omega(q)}}$.

    Adesso, notiamo che l'insieme $K=\{z\in\Omega\mid\delta_\Omega(z)\ge\epsilon\}$ è compatto. Dato $x_0\in\Omega$, basta allora prendere $\alpha=1/2$ e $C=\frac{1}{2}\log{\epsilon}+\text{diam}_{k_\Omega}(K)$ se $x_0\in K$, altrimenti si prende $C=\max\{\frac{1}{2}\log{\epsilon},D_{k_{\Omega}}(x_0,K)\}$ dove abbiamo posto $D_{k_\Omega}(x_0,K)=\max\{k_\Omega(x_0,z)\mid z\in K\}$. Segue dunque che $\Omega$ soddisfa la condizione (2) nella Definizione \ref{gold}.
\end{proof}