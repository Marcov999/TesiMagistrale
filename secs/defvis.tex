Nella sezione precedente abbiamo visto come l'ipotesi di varietà taut ci permette di dire, se le orbite di una certa funzione non sono relativamente compatte, che la successione delle iterate è compattamente divergente.

Per ottenere un risultato di tipo Wolff-Denjoy, nel caso in cui le iterate siano compattamente divergenti dobbiamo dire due cose: che le iterate convergono uniformemente sui compatti a una funzione a valori nel bordo euclideo, e che in realtà tale funzione è una costante.

Per ottenere la convergenza uniforme al bordo ci basterà supporre che la varietà sia embeddata in un qualche $\mathbb{C}^d$ e limitata, dopodiché si applica il teorema di Montel. Per dire che la funzione è costante, invece, ci serviranno delle ipotesi aggiuntive di tipo geometrico: la condizione di visibilità per le simil-geodetiche.

\begin{defn}
    Sia $X$ una varietà connessa complessa embeddata in $\mathbb{C}^d$ e limitata, e fissiamo $\lambda \ge 1$ e $\kappa \ge 0$. Sia $I\subseteq \mathbb{R}$ un intervallo; una curva $\sigma:I \longrightarrow X$ è detta una \textit{$(\lambda,\kappa)$-simil-geodetica} se
    \begin{enumerate}
        \item per ogni $s,t \in I$ si ha
        $$\frac{1}{\lambda}|t-s|-\kappa \le k_X\big(\sigma(s),\sigma(t)\big)\le\lambda|t-s|+\kappa;$$
        \item $\sigma$ è assolutamente continua (quindi $\sigma'(t)$ esiste per quasi ogni $t \in I$) e per quasi ogni $t \in I$ si ha
        $$K_X\big(\sigma(t);\sigma'(t)\big) \le \lambda.$$
    \end{enumerate}
\end{defn}

\begin{defn} \label{visibility}
    Sia $X$ una varietà connessa complessa embeddata in $\mathbb{C}^d$ e limitata, e fissiamo $\lambda \ge 1$ e $\kappa \ge 0$. Diciamo che $X$ ha la \textit{condizione di visibilità rispetto alle $(\lambda,\kappa)$-simil-geodetiche} se
    \begin{enumerate}
        \item ogni due punti distinti di $X$ possono essere collegati da una $(\lambda,\kappa)$-simil-geodetica;
        \item per ogni coppia di punti $p,q\in\partial X$ con $p\not=q$, esistono in $\mathbb{C}^d$ due intorni $V$ e $W$ di $p$ e $q$ rispettivamente e un compatto $K$ di $X$ tali che: $\overline{V}\cap\overline{W}=\emptyset$; ogni $(\lambda,\kappa)$-simil-geodetica in $X$ che collega un punto di $V$ a un punto di $W$ interseca $K$.
    \end{enumerate}
\end{defn}

Nel caso di un dominio limitato con bordo regolare, l'ipotesi di essere strettamente pseudoconvesso permetteva di concludere la condizione geometrica di Gromov-iperbolicità. Inoltre, in tal caso il dominio è proprio e completo (si veda \cite[Paragraph 3.3]{G}); dunque per il teorema di Hopf-Rinow (\cite[Part I, Paragraph 3, Proposition 3.7]{BH}) è uno spazio geodetico, ovvero ogni coppia di punti è collegata da una geodetica. Si può dimostrare che gli spazi Gromov-iperbolici, propri e geodetici soddisfano la visibilità sia per le geodetiche che per le simil-geodetiche: per la prima si può ragionare come in \cite[Proposition 2.5]{BNT} usando il fatto che per ogni coppia di punti distinti di $\partial_G X$ esistono due loro intorni disgiunti in $X\cup\partial_GX$ (si veda \cite[Part III, Chapter H, Paragraph 3, Lemma 3.6]{BH}); la seconda segue da \cite[Part III, Chapter H, Paragraph 1, Theorem 1.7]{BH}.

Tuttavia, nella prossima sezione vedremo esempi di domini che soddisfano la condizione di visibilità per le simil-geodetiche ma che non sono Gromov-iperbolici. Segue dunque che i risultati che andremo a dimostrare sono, in un certo senso, più generali. In particolare, il Teorema \ref{abate_wd} sarà un corollario del teorema che dimostreremo. \\

Le simil-geodetiche sono delle curve che, a meno di costanti moltiplicative e additive, si comportano come le geodetiche, cioè le curve che minimizzano la lunghezza.

\marginpar{Qua ci va un bel disegnino in inkscape, sia del caso vietato che del caso previsto dalla definizione}
Quello che chiediamo, euristicamente, nella Definizione \ref{visibility} è che, se vogliamo andare da un punto a un altro del bordo con tali curve, allora non possiamo stare arbitrariamente vicini al bordo, ma siamo costretti a ``piegarci'' verso l'interno; in pratica, stiamo chiedendo che ci sia una sorta di curvatura negativa. \\

\begin{ftt}
    Il dominio $\Omega$ definito nell'Esempio \ref{servetaut} soddisfa la condizione di visibilità per le simil-geodetiche. Per vederlo, consideriamo due casi:
    \begin{nlist}
        \item uno dei due punti è l'origine. Allora basta prendere come compatto un qualsiasi insieme della forma $\{r \le |z| \le R\}$ con $0<r<R<1$ e i due intorni aperti sufficientemente piccoli;
        \item i due punti sono entrambi sulla sfera unitaria. Per \cite[Proposition 6]{NTT} è facile vedere che, se la palla unitaria soddisfa la condizione di visibilità per simil-geodetiche, allora $\Omega$ la soddisfa in questo caso. Ma la palla unitaria è limitata ed è facile vedere che è strettamente pseudoconvessa, quindi, per quanto detto sopra, soddisfa la condizione voluta.
    \end{nlist}
    Perciò, l'ipotesi che la varietà sia taut è necessaria per ottenere un risultato di tipo Wolff-Denjoy, anche con la condizione di visibilità.
\end{ftt}