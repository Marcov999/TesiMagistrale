Nella sezione precedente abbiamo visto come l'ipotesi di varietà taut ci permette di dire, se le orbite di una certa funzione non sono relativamente compatte, che la successione delle iterate è compattamente divergente.

Per ottenere un risultato di tipo Wolff-Denjoy, nel caso in cui le iterate siano compattamente divergenti dobbiamo dire due cose: che le iterate convergono uniformemente sui compatti a una funzione a valori nel bordo euclideo, e che in realtà tale funzione è una costante.

Per ottenere la convergenza uniforme al bordo ci basterà supporre che la varietà sia embeddata in un qualche $\mathbb{C}^d$ e limitata, dopodiché si applica il teorema di Montel. Per dire che la funzione è costante, invece, ci serviranno delle ipotesi aggiuntive di tipo geometrico: la condizione di visibilità per le simil-geodetiche. \\

Nel caso di un dominio con bordo regolare, l'ipotesi di essere strettamente pseudoconvesso permetteva di concludere la condizione geometrica di Gromov-iperbolicità; sotto l'ipotesi di Gromov-iperbolicità, si può dimsotrare che il dominio soddisfa una condizione di visibilità per le geodetiche.

Tuttavia, nella prossima sezione vedremo esempi di domini che soddisfano la condizione di visibilità per le simil-geodetiche ma che non sono Gromov-iperbolici. (CONTROLLARE LA VERIDICITÀ DELLA SEGUENTE FRASE) Inoltre, non vale neanche l'altra implicazione: esistono domini Gromov-iperbolici per i quali non vale la condizione di visibilità per le circa geodetiche.

Dunque non c'è implicazione né in un senso né nell'altro, e il risultato che otteniamo è indipendente da quello per domini Gromov-iperbolici. Prima di enunciarlo, ci serviranno alcune definizioni, partendo da quella di simil-geodetica.

\begin{defn}
    Sia $X$ una varietà connessa complessa embeddata in $\mathbb{C}^d$ e limitata, e fissiamo $\lambda \ge 1$ e $\kappa \ge 0$. Sia $I\subseteq \mathbb{R}$ un intervallo; una curva $\sigma:I \longrightarrow X$ è detta una \textit{$(\lambda,\kappa)$-simil-geodetica} se
    \begin{enumerate}
        \item per ogni $s,t \in I$ si ha
        $$\frac{1}{\lambda}|t-s|-\kappa \le k_X\big(\sigma(s),\sigma(t)\big)\le\lambda|t-s|+\kappa;$$
        \item $\sigma$ è assolutamente continua (quindi $\sigma'(t)$ esiste per quasi ogni $t \in I$) e per quasi ogni $t \in I$ si ha
        $$K_X\big(\sigma(t);\sigma'(t)\big) \le \lambda.$$
    \end{enumerate}
\end{defn}

Diamo adesso la definizione di condizione di visibilità.

\begin{defn} \label{visibility}
    Sia $X$ una varietà connessa complessa embeddata in $\mathbb{C}^d$ e limitata, e fissiamo $\lambda \ge 1$ e $\kappa \ge 0$. Diciamo che $X$ ha la \textit{condizione di visibilità rispetto alle $(\lambda,\kappa)$-simil-geodetiche} se
    \begin{enumerate}
        \item ogni due punti distinti di $X$ possono essere collegati da una $(\lambda,\kappa)$-simil-geodetica;
        \item per ogni coppia di punti $p,q\in\partial X$ con $p\not=q$, esistono in $\mathbb{C}^d$ due intorni $V$ e $W$ di $p$ e $q$ rispettivamente e un compatto $K$ di $X$ tali che: $\overline{V}\cap\overline{W}=\emptyset$; ogni $(\lambda,\kappa)$-simil-geodetica in $X$ che collega un punto di $V$ a un punto di $W$ interseca $K$.
    \end{enumerate}
\end{defn}

Le simil-geodetiche sono delle curve che, a meno di costanti moltiplicative e additive, si comportano come le geodetiche, cioè le curve che minimizzano la lunghezza.

Quello che chiediamo nella Definizione \ref{visibility} è che, per andare da un punto a un altro del bordo con tali curve, allora non possiamo stare arbitrariamente vicini al bordo, ma siamo costretti a ``piegarci'' verso l'interno; in pratica, stiamo chiedendo che ci sia una sorta di curvatura negativa. QUA CI VA UN BEL DISEGNINO IN INKSCAPE, SIA DEL CASO VIETATO CHE DEL CASO PREVISTO DALLA DEFINIZIONE

\begin{ftt}
    Il dominio $\Omega$ definito nell'Esempio \ref{servetaut} soddisfa la condizione di visibilità per le simil-geodetiche. Per vederlo, consideriamo due casi:
    \begin{nlist}
        \item uno dei due punti è l'origine. Allora basta prendere come compatto un qualsiasi insieme della forma $\{r \le |z| \le R\}$ con $0<r<R<1$ e i due intorni aperti sufficientemente piccoli;
        \item i due punti sono entrambi sulla sfera unitaria. Per \cite[Proposition 6]{NTT}, è facile vedere che, se la palla unitaria soddisfa la condizione di visibilità per simil-geodetiche, allora $\Omega$ la soddisfa in questo caso. (LA FRASE SEGUENTE È FALSISSIMA, COME SI FA??? DA CONTROLLARE: DOVREBBE BASTARE FARE COME PER I GOLDILOCK, MI PARE CHE SI USI, DELLE DUE PROPRIETÀ DELLA DEF, QUELLA FACILE DA VERIFICARE PER LA PALLA UNITARIA) Non è difficile convincersi che la condizione è soddisfatta per la palla unitaria.
    \end{nlist}
\end{ftt}