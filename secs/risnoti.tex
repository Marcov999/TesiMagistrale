Vediamo ora alcuni risultati noti della teoria che ci saranno utili nelle nostre dimostrazioni. Cominciamo con alcuni teoremi noti dell'analisi complessa in più variabili.

\begin{thm}
    (Montel, INSERIRE CIT) Sia $\Omega \subseteq \mathbb{C}^n$ un dominio. Una famiglia $\mathcal{F}$ è relativamente compatta in $\mathcal{O}(\Omega)$ se e solo se è uniformemente limitata sui compatti.
\end{thm}

WEIERSTRASS SERVE? INOLTRE, SE ASCOLI-ARZELÀ È INEVITABILE CAMBIA LA FRASE SOPRA\\

Vediamo adesso l'espressione esplicita per $k_X$ in un paio di casi espliciti, dalla quale discende un'importante conseguenza.

\begin{prop}
    (METTERE UNA CIT) Si ha che:
    \begin{nlist}
        \item la distanza di Poincaré e la pseudodistanza di Kobayashi coincidono per $\mathbb{D}$;
        \item dati $z=(z_1,\dots,z_n)$ e $w=(w_1,\dots,w_n)$ in $\mathbb{D}^n$, si ha
        $$k_{\mathbb{D}^n}(z,w)=\max_{j=1,\dots,n}\{k_{\mathbb{D}}(z_j,w_j)\};$$
        \item L'ESPRESSIONE ESPLICITA PER LA PALLA METTILA SOLO SE SERVE.
    \end{nlist}
\end{prop}
\begin{proof}
    \begin{nlist}
        \item Che la pseudodistanza di Kobayashi sia almeno la distanza di Poincaré segue dal lemma di Schwarz-Pick e dalla disuguaglianza triangolare per la distanza di Poincaré; per avere l'uguaglianza, basta notare che il minimo nella definizione della pseudodistanza di Kobayashi è effettivamente raggiunto usando l'identità.
        \item Siano $\varphi_1,\dots,\varphi_m \in \text{Hol}(\mathbb{D},\mathbb{D}^n)$ e $\zeta_0,\dots,\zeta_m \in \mathbb{D}$ tali che $\varphi_1(\zeta_0)=z$ e $\varphi_m(\zeta_m)=w$. Allora, chiamando $\pi_j$ la proiezione sulla $j$-esima coordinata, si ha
        \begin{align*}
            \sum_{h=1}^m k_{\mathbb{D}}(\zeta_{j-1},\zeta_j) &\ge \sum_{h=1}^m k_{\mathbb{D}}\big((\pi_j\circ \varphi_h)(\zeta_{j-1}),(\pi_j\circ \varphi_h)(\zeta_j)\big)\\
            & \ge k_{\mathbb{D}}\big((\pi_j\circ \varphi_1)(\zeta_0),(\pi_j\circ \varphi_m)(\zeta_m)\big)=k_{\mathbb{D}}(z_j,w_j),
        \end{align*}
        dove nella prima disuguaglianza abbiamo usato il lemma di Schwarz-Pick e nella seconda la disuguaglianza triangolare per $k_{\mathbb{D}}$. Segue dunque che $k_{\mathbb{D}^n}(z,w)\ge\max_{j=1,\dots,n}\{k_{\mathbb{D}}(z_j,w_j)\}$.
        Per mostrare che il minimo è effettivamente raggiunto, ricordiamo dall'Osservazione \ref{k_lip} che $k_{\mathbb{D}^n}$ è una semicontrazione rispetto alle funzioni olomorfe, in particolare è invariante per biolomorfismo, così come lo è $k_{\mathbb{D}}$ dal lemma di Schwarz-Pick. Componendo quindi con $f_1\times\dots\times f_n$ dove $f_j \in \text{Aut}(\mathbb{D})$ è tale che $f_j(z_j)=0$, abbiamo che sia $k_{\mathbb{D}^n}(z,w)$ che $\max_{j=1,\dots,n}\{k_{\mathbb{D}}(z_j,w_j)\}$ rimangono invariati. Allora, detto $j_0$ l'indice per cui $k_{\mathbb{D}}(0,w_{j_0})$ è massimo, possiamo considerare $\varphi=g_1\times\dots\times g_n$ dove $g_j \in\text{Hol}(\mathbb{D},\mathbb{D})$ è l'identità per $j=j_0$ e un'opportuna composizione di una rotazione e una contrazione per $j\not=j_0$, di modo che $g_j(w_{j_0})=w_j$. In tal modo, $\varphi(z_{j_0})=\varphi(0)=0=z$ e $\varphi(w_{j_0})=w$.
    \end{nlist}
\end{proof}

\begin{cor}
    La pseudodistanza di Kobayashi è effettivamente una distanza per i domini limitati.
\end{cor}
\begin{proof}
    Discende tutto dalla Proposizione precedente e dall'Osservazione \ref{k_lip}. Innanzitutto, dalla Proposizione abbiamo che $k_{\mathbb{D}^n}$ è effettivamente una distanza; poiché dall'Osservazione sappiamo che $k_X$ è invariante per biolomorfismi, segue che $k_{\mathbb{D}_r^n}$ è una distanza per ogni $r>0$. Se $\Omega\subseteq\mathbb{C}^n$ è un dominio limitato, esiste $r>0$ tale che $\Omega\subseteq\mathbb{D}_r^n$. In tal caso, l'inclusione è una funzione olomorfa. Allora, sempre dall'Osservazione, si ha che se $x,y \in \Omega$ con $x\not=y$ allora $0<k_{\mathbb{D}_r^n}(x,y) \le k_{\Omega}(x,y)$. Segue dunque che $k_\Omega$ è una distanza, come voluto.
\end{proof}

\begin{oss}
    Con la stessa dimostrazione, si ottiene anche che le sottovarietà di varietà Kobayashi-iperboliche sono Kobayashi-iperboliche.
\end{oss}

Citiamo ora un risultato che lega pseudometrica e pseudodistanza di Kobayashi; per una dimostrazione si rimanda a \cite{R}.

\begin{thm}
    \cite[Result 2.1]{CMS} Sia $X$ una sottovarietà complessa connessa embeddata in $\mathbb{C}^d$. Per ogni $z,w \in X$, abbiamo
    \begin{nlist}
        \item $k_X(z,w)=\inf\{l_X(\gamma) \mid \gamma:[a,b] \longrightarrow X\text{ è $C^1$ a tratti, }\gamma(a)=z,\gamma(b)=w\}$;
        \item $k_X(z,w)=\inf\{l_X(\gamma) \mid \gamma:[a,b] \longrightarrow X\text{ è assolutamente continua,}$\\
        $\text{    }\qquad\qquad\,\,\,\,\gamma(a)=z,\gamma(b)=w\}$;
    \end{nlist}
    qui, $l_X(\gamma)=\displaystyle\int_a^b K_X\big(\gamma(t);\gamma'(t)\big)\diff t$ è ben definito in entrambi i casi.
\end{thm}

Adesso vogliamo arrivare a dire che l'ipotesi taut ci permette di ottenere la dicotomia sul comportamento delle iterate delle funzioni olomorfe. Per farlo, ci servono prima alcuni risultati.

\begin{lm}
    Siano $X$ una varietà Kobayashi-iperbolica, $z_0 \in X$ e $r_1,r_2>0$. Allora $B_X\big(B_X(z_0,r_1),r_2\big)=B_X(z_0,r_1+r_2)$, dove $B_X(x,r)$ è la palla di centro $x \in X$ e raggio $r>0$ rispetto alla distanza di Kobayashi e, dato $A\subseteq X$, poniamo $B_X(A,r)=\displaystyle\bigcup_{x \in A}B_X(x,r)$.
\end{lm}
\begin{proof}
    L'inclusione $B_X\big(B_X(z_0,r_1),r_2\big)\subseteq B_X(z_0,r_1+r_2)$ segue dalla disuguaglianza triangolare.

    Per l'altra inclusione, consideriamo $z \in B_X(z_0,r_1+r_2)$ e prendiamo $\epsilon>0$ tale che $3\epsilon=r_1+r_2-k_X(z_0,z)$. Adesso, se $k_X(z_0,z)<r_1$ la conclusione è immediata; assumiamo dunque che $k_X(z_0,z)\ge r_1$, allora si ha $r_2-\epsilon \ge 0$. Supponiamo che $r_1 \le \epsilon$. Allora $k_X(z_0,z)=r_1+r_2-3\epsilon<r_2$ e anche in questo caso la conclusione segue. Assumiamo quindi anche che $r_1-\epsilon>0$.

    Dalla definizione di $k_X$, esistono $\varphi_1,\dots,\varphi_m \in \text{Hol}(\mathbb{D},X)$ e $\zeta_0,\dots,\zeta_m \in \mathbb{D}$ tali che $\varphi_1(\zeta_0)=z_0, \varphi_m(\zeta_m)=z$ e
    $$\sum_{j=1}^m k_{\mathbb{D}}(\zeta_{j-1},\zeta_j)<r_1+r_2-2\epsilon.$$
    Sia $\mu \le m$ il più grande intero tale che
    $$\sum_{j=1}^{\mu-1} k_{\mathbb{D}}(\zeta_{j-1},\zeta_j)<r_1-\epsilon,$$
    che esiste perché $r_1-\epsilon>0$. Prendiamo $\eta_{\mu}$ il punto sulla geodetica congiungente $\zeta_{\mu-1}$ e $\zeta_{\mu}$ tale che
    $$\sum_{j=1}^{\mu-1} k_{\mathbb{D}}(\zeta_{j-1},\zeta_j)+k_{\mathbb{D}}(\zeta_{\mu-1},\eta_{\mu})=r_1-\epsilon,$$
    che esiste perché $r_1+r_2-2\epsilon \ge r_1-\epsilon$. Prendendo dunque $w=\varphi_{\mu}(\eta_{\mu})$ abbiamo $k_X(z_0,w)<r_1$, inoltre per come è stato scelto $\eta_{\mu}$ si ha
    \begin{align*}
        \sum_{j=1}^m k_{\mathbb{D}}(\zeta_{j-1},\zeta_j)&=\sum_{j=1}^{\mu-1} k_{\mathbb{D}}(\zeta_{j-1},\zeta_j)+k_{\mathbb{D}}(\zeta_{\mu-1},\zeta_{\mu})+\sum_{j=\mu+1}^m k_{\mathbb{D}}(\zeta_{j-1},\zeta_j)\\
        &=\sum_{j=1}^{\mu-1} k_{\mathbb{D}}(\zeta_{j-1},\zeta_j)+k_{\mathbb{D}}(\zeta_{\mu-1},\eta_{\mu})+\\
        &+k_{\mathbb{D}}(\eta_{\mu},\zeta_{\mu})+\sum_{j=\mu+1}^m k_{\mathbb{D}}(\zeta_{j-1},\zeta_j),
    \end{align*}
    da cui
    $$k_{\mathbb{D}}(\eta_{\mu},\zeta_{\mu})+\sum_{j=\mu+1}^m k_{\mathbb{D}}(\zeta_{j-1},\zeta_j)=\sum_{j=1}^m k_{\mathbb{D}}(\zeta_{j-1},\zeta_j)-(r_1-\epsilon)<r_2-\epsilon,$$
    perciò $k_X(w,z)<r_2$ e di conseguenza $z \in B_X\big(B_X(z_0,r_1),r_2\big)$, come voluto.
\end{proof}

Vogliamo dare una caratterizzazione equivalente all'essere taut per una varietà.

\begin{oss} \label{c0yx}
    Dalla definizione della compattificazione di Alexandroff, una successione in $C^0(Y,X)$ converge in $C^0(Y,X^*)$ alla funzione costante a $\infty$ se e solo se è compattamente divergente. Inoltre, se $X$ e $Y$ sono varietà (ma in realtà bastano ipotesi meno stringenti) un sottoinsieme di $C^0(Y,X^*)$ è compatto se e solo se è compatto per successioni.
\end{oss}

\begin{cor}
    Una varietà complessa connessa $X$ è taut se e solo se la famiglia $\text{Hol}(\mathbb{D},X)$ è normale.
\end{cor}
\begin{proof}
    Supponiamo che $X$ sia taut e prendiamo una successione $\{f_{\nu}\}_{\nu\in\mathbb{N}}$ in $\text{Hol}(\mathbb{D},X)$. Per la Proposizione \ref{alex}, la chiusura di $\text{Hol}(\mathbb{D},X)$ è compatta in $C^0(\mathbb{D},X^*)$, e per l'Osservazione \ref{c0yx} è compatta per successioni. Allora esiste una sottosuccessione $\{f_{\nu_j}\}_{j \in \mathbb{N}}$ che converge, uniformemente sui compatti, a una qualche funzione $f$. Se $f \in \text{Hol}(\mathbb{D},X)$ abbiamo concluso; altrimenti, poiché $X$ è taut, $f$ è la funzione costante a $\infty$. Ma, per l'Osservazione \label{c0yx}, questo significa che $\{f_{\nu_j}\}_{j \in \mathbb{N}}$ è compattamente divergente. In ogni caso, possiamo concludere che $\text{Hol}(\mathbb{D},X)$ è normale.

    Supponiamo adesso che $\text{Hol}(\mathbb{D},X)$ sia normale. Se $f$ è una funzione nella sua chiusura in $C^0(\mathbb{D},X^*)$, allora è il limite di una successione in $\text{Hol}(\mathbb{D},X)$. Poiché questa famiglia è normale, possiamo trovare una sottosuccessione che converge uniformemente sui compatti oppure è compattamente divergente, ma dovrà comunque convergere a $f$. Allora nel primo caso $f \in \text{Hol}(\mathbb{D},X)$, mentre nel secondo è la funzione costante a $\infty$. Dunque $X$ è taut.
\end{proof}

Si può dimostrare qualcosa di più.

\begin{prop}
    (INSERIRE CIT) Sia $X$ una varietà taut. Allora $\text{Hol}(Y,X)$ è una famiglia normale per ogni varietà complessa $Y$.
\end{prop}

Prima di enunciare il risultato che, data una varietà taut, ci dà la dicotomia che cerchiamo per il comportamento delle iterate delle funzioni olomorfe, dobbiamo studiarne alcune proprietà.

\begin{lm}
    Sia $X$ una varietà complessa e $\rho\in\text{Hol}(X,X)$ una retrazione olomorfa di $X$. Allora l'immagine di $\rho$ è una sottovarietà chiusa di $X$.
\end{lm}
\begin{proof}
    Sia $M=\rho(X)$ e consideriamo $z_0 \in M$. Prendiamo un intorno aperto $U$ di $z_0$ in $X$ che sia contenuto in una carta locale di $X$ in $z_0$. Allora $V=\rho^{-1}(U)\cap U$ è un intorno aperto di $z_0$ tale che $\rho(V) \subseteq V$. Possiamo dunque supporre senza perdita di generalità che $X$ sia un dominio limitato $\Omega\subseteq\mathbb{C}^n$.

    Sia $P=D\rho(z_0):\mathbb{C}^n \longrightarrow \mathbb{C}^n$ e definiamo $\varphi:\Omega \longrightarrow \mathbb{C}^n$ come
    $$\varphi=\id+(2P-\id)\circ(\rho-P).$$
    Poiché $D\varphi(z_0)=\id$, la funzione $\varphi$ definisce una carta locale in un intorno di $z_0$. Adesso, dato che $P^2=P$ e $\rho^2=\rho$ si ha
    \begin{align*}
        \varphi\circ\rho&=\rho+(2P-\id)\circ\rho^2-(2P-\id)\circ P\circ\rho\\
        &=P\circ\rho=P+P\circ(2P-\id)\circ(\rho-P)=P\circ\varphi.
    \end{align*}
    Allora letta in questa carta $\rho$ diventa lineare, perciò $M$ è una sottovarietà vicino a $z_0$. Per arbitrarietà di $z_0$, segue che $M$ è una varietà. È chiusa perché $\rho(X)=\text{Fix}(\rho)$.
\end{proof}

\begin{thm}
    (INSERIRE CIT) Siano $X$ una varietà taut e $f\in\text{Hol}(X,X)$. Supponiamo che la successione $\{f^k\}_{k\in\mathbb{N}}$ delle iterate di $f$ non sia compattamente divergente. Allora esiste un unica retrazione olomorfa $\rho\in\Gamma(f)$ su una sottovarietà $M$ di $X$ tale che ogni funzione limite di $f$ è della forma $h=\gamma\circ\rho$.

    Inoltre, $\varphi=f\restrict{M}\in\text{Aut}(M)$ e $\Gamma(f)$ è isomorfo al sottogruppo di $\text{Aut}(M)$ dato dalla chiusura di $\{\varphi^k\}_{k \in\mathbb{N}}$.
\end{thm}

\begin{proof}
    Poiché la successione delle iterate non è compattamente divergente, esistono due compatti $H,K\subseteq X$ e una sottosuccessione delle iterate tali che l'intersezione di $K$ con l'immagine di $H$ tramite le funzioni della sottosuccessione non è mai vuota. Dato che $X$ è tesa, possiamo estrarre una sottosottosuccessione che converge uniformemente sui compatti o è compattamente divergente; per costruzione non può essere il secondo caso, dunque abbiamo trovato una sottosuccessione $\{f^{k_{\nu}}\}_{\nu\in\mathbb{N}}$ che converge uniformemente sui compatti a $h\in\text{Hol}(X,X)$. Possiamo anche assumere che $p_\nu=k_{\nu+1}-k_\nu$ e $q_\nu=p_\nu-k_\nu$ tendano a $+\infty$ per $\nu\longrightarrow+\infty$. A meno di prendere ulteriori sottosuccessioni, possiamo anche supporre che $\{f^{p_\nu}\}_{\nu\in\mathbb{N}}$ e $\{f^{q_\nu}\}_{\nu\in\mathbb{N}}$ convergano uniformemente sui compatti o siano compattamente divergenti (non necessariamente la stessa cosa per entrambe); è facile vedere che i ragionamenti che andremo a fare vanno bene considerando anche eventuali sottosuccessioni, quindi non perdiamo di generalità. Allora
    $$\lim_{\nu\longrightarrow+\infty}f^{p_\nu}\big(f^{k_\nu}(z)\big)=\lim_{\nu\longrightarrow+\infty}f^{k_{\nu+1}}(z)=h(z)$$
    per ogni $z \in X$; poiché l'orbita di $z$ tramite $\{f^{k_\nu}\}_{\nu\in\mathbb{N}}$ tende a qualcosa, è relativamente compatta, dunque $\{f^{p_\nu}\}_{\nu\in\mathbb{N}}$ non può essere compattamente divergente. Allora converge a una $\rho\in\text{Hol}(X,X)$ tale che
    \begin{equation} \label{roacca}
        h\circ\rho=\rho\circ h=h.
    \end{equation}
    Similmente troviamo che $\{f^{q_\nu}\}_{\nu\in\mathbb{N}}$ converge a una $g\in\text{Hol}(X,X)$ tale che
    \begin{equation} \label{giacca}
        g\circ h=h\circ g=\rho.
    \end{equation}
    In particolare, $\rho^2=\rho\circ\rho=g\circ h\circ\rho=g\circ h=\rho$, perciò $\rho$ è una retrazione di $X$ su una sottovarietà $M$. Dalla \eqref{roacca} abbiamo $h(X)\subseteq M$, inoltre $g\circ\rho=\rho\circ g$, da cui $g(M)\subseteq M$; allora la \eqref{giacca} ci dà $g\circ h\restrict{M}=h\circ g\restrict{M}=\id_M$. Dunque ponendo $\gamma=h\restrict{M}$ otteniamo $h=\gamma\circ\rho$ con $\gamma\in\text{Aut}(M)$. Dobbiamo mostrare che $\rho$ non dipende da $h$, in particolare non dipende dalla sottosuccessione scelta.

    Sia $\{f^{k'_\nu}\}_{\nu\in\mathbb{N}}$ un'altra sottosuccessione convergente a $h'\in\text{Hol}(X,X)$. Ragionando come sopra, possiamo supporre che $s_\nu=k'_\nu-k_\nu$ e $t_\nu=k_{\nu+1}-k'_\nu$ convergano a $+\infty$ per $\nu\longrightarrow+\infty$ e che $\{f^{s_\nu}\}_{\nu\in\mathbb{N}}$ e $\{f^{t_\nu}\}_{\nu\in\mathbb{N}}$ convergano rispettivamente a $\alpha,\beta \in \text{Hol}(X,X)$ tali che
    \begin{equation}\label{accaprimo}
        \alpha\circ h=h\circ\alpha=h' \qquad\text{ e }\qquad \beta\circ h'=h'\circ\beta=h.
    \end{equation}
    Allora $h(X)=h'(X)$, dunque $M$ non dipende dalla sottosuccessione scelta. Adesso scriviamo $h=\gamma_1\circ\rho_1,h'=\gamma_2\circ\rho_2,\alpha=\gamma_3\circ\rho_3,\beta=\gamma_4\circ\rho_4$ dove $\rho_1,\rho_2,\rho_3,\rho_4$ sono delle retrazioni olomorfe di $X$ su $M$ e $\gamma_1,\gamma_2,\gamma_3,\gamma_4\in\text{Aut}(M)$. Vogliamo dire che $\rho_1=\rho_2$. Notiamo che $h\circ h'=h'\circ h$ e $\alpha\circ\beta=\beta\circ\alpha$, che insieme alla \eqref{accaprimo} ci dà
    \begin{equation}\label{gammaro}\begin{split}
        \gamma_1\circ\gamma_2\circ\rho_2=\gamma_2\circ\gamma_1\circ\rho_1,\\
        \gamma_3\circ\gamma_1\circ\rho_1=\gamma_1\circ\gamma_3\circ\rho_3=\gamma_2\circ\rho_2,\\
        \gamma_4\circ\gamma_2\circ\rho_2=\gamma_2\circ\gamma_4\circ\rho_4=\gamma_1\circ\rho_1,\\
        \gamma_3\circ\gamma_4\circ\rho_4=\gamma_4\circ\gamma_3\circ\rho_3.
    \end{split}\end{equation}
    Usando la prima equazione in \eqref{gammaro} scriviamo $\rho_2$ in funzione di $\rho_1$, e sostituendo nella seconda troviamo $\gamma_3=\gamma_1^{-1}\circ\gamma_2$.
\end{proof}

SAREBBE BELLO FARE LA DIM, MA RICHIEDE UN PO' DI LAVORO EXTRA; VALUTARE IN SEGUITO, E IN CASO METTERE TUTTE LE COSE EQUIVALENTI

\begin{thm}
    (INSERIRE CIT) Sia $X$ una varietà taut e $f \in \text{Hol}(X,X)$. Le seguenti affermazioni sono equivalenti:
    \begin{nlist}
        \item la successione delle iterate $\{f^k\}_{k \in \mathbb{N}}$ non è compattamente divergente;
        \item l'orbita di $z$ è relativamente compatta in $X$ per ogni $z \in X$.
    \end{nlist}
\end{thm}

Per finire, vediamo un risultato di tipo Wolff-Denjoy per domini strettamente pseudoconvessi (quindi con bordo $C^2$), in particolare faremo riferimento a una dimostrazione che sfrutta fatti geometrici quali la Gromov-iperbolicità. Il Teorema, che già era noto ancora prima che venisse mostrata la Gromov-iperbolicità, è il seguente.

\begin{thm} \label{abate_wd}
    (Abate, \cite[Theorem 0.5]{A}) Sia $\Omega \subseteq \mathbb{C}^n$ un dominio limitato e strettamente pseudoconvesso e $f \in \text{Hol}(\Omega,\Omega)$. Allora vale esattamente una delle seguenti affermazioni:
    \begin{itemize}
        \item le orbite di $f$ sono relativamente compatte in $\Omega$;
        \item esiste un unico punto di $\partial\Omega$ tale che le orbite di $f$ convergono tutte, uniformemente sui compatti, a quel punto.
    \end{itemize}
\end{thm}

Per dimostrarlo usando la Gromov-iperbolicità, è prima necessario mostrare che $(\Omega,k_{\Omega})$ è Gromov-iperbolico. Citiamo l'articolo di Balogh e Bonk in cui si trova la dimostrazione.

\begin{thm} \label{balogh_bonk}
    (Balogh, Bonk \cite[Theorem 1.4]{BB}) Sia $\Omega \subseteq \mathbb{C}^n$ con $n \ge 2$ un dominio limitato e strettamente pseudoconvesso. Allora $(\Omega,k_{\Omega})$ è Gromov-iperbolico. Inoltre, il bordo iperbolico $\partial_G\Omega$ è identificato con il bordo euclideo $\partial\Omega$.
\end{thm}

Serve anche un Teorema dovuto a Karlsson.

\begin{thm} \label{karlsson}
    (Karlsson, \cite[Corollary 3.7]{Ka}) Sia $(X,d)$ uno spazio metrico proprio tale che
    \begin{nlist}
        \item è un aperto denso di uno spazio topologico $\overline{X}$ compatto e di Hausdorff la cui topologia di sottospazio coincide con la topologia di spazio metrico. Inoltre, dati $x \in X$ e $x_n$ una successione in $X$ che converge a un punto di $\overline{X}\setminus X$, si ha $\displaystyle\lim_{n\longrightarrow+\infty}d(x,x_n)=+\infty$;
        \item date $x_n$ e $y_n$ due successioni convergenti a due punti distinti di $\overline{X}\setminus X$ e $z \in X$, si ha $\displaystyle\lim_{n\longrightarrow+\infty} d(x_n,y_n)-\max\{d(x_n,z),d(y_n,z)\}=+\infty$.
    \end{nlist}
    Sia $\phi:X \longrightarrow X$ una semicontrazione. Allora vale esattamente una delle seguenti affermazioni:
    \begin{itemize}
        \item le orbite di $\phi$ sono limitate;
        \item esiste un unico punto di $\overline{X}\setminus X$ a cui convergono tutte le orbite.
    \end{itemize}
\end{thm}

L'ipotesi (ii) del Teoerema \ref{karlsson} è sempre verificata dagli spazi Gromov-iperbolici, mentre segue dal Teorema \ref{balogh_bonk} che la (i) è vera per i domini limitati e strettamente pseudoconvessi. Usando anche il teorema di Montel, si ottiene così il Teorema \ref{abate_wd}.

Tuttavia, come già anticipato nell'introduzione, quello che noi andremo a vedere è un risultato che vale anche per domini con bordo non necessariamente regolare. L'ipotesi di tipo geometrico che andremo ad utilizzare è il concetto di visibilità, di cui discuteremo anche il rapporto con la Gromov-iperbolicità.