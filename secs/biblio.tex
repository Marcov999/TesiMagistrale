\begin{thebibliography}{widest entry}
  \bibitem[A]{A} M. Abate: Iteration theory, compactly divergent sequences and commuting holomorphic maps. \textit{Annali della Scuola Normale Superiore di Pisa. Classe di Scienze. Serie V}, \textbf{18} (1991), no. 2, 167--191
  \bibitem[BB]{BB} Z. M. Balogh, M. Bonk: Gromov hyperbolicity and the Kobayashi metric on strictly pseudoconvex domains. \textit{Commentarii Mathematici Helvetici}, \textbf{75} (2000), no. 3, 504--533
  \bibitem[CMS]{CMS} V. S. Chandel, A. Maitra, A. D. Sarkar: Notions of Visibility with respect to the Kobayashi distance: Comparison and Applications. Preprint, arXiv:2111.00549v1 (2021)
  \bibitem[Ka]{Ka} A. Karlsson: Non-expanding maps and Busemann functions. \textit{Ergodic Theory and Dynamical Systems}, \textbf{21} (2001), no. 5, 1447--1457
  \bibitem[K1]{K1} S. Kobayashi: \textbf{Hyperbolic Manifolds and Holomorphic Mappings (Second Edition)}. World Scientific Publishing Co., Singapore (MA FA FEDE LA SEDE UFFICIALE, O DOV'È STATA STAMPATA LA `MIA COPIA'?), 2005
  \bibitem[K2]{K2} S. Kobayashi: Invariant distances on complex manifolds and holomorphic mappings. \textit{Journal of the Mathematical Society of Japan}, \textbf{19} (1967), 460--480
  \bibitem[R]{R} H. L. Royden: Remarks on the Kobayashi metric, in \textbf{Several Complex Variables II}, \textit{Proceedings of the International Mathematical Conference} (College Park, MD, 1970), pp. 125--137, \textit{Lecture Notes in Mathematics}, \textbf{185}, Springer, Berlin, 1971
\end{thebibliography}
