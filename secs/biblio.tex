\begin{thebibliography}{widest entry}
  \bibitem[A1]{A1} M. Abate: \textbf{Iteration theory of holomorphic maps on taut manifolds}. Mediterranean Press, Cosenza, 1989 [\url{http://www.dm.unipi.it/˜abate/libri/libriric/libriric.html}]
  \bibitem[A2]{A2} M. Abate: Iteration theory, compactly divergent sequences and commuting holomorphic maps. \textit{Annali della Scuola Normale Superiore di Pisa. Classe di Scienze. Serie V}, \textbf{18} (1991), no. 2, 167--191
  \bibitem[A3]{A3} M. Abate: A characterization of hyperbolic manifolds. \textit{Proceedings of the American Mathematical Society}, \textbf{117} (1993), no. 3, 789--793
  \bibitem[A4]{A4} M. Abate: Dynamics in several complex variables. In Metrical and dynamical aspects in complex analysis, Ed. L. Blanc-Centi, Lecture Notes in Mathematics 2195, Springer, Berlin, 2017, pp. 25--54
  \bibitem[B]{B} T. J. Barth: The Kobayashi distance induces the standard topology. \textit{Proceedings of the American Mathematical Society}, \textbf{35} (1972), 439--441
  \bibitem[BB]{BB} Z. M. Balogh, M. Bonk: Gromov hyperbolicity and the Kobayashi metric on strictly pseudoconvex domains. \textit{Commentarii Mathematici Helvetici}, \textbf{75} (2000), no. 3, 504--533
  \bibitem[CMS]{CMS} V. S. Chandel, A. Maitra, A. D. Sarkar: Notions of Visibility with respect to the Kobayashi distance: Comparison and Applications. Preprint, arXiv:2111.00549v1 (2021)
  \bibitem[N]{N} R. Narasimhan: \textbf{Several Complex Variables}. University of Chicago Press, Chicago, 1971
  \bibitem[NTT]{NTT} N. Nikolov, P. J. Thomas, M. Trybuła: Gromov (non)hyperbolicity of certain domains in $\mathbb{C}^2$. Preprint, arXiv:1403.7673v2 (2015)
  \bibitem[Ka]{Ka} A. Karlsson: Non-expanding maps and Busemann functions. \textit{Ergodic Theory and Dynamical Systems}, \textbf{21} (2001), no. 5, 1447--1457
  \bibitem[Kr]{Kr} S. G. Krantz: \textbf{Function Theeory of Several Complex Variables: Second Edition}. AMS Chelsea Publishing, Providence, 2001
  \bibitem[K1]{K1} S. Kobayashi: Invariant distances on complex manifolds and holomorphic mappings. \textit{Journal of the Mathematical Society of Japan}, \textbf{19} (1967), 460--480
  \bibitem[K2]{K2} S. Kobayashi: \textbf{Hyperbolic Manifolds and Holomorphic Mappings: An Introduction (Second Edition)}. World Scientific Publishing Co., Singapore, 2005
  \bibitem[R]{R} H. L. Royden: Remarks on the Kobayashi metric. In Several Complex Variables II, Proceedings of the International Mathematical Conference, Lecture Notes in Mathematics \textbf{185}, Springer, Berlin, 1971, pp. 125--137
  \bibitem[W]{W} H. Wu: Normal families of holomorphic mappings. \textit{Acta Mathematica}. \textbf{119} (1967), 193--233.
\end{thebibliography}
