\begin{thebibliography}{widest entry}
  \bibitem[A1]{A1} M. Abate: \textbf{Iteration theory of holomorphic maps on taut manifolds}. Mediterranean Press, Cosenza, 1989 [\url{http://www.dm.unipi.it/˜abate/libri/libriric/libriric.html}]
  \bibitem[A2]{A2} M. Abate: Iteration theory, compactly divergent sequences and commuting holomorphic maps. \textit{Annali della Scuola Normale Superiore di Pisa. Classe di Scienze. Serie IV}, \textbf{18} (1991), no. 2, 167--191
  \bibitem[A3]{A3} M. Abate: A characterization of hyperbolic manifolds. \textit{Proceedings of the American Mathematical Society}, \textbf{117} (1993), no. 3, 789--793
  \bibitem[A4]{A4} M. Abate: Dynamics in several complex variables. In \textbf{Metrical and dynamical aspects in complex analysis}, Ed. L. Blanc-Centi, Lecture Notes in Mathematics \textbf{2195}, Springer, Berlin, 2017, pp. 25--54
  \bibitem[A5]{A5} M. Abate:\marginpar{Non sono sicuro del Berlin; Berlin o Leck?} \textbf{Holomorphic Dynamics on Hyperbolic Riemann Surfaces}. De Gruyter, Berlin, 2023
  \bibitem[B]{B} T. J. Barth: The Kobayashi distance induces the standard topology. \textit{Proceedings of the American Mathematical Society}, \textbf{35} (1972), 439--441
  \bibitem[BB]{BB} Z. M. Balogh, M. Bonk: Gromov hyperbolicity and the Kobayashi metric on strictly pseudoconvex domains. \textit{Commentarii Mathematici Helvetici}, \textbf{75} (2000), no. 3, 504--533
  \bibitem[BH]{BH} M. R. Bridson, A. Haefliger: \textbf{Metric-Spaces of Non-Positive Curvature}. Springer, Berlin, 1999
  \bibitem[BM]{BM} G. Bharali, A. Maitra: A weak notion of visibility, a family of examples, and Wolff-Denjoy theorems. \textit{Annali della Scuola Normale Superiore di Pisa. Classe di Scienze. Serie V}, \textbf{22} (2021), no. 1, 195--240
  \bibitem[BNT]{BNT} F. Bracci, N. Nikolov, P. J. Thomas: Visibility of Kobayashi geodesics in convex domains and related properties. \textit{Mathematische Zeitschrift}, \textbf{301} (2022), no. 2, 2011--2035
  \bibitem[BZ1]{BZ1} G. Bharali, A. Zimmer: Goldilocks domains, a weak notion of visibility, and applications. \textit{Advances in Mathematics}, \textbf{310} (2017), 377--425
  \bibitem[Ca]{Ca} H. Cartan: Sur les rétractions d'une variété. \textit{Comptes Rendus des Séances de l'Académie des Sciences. Série I. Mathématique}, \textbf{303} (1986), no. 14, 715
  \bibitem[Ch]{Ch} S. Cho: A lower bound on the Kobayashi metric near a point of finite type in $\mathbb{C}^n$. \textit{Journal of Geometric Analysis}, \textbf{2} (1992), no. 4, 317--325
  \bibitem[CMS]{CMS} V. S. Chandel, A. Maitra, A. D. Sarkar: Notions of Visibility with respect to the Kobayashi distance: Comparison and Applications. Preprint, arXiv:2111.00549v1 (2021)
  \bibitem[D'A]{D'A} J. P. D'Angelo: Real hypersurfaces, orders of contact, and applications. \textit{Annals of Mathematics. Second Series}, \textbf{115} (1982), no. 3, 615--637
  \bibitem[FR]{FR} F. Forstnerič, J.-P. Rosay: Localization of the Kobayashi metric and the boundary continuity of proper holomorphic mappings. \textit{Mathematische Annalen}, \textbf{279} (1987), no. 2, 239--252
  \bibitem[G]{G} I. Graham: Boundary behavior of the Carathéodory and Kobayashi metrics on strongly pseudoconvex domains in $\mathbb{C}^n$ with smooth boundary. \textit{Transactions of the American Mathematical Society}, \textbf{207} (1975), 219--240
  \bibitem[H]{H} L. Hörmander: \textbf{An Introduction to Complex Analysis in Several Variables}. Elsevier Science Publishers B. V., Amsterdam, 1990
  \bibitem[Ka]{Ka} A. Karlsson: Non-expanding maps and Busemann functions. \textit{Ergodic Theory and Dynamical Systems}, \textbf{21} (2001), no. 5, 1447--1457
  \bibitem[Ke]{Ke} J. L. Kelley: \textbf{General Topology}. Springer, New York, 1975
  \bibitem[Ko1]{Ko1} S. Kobayashi: Invariant distances on complex manifolds and holomorphic mappings. \textit{Journal of the Mathematical Society of Japan}, \textbf{19} (1967), 460--480
  \bibitem[Ko2]{Ko2} S. Kobayashi: \textbf{Hyperbolic Manifolds and Holomorphic Mappings: An Introduction (Second Edition)}. World Scientific Publishing, Singapore, 2005
  \bibitem[Kr]{Kr} S. G. Krantz: \textbf{Function Theory of Several Complex Variables: Second Edition}. AMS Chelsea Publishing, Providence, 2001
  \bibitem[KR]{KR} N. Kerzman, J.-P. Rosay: Fonctions plurisousharmoniques d'exhaustion bornées et domaines taut. \textit{Mathematische Annalen}, \textbf{257} (1981), no. 2, 171--184
  \bibitem[N]{N} R. Narasimhan: \textbf{Several Complex Variables}. University of Chicago Press, Chicago, 1971
  \bibitem[NTT]{NTT} N. Nikolov, P. J. Thomas, M. Trybuła: Gromov (non-)hyperbolicity of certain domains in $\mathbb{C}^2$. \textit{Forum Mathematicum}, \textbf{28} (2016), no. 4, 783--794
  \bibitem[Rosa]{Rosa} J.-P. Rosay: Un exemple d'ouvert borné de $\mathbb{C}^3$ ``taut'' mais non hyperbolique complet. \textit{Pacific Journal of Mathematics}, \textbf{98} (1982), no. 1, 153--156
  \bibitem[Ross]{Ross} H. Rossi: Vector fields on analytic spaces. \textit{Annals of Mathematics. Second Series}, \textbf{78} (1963), 455--467
  \bibitem[Roy]{Roy} H. L. Royden: Remarks on the Kobayashi metric. In \textbf{Several Complex Variables II}, Proceedings of the International Mathematical Conference, Lecture Notes in Mathematics \textbf{185}, Springer, Berlin, 1971, pp. 125--137
  \bibitem[S]{S} M. Spivak: \textbf{A Comprehensive Introduction to Differential Geometry, Volume I, Third edition}. Publish or Perish, Inc., Houston, 1999
  \bibitem[V]{V} S. Venturini: Pseudodistances and pseudometrics on real and complex manifolds. \textit{Annali di Matematica Pura ed Applicata. Serie Quarta}, \textbf{154} (1989), 385--402
  \bibitem[W]{W} H. Wu: Normal families of holomorphic mappings. \textit{Acta Mathematica}, \textbf{119} (1967), 193--233
\end{thebibliography}
