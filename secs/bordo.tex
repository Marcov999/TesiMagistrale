Elenchiamo ora, senza dimostrarli, alcuni teoremi che legano l'ipotesi di visibilità alle estensioni al bordo di funzioni. Risultati di estensione erano noti da tempo per il caso di mappe tra domini regolari, per esempio strettamente pseudoconvessi o con bordo liscio. Gli enunciati che vedremo hanno ipotesi che permettono una minore regolarità del bordo (abbiamo visto, tra gli altri esempi, domini con visibilità che hanno delle cuspidi), ma tornerà in gioco la Gromov-iperbolicità, che nei teoremi di tipo ``Wolff-Denjoy'' non era presente tra le ipotesi.

\begin{defn}
    Un \textit{cono circolare aperto con apertura $\theta$} è un aperto di $\mathbb{C}^d$ della forma
    $$\Gamma(v,\theta):=\{z\in\mathbb{C}^d\mid \mathfrak{Re}\langle z,v\rangle>\cos(\theta/2)\|z\|\},$$
    dove $v$ è un vettore unitario di $\mathbb{C}^d$ e $\theta\in(0,\pi)$. Per ogni punto $p\in\mathbb{C}^d$, l'\textit{asse} del cono traslato $p+\Gamma(v,\theta)$ è l'insieme $\{p+tv\mid t>0\}$.
\end{defn}

\begin{defn}
    Sia $\Omega$ un dominio di $\mathbb{C}^d$. Diciamo che $\Omega$ soddisfa una \textit{condizione di cono interno} se esistono due costanti $r>0$ e $\theta\in(0,\pi)$, e un compatto $K\subseteq\Omega$, tali che per ogni $x\in\Omega\setminus K$ esistono un punto $\xi_x\in\partial\Omega$ e un vettore unitario $v_x$ tali che:
    \begin{itemize}
        \item il punto $x$ appartiene all'asse del cono traslato $\xi_x+\Gamma(v_x,\theta)$;
        \item si ha $\big(\xi_x+\Gamma(v_x,\theta)\big)\cap B_r(\xi_x)\subseteq\Omega$.
    \end{itemize}
\end{defn}

\begin{thm}
    (\cite[Theorem 1.5]{BZ1}) Siano $D$ e $\Omega$ due domini limitati di $\mathbb{C}^d$. Supponiamo che $D$ sia pseudoconvesso con bordo $C^2$, e che $\Omega$ sia un dominio Goldilocks che soddisfa una condizione di cono interno. Allora ogni funzione olomorfa propria $F:D\longrightarrow\Omega$ si estende a una funzione continua su $\overline{D}$.
\end{thm}

I teoremi di estensione si hanno non solo per le funzioni olomorfe proprie, ma anche per le quasi-isometrie.

\begin{defn}
    Siano $(X,d_1)$ e $(Y,d_2)$ due spazi metrici, e fissiamo $\lambda\ge 1$ e $\kappa>0$. Una funzione $F:X\longrightarrow Y$ è detta \textit{embedding $(\lambda,\kappa)$-quasi-isometrico rispetto a $d_1$ e $d_2$} se si ha
    $$\frac{1}{\lambda}d_2\big(F(x_1),F(x_2)\big)-\kappa \le d_1(x_1,x_2) \le \lambda d_2\big(F(x_1),F(x_2)\big)+\kappa$$
    per ogni $x_1,x_2\in X$.
\end{defn}

Risultati di estensione sono ben noti per gli embedding quasi-isometrici tra spazi Gromov-iperbolici, si veda \cite[Part III, Chapter H, Theorem 3.9]{BH}.

\begin{thm}
    (\cite[Theorem 1.7]{BZ1}) Siano $D$ un dominio limitato di $\mathbb{C}^k$ e $\Omega$ un dominio Goldilocks di $\mathbb{C}^d$. Supponiamo che $(D,k_D)$ sia uno spazio metrico proprio e Gromov-iperbolico. Sia $F:D\longrightarrow\Omega$ un embedding quasi-isometrico rispetto alle distanze di Kobayashi e continuo; allora esiste un'estensione continua $\tilde{F}:D\cup\partial^GD\longrightarrow\overline{\Omega}$.
\end{thm}

Bharali e Zimmer hanno dimostrato teoremi di estensione anche per domini con visibilità non necessariamente limitati.

\begin{defn}
    Sia $\Omega$ un dominio di $\mathbb{C}^d$. Diciamo che \textit{$\Omega$ ha delle buone geodetiche} se:
    \begin{itemize}
        \item è Kobayashi-iperbolico e completo rispetto a $k_\Omega$;
        \item per ogni coppia di successioni $\{z_n\}_{n\in\mathbb{N}},\{w_n\}_{n\in\mathbb{N}}\subseteq\Omega$ tali per cui si abbia $\displaystyle\lim_{n\longrightarrow+\infty}z_n=\displaystyle\lim_{n\longrightarrow+\infty}w_n=\xi\in\partial^\mathcal{E}\Omega$, e date $\sigma_n$ delle geodetiche congiungenti $z_n$ a $w_n$ per ogni $n\in\mathbb{N}$, si ha che esiste (e quindi per ogni) $o\in\Omega$ tale che $\displaystyle\lim_{n\longrightarrow+\infty}k_\Omega(o,\sigma_n)=+\infty$.
    \end{itemize}
\end{defn}

\begin{thm}
    (\cite[Theorem 1.6]{BZ2}) Siano $\Omega_1\subseteq\mathbb{C}^{d_1}$ e $\Omega_2\subseteq\mathbb{C}^{d_2}$ due domini tali che:
    \begin{enumerate}[label={(\arabic*)}]
        \item il dominio $\Omega_1$ ha delle buone geodetiche;
        \item il dominio $\Omega_2$ è $(\lambda,\kappa)$-visibile per ogni $\lambda\ge 1$ e $\kappa\ge 0$.
    \end{enumerate}

    Sia $f:\Omega_1\longrightarrow\Omega_2$ un embedding quasi-isometrico rispetto alle distanze di Kobayashi; allora $f$ si estende a una funzione continua $\tilde{f}:\overline{\Omega}_1^\mathcal{E}\longrightarrow\overline{\Omega}_2^\mathcal{E}$.
\end{thm}

\begin{defn}
    Sia $\Omega\subseteq\mathbb{C}^d$ un dominio. Diciamo che $\Omega$ è un \textit{dominio lipschitziano} (rispettivamente, \textit{dominio $C^0$}) se per ogni $x\in\Omega$ esistono un intorno $U_x$ di $x$, un cambio di coordinate unitarie centrato in $x$ e una funzione lipschitziana (rispettivamente, una funzione continua) $\varphi_x$, definita in un qualche intorno aperto dell'origine in $\mathbb{C}^{d-1}\times\mathbb{R}$, tali che, se $w=(w_1,\dots,w_n)$ sono le suddette coordinate e $W_x$ è l'immagine di $U_x$ tramite esse, allora $U_x\cap\Omega$ in coordinate è dato da
    $$\{w\in W_x\mid \mathfrak{Im}w_n>\varphi_x(w_1,\dots,w_{n-1},\mathfrak{Re}w_n)\}.$$
\end{defn}

\begin{thm}
    (\cite[Theorem 1.9]{BZ2}) Siano $\Omega_1,\Omega_2\subsetneq\mathbb{C}$ due domini lipschitziani. Sia $f:\Omega_1\longrightarrow\Omega_2$ un biolomorfismo; allora $f$ si estende a un omeomorfismo $\tilde{f}:\overline{\Omega}_1^\mathcal{E}\longrightarrow\overline{\Omega}_2^\mathcal{E}$.
\end{thm}

Il prossimo risultato permette di andare in senso opposto: data un estensione al bordo, dedurne la visibilità del dominio.

\begin{thm}
    (\cite[Theorem 1.10]{BZ2}) Sia $\Omega\subseteq\mathbb{C}^d$ un dominio Kobayashi-iperbolico, e supponiamo che $(\Omega,k_\Omega)$ sia Gromov-iperbolico. Se l'identità $\id_\Omega$ si estende a un omeomorfismo da $\Omega\cup\partial^G\Omega$ in $\overline{\Omega}^\mathcal{E}$ allora:
    \begin{enumerate}[label={(\arabic*)}]
        \item si ha che $\Omega$ è completo rispetto alla distanza di Kobayashi;
        \item si ha che $\Omega$ è $(1,\kappa)$-visibile per ogni $\kappa\ge0$.
    \end{enumerate}
\end{thm}

Passiamo ora ai teoremi di \cite{CMS}, che trattano il caso di sottovarietà qualsiasi.

\begin{defn}
    Sia $X$ una sottovarietà complessa, connessa e limitata di $\mathbb{C}^d$. Un sottoinsieme $S$ di $X$ si dice \textit{sottospazio geodetico} se:
    \begin{itemize}
        \item lo spazio metrico $(S,k_X\restrict{S\times S})$ è completo;
        \item per ogni coppia di punti distinti di $S$, esiste ua geodetica di $(X,k_X)$ che li congiunge e che sia tutta contenuta in $S$.
    \end{itemize}
\end{defn}

\begin{defn}
    Sia $X$ una sottovarietà complessa, connessa e limitata di $\mathbb{C}^d$. Un sottospazio geodetico $S$ si dice \textit{sottospazio di visibilità} se per ogni $p,q\in\overline{S}\setminus S$ con $p\not=q$ esistono in $\mathbb{C}^d$ due intorni $U$ e $V$, rispettivamente di $p$ e di $q$, e un compatto $K\subseteq S$ tali che $\overline{U}\cap\overline{V}=\emptyset$ e, per ogni geodetica $\sigma:[a,b]\longrightarrow S$ di $(X,k_X)$ che collega un punto di $U\cap S$ a un punto di $V\cap S$, si ha $\sigma([a,b])\cap K=\emptyset$.
\end{defn}

\begin{defn}
    Sia $(X,d)$ uno spazio metrico e $(\iota,\tilde{X})$ una sua compattificazione. Un \textit{loop geodetico di $X$ in $\tilde{X}$} è una geodetica $\sigma:\mathbb{R}\longrightarrow X$ (con la distanza euclidea in partenza e $d$ in arrivo) tale che l'insieme dei suoi punti limite a $+\infty$ è uguale a quello dei punti limite a $+\infty$.
\end{defn}

\begin{thm}
    (\cite[Theorem 1.4]{CMS}) Sia $X$ una sottovarietà complessa, connessa e limitata di $\mathbb{C}^d$. Sia $S\subseteq X$ un sottospazio geodetico di $M$ tale che $(S,k_X\restrict{S\times S})$ sia Gromov-iperbolico. Allora $S$ è sottospazio di visibilità se e solo se l'identità $\id_S$ si estende a una funzione continua e suriettiva $\tilde{id}_S:S\cup\partial^GS\longrightarrow\overline{S}$.

    Inoltre, tale estensione è un omeomorfismo se e solo se $S$ non ha loop geodetici in $\overline{S}$.
\end{thm}

\begin{thm}
    (\cite[Theorem 1.5]{CMS}) Siano $X\subseteq\mathbb{C}^m$ e $Y\subseteq\mathbb{C}^n$ due sottovarietà complesse, connesse e limitate, e supponiamo che $(X,k_X)$ sia completo e Gromov-iperbolico. Sia $f:X\longrightarrow Y$ un'isometria rispetto alle distanze di Kobayashi, e supponiamo che $S:=f(X)$ sia un sottospazio di visibilità di $Y$. Allora $f$ si estende a una funzione continua $\tilde{f}:X\cup\partial^GX\longrightarrow\overline{Y}$.

    Inoltre, se $S$ non ha loop geodetici in $\overline{S}$ allora $\tilde{f}$ è un omeomorfismo tra $X\cup\partial^GX$ e $\overline{S}$.
\end{thm}