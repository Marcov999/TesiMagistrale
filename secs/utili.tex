\begin{prop} \label{metrica_bilip}
    Sia $X$ una varietà connessa complessa embeddata in $\mathbb{C}^d$ e limitata di dimensione $n$. Allora
    \begin{enumerate}[label={(\arabic*)}]
        \item esiste $c>0$ tale che $c\|Z\| \le K_X(z;Z)$ per ogni $z \in X$ e $Z \in T_zX$;
        \item per ogni compatto $K\subseteq X$, esiste una costante $C_1=C_1(K)>0$ tale che $K_X(z;Z) \le C_1\|Z\|$ per ogni $z \in X$ e $Z \in T_zX$.
    \end{enumerate}
\end{prop}

\begin{proof}
    POI, VEDERE BENE QUANTO A FONDO SI PUÒ ANDARE NEI DETTAGLI.
\end{proof}

\begin{lm} \label{restrdis}
    Sia $X$ una varietà connessa complessa embeddata in $\mathbb{C}^d$ e limitata e $\sigma:[a,b] \longrightarrow X$ una curva assolutamente continua. Se
    $$l_X(\sigma) \le k_X\big(\sigma(a),\sigma(b)\big)+\kappa$$
    allora, per ogni $a \le s \le t \le b$, si ha
    $$l_X(\sigma\restrict{[s,t]}) \le k_X\big(\sigma(s),\sigma(t)\big)+\kappa.$$
\end{lm}

\begin{proof}
    Siano $s$ e $t$ come sopra. Allora
    $$l_X(\sigma\restrict{[s,t]})=l_X(\sigma)-l_X(\sigma\restrict{[a,s]})-l_X(\sigma\restrict{[t,b]}).$$
    Usando la nostra ipotesi e il punto (ii) del Teorema \ref{lung_int}, troviamo
    $$l_X(\sigma\restrict{[s,t]}) \le k_X\big(\sigma(a),\sigma(b)\big)+\kappa-k_X\big(\sigma(a),\sigma(s)\big)-k_X\big(\sigma(t),\sigma(b)\big).$$
    Applicando la disuguaglianza triangolare, si ottiene la tesi.
\end{proof}

\begin{thm} \label{similgeo_conn}
    Sia $X$ una varietà connessa complessa embeddata in $\mathbb{C}^d$ e limitata. Per ogni $z,w \in X$ e ogni $\kappa>0$ esiste una $(1,\kappa)$-simil-geodetica $\sigma:[a,b]\longrightarrow X$ tale che $\sigma(a)=z$ e $\sigma(b)=w$.
\end{thm}

\begin{proof}
    Per il punto (i) del Teorema \ref{lung_int}, a meno di riparametrizzare esiste una curva $C^1$ a tratti $\gamma:[0,1] \longrightarrow X$ tale che $\gamma(0)=z,\gamma(1)=w$ e
    $$l_X(\gamma)<k_X(z,w)+\kappa.$$
    Inoltre, a meno di perturbare di poco la curva, possiamo assumere che sia $C^1$ e che $\gamma'(t)\not=0$ per ogni $t\in[0,1]$. Consideriamo la funzione $f:[0,1] \longrightarrow [0,+\infty)$ data da
    $$f(t)=\int_0^t K_X\big(\gamma(r);\gamma'(r)\big)\diff r.$$
    Poiché $\gamma([0,1])$ è compatto in $X$, per i punti (1) e (2) della Proposizione \ref{metrica_bilip} esiste $C>0$ tale che
    $$\frac{1}{C}\|\gamma'(t)\| \le K_X\big(\gamma(t);\gamma'(t)\big) \le C\|\gamma'(t)\|\text{ per ogni }t\in[0,1].$$
    Dato che $\|\gamma'(t)\|>0$ per ogni $t\in[0,1]$ e $\gamma'$ è continua, esistono $A,B>0$ tali che $A \le \|\gamma'(t)\le B$ per ogni $t\in[0,1]$. Dunque $f$ è una funzione bilipschitziana e, di conseguenza, strettamente crescente. Chiamiamo $g:[0,l_X(\gamma)] \longrightarrow [0,1]$ l'inversa di $f$. Il nostro claim è che la curva $\sigma=\gamma\circ g:[0,l_X(\gamma)] \longrightarrow X$ sia una $(1,\kappa)$-simil-geodetica.

    Poiché $g$ è bilipschitziana (perché lo è la sua inversa) e $\gamma$ è $C^1$, abbiamo che $\sigma$ è lipschitziana e quindi assolutamente continua. Allora, per i $t$ per i quali $g'(t)$ esiste si ha $\sigma'(t)=\gamma'\big(g(t)\big)\gamma'(t)$. Inoltre, per tali $t$ anche $f'\big(g(t)\big)$ esiste ed è non-nullo, e $g'(t)=1/f'\big(g(t)\big)>0$. Per il teorema fondamentale del calcolo integrale di Lebesgue, si ha che $f'$ esiste per quasi ogni $s\in[0,1]$ e $f'(s)=K_X\big(\gamma(s);\gamma'(s)\big)$. Siccome $g$ è bilipschitziana, la preimmagine degli $s\in[0,1]$ per cui esiste $f'(s)$ è un sottoinsieme di $[0,l_X(\gamma)]$ di misura piena. Visto che $\gamma'(s)\not=0$ per ogni $s\in[0,1]$, otteniamo che
    $$g'(t)=\frac{1}{K_X\Big(\gamma\big(g(t)\big);\gamma'\big(g(t)\big)\Big)}$$
    per quasi ogni $t\in[0,l_X(\gamma)]$. Per tali $t$ si ha che
    $$K_X\big(\sigma(t);\sigma'(t)\big)=K_X\Big(\gamma\big(g(t)\big);\gamma'\big(g(t)\big)g'(t)\Big)=1.$$
    Quindi $l_X(\sigma)=l_X(\gamma) \le k_X(z,w)+\kappa$. Per il Lemma \ref{restrdis} si ha, per ogni $0 \le s \le t \le l_X(\gamma)$, che
    $$|t-s|=l_X(\sigma\restrict{[s,t]}) \le k_X\big(\sigma(s),\sigma(t)\big)+\kappa.$$
    Dato che $\sigma$ è assolutamente continua, per il punto (ii) del Teorema \ref{lung_int} abbiamo anche che
    $$k_X\big(\sigma(s),\sigma(t)\big) \le l_X(\sigma\restrict{[s,t]})=|s-t|$$
    per ogni $0 \le s \le t \le l_X(\gamma)$. Segue dunque che $\sigma$ è una $(1,\kappa)$-simil-geodetica.
\end{proof}

\begin{thm} \label{lim_is_const}
    Sia $X$ una varietà connessa complessa embeddata in $\mathbb{C}^d$ e limitata. Supponiamo che esista un $\kappa_0>0$ tale che $X$ soddisfi la condizione di visibilità rispetto alle $(1,\kappa_0)$-simil-geodetiche. Sia $\{F_\nu\}_{\nu\in\mathbb{N}} \subseteq \mathcal{O}(X,X)$  una successione che converge uniformemente sui compatti di $X$ a una $\psi\in\mathcal{O}(X,\partial X)$. Allora $\psi$ è costante.
\end{thm}

\begin{proof}
    Supponiamo, per assurdo, che $\psi$ non sia costante. Allora, dati $x \in X$ e $\epsilon>0$, la restrizione di $\psi$ alla palla aperta rispetto alla distanza di Kobayashi di centro $x$ e raggio $\epsilon$, $B_{k_X}(x,\epsilon)$, non è costante. Infatti, tale palla è un aperto di $X$, e se $\psi$ fosse costante su di essa lo sarebbe su tutta $X$ per il principio di identità delle funzioni olomorfe, poiché $X$ è connessa. Questa, però, sarebbe una contraddizione alla nostra assunzione.

    Fissiamo $x_0 \in X$ e poniamo $\epsilon_0=\kappa_0/3$. Per quanto detto, deve esistere un $x_1 \in B_{k_X}(x_0,\epsilon_0)$ tale che $\{F_\nu(x_0)\}_{\nu\in\mathbb{N}}$ e $\{F_\nu(x_1)\}_{\nu\in\mathbb{N}}$ convergono rispettivamente a $\xi$ e $\eta$, con $\xi,\eta \in \partial X$ e $\xi\not=\eta$. Consideriamo adesso una $(1,\kappa_0/3)$-simil-geodetica $\gamma:[a,b]\longrightarrow X$ tale che $\gamma(a)=x_0$ e $\gamma(b)=x_1$, che esiste per il Teorema \ref{similgeo_conn}. Per definizione si ha che
    $$|a-b|-\kappa_0/3 \le k_X(x_0,x_1) \implies |a-b| \le k_X(x_0,x_1)+\kappa_0/3 \le 2\kappa_0/3.$$
    Per ogni $\nu\in\mathbb{N}$ definiamo $\sigma_\nu:[a,b]\longrightarrow X$ come $\sigma_\nu=F_\nu\circ\gamma$. Mostriamo che $\sigma_\nu$ è una $(1,\kappa_0)$-simil-geodetica per ogni $\nu\in\mathbb{N}$. Per ogni $s,t \in [a,b]$ si ha
    \begin{align*}k_X\big(\sigma_\nu(s),\sigma_\nu(t)\big)&=k_X\Big(F_\nu\big(\gamma(s)\big),F_\nu\big(\gamma(t)\big)\Big)\\
        &\le k_X\big(\gamma(s),\gamma(t)\big) \le |s-t|+\kappa_0/3 \le |s-t|+\kappa_0,
    \end{align*}
    dove abbiamo usato che le funzioni olomorfe sono delle semicontrazioni rispetto alla distanza di Kobayashi e che $\gamma$ è una $(1,\kappa_0/3)$-simil-geodetica. Inoltre, si ha che $|s-t|-\kappa_0 \le |a-b|-\kappa_0 <0$ per ogni $s,t \in [a,b]$, dunque
    $$|s-t|-\kappa_0 \le k_X\big(\sigma_\nu(s),\sigma_\nu(t)\big) \le |s-t|+\kappa_0.$$
    Infine, per ogni $t$ si ha
    $$K_X\big(\sigma_\nu(t);\sigma_\nu'(t)\big)=K_X\Big(F_\nu\big(\gamma(s)\big);DF_\nu\big(\gamma(t)\big)\gamma'(t)\Big) \le K_X\big(\gamma(t);\gamma'(t)\big) \le 1,$$
    dove abbiamo usato che le funzioni olomorfe non aumentano la metrica di Kobayashi e che $\gamma$ è una $(1,\kappa_0/3)$-simil-geodetica. Adesso, poiché $\{F_\nu\}_{\nu\in\mathbb{N}}$ converge uniformemente sui compatti di $X$ a una funzione olomorfa $\psi$ a valori in $\partial X$, ne consegue che per ogni compatto $K\subseteq X$ esiste un $\nu(K)\in\mathbb{N}$ tale che per ogni $\nu\ge\nu(K)$ si ha $\sigma_\nu([a,b])\cap K=\emptyset$. Anche in questo caso troviamo una contraddizione, poiché $X$ soddisfa la condizione di visibilità rispetto alle $(1,\kappa_0)$-simil-geodetiche.
\end{proof}