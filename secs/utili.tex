Prima di andare a vedere il teorema di tipo ``Wolff-Denjoy'', dobbiamo mostrare diversi risultati preliminari. Visto che andremo a dimostrare la versione del teorema che si trova in \cite{CMS}, tali risultati sono per la maggior parte dimostrati, e il resto citati, nel suddetto articolo. \\

Cominciamo con delle stime dal basso e dall'alto per la metrica di Kobayashi, che permettono anche di ottenere la lipschitzianità delle simil-geodetiche.

\begin{lm} \label{polycpt}
    Sia $X$ una varietà complessa. Se un sottoinsieme compatto $K\subseteq X$ è contenuto nel polidisco di una carta di $X$, allora esiste una costante $C=C(K)>0$ tale che $K_X(z;Z) \le C\|Z\|_X$ per ogni $z \in K$ e $Z \in T_zX$.
\end{lm}

\begin{proof}
    Siano $n=\dim{X}$ e $D=\mathbb{D}_{r_1}\times\dots\times\mathbb{D}_{r_n}$ il polidisco che contiene $K$. Applicando la Proposizione \ref{metrdecr} all'inclusione e passando in coordinate, per ogni $z \in K$ e $Z \in T_zX$ si ha che
    $$K_X(z;Z) \le K_D(z;Z) \le \max_{j=1,\dots,n} \frac{r_j|Z_j|}{r_j^2-|z_j|^2},$$
    dove $|\cdot|$ è il modulo, cioè la norma euclidea delle coordinate della carta, e l'ultima disuguaglianza è il punto (iii) della Proposizione \ref{k_polidisco}. Poiché, per compattezza di $K$, la quantità $r_j^2-|z_j|^2$ è limitata dal basso da una costante positiva per $j=1,\dots,n$, esiste una costante $C_0>0$ tale che
    $$K_X(z;Z) \le C_0\max_{j=1,\dots,n}\{|Z_j|\}.$$
    
    
    Consideriamo la norma hermitiana come una funzione continua
    $$\|\cdot\|_X:K\times\big\{v\in\mathbb{C}^n\mid\max_{j=1,\dots,n}\{|v_j|\}=1\big\}\longrightarrow (0,+\infty),$$
    dove il secondo fattore nel dominio della funzione è visto come sottoinsieme del tangente al variare dei punti nel primo fattore. Per compattezza, tale funzione ammette un minimo $c>0$; a meno di riscalare abbiamo che
    $$\|Z\|_X \ge c\max_{j=1,\dots,n}\{|Z_j|\}$$
    per ogni $z\in K$ e $Z\in T_zX$. Basta allora prendere $C=C_0/c$.
\end{proof}

\begin{prop} \label{metrica_bilip}
    Sia $X$ una varietà complessa. Allora:
    \begin{enumerate}[label={(\arabic*)}]
        \item se $X$ è connessa, è Kobayashi-iperbolica se e solo se per ogni compatto $K\subseteq X$ esiste una costante $c=c(K)>0$ tale che $c\|Z\|_X \le K_X(z;Z)$ per ogni $z \in K$ e $Z \in T_zX$;
        \item per ogni compatto $K\subseteq X$ esiste una costante $C=C(K)>0$ tale che $K_X(z;Z) \le C\|Z\|_X$ per ogni $z \in X$ e $Z \in T_zX$;
        \item se $X$ è una sottovarietà limitata di $\mathbb{C}^n$ esiste una costante $c>0$ tale che $c\|Z\| \le K_X(z;Z)$ per ogni $z \in X$ e $Z \in T_zX$.
    \end{enumerate}
\end{prop}

\begin{proof}
    (1) Mostriamo che se vale quella condizione allora $X$ è Kobayashi-iperbolica. Prendiamo due punti distinti $z_0,w_0\in X$, per cui esiste un intorno compatto $K$ di $z_0$ con $w_0\not\in K$; in particolare, ogni curva da $z_0$ a $w_0$ deve uscire da $K$. Quindi per il Teorema \ref{lung_int} si ha
    $$k_X(z_0,w_0) \ge c\cdot d_X\big(z_0,\partial_X K)>0,$$
    da cui $k_X$ è una distanza e $X$ è Kobayashi-iperbolica.

    Viceversa, supponiamo che $X$ sia Kobayashi-iperbolica. Prendiamo un punto $z_0\in X$ e fissiamo un intorno $U$ di $X$ contenuto in una carta centrata in $z_0$ e che sia biolomorfo a $\mathbb{B}^d$, dove $d=\dim{X}$. Consideriamo l'aperto $V\subseteq U$ corrispondente a $\mathbb{B}_{1/2}^d$ tramite il biolomorfismo, che è ancora un intorno di $z_0$. Scegliamo inoltre $\epsilon>0$ tale che $B_X(z_0,2\epsilon)\subset\subset V$ (ricordiamo che con $B_X$ si intendono le palle rispetto a $k_X$). In particolare, per la Definizione \ref{poidist}, il punto (i) della Proposizione \ref{k_polidisco} e la Proposizione \ref{semicontr}, per ogni $\varphi\in\text{Hol}(\mathbb{D},X)$ tale che $\varphi(0)\in B_X(z_0,\epsilon)$ si ha $\varphi(\mathbb{D}_{\tanh{\epsilon}})\subseteq B_X(z_0,2\epsilon)$.

    Dati $z\in B_X(z_0,\epsilon)$ e $Z\in T_zX$, siano $\varphi\in\text{Hol}(\mathbb{D},X)$ e $v\in\mathbb{C}$ tali che $\varphi(0)=z$ e $\diff_0\varphi(v)=Z$. Allora, ponendo $\psi(\zeta)=\varphi\big((\tanh{\epsilon})\zeta\big)$, abbiamo che $\psi\in\text{Hol}(\mathbb{D},V)$, $\psi(0)=z$ e $\diff_0\psi(v)=(\tanh{\epsilon})Z$. Dalla definizione della pseudometrica di Kobayashi segue che
    $$(\tanh{\epsilon})K_V(z;Z) \le k_X(z;Z)$$
    per ogni $z\in B_X(z_0,\epsilon)$ e $Z\in T_zX$. Dal punto (3), che dimostreremo indipendentemente tra poco, segue che la condizione voluta è vera in $\mathbb{B}_{1/2}^d$, e dunque in $V$, senza che la costante dipenda da un compatto, ma rispetto alla metrica euclidea. Tuttavia, considerano il compatto $\overline{\mathbb{B}_{1/2}^d}$ (motivo per cui abbiamo dovuto prendere la palla più piccola) e chiamando $E_j$ i vettori delle base canonica di $T_wX$ identificato, al variare di $w\in\overline{\mathbb{B}_{1/2}^d}$, con $\mathbb{C}^d$, si ha
    \begin{align*}
        \|Z\|_X \le \sum_{j=1}^d |a_j|\cdot\|E_j\|_X &\le \max_{\substack{w\in\overline{\mathbb{B}_{1/2}^d},\\j=1,\dots,d}}\|E_j\|_X \sum_{j=1}^d |a_j| \\
        &\le C_1\sqrt{\sum_{j=1}^d |a_j|^2}=C_1\|Z\|
    \end{align*}
    per ogni $w\in\overline{\mathbb{B}_{1/2}^d}$ e $Z\in T_wX$ della forma $Z=\displaystyle\sum_{j=1}^d a_jE_j$, e per una qualche costante $C_1>0$, per cui la condizione vale anche con la metrica $\|\cdot\|_X$. Dato allora un compatto $K$, basta ricoprirlo con un numero finito di intorni della forma $B_X(z_0,\epsilon)$ e prendere la costante più piccola al variare di tali intorni.

    (2) Per ogni $z \in K$, scegliamo un polidisco $U_z$ centrato in $z$ e contenuno in una carta di $X$; sia $U_z'\subseteq U_z$ un altro polidisco, nella stessa carta, centrato in $z$ e relativamente compatto in $U_z$ per ogni $z \in K$. Dato che $K$ è compatto, esistono $z_1,\dots,z_l$ tali che $K\subseteq\displaystyle\bigcup_{j=1}^l U_{z_j}'$. Allora, poiché $\overline{U}_{z_j}'$ è un sottoinsieme compatto di $U_{z_j}$ per $j=1,\dots,l$, per il Lemma \ref{polycpt} abbiamo
    $$K_X(z,Z) \le C_j\|Z\|_X$$
    per ogni $z \in U_{z_j}'$ e $Z\in T_zX$, dove $C_j>0$ è una costante che dipende dal compatto $\overline{U}_{z_j}'$. Basta allora porre $C(K)=\displaystyle\max_{j=1,\dots,l}\{C_j\}$.

    (3) Supponiamo per assurdo che esistano $z_j \in X$ e $Z_j \in T_{z_j}X$, con $Z_j\not=0$, tali che $\displaystyle \lim_{j \longrightarrow +\infty} K_X(z_j;Z_j)/\|Z_j\| \longrightarrow 0$. Senza perdita di generalità possiamo supporre $\|Z_j\|=1$ per ogni $j$. Per definizione di $K_X$, esistono delle funzioni $f_j\in\text{Hol}(\mathbb{D},X)$ e dei $v_j \in \mathbb{C}$ tali che $f_j(0)=z_j$ e
    $$|v_j| \le K_X(z_j;Z_j)+1/j\quad\text{ e }\quad \diff_0 f_j(v_j)=Z_j.$$

    Segue che $\displaystyle \lim_{j \longrightarrow +\infty}\|\diff_0f_j(1)\| = \lim_{j \longrightarrow +\infty}1/|v_j|=+\infty$. A meno di sottosuccessioni e di riordinare le coordinate, possiamo supporre che siano le prime componenti dei vettori $\diff_0f_j(1)$ a tendere a $+\infty$. Chiamiamo $g_j$ la prima componente di $f_j$, cosicché $g_j'$ è la prima componente di $\diff f_j(1)$. Le $g_j$ sono le composizioni delle $f_j$ con un embedding e una proiezione, dunque sono olomorfe; inoltre, poiché $X$ è limitata, sono equilimitate. Esiste quindi un $r>0$ tale che $g_j \in \text{Hol}(\mathbb{D},\mathbb{D}_r)$ per ogni $j$. Adesso, noi sappiamo che $\displaystyle\lim_{j\longrightarrow +\infty} |g_j'(0)|=+\infty$; basta allora applicare il lemma di Schwarz a $\dfrac{g_j-g_j(0)}{2r}$ con $j$ sufficientemente grande per ottenere una contraddizione.
\end{proof}

\begin{oss} \label{basta_herm}
    Per il punto (3) della Proposizione precedente non è necessario che la metrica sia quella euclidea, basta una metrica hermitiana qualsiasi. Per vederlo, basta usare il punto (1) e la Proposizione \ref{metrdecr}.
\end{oss}

\begin{cor} \label{dist_da_sotto}
    Sia $X$ una sottovarietà complessa, connessa e relativamente compatta di una varietà Kobayashi-iperbolica $Y$. Allora esiste $c>0$ tale che $c\cdot d_Y(z,w) \le k_X(z,w)$ per ogni $z,w \in X$.
\end{cor}

\begin{proof}
    Per il Teorema \ref{lung_int} ci basta mostrare che $c\cdot d_Y(z,w) \le l_X(\gamma)$ per ogni curva $C^1$ a tratti $\gamma:[a,b] \longrightarrow X$ tale che $\gamma(a)=z$ e $\gamma(b)=w$. Prendendo $c=c(\overline{X})>0$ dato dal punto (1) della Proposizione \ref{metrica_bilip} e usando la Proposizione \ref{metrdecr}, abbiamo che
    \begin{align*}
        l_X(\gamma)&=\int_a^b K_X\big(\gamma(t);\gamma'(t)\big)\diff t \ge \int_a^b K_Y\big(\gamma(t);\gamma'(t)\big)\diff t\\
        &\ge \int_a^b c\|\gamma'(t)\|_Y\diff t c\cdot d_Y(z,w),
    \end{align*}
    come voluto.
\end{proof}

\begin{prop} \label{simillip}
    Sia $X$ una sottovarietà complessa, connessa e relativamente compatta di una varietà Kobayashi-iperbolica $Y$. Allora si ha che per ogni $\lambda \ge 1$ esiste una costante $C=C(\lambda)>0$ tale che ogni $(\lambda,\kappa)$-simil-geodetica è $C$-lipschitziana rispetto a $d_Y$.
\end{prop}

\begin{proof}
    Ogni $(\lambda,\kappa)$-simil-geodetica $\sigma:I\longrightarrow X$ è, per definizione, assolutamente continua rispetto a $d_X$. Allora, per il teorema fondamentale del calcolo integrale, per ogni $s,t\in I$ abbiamo che
    $$\sigma(t)=\sigma(s)+\int_s^t\sigma'(r)\diff r.$$
    Per il punto (1) della Proposizione \ref{metrica_bilip}, esiste una costante $c=c(\overline{X})>0$ tale che $c\|Z\|_Y \le K_Y(z;Z)$ per ogni $z\in X$ e $Z\in T_zX$, e per definizione di $(\lambda,\kappa)$-simil-geodetica $K_X\big(\sigma(t);\sigma'(t)\big) \le \lambda$ per quasi ogni $t\in I$. Dunque, usando anche la Proposizione \ref{metrdecr}, si ha che $\|\sigma'(t)\|_Y \le \lambda/c$ per quasi ogni $t\in I$, da cui
    $$d_Y\big(\sigma(t),\sigma(s)\big) \le \int_s^t\|\sigma'(r)\|_Y\diff r \le \frac{\lambda}{c}|t-s|,$$
    cioè $\sigma$ è $\lambda/c$-lipschitziana rispetto a $d_Y$.
\end{proof}

Il seguente Lemma è un fatto tecnico che ci servirà tra poco.

\begin{lm} \label{restrdis}
    Siano $X$ una varietà complessa e connessa e $\sigma:[a,b] \longrightarrow X$ una curva assolutamente continua rispetto a $d_X$. Se
    $$l_X(\sigma) \le k_X\big(\sigma(a),\sigma(b)\big)+\kappa,$$
    allora, per ogni $a \le s \le t \le b$, si ha
    $$l_X(\sigma\restrict{[s,t]}) \le k_X\big(\sigma(s),\sigma(t)\big)+\kappa.$$
    \vspace*{-\baselineskip}
\end{lm}
\begin{proof}
    Siano $s$ e $t$ come sopra. Allora
    $$l_X(\sigma\restrict{[s,t]})=l_X(\sigma)-l_X(\sigma\restrict{[a,s]})-l_X(\sigma\restrict{[t,b]}).$$

    Usando la nostra ipotesi e il punto (ii) del Teorema \ref{lung_int}, troviamo
    $$l_X(\sigma\restrict{[s,t]}) \le k_X\big(\sigma(a),\sigma(b)\big)+\kappa-k_X\big(\sigma(a),\sigma(s)\big)-k_X\big(\sigma(t),\sigma(b)\big);$$
    applicando la disuguaglianza triangolare, si ottiene la tesi.
\end{proof}

 Adesso vogliamo mostrare che le varietà Kobayashi-iperboliche sono connesse per archi simil-geodetici.

\begin{thm} \label{similgeo_conn}
    Sia $X$ una varietà Kobayashi-iperbolica. Per ogni $z,w \in X$ e ogni $\kappa>0$ esiste una $(1,\kappa)$-simil-geodetica $\sigma:[a,b]\longrightarrow X$ tale che $\sigma(a)=z$ e $\sigma(b)=w$.
\end{thm}

\begin{proof}
    Per il punto (i) del Teorema \ref{lung_int}, a meno di riparametrizzare esiste una curva $C^1$ a tratti $\gamma:[0,1] \longrightarrow X$ tale che $\gamma(0)=z,\gamma(1)=w$ e
    $$l_X(\gamma)<k_X(z,w)+\kappa;$$
    inoltre, a meno di perturbare di poco la curva, possiamo assumere che sia $C^1$ e che $\gamma'(t)\not=0$ per ogni $t\in[0,1]$. Consideriamo la funzione $f:[0,1] \longrightarrow [0,+\infty)$ data da
    $$f(t)=\int_0^t K_X\big(\gamma(r);\gamma'(r)\big)\diff r.$$

    Poiché $\gamma([0,1])$ è compatto in $X$, per i punti (1) e (2) della Proposizione \ref{metrica_bilip} esiste $C>0$ tale che
    $$\frac{1}{C}\|\gamma'(t)\|_X \le K_X\big(\gamma(t);\gamma'(t)\big) \le C\|\gamma'(t)\|_X\text{ per ogni }t\in[0,1].$$

    Dato che $\|\gamma'(t)\|_X>0$ per ogni $t\in[0,1]$ e $\gamma'$ è continua, esistono ${A,B>0}$ tali che $A \le \|\gamma'(t)\|_X\le B$ per ogni $t\in[0,1]$. Quindi $f$ è una funzione bilipschitziana. Sia dunque $g:[0,l_X(\gamma)] \longrightarrow [0,1]$ l'inversa di $f$. Vogliamo dire che la curva $\sigma=\gamma\circ g:[0,l_X(\gamma)] \longrightarrow X$ è una $(1,\kappa)$-simil-geodetica; sostanzialmente, $\sigma$ è la riparametrizzazione per lunghezza d'arco di $\gamma$.

    Poiché $g$ è bilipschitziana (perché lo è la sua inversa) e $\gamma$ è $C^1$, abbiamo che $\sigma$ è lipschitziana, per cui anche assolutamente continua, rispetto a $d_X$; allora, per i $t$ per i quali $g'(t)$ esiste, si ha $\sigma'(t)=\gamma'\big(g(t)\big)g'(t)$. Inoltre, per tali $t$ anche $f'\big(g(t)\big)$ esiste ed è non-nullo, e $g'(t)=1/f'\big(g(t)\big)>0$. Per il teorema fondamentale del calcolo integrale, si ha che $f'$ esiste per quasi ogni $s\in[0,1]$ e $f'(s)=K_X\big(\gamma(s);\gamma'(s)\big)$. Siccome $g$ è bilipschitziana, la preimmagine degli $s\in[0,1]$ per cui $f'(s)$ esiste è un sottoinsieme di $[0,l_X(\gamma)]$ di misura piena. Visto che $\gamma'(s)\not=0$ per ogni $s\in[0,1]$, otteniamo che
    $$g'(t)=\frac{1}{K_X\Big(\gamma\big(g(t)\big);\gamma'\big(g(t)\big)\Big)}$$
    per quasi ogni $t\in[0,l_X(\gamma)]$. Per tali $t$ si ha che
    $$K_X\big(\sigma(t);\sigma'(t)\big)=K_X\Big(\gamma\big(g(t)\big);\gamma'\big(g(t)\big)g'(t)\Big)=1;$$
    quindi $l_X(\sigma)=l_X(\gamma) \le k_X(z,w)+\kappa$. Per il Lemma \ref{restrdis} si ha, per ogni $0 \le s \le t \le l_X(\gamma)$, che
    $$|t-s|=l_X(\sigma\restrict{[s,t]}) \le k_X\big(\sigma(s),\sigma(t)\big)+\kappa.$$

    Dato che $\sigma$ è assolutamente continua, per il punto (ii) del Teorema \ref{lung_int} abbiamo anche che
    $$k_X\big(\sigma(s),\sigma(t)\big) \le l_X(\sigma\restrict{[s,t]})=|s-t|$$
    per ogni $0 \le s \le t \le l_X(\gamma)$. Segue dunque che $\sigma$ è una $(1,\kappa)$-simil-geodetica.
\end{proof}

Adesso ci servirà un lemma quasi ovvio.

\begin{lm} \label{quasiovvio}
    Sia $X$ una varietà complessa e connessa. Se $\sigma:[a,b] \longrightarrow X$ è una $(1,\kappa)$-simil-geodetica per qualche $\kappa>0$, allora per ogni $t \in [a,b]$ si ha
    $$k_X\big(\sigma(a),\sigma(t)\big)+k_X\big(\sigma(t),\sigma(b)\big) \le k_X\big(\sigma(a),\sigma(b)\big)+3\kappa.$$
    \vspace*{-\baselineskip}
\end{lm}
\begin{proof}
    È un'immediata conseguenza della definizione di $(1,\kappa)$-simil-geodetica.
\end{proof}

Il seguente lemma, invece, ci servirà per la prossima dimostrazione. È un risultato sulla convergenza puntuale, mentre più avanti ne vedremo uno sulla convergenza uniforme sui compatti per varietà $(1,\kappa_0)$-visibili per qualche $\kappa_0>0$.

\begin{lm} \label{lemmasottosotto1}
    Siano $X$ uno spazio topologico primo numerabile e ${\{x_n\}_{n\in\mathbb{N}}\subseteq X}$ una successione con chiusura compatta avente un unico punto di accumulazione $\xi$. Allora $\displaystyle\lim_{n\longrightarrow+\infty}x_n=\xi$.
\end{lm}

\begin{proof}
    Supponiamo per assurdo che non valga la tesi. Allora esistono un intorno $U$ di $\xi$ e una sottosuccessione $\{x_{n_j}\}_{j\in\mathbb{N}}$ tali che $x_{n_j}\not\in U$ per ogni $j\in\mathbb{N}$. Poiché $\overline{\{x_n\}_{n\in\mathbb{N}}}$ è compatta in $X$ che è primo numerabile, è compatta per successioni. Dunque, a meno di ulteriori sottosuccessioni, possiamo supporre che $\displaystyle\lim_{j\longrightarrow+\infty}x_{n_j}=\xi'$, ma poiché la successione ha un unico punto di accumulazione dev'essere $\xi'=\xi$, contraddizione in quanto $x_{n_j}\not\in U$ intorno di $\xi$ per ogni $j\in\mathbb{N}$.
\end{proof}