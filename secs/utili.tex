Prima di andare a vedere il teorema di tipo ``Wolff-Denjoy'', dobbiamo mostrare diversi risultati preliminari. Visto che andremo a dimostrare la versione del teorema che si trova in \cite{CMS}, tali risultati sono per la maggior parte dimostrati, e il resto citati, nel suddetto articolo. \\

Cominciamo con delle stime dal basso e dall'alto per la metrica di Kobayashi, che permettono anche di ottenere la lipschitzianità delle simil-geodetiche.

\begin{lm} \label{polycpt}
    Sia $X$ una varietà complessa. Se un sottoinsieme compatto $K\subseteq X$ è contenuto nel polidisco di una carta di $X$, allora esiste una costante $C=C(K)>0$ tale che $K_X(z;Z) \le C\|Z\|$ per ogni $z \in K$ e $Z \in T_zX$.
\end{lm}

\begin{proof}
    Siano $n=\dim{X}$ e $D=\mathbb{D}_{r_1}\times\dots\times\mathbb{D}_{r_n}$ il polidisco che contiene $K$. Applicando l'Osservazione \ref{metr_noncr} all'inclusione e passando in coordinate, per ogni $z \in K$ e $Z \in T_zX$ si ha che
    $$K_X(z;Z) \le K_D(z;Z) \le \max_{j=1,\dots,n} \frac{r_j|Z_j|}{r_j^2-|z_j|^2}.$$
    
    Per ottenere la seconda disuguaglianza, ragioniamo al seguente modo: a meno di riscalare tutto per una costante, possiamo supporre che il membro destro sia uguale a $1$ (se fosse $0$, avremmo $Z=0$ e la tesi sarebbe immediata). Consideriamo adesso la funzione $f\in\text{Hol}(\mathbb{D},D)$ che manda $\zeta\in\mathbb{D}$ nell'elemento di $D$ che ha come $j$-esima coordinata $\dfrac{r_j\alpha_j\zeta+z_j}{1+\bar{z}_j\alpha_j\zeta/r_j}$, dove $\alpha_j=\dfrac{r_jZ_j}{r_j^2-|z_j|^2}$; allora $f(0)=z$ e si ha che $Df(0)\cdot 1=Z$ e la disuguaglianza discende dalla definizione di $K_D$. Poiché, per compattezza di $K$, la quantità $r_j^2-|z_j|^2$ è limitata dal basso da una costante positiva per $j=1,\dots,n$, la tesi segue facilmente.
\end{proof}

\begin{prop} \label{metrica_bilip}
    Sia $X$ una varietà complessa, connessa, embeddata in $\mathbb{C}^d$ e limitata. Allora
    \begin{enumerate}[label={(\arabic*)}]
        \item esiste $c>0$ tale che $c\|Z\| \le K_X(z;Z)$ per ogni $z \in X$ e $Z \in T_zX$;
        \item per ogni compatto $K\subseteq X$, esiste una costante $C=C(K)>0$ tale che $K_X(z;Z) \le C\|Z\|$ per ogni $z \in X$ e $Z \in T_zX$.
    \end{enumerate}
\end{prop}

\begin{proof}
    (1) Supponiamo per assurdo che esistano $z_j \in X$ e $Z_j \in T_{z_j}X$, con $Z_j\not=0$, tali che $\displaystyle \lim_{j \longrightarrow +\infty} K_X(z_j;Z_j)/\|Z_j\| \longrightarrow 0$. Senza perdita di generalità possiamo supporre $\|Z_j\|=1$ per ogni $j$. Per definizione di $K_X$, esistono delle funzioni $f_j\in\text{Hol}(\mathbb{D},X)$ e dei $v_j \in \mathbb{C}$ tali che $f_j(0)=z_j$ e
    $$|v_j| \le K_X(z_j;Z_j)+1/j\quad\text{ e }\quad Df_j(0)v_j=Z_j.$$

    Segue che $\displaystyle \lim_{j \longrightarrow +\infty}\|Df_j(0)\|=\lim_{j \longrightarrow +\infty}1/|v_j|=+\infty$. A meno di sottosuccessioni e di riordinare le coordinate, possiamo supporre che siano le prime componenti dei vettori $Df_j(0)$ a tendere a $+\infty$. Chiamiamo $g_j$ la prima componente di $f_j$, cosicché $g_j'$ è la prima componente di $Df_j$. Le $g_j$ sono le composizioni delle $f_j$ con un embedding e una proiezione, dunque sono olomorfe; inoltre, poiché $X$ è limitata, sono equilimitate. Esiste quindi un $r>0$ tale che $g_j \in \text{Hol}(\mathbb{D},\mathbb{D}_r)$ per ogni $j$. Adesso, noi sappiamo che $\displaystyle\lim_{j\longrightarrow +\infty} |g_j'(0)|=+\infty$; basta allora applicare il lemma di Schwarz per ottenere una contraddizione.

    (2) Per ogni $z \in K$, scegliamo un polidisco $U_z$ centrato in $z$ e contenuno in una carta di $X$; sia $U_z'\subseteq U_z$ un altro polidisco, nella stessa carta, centrato in $z$ e relativamente compatto in $U_z$ per ogni $z \in K$. Dato che $K$ è compatto, esistono $z_1,\dots,z_l$ tali che $K\subseteq\displaystyle\bigcup_{j=1}^l U_{z_j}'$. Allora, poiché $\overline{U}_{z_j}'$ è un sottoinsieme compatto di $U_{z_j}$ per $j=1,\dots,l$, per il Lemma \ref{polycpt} abbiamo
    $$K_X(z,Z) \le C_i\|Z\|$$
    per ogni $z \in U_{z_j}'$ e $Z\in T_zX$, dove $C_i>0$ è una costante che dipende dal compatto $\overline{U}_{z_j}'$. Basta allora porre $C(K)=\displaystyle\max_{j=1,\dots,l}\{C_j\}$.
\end{proof}

\begin{prop} \label{simillip}
    Sia $X$ una varietà complessa, connessa, embeddata in $\mathbb{C}^d$ e limitata. Per ogni $\lambda \ge 1$ esiste una costante $C=C(\lambda)>0$ tale che ogni $(\lambda,\kappa)$-simil-geodetica è $C$-lipschitziana rispetto alla distanza euclidea.
\end{prop}

\begin{proof}
    Ogni $(\lambda,\kappa)$-simil-geodetica $\sigma:I\longrightarrow X$ è, per definizione, assolutamente continua rispetto alla distanza euclidea. Allora, per il teorema fondamentale del calcolo integrale di Lebesgue, per ogni $s,t\in I$ abbiamo che
    $$\sigma(t)=\sigma(s)+\int_s^t\sigma'(r)\diff r.$$
    
    Per il punto (1) della Proposizione \ref{metrica_bilip}, esiste una costante $c>0$ tale che $c\|Z\| \le K_X(z;Z)$ per ogni $z\in X$ e $Z\in T_zX$, e per definizione di $(\lambda,\kappa)$-simil-geodetica $K_X\big(\sigma(t);\sigma'(t)\big) \le \lambda$ per quasi ogni $t\in I$. Dunque $\|\sigma'(t)\| \le \lambda/c$ per quasi ogni $t\in I$, da cui
    $$\|\sigma(t)-\sigma(s)\| =\|\int_s^t\sigma'(r)\diff r\| \le \int_s^t\|\sigma'(r)\|\diff r \le \frac{\lambda}{c}|t-s|,$$
    cioè $\sigma$ è $\lambda/c$-lipschitziana rispetto alla distanza euclidea.
\end{proof}

Il seguente Lemma è un fatto tecnico che ci servirà tra poco.

\begin{lm} \label{restrdis}
    Siano $X$ una varietà complessa, connessa, embeddata in $\mathbb{C}^d$ e limitata e $\sigma:[a,b] \longrightarrow X$ una curva assolutamente continua. Se
    $$l_X(\sigma) \le k_X\big(\sigma(a),\sigma(b)\big)+\kappa,$$
    allora, per ogni $a \le s \le t \le b$, si ha
    $$l_X(\sigma\restrict{[s,t]}) \le k_X\big(\sigma(s),\sigma(t)\big)+\kappa.$$
\end{lm}

\begin{proof}
    Siano $s$ e $t$ come sopra. Allora
    $$l_X(\sigma\restrict{[s,t]})=l_X(\sigma)-l_X(\sigma\restrict{[a,s]})-l_X(\sigma\restrict{[t,b]}).$$

    Usando la nostra ipotesi e il punto (ii) del Teorema \ref{lung_int}, troviamo
    $$l_X(\sigma\restrict{[s,t]}) \le k_X\big(\sigma(a),\sigma(b)\big)+\kappa-k_X\big(\sigma(a),\sigma(s)\big)-k_X\big(\sigma(t),\sigma(b)\big);$$
    applicando la disuguaglianza triangolare, si ottiene la tesi.
\end{proof}

Quello che vogliamo mostrare adesso è che le varietà complesse, connesse, embettate in $\mathbb{C}^d$ e limitate sono anche connesse per archi simil-geodetici.

\begin{thm} \label{similgeo_conn}
    Sia $X$ una varietà complessa, connessa, embeddata in $\mathbb{C}^d$ e limitata. Per ogni $z,w \in X$ e ogni $\kappa>0$ esiste una $(1,\kappa)$-simil-geodetica $\sigma:[a,b]\longrightarrow X$ tale che $\sigma(a)=z$ e $\sigma(b)=w$.
\end{thm}

\begin{proof}
    Per il punto (i) del Teorema \ref{lung_int}, a meno di riparametrizzare esiste una curva $C^1$ a tratti $\gamma:[0,1] \longrightarrow X$ tale che $\gamma(0)=z,\gamma(1)=w$ e
    $$l_X(\gamma)<k_X(z,w)+\kappa;$$
    inoltre, a meno di perturbare di poco la curva, possiamo assumere che sia $C^1$ e che $\gamma'(t)\not=0$ per ogni $t\in[0,1]$. Consideriamo la funzione $f:[0,1] \longrightarrow [0,+\infty)$ data da
    $$f(t)=\int_0^t K_X\big(\gamma(r);\gamma'(r)\big)\diff r.$$

    Poiché $\gamma([0,1])$ è compatto in $X$, per i punti (1) e (2) della Proposizione \ref{metrica_bilip} esiste $C>0$ tale che
    $$\frac{1}{C}\|\gamma'(t)\| \le K_X\big(\gamma(t);\gamma'(t)\big) \le C\|\gamma'(t)\|\text{ per ogni }t\in[0,1].$$

    Dato che $\|\gamma'(t)\|>0$ per ogni $t\in[0,1]$ e $\gamma'$ è continua, esistono $A,B>0$ tali che $A \le \|\gamma'(t)\|\le B$ per ogni $t\in[0,1]$. Dunque $f$ è una funzione bilipschitziana e, di conseguenza, strettamente crescente. Chiamiamo $g:[0,l_X(\gamma)] \longrightarrow [0,1]$ l'inversa di $f$. Vogliamo dire che la curva $\sigma=\gamma\circ g:[0,l_X(\gamma)] \longrightarrow X$ è una $(1,\kappa)$-simil-geodetica.

    Poiché $g$ è bilipschitziana (perché lo è la sua inversa) e $\gamma$ è $C^1$, abbiamo che $\sigma$ è lipschitziana, quindi è assolutamente continua; allora, per i $t$ per i quali $g'(t)$ esiste, si ha $\sigma'(t)=\gamma'\big(g(t)\big)g'(t)$. Inoltre, per tali $t$ anche $f'\big(g(t)\big)$ esiste ed è non-nullo, e $g'(t)=1/f'\big(g(t)\big)>0$. Per il teorema fondamentale del calcolo integrale di Lebesgue, si ha che $f'$ esiste per quasi ogni $s\in[0,1]$ e $f'(s)=K_X\big(\gamma(s);\gamma'(s)\big)$. Siccome $g$ è bilipschitziana, la preimmagine degli $s\in[0,1]$ per cui $f'(s)$ esiste è un sottoinsieme di $[0,l_X(\gamma)]$ di misura piena. Visto che $\gamma'(s)\not=0$ per ogni $s\in[0,1]$, otteniamo che
    $$g'(t)=\frac{1}{K_X\Big(\gamma\big(g(t)\big);\gamma'\big(g(t)\big)\Big)}$$
    per quasi ogni $t\in[0,l_X(\gamma)]$. Per tali $t$ si ha che
    $$K_X\big(\sigma(t);\sigma'(t)\big)=K_X\Big(\gamma\big(g(t)\big);\gamma'\big(g(t)\big)g'(t)\Big)=1;$$
    quindi $l_X(\sigma)=l_X(\gamma) \le k_X(z,w)+\kappa$. Per il Lemma \ref{restrdis} si ha, per ogni $0 \le s \le t \le l_X(\gamma)$, che
    $$|t-s|=l_X(\sigma\restrict{[s,t]}) \le k_X\big(\sigma(s),\sigma(t)\big)+\kappa.$$

    Dato che $\sigma$ è assolutamente continua, per il punto (ii) del Teorema \ref{lung_int} abbiamo anche che
    $$k_X\big(\sigma(s),\sigma(t)\big) \le l_X(\sigma\restrict{[s,t]})=|s-t|$$
    per ogni $0 \le s \le t \le l_X(\gamma)$. Segue dunque che $\sigma$ è una $(1,\kappa)$-simil-geodetica.
\end{proof}

Adesso ci servirà un Lemma quasi ovvio.

\begin{lm} \label{quasiovvio}
    Sia $X$ una varietà complessa, connessa, embeddata in $\mathbb{C}^d$ e limitata. Se $\sigma:[a,b] \longrightarrow X$ è una $(1,\kappa)$-simil-geodetica per qualche $\kappa>0$, allora per ogni $t \in [a,b]$ si ha
    $$k_X\big(\sigma(a),\sigma(t)\big)+k_X\big(\sigma(t),\sigma(b)\big) \le k_X\big(\sigma(a),\sigma(b)\big)+3\kappa.$$
\end{lm}

\begin{proof}
    È un'immediata conseguenza della definizione di $(1,\kappa)$-simil-geodetica.
\end{proof}

Quello che andremo ora a dimostrare è uno dei fatti cruciali per ottenere il teorema di tipo ``Wolff-Denjoy''. Esso afferma che, sotto condizioni di visibilità per le simil-geodetiche, le sottosuccessioni di iterate di una funzione olomorfa che ``tendono a infinito'' convergono tutte, puntualmente, a un unico punto del bordo.

\begin{prop} \label{inf_impl_ugu}
    Sia $X$ una varietà complessa, connessa, embeddata in $\mathbb{C}^d$ e limitata. Supponiamo che esista un $\kappa_0>0$ tale che $X$ soddisfi la condizione di visibilità rispetto alle $(1,\kappa_0)$-simil-geodetiche. Siano $\nu,\mu:\mathbb{N}\longrightarrow\mathbb{N}$ due funzioni strettamente crescenti e $F\in\textnormal{Hol}(X,X)$ per cui esiste $x_0 \in X$ tale che
    \begin{equation}
        \lim_{j\longrightarrow+\infty} k_X\big(F^{\nu(j)}(x_0),x_0\big)=\lim_{j\longrightarrow+\infty} k_X\big(F^{\mu(j)}(x_0),x_0\big)=+\infty;
    \end{equation}
    allora esiste $\xi\in\partial X$ tale che
    \begin{equation}
        \lim_{j\longrightarrow+\infty} F^{\nu(j)}(z)=\lim_{j\longrightarrow+\infty} F^{\mu(j)}(z)=\xi
    \end{equation}
    per ogni $z \in X$.
\end{prop}

\begin{proof}
    Usando la disuguaglianza triangolare e il fatto che le funzioni olomorfe sono semicontrazioni rispetto a $k_X$, troviamo che
    \begin{gather*}
        k_X\big(F^{\nu(j)}(z),z\big) \ge k_X\big(F^{\nu(j)}(x_0),x_0\big)-2k_X(x_0,z)\text{ e}\\
        k_X\big(F^{\mu(j)}(z),z\big) \ge k_X\big(F^{\mu(j)}(x_0),x_0\big)-2k_X(x_0,z);
    \end{gather*}
    segue che $\displaystyle\lim_{j\longrightarrow+\infty}k_X\big(F^{\nu(j)}(z),z\big)=\lim_{j\longrightarrow+\infty}k_X\big(F^{\mu(j)}(z),z\big)=+\infty$ per ogni $z \in X$. Usando l'ipotesi e il fatto che $X$ è limitata in $\mathbb{C}^d$, possiamo trovare una funzione strettamente crescente $\tau:\mathbb{N} \longrightarrow \mathbb{N}$ tale che
    \begin{itemize}
        \item si ha $k_X\big(F^{(\nu\circ\tau)(j)}(x_0),x_0\big) \ge k_X\big(F^k(x_0),x_0\big)$ per ogni $j \in \mathbb{N}$ e per ogni $k \le (\nu\circ\tau)(j)$;
        \item la successione $\{F^{(\nu\circ\tau)(j)}(x_0)\}_{j\in\mathbb{N}}$ converge a un certo $\xi\in\partial X$.
    \end{itemize}

    Vogliamo ora mostrare la seguente asserzione.

    Siano $\{m_j\}_{j\in\mathbb{N}}$ e $\{m'_j\}_{j\in\mathbb{N}}$ due successioni strettamente crescenti di numeri naturali e $z_0,z'_0\in X$ tali che:
    \begin{enumerate}[label={(\arabic*)}]
        \item per ogni $j\in\mathbb{N}$ si ha $m_j \ge m'_j$;
        \item per ogni $j\in\mathbb{N}$ e $k \le m_j$ si ha $k_X\big(F^{m_j}(z_0),z_0\big) \ge k_X\big(F^k(z_0),z_0\big)$;
        \item si ha $\displaystyle\lim_{j\longrightarrow+\infty}k_X\big(F^{m_j}(z_0),z_0\big)=\lim_{j\longrightarrow+\infty}k_X\big(F^{m'_j}(z'_0),z_0\big)=+\infty$;
        \item le successioni $\{F^{m_j}(z_0)\}_{j\in\mathbb{N}}$ e $\{F^{m'_j}(z'_0)\}_{j\in\mathbb{N}}$ convergono, rispettivamente, a $\zeta$ e $\zeta'$ in $\partial X$;
    \end{enumerate}
    allora $\zeta=\zeta'$.

    Supponiamo per assurdo che $\zeta\not=\zeta'$. Grazie al Teorema \ref{similgeo_conn} possiamo scegliere, per ogni $j\in\mathbb{N}$, una $(1,\kappa_0)$-simil-geodetica $\sigma_j:[0,T_j] \longrightarrow X$ tale che $\sigma_j(0)=F^{m_j}(z_0)$ e $\sigma_j(T_j)=F^{m'_j}(z'_0)$. Adesso, dato che abbiamo assunto che $\{F^{m_j}(z_0)\}_{j\in\mathbb{N}}$ e $\{F^{m'_j}(z'_0)\}_{j\in\mathbb{N}}$ convergano a due punti di $\partial X$ distinti e $X$ ha la visibilità rispetto alle $(1,\kappa_0)$-simil-geodetiche, esistono una costante $0<R<+\infty$ e, per ogni $j\in\mathbb{N}$, un $t_j\in[0,T_j]$ tali che $k_X\big(z_0,\sigma_j(t_j)\big)<R$. Per il Lemma \ref{quasiovvio} si ha dunque che
    \begin{equation} \label{3k_02R}
        \begin{aligned}
            k_X\big(F^{m_j}(z_0),F^{m'_j}(z'_0)\big) &\ge k_X\big(F^{m_j}(z_0),\sigma_j(t_j)\big)+k_X\big(\sigma_j(t_j),F^{m'_j}(z'_0)\big)-3\kappa_0\\
            &\ge k_X\big(F^{m_j}(z_0),z_0\big)-k_X\big(z_0,\sigma_j(t_j)\big)+\\
            &+k_X\big(F^{m'_j}(z'_0),z_0\big)-k_X\big(z_0,\sigma_j(t_j)\big)-3\kappa_0\\
            &\ge k_X\big(F^{m_j}(z_0),z_0\big)+k_X\big(F^{m'_j}(z'_0),z_0\big)-3\kappa_0-2R;
        \end{aligned}
    \end{equation}
    d'altra parte, abbiamo anche che
    \begin{equation} \label{-3k_02R}
        \begin{aligned}
            k_X\big(F^{m_j}(z_0),F^{m'_j}(z'_0)\big) &\le k_X\big(F^{m_j-m'_j}(z_0),z'_0\big) \\
            &\le k_X\big(F^{m_j-m'_j}(z_0),z_0\big)+k_X(z_0,z'_0) \\
            &\le k_X\big(F^{m_j}(z_0),z_0\big)+k_X(z_0,z'_0),
        \end{aligned}
    \end{equation}
    dove per la prima e la terza disuguaglianza abbiamo usato, rispettivamente, le condizioni (1) e (2) sulle successioni $\{m_j\}_{j\in\mathbb{N}}$ e $\{m'_j\}_{j\in\mathbb{N}}$; nella prima, abbiamo anche usato che le funzioni olomorfe sono semicontrazioni rispetto a $k_X$. Concatenando la \eqref{3k_02R} e la \eqref{-3k_02R} e riarrangiando i termini, otteniamo
    $$k_X\big(F^{m'_j}(z'_0),z_0\big) \le k_X(z_0,z'_0)+3\kappa_0+2R,$$
    che è in contraddizione con la condizione (3).\\

    Adesso che l'asserzione è stata dimostrata, possiamo concludere la dimostrazione. Mostriamo innanzitutto che $\displaystyle\lim_{j\longrightarrow+\infty}F^{\nu(j)}(z)=\xi$ per ogni $z\in X$. Fissiamo uno $z\in X$, e prendiamo $\xi'$ punto limite di $\{F^{\nu(j)}(z)\}_{j\in\mathbb{N}}$. Allora esiste una $\tau':\mathbb{N}\longrightarrow\mathbb{N}$, strettamente crescente, tale che $\displaystyle \lim_{j\longrightarrow+\infty} F^{(\nu\circ\tau')(j)}(z)=\xi'$. Scegliamo una $\tilde{\tau}:\mathbb{N}\longrightarrow\mathbb{N}$, strettamente crescente, tale che $\nu\circ\tau\circ\tilde{\tau} \ge \nu\circ\tau'$ e applichiamo l'asserzione dimostrata sopra a $m_j=(\nu\circ\tau\circ\tilde{\tau})(j), m'_j=(\nu\circ\tau')(j)$ e $z_0=x_0,z'_0=z$. Troviamo così $\xi'=\xi$, da cui segue facilmente che tutta la successione deve tendere a $\xi$. Ragionando in modo analogo, troviamo che vale lo stesso anche per $\{F^{\mu(j)}(z)\}_{j\in\mathbb{N}}$, come voluto.
\end{proof}

Anche il seguente teorema ci aiuterà nella nostra dimostrazione. Esso afferma che, sotto condizioni di visibilità per le simil-geodetiche, le successioni di funzioni olomorfe che convergono uniformemente sui compatti di $X$ devono necessariamente convergere a una costante.

\begin{thm} \label{lim_is_const}
    Sia $X$ una varietà complessa, connessa, embeddata in $\mathbb{C}^d$ e limitata. Supponiamo che esista un $\kappa_0>0$ tale che $X$ soddisfi la condizione di visibilità rispetto alle $(1,\kappa_0)$-simil-geodetiche. Sia $\{F_\nu\}_{\nu\in\mathbb{N}} \subseteq \textnormal{Hol}(X,X)$  una successione che converge uniformemente sui compatti di $X$ a una $\psi\in\textnormal{Hol}(X,\partial X)$. Allora $\psi$ è costante.
\end{thm}

\begin{proof}
    Supponiamo, per assurdo, che $\psi$ non sia costante. Allora, dati $x \in X$ e $\epsilon>0$, la restrizione di $\psi$ a $B_X(x,\epsilon)$, la palla aperta rispetto alla distanza di Kobayashi di centro $x$ e raggio $\epsilon$, non è costante. Infatti, tale palla è un aperto di $X$, e se $\psi$ fosse costante su di essa lo sarebbe su tutta $X$ per il principio di identità delle funzioni olomorfe, poiché $X$ è connessa. Questa, però, sarebbe una contraddizione alla nostra assunzione.

    Fissiamo $x_0 \in X$ e poniamo $\epsilon_0=\kappa_0/3$. Per quanto detto, deve esistere un $x_1 \in B_X(x_0,\epsilon_0)$ tale che $\{F_\nu(x_0)\}_{\nu\in\mathbb{N}}$ e $\{F_\nu(x_1)\}_{\nu\in\mathbb{N}}$ convergono, rispettivamente, a $\xi$ e $\eta$, con $\xi,\eta \in \partial X$ e $\xi\not=\eta$. Consideriamo adesso una $(1,\kappa_0/3)$-simil-geodetica $\gamma:[a,b]\longrightarrow X$ tale che $\gamma(a)=x_0$ e $\gamma(b)=x_1$, che esiste per il Teorema \ref{similgeo_conn}; da \eqref{simil-geo1} nella definizione si ha che
    $$|a-b|-\kappa_0/3 \le k_X(x_0,x_1) \implies |a-b| \le k_X(x_0,x_1)+\kappa_0/3 \le 2\kappa_0/3.$$

    Per ogni $\nu\in\mathbb{N}$, definiamo $\sigma_\nu:[a,b]\longrightarrow X$ come $\sigma_\nu=F_\nu\circ\gamma$. Mostriamo che $\sigma_\nu$ è una $(1,\kappa_0)$-simil-geodetica per ogni $\nu\in\mathbb{N}$. Per ogni $s,t \in [a,b]$ si ha
    \begin{align*}k_X\big(\sigma_\nu(s),\sigma_\nu(t)\big)&=k_X\Big(F_\nu\big(\gamma(s)\big),F_\nu\big(\gamma(t)\big)\Big)\\
        &\le k_X\big(\gamma(s),\gamma(t)\big) \le |s-t|+\kappa_0/3 \le |s-t|+\kappa_0,
    \end{align*}
    dove abbiamo usato che le funzioni olomorfe sono delle semicontrazioni rispetto alla distanza di Kobayashi e, dato che $\gamma$ è una $(1,\kappa_0/3)$-simil-geodetica, la \eqref{simil-geo1}. Inoltre, si ha che $|s-t|-\kappa_0 \le |a-b|-\kappa_0 \le -\kappa_0/3<0$ per ogni $s,t \in [a,b]$, dunque
    \begin{equation} \label{snu1}
        |s-t|-\kappa_0 \le k_X\big(\sigma_\nu(s),\sigma_\nu(t)\big) \le |s-t|+\kappa_0.
    \end{equation}

    Infine, per ogni $t$ si ha
    \begin{equation} \label{snu2}
        \begin{aligned}
            K_X\big(\sigma_\nu(t);\sigma_\nu'(t)\big)&=K_X\Big(F_\nu\big(\gamma(s)\big);DF_\nu\big(\gamma(t)\big)\gamma'(t)\Big) \\
            &\le K_X\big(\gamma(t);\gamma'(t)\big) \le 1,
        \end{aligned}
    \end{equation}
    dove abbiamo usato che le funzioni olomorfe non aumentano la metrica di Kobayashi e, visto che $\gamma$ è una $(1,\kappa_0/3)$-simil-geodetica, la \eqref{simil-geo2}. Si ha anche che la curva $\sigma_\nu$ è assolutamente continua per ogni $\nu$, poiché composizione di $\gamma$, che è assolutamente continua per definizione di $(1,\kappa_0/3)$-simil-geodetica, e $F_\nu$, che è olomorfa e dunque lipschitziana sul compatto $\gamma([a,b])$. Grazie alla \eqref{snu1} e alla \eqref{snu2}, possiamo concludere che $\sigma_\nu$ è una $(1,\kappa_0)$-simil-geodetica per ogni $\nu$.
    
    Adesso, poiché $\{F_\nu\}_{\nu\in\mathbb{N}}$ converge uniformemente sui compatti di $X$ a una funzione olomorfa $\psi$ a valori in $\partial X$, ne consegue che per ogni compatto $K\subseteq X$ esiste un $\nu(K)\in\mathbb{N}$ tale che, per ogni $\nu\ge\nu(K)$, si ha $\sigma_\nu([a,b])\cap K=\emptyset$. Ma così troviamo una contraddizione, perché $X$ soddisfa la condizione di visibilità rispetto alle $(1,\kappa_0)$-simil-geodetiche.
\end{proof}