Il primo esempio di una classe di domini di $\mathbb{C}^n$ con visibilità è quello, introdotto in \cite{BZ1}, dei domini Goldilocks. Prima di darne la definizione, introduciamo, dati un dominio limitato $\Omega\subseteq\mathbb{C}^n$, un sottoinsieme $U\subseteq\mathbb{C}^n$ e $r>0$, la seguente quantità:
$$M_{\Omega,U}(r):=\sup\left\{\frac{1}{K_\Omega(x;v)}\mid x\in\Omega\cap U,\delta_\Omega(x) \le r\text{ e } \|v\|=1\right\}.$$

Poniamo $M_\Omega:=M_{\Omega,\mathbb{C}^n}$. La funzione $M_{\Omega,U}$ è monotona crescente, dunque misurabile secondo Lebesgue; inoltre, segue dal punto (3) della Proposizione \ref{metrica_bilip} che è anche limitata. Perciò ha senso la definizione che stiamo per dare.

\begin{defn} \label{gold}
    Un dominio limitato $\Omega\subseteq\mathbb{C}^n$ è detto \textit{dominio Goldilocks} se:
    \begin{enumerate}[label={(\arabic*)}]
        \item esiste (e quindi per ogni) $\epsilon>0$ tale che $\displaystyle\int_0^{\epsilon} \frac{1}{r}M_\Omega(r)\diff r<+\infty$;
        \item per ogni $x_0\in\Omega$ esistono due costanti $C,\alpha>0$ (che dipendono da $x_0$) tali che $k_\Omega(x_0,x) \le C+\alpha\log{\dfrac{1}{\delta_\Omega(x)}}$ per ogni $x\in\Omega$.
    \end{enumerate}
\end{defn}

\begin{oss}
    Il nome particolare, domini Goldilocks (Riccioli d'oro, in italiano), è dovuto al fatto che, come la protagonista della fiaba, tali domini evitano due estremi ``sgradevoli'': il bordo non ha cuspidi rivolte verso l'esterno né punti in cui è piatto fino a ordine infinito. Il primo caso è escluso dalla condizione (2) nella definizione. La condizione (1) implica che il dominio è pseudoconvesso (\cite[Proposition 2.15]{BZ1}).
\end{oss}

Per vedere che i domini Goldilocks soddisfano la condizione di visibilità, iniziamo mostrando un risultato che ci servirà anche nei prossimi esempi.

\begin{thm} \label{extvis}
    (\cite[Theorem 1.9]{CMS}) Sia $\Omega$ un dominio limitato di $\mathbb{C}^d$. Sia $S\subseteq\partial\Omega$ un insieme chiuso tale che per ogni $p,q\in\partial\Omega$ con $p\not=q$ esistono $p'\in\partial\Omega$ e $r>0$ tali, detta $B(p',r)$ la palla euclidea di centro $p'$ e raggio $r$, che:
    \begin{nlist}
        \item si ha $p\in B(p',r)$ e $q\in\partial\Omega\setminus\overline{B(p',r)}$;
        \item si ha $S\cap\partial B(p',r)=\emptyset$.
    \end{nlist}
    
    Inoltre, supponiamo che per ogni $\xi\in\partial\Omega\setminus S$ esistano un intorno $U$ di $\xi$, uno $z_0\in\Omega$ e una funzione $C^1$ strettamente crescente $f:(0,+\infty)\longrightarrow\mathbb{R}$, con $f(t)\longrightarrow+\infty$ per $t\longrightarrow+\infty$, tali che:
    \begin{enumerate}[label={(\arabic*)}]
        \item si ha $k_\Omega(z_0,z) \le f\big(1/\delta_\Omega(z)\big)$ per ogni $z\in\Omega\cap U$;
        \item si ha $M_{\Omega,U}(r)\longrightarrow 0$ per $r\longrightarrow 0$;
        \item esiste $r_0>0$ tale che $\displaystyle\int_0^{r_0}\frac{M_{\Omega,U}(r)}{r^2}f'\left(\frac{1}{r}\right)\diff r<+\infty$.
    \end{enumerate}

    Allora $\Omega$ è $(\lambda,\kappa)$-visibile per ogni $\lambda \ge 1$ e $\kappa>0$.
\end{thm}

\begin{proof}
    Supponiamo per assurdo che esistano $\lambda \ge 1$ e $\kappa>0$ tali che $\Omega$ non sia $(\lambda,\kappa)$-visibile. Allora esistono $p,q\in\partial\Omega$ con $p\not=q$, due successioni $\{p_n\}_{n\in\mathbb{N}}$ e $\{q_n\}_{n\in\mathbb{N}}$, convergenti rispettivamente a $p$ e $q$, e una successione $\{\gamma_n\}_{n\in\mathbb{N}}$ di $(\lambda,\kappa)$-simil-geodetiche, con $\gamma_n:[a_n,b_n]\longrightarrow\Omega$ e $\gamma_n(a_n)=p_n$ e $\gamma_n(b_n)=q_n$ per ogni $n\in\mathbb{N}$, tali che
    $$\max_{a_n\le t\le b_n}\delta_\Omega\big(\gamma_n(t)\big)\longrightarrow 0\text{ per }n\longrightarrow+\infty.$$

    Per ipotesi esistono $p'\in\partial\Omega$ e $r>0$ tali che valgano (i) e (ii). Poiché $p_n\longrightarrow p$ e $q_n\longrightarrow q$ per $n\longrightarrow+\infty$, possiamo assumere senza perdita di generalità che $\{p_n\}_{n\in\mathbb{N}}\subseteq B(p',r)$ e $\{q_n\}_{n\in\mathbb{N}}\subseteq \Omega\setminus\overline{B(p',r)}$. Poiché $\gamma_n$ è una curva continua che collega $p_n$ e $q_n$, deve esistere $\alpha_n\in(a_n,b_n)$ tale che $\xi_n:=\gamma_n(\alpha_n)\in\partial B(p',r)$; a meno di sottosuccessioni, possiamo assumere che $\xi_n\longrightarrow\xi\in\partial\Omega\cap\partial B(p',r)$ per $n\longrightarrow+\infty$. Per (ii) si ha che $\xi\in\partial\Omega\setminus S$; allora esistono, per ipotesi, un intorno $U$ di $\xi$, uno $z_0\in\Omega$ e una funzione $C^1$ strettamente crescente $f:(0,+\infty)\longrightarrow\mathbb{R}$ tali che valgano (1), (2) e (3). Osserviamo che tali ipotesi sono ancora soddisfatte se prendiamo un intorno $V \subseteq U$ di $\xi$, per cui, a meno di prendere un intorno più piccolo, possiamo supporre che $\overline{U}\cap(S\cup\{p,q\})=\emptyset$; inoltre, di nuovo a meno di sottosuccessioni, possiamo anche supporre che $q_n\not\in\overline{U}$.

    Sia $\epsilon>0$ tale che $\overline{B(\xi,\epsilon)}\subseteq U$; poiché $\xi_n\longrightarrow \xi$ per $n\longrightarrow+\infty$, possiamo assumere senza perdita di generalità che $\{\xi_n\}_{n\in\mathbb{N}}\subseteq B(\xi,\epsilon)$. Per ogni $n\in\mathbb{N}$ poniamo
    $$\beta_n:=\inf\{t\in[\alpha_n,b_n]\mid \gamma_n(t)\in\partial B(\xi,\epsilon)\};$$
    per definizione di $\beta_n$ e per il fatto che $\partial B(\xi,\epsilon)$ è chiuso, si ha $\gamma_n(\beta_n)\in\partial B(\xi,\epsilon)$ e $a_n<\alpha_n<\beta_n<b_n$. Poniamo $\sigma_n:=\gamma_n\restrict{[\alpha_n,\beta_n]}:[\alpha_n,\beta_n]\longrightarrow\Omega$ per ogni $n\in\mathbb{N}$. Allora, poiché $\gamma_n(\alpha_n)=\xi_n\in B(\xi,\epsilon)$, per definizione di $\beta_n$ dev'essere $\sigma_n([\alpha_n,\beta_n])\subseteq\overline{B(\xi,\epsilon)}\subseteq U$. Notiamo che, essendo la restrizione della $(\lambda,\kappa)$-simil-geodetica $\gamma_n$, anche $\sigma_n$ è una $(\lambda,\kappa)$-simil-geodetica per ogni $n\in\mathbb{N}$; inoltre, si ha
    $$\max_{\alpha_n\le t\le \beta_n}\delta_\Omega\big(\sigma_n(t)\big)\le\max_{a_n\le t\le b_n}\delta_\Omega\big(\gamma_n(t)\big)\longrightarrow 0\text{ per }n\longrightarrow+\infty.$$

    Adesso, a meno di riparametrizzare le curve $\sigma_n$, possiamo assumere che $\alpha_n\le 0\le\beta_n$ e che
    $$\max_{\alpha_n\le t\le \beta_n}\delta_\Omega\big(\sigma_n(t)\big)=\delta_\Omega\big(\sigma_n(0)\big)$$
    per ogni $n\in\mathbb{N}$. Per la Proposizione \ref{simillip} esiste una costante $C>0$, che dipende solo da $\lambda$, tale che le $(\lambda,\kappa)$-simil-geodetiche di $\Omega$ sono $C$-lipschitziane rispetto alla distanza euclidea. Allora, applicando il teorema di Ascoli-Arzelà e passando a un'opportuna sottosuccessione con un procedimento diagonale, possiamo assumere che:
    \begin{itemize}
        \item si ha $\alpha_n\longrightarrow\alpha\in[-\infty,0]$ e $\beta_n\longrightarrow\beta\in[0,+\infty]$ per $n\longrightarrow+\infty$;
        \item la successione $\{\sigma_n\}_{n\in\mathbb{N}}$ converge uniformemente sui compatti di $(\alpha,\beta)$ a una curva continua $\sigma:(\alpha,\beta)\longrightarrow \overline{B(\xi,\epsilon)}\subseteq U$;
        \item si ha $\sigma_n(\alpha_n)=\xi_n\longrightarrow\xi$ e $\sigma_n(\beta_n)=\eta_n\longrightarrow\eta$ per $n\longrightarrow+\infty$, con $\xi\in\partial\Omega\cap\partial B(p',r)$ e $\eta\in\partial\Omega\cap\partial B(\xi,\epsilon)$.
    \end{itemize}

    Ovviamente dev'essere $\xi\not=\eta$; quindi, dato che $\|\sigma(\alpha_n)-\sigma(\beta_n)\| \le C(\beta_n-\alpha_n)$ per ogni $n\in\mathbb{N}$, si ha che $C(\beta-\alpha)\ge\|\xi-\eta\|>0$, per cui $\beta>\alpha$. \\

    Mostriamo adesso che $\sigma$ è costante. Vediamo innanzitutto che per ogni $n\in\mathbb{N}$ e per quasi ogni $t\in(\alpha_n,\beta_n)$ si ha
    $$\|\sigma_n'(t)\| \le \lambda M_{\Omega,U}\Big(\delta_\Omega\big(\sigma_n(t)\big)\Big).$$
    
    Sia $t\in(\alpha_n,\beta_n)$ tale che $\sigma_n'(t)$ esiste e $\|\sigma_n'(t)\|\not=0$ (altrimenti la disuguaglianza è immediata). Ricordiamo che ogni $\sigma_n$ è una $(\lambda,\kappa)$-simil-geodetica, per cui $K_\Omega\big(\sigma_n(t);\sigma_n'(t)\big)\le\lambda$; quindi, dato che $\sigma_n([\alpha_n,\beta_n])\subseteq U$, si ha
    $$\|\sigma_n'(t)\|\le \frac{\lambda}{K_\Omega\left(\sigma_n(t);\frac{\sigma_n'(t)}{\|\sigma_n'(t)\|}\right)}\le\lambda M_{\Omega,U}\Big(\delta_\Omega\big(\sigma_n(t)\big)\Big),$$
    come voluto. Adesso, poiché $\displaystyle\max_{\alpha_n\le t\le \beta_n}\delta_\Omega\big(\sigma_n(t)\big)\longrightarrow 0$ per $n\longrightarrow+\infty$, dal fatto che $M_{\Omega,U}$ è crescente e dall'ipotesi (2) si ha che $M_{\Omega,U}\Big(\delta_\Omega\big(\sigma_n(t)\big)\Big)\longrightarrow 0$ uniformemente. Ma allora, dati $\alpha<u\le w<\beta$, si ha
    \begin{align*}
        \|\sigma(u)-\sigma(w)\|&\le \lim_{n\longrightarrow+\infty}\|\sigma_n(u)-\sigma_n(w)\| \\
        =&\lim_{n\longrightarrow+\infty}\left\|\int_u^w\sigma_n'(t)\diff t\right\|\le \limsup_{n\longrightarrow+\infty} \int_u^w \|\sigma_n'(t)\|\diff t\\
        &\le \lambda\limsup_{n\longrightarrow+\infty} \int_u^w M_{\Omega,U}\Big(\delta_\Omega\big(\sigma_n(t)\big)\Big)\diff t=0;
    \end{align*}
    dunque $\sigma$ è costante su $(\alpha,\beta)$. \\

    Vogliamo ottenere una contraddizione mostrando anche che $\sigma$ non è costante. Distinguiamo due casi.

    Caso 1: sia $\alpha$ che $\beta$ sono finiti. Per ogni $n\in\mathbb{N}$, definiamo al seguente modo la curva $\tilde{\sigma}_n:[\alpha,\beta]\longrightarrow\Omega$: restringiamo $\sigma_n$ all'intervallo $[\alpha_n,\beta_n]\cap[\alpha,\beta]$ ed estendiamola ad una costante sugli intervalli $[\alpha,\alpha_n]$ e $[\beta_n,\beta]$ se $\alpha<\alpha_n$ o $\beta_n<\beta$. È facile vedere che le $\tilde{\sigma}_n$ sono ancora $C$-lipschitziane, per cui, applicando di nuovo Ascoli-Arzelà, a meno di sottosuccessioni convergono, uniformemente sui compatti, a una curva continua $\tilde{\sigma}:[\alpha,\beta]\longrightarrow\overline{\Omega}$. Questa curva estende in modo continuo $\sigma$ a tutto $[\alpha,\beta]$ e $\tilde{\sigma}(\alpha)=\xi\not=\eta=\tilde{\sigma}(\beta)$, per cui $\tilde{\sigma}$ non è costante, e di conseguenza non lo è neanche $\sigma$.

    Caso 2: $\alpha=-\infty$ o $\beta=+\infty$. Ricordiamo che $\sigma_n$ è una $(\lambda,\kappa)$-simil-geodetica per ogni $n\in\mathbb{N}$; dunque si ha
    \begin{equation}\label{disconf}
        \begin{aligned}
            \frac{1}{\lambda}|t|-\kappa&\le k_\Omega\big(\sigma_n(0),\sigma_n(t)\big)\\
            &\le k_\Omega\big(\sigma_n(0),z_0\big)+k_\Omega\big(z_0,\sigma_n(t)\big) \le 2f\left(\frac{1}{\delta_\Omega\big(\sigma_n(t)\big)}\right),
        \end{aligned}
    \end{equation}
    dove l'ultima disuguaglianza segue dall'ipotesi (1), dal fatto che per ogni $n\in\mathbb{N}$ si ha $\sigma_n([\alpha_n,\beta_n])\subseteq\Omega\cap U$ e da $\displaystyle\max_{\alpha_n\le t\le \beta_n}\delta_\Omega\big(\sigma_n(t)\big)=\delta_\Omega\big(\sigma_n(0)\big)$.

    Consideriamo il caso $\beta=+\infty$. Poiché la successione $\{\beta_n\}_{n\in\mathbb{N}}$ diverge a $+\infty$ e $f$ è continua con $\displaystyle\lim_{n\longrightarrow+\infty}f(t)=+\infty$, esistono un naturale $N\in\mathbb{N}$ e una costante $B>0$ tali che per ogni $n\ge N$ e $t\in(B,\beta_n]$ si ha $\dfrac{|t|}{2\lambda}-\dfrac{\kappa}{2}\in f\big((0,+\infty)\big)$. Usando anche la disuguaglianza \eqref{disconf} e il fatto che $f$ è strettamente crescente, troviamo che
    $$f^{-1}\left(\frac{|t|}{2\lambda}-\frac{\kappa}{2}\right) \le \frac{1}{\delta_\Omega\big(\sigma_n(t)\big)}$$
    per ogni $n\ge N$ e $t\in(B,\beta_n]$. Se $\alpha=-\infty$, ragionando allo stesso modo troviamo un intero $N'$ e una costante $A>0$ tali che
    $$f^{-1}\left(\frac{|t|}{2\lambda}-\frac{\kappa}{2}\right) \le \frac{1}{\delta_\Omega\big(\sigma_n(t)\big)}$$
    per ogni $n\ge N'$ e $t\in[\alpha_n,-A)$. Vediamo il caso $\alpha=-\infty$ e $\beta=+\infty$. Dalle due disuguaglianze appena mostrate, usando anche che $\|\sigma_n'(t)\| \le \lambda M_{\Omega,U}\Big(\delta_\Omega\big(\sigma_n(t)\big)\Big)$ e che $M_{\Omega,U}$ è crescente, troviamo che
    $$\|\sigma_n'(t)\| \le \lambda M_{\Omega,U}\left(\frac{1}{f^{-1}\left(\frac{|t|}{2\lambda}-\frac{\kappa}{2}\right)}\right)$$
    per ogni $n\ge\max\{N,N'\}$ e per quasi ogni $t\in[\alpha_n,-A)\cup(B,\beta_n]$. Usando l'ipotesi (3) e il cambio di variabile $r=\dfrac{1}{f^{-1}\left(\frac{|t|}{2\lambda}-\frac{\kappa}{2}\right)}$, si ha che esistono due costanti $c\in(-\infty,-A)$ e $d\in(B,+\infty)$ tali che
    \begin{align*}
        \lambda\int_{-\infty}^c M_{\Omega,U}\left(\frac{1}{f^{-1}\left(\frac{|t|}{2\lambda}-\frac{\kappa}{2}\right)}\right)\diff t&+\lambda\int_d^{+\infty}M_{\Omega,U}\left(\frac{1}{f^{-1}\left(\frac{|t|}{2\lambda}-\frac{\kappa}{2}\right)}\right)\diff t\\
        &<\|\xi-\eta\|.
    \end{align*}

    Allora, usando le ultime due disuguaglianze, otteniamo
    \begin{align*}
        \|\sigma(d)-\sigma(c)\|&=\lim_{n\longrightarrow+\infty}\|\sigma_n(d)-\sigma_n(c)\|\\
        &\ge\begin{aligned}[t]\liminf_{n\longrightarrow+\infty}(&\|\sigma_n(\beta_n)-\sigma_n(\alpha_n)\|-\|\sigma_n(\alpha_n)-\sigma_n(c)\|\\
        &-\|\sigma_n(\beta_n)-\sigma_n(d)\|)\end{aligned}\\
        &\ge\begin{aligned}[t]&\lim_{n\longrightarrow+\infty}\|\sigma_n(\beta_n)-\sigma_n(\alpha_n)\|-\limsup_{n\longrightarrow+\infty}\left\|\int_{\alpha_n}^c \sigma_n'(t)\diff t\right\|\\
        &-\limsup_{n\longrightarrow+\infty}\left\|\int_d^{\beta_n} \sigma_n'(t)\diff t\right\|\end{aligned}\\
        &\ge\|\xi-\eta\|-\limsup_{n\longrightarrow+\infty}\int_{\alpha_n}^c \|\sigma_n'(t)\|\diff t-\limsup_{n\longrightarrow+\infty}\int_d^{\beta_n} \|\sigma_n'(t)\|\diff t\\
        &\ge\begin{aligned}[t]\|\xi-\eta\|&-\limsup_{n\longrightarrow+\infty}\lambda\int_{\alpha_n}^c M_{\Omega,U}\left(\frac{1}{f^{-1}\left(\frac{|t|}{2\lambda}-\frac{\kappa}{2}\right)}\right)\diff t\\
        &-\limsup_{n\longrightarrow+\infty}\lambda\int_d^{\beta_n} M_{\Omega,U}\left(\frac{1}{f^{-1}\left(\frac{|t|}{2\lambda}-\frac{\kappa}{2}\right)}\right)\diff t\end{aligned}\\
        &=\begin{aligned}[t]\|\xi-\eta\|&-\lambda\int_{-\infty}^c M_{\Omega,U}\left(\frac{1}{f^{-1}\left(\frac{|t|}{2\lambda}-\frac{\kappa}{2}\right)}\right)\diff t\\
        &-\lambda\int_d^{+\infty} M_{\Omega,U}\left(\frac{1}{f^{-1}\left(\frac{|t|}{2\lambda}-\frac{\kappa}{2}\right)}\right)\diff t>0;\end{aligned}
    \end{align*}
    dunque in questo caso $\sigma$ non è costante. Se invece $a<-\infty$ e $b=+\infty$ (il caso $a=-\infty$ e $b<+\infty$ è analogo), ragionando come nel caso 1 estendiamo le $\sigma_n$ a delle $\tilde{\sigma}_n$ che, a meno di sottosuccessioni, convergono uniformemente sui compatti a una curva continua $\tilde{\sigma}:[\alpha,+\infty)\longrightarrow\overline{\Omega}$ che estende $\sigma$. Allora basta ripetere la stima precedente con $\tilde{\sigma}_n, \tilde{\sigma}$ e $\alpha$ al posto di $\sigma_n, \sigma$ e $c$ e scegliendo un $d$ opportuno, trovando così che $\tilde{\sigma}$, e di conseguenza $\sigma$, non è costante. \\

    Poiché la nostra assunzione porta a una contraddizione, dev'essere falsa, da cui la tesi.
\end{proof}

Adesso mostriamo che i domini Goldilocks sono visibili per le simil-geodetiche.

\begin{cor} \label{gold_is_vis}
    Un dominio limitato Goldilocks $\Omega\subseteq\mathbb{C}^n$ è $(\lambda,\kappa)$-visibile per ogni $\lambda\ge 1$ e $\kappa\ge 0$.
\end{cor}

\begin{proof}
    Basta applicare il Teorema \ref{extvis} con $S=\emptyset$, $U=\mathbb{C}^n$, per cui abbiamo che $f(x)=C+\alpha\log{x}$, con $C$ e $\alpha$ dati dal punto (2) della Definizione \ref{gold} per un punto fissato.
\end{proof}

Un esempio esplicito di domini Goldilocks sono i domini limitati, pseudoconvessi e di tipo finito.