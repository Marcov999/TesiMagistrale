Il secondo esempio, introdotto in \cite{BM}, è quello dei domini Caltrops.

\begin{defn}
    Un dominio limitato $\Omega\subseteq\mathbb{C}^n$, con $n\ge 2$, è detto \textit{dominio Caltrop} se esiste un insieme finito di punti $\{q_1,\dots,q_N\}\subseteq\partial\Omega$ tale che:
    \begin{itemize}
        \item il sottoinsieme del bordo $\partial\Omega\setminus\{q_1,\dots,q_N\}$ è $C^2$ e $\Omega$ è strettamente pseudoconvesso in ogni punto di tale insieme;
        \item per ogni $j=1,\dots, N$ esiste un intorno aperto e connesso $V_j\ni q_j$ tale che esistono due costanti $p_j\in(1,3/2)$ e $C_j>1$, una trasformazione unitaria $\mathbb{U}^{(j)}$ e una funzione continua $\psi_j:[0,A_j]\longrightarrow[0,+\infty)$, con $A_j>0$, tali che $\mathbb{U}_j(\Omega\cap V_j)$ è un ``solido di rivoluzione'' dato da
        \begin{align*}
            \mathbb{U}_j(\Omega\cap V_j)=&\Bigg\{(z_1,\dots,z_n)\in\mathbb{C}^n\mid \mathfrak{Re}z_n\in (0,A_j),\\
            &\left.(\mathfrak{Im}z_n)^2+\sum_{j=1}^{n-1}|z_j|^2 \le \psi_j(\mathfrak{Re}z_n)^2\right\}
        \end{align*}
        dove $\mathbb{U}_j(z)=\mathbb{U}^{(j)}(z-q_j)$ per ogni $z\in\mathbb{C}^n$. Inoltre, $\psi_j$ ha le seguenti proprietà:
        \begin{itemize}
            \item è di classe $C^2$ su $(0,A_j)$;
            \item per ogni $x\in[0,A_j]$ si ha $(1/C_j)x^{p_j} \le \psi_j(x) \le C_jx^{p_j}$;
            \item si ha che $\psi_j$ è strettamente crescente e $\psi_j'$ è crescente su $(0,A_j)$;
            \item si ha $\displaystyle\lim_{x\longrightarrow0^+}\psi_j(x)\psi_j''(x)=0$.
        \end{itemize}
    \end{itemize}
\end{defn}

\begin{oss}
    Il nome, che in italiano può essere tradotto come tribolo o ``piede di corvo'' (un'arma da lancio a quattro punte), rimanda al fatto che, vicino ai punti in cui non è liscio, il bordo di tali domini assume una forma simile a quella di una cuspide hölderiana non eccessivamente appuntita.
\end{oss}

Ci occupiamo adesso di mostrare che i domini Caltrops esistono. Vediamo l'esempio di un dominio Caltrop con una sola punta in $\mathbb{C}^2$. Siano $A,\beta>0$ e sia $\psi:[-A,\beta]\longrightarrow[0,+\infty)$ una funzione continua di classe $C^2$ su $(-A,\beta)$ tale che:
\begin{enumerate}[label={(\arabic*)}]
    \item per ogni $t\in(-A,-B)$ si ha $\psi(t)=(t+A)^p$;
    \item per ogni $t\in(0,\beta)$ si ha $\psi(t)=\sqrt{\beta^2-t^2}$,
\end{enumerate}
dove $B\in(0,A)$ e $p\in(1,3/2)$. Consideriamo il ``solido di rivoluzione'' dato da
$$\Omega:=\{(z,w)\in\mathbb{C}^2\mid |z|^2+|\mathfrak{Im}w|^2<C\psi(\mathfrak{Re}w)^2,-A<\mathfrak{Re}w<\beta\},$$
dove $C>0$ è una costante che sceglieremo più avanti. Poniamo inoltre
$$\rho(z,w):=|z|^2+|\mathfrak{Im}w|^2-C\psi(\mathfrak{Re}w)^2,$$
considerata sull'insieme $\{(z,w)\in\mathbb{C}^2\mid -A<\mathfrak{Re}w<\beta+\epsilon\}$, dove $\epsilon>0$ è fissato e $\psi^2$ è estesa nel modo ovvio su $(\beta,\epsilon)$. Si verifica che $\rho$ è una funzione $C^2$ avente l'ipersuperficie reale $\partial\Omega\cap\{(z,w)\in\mathbb{C}^2\mid -A<\mathfrak{Re}w\}$ come luogo di zeri. Calcoliamo le seguenti derivati parziali seconde:
\begin{gather*}
    \partial^2_{z\bar{z}}\rho\equiv 1;\\
    \partial^2_{z\bar{w}}\rho=\partial^2_{\bar{z}w}\rho\equiv 0;\\
    \partial^2_{w\bar{w}}\rho(z,w)=\frac{1}{2}-\frac{C}{2}\big(\psi''(\mathfrak{Re}w)\psi(\mathfrak{Re}w)+\psi'(\mathfrak{Re}w)^2\big).
\end{gather*}

In particolare, si ha che
$$\partial^2_{w\bar{w}}\rho(z,w)-\frac{1}{2}=-\frac{Cp(2p-1)}{2}(\mathfrak{Re}w+A)^{2(p-1)}$$
per $\mathfrak{Re}w$ sufficientemente vicino a $-A$, che tende crescendo a $0$ per $\mathfrak{Re}w$ che tende descrescendo a $-A$. Allora, essendo $\psi$ di classe $C^2$ su $(-A,\beta)$, scegliendo $C$ sufficientemente piccolo possiamo imporre che $\partial^2_{w\bar{w}}\rho(z,w)\ge\dfrac{1}{4}$ per ogni $w$ tale che $-A<\mathfrak{Re}w\le 0$. Segue che $\partial\Omega\cap\{(z,w)\in\mathbb{C}^2\mid -A<\mathfrak{Re}w\le 0\}$ è un sottoinsieme di punti strettamente pseudoconvessi del bordo di $\Omega$. Per la condizione (2) su $\psi$, anche $\partial\Omega\cap\{(z,w)\in\mathbb{C}^2\mid \mathfrak{Re}w>0\}$ lo è. Le altre proprietà di dominio Caltrop seguono dalla condizione (1) su $\psi$; la punta è in $(0,-A)$.

In \cite[Section 3.2]{CMS} vengono costruiti domini Caltrops con un numero arbitrario di punte. \\

Vediamo adesso che i domini Caltrops hanno le proprietà volute. Cominciamo con la condizione di visibilità, mostrando un risultato che ci servirà anche nel prossimo esempio. Poniamo
$$M_{\Omega,U}(r):=\sup\left\{\frac{1}{K_\Omega(x;v)}\mid x\in\Omega\cap U,\delta_\Omega(x) \le r, \|v\|=1\right\}.$$

\begin{thm} \label{extvis}
    (\cite[Theorem 1.9]{CMS}) Sia $\Omega$ un dominio limitato di $\mathbb{C}^d$. Sia $S\subseteq\partial\Omega$ un insieme chiuso tale che per ogni $p,q\in\partial\Omega$ con $p\not=q$ esistono $p'\in\partial\Omega$ e $r>0$ tali, detta $B(p',r)$ la palla euclidea di centro $p'$ e raggio $r$, che:
    \begin{nlist}
        \item si ha $p\in B(p',r)$ e $q\in\partial\Omega\setminus\overline{B(p',r)}$;
        \item si ha $S\cap\partial B(p',r)=\emptyset$.
    \end{nlist}
    
    Inoltre, supponiamo che per ogni $q'\in\partial\Omega\setminus S$ esistono un intorno $U$ di $q'$, uno $z_0\in\Omega$ e una funzione $C^1$ strettamente crescente $f:(0,+\infty)\longrightarrow\mathbb{R}$, con $f(t)\longrightarrow+\infty$ per $t\longrightarrow+\infty$, tali che:
    \begin{enumerate}[label={(\arabic*)}]
        \item si ha $k_\Omega(z_0,z) \le f\big(1/\delta_\Omega(z)\big)$ per ogni $z\in\Omega\cap U$;
        \item si ha $M_{\Omega,U}(r)\longrightarrow 0$ per $r\longrightarrow 0$;
        \item esiste $r_0>0$ tale che $\displaystyle\int_0^{r_0}\frac{M_{\Omega,U}(r)}{r^2}f'\left(\frac{1}{r}\right)\diff r<+\infty$.
    \end{enumerate}

    Allora $\Omega$ è $(\lambda,\kappa)$-visibile per ogni $\lambda \ge 1$ e $\kappa>0$.
\end{thm}

\begin{proof}
    La dimostrazione è molto simile a quella del Teorema \ref{gold_is_vis}. Supponiamo per assurdo che esistano $\lambda \ge 1$ e $\kappa>0$ tali che $\Omega$ non sia $(\lambda,\kappa)$-visibile. Allora esistono $p,q\in\partial\Omega$ con $p\not=q$, due successioni $\{p_n\}_{n\in\mathbb{N}}$ e $\{q_n\}_{n\in\mathbb{N}}$, convergenti rispettivamente a $p$ e $q$, e una successione $\{\gamma_n\}_{n\in\mathbb{N}}$ di $(\lambda,\kappa)$-simil-geodetiche, con $\gamma_n:[a_n,b_n]\longrightarrow\Omega$ e $\gamma_n(a_n)=p_n$ e $\gamma_n(b_n)=q_n$ per ogni $n\in\mathbb{N}$, tali che
    $$\max_{a_n\le t\le b_n}\delta_\Omega\big(\gamma_n(t)\big)\longrightarrow 0\text{ per }n\longrightarrow+\infty.$$

    Per ipotesi esistono $p'\in\partial\Omega$ e $r>0$ tali che valgano (i) e (ii). Poiché $p_n\longrightarrow p$ e $q_n\longrightarrow q$ per $n\longrightarrow+\infty$, possiamo assumere senza perdita di generalità che $\{p_n\}_{n\in\mathbb{N}}\subseteq B(p',r)$ e $\{q_n\}_{n\in\mathbb{N}}\subseteq \Omega\setminus\overline{B(p',r)}$. Poiché $\gamma_n$ è una curva continua che collega $p_n$ e $q_n$, deve esistere $\alpha_n\in(a_n,b_n)$ tale che $\xi_n:=\gamma_n(\alpha_n)\in\partial B(p',r)$; a meno di sottosuccessioni, possiamo assumere che $\xi_n\longrightarrow\xi\in\partial\Omega\cap\partial B(p',r)$ per $n\longrightarrow+\infty$. Per (ii) si ha che $\xi\in\partial\Omega\setminus S$; allora esistono, per ipotesi, un intorno $U$ di $\xi$, uno $z_0\in\Omega$ e una funzione $C^1$ strettamente crescente $f:(0,+\infty)\longrightarrow\mathbb{R}$ tali che valgano (1), (2) e (3). Osserviamo che tali ipotesi sono ancora soddisfatte se prendiamo un intorno $V \subseteq U$ di $\xi$, per cui, a meno di prendere un intorno più piccolo, possiamo supporre che $\overline{U}\cap(S\cup\{p,q\})=\emptyset$; inoltre, di nuovo a meno di sottosuccessioni, possiamo anche supporre che $q_n\not\in\overline{U}$.

    Sia $\epsilon>0$ tale che $\overline{B(\xi,\epsilon)}\subseteq U$; poiché $\xi_n\longrightarrow \xi$ per $n\longrightarrow+\infty$, possiamo assumere senza perdita di generalità che $\{\xi_n\}_{n\in\mathbb{N}}\subseteq B(\xi,\epsilon)$. Per ogni $n\in\mathbb{N}$ poniamo
    $$\beta_n:=\inf\{t\in[\alpha_n,b_n]\mid \gamma_n(t)\in\partial B(\xi,\epsilon)\};$$
    per definizione di $\beta_n$ e per il fatto che $\partial B(\xi,\epsilon)$ è chiuso, si ha $\gamma_n(\beta_n)\in\partial B(\xi,\epsilon)$ e $a_n<\alpha_n<\beta_n<b_n$. Poniamo $\sigma_n:=\gamma_n\restrict{[\alpha_n,\beta_n]}:[\alpha_n,\beta_n]\longrightarrow\Omega$ per ogni $n\in\mathbb{N}$. Allora, poiché $\gamma_n(\alpha_n)=\xi_n\in B(\xi,\epsilon)$, per definizione di $\beta_n$ dev'essere $\sigma_n([\alpha_n,\beta_n])\subseteq\overline{B(\xi,\epsilon)}\subseteq U$. Notiamo che, essendo la restrizione della $(\lambda,\kappa)$-simil-geodetica $\gamma_n$, anche $\sigma_n$ è una $(\lambda,\kappa)$-simil-geodetica per ogni $n\in\mathbb{N}$; inoltre, si ha
    $$\max_{\alpha_n\le t\le \beta_n}\delta_\Omega\big(\sigma_n(t)\big)\le\max_{a_n\le t\le b_n}\delta_\Omega\big(\gamma_n(t)\big)\longrightarrow 0\text{ per }n\longrightarrow+\infty.$$

    Adesso, a meno di riparametrizzare le curve $\sigma_n$, possiamo assumere che $\alpha_n\le 0\le\beta_n$ e che
    $$\max_{\alpha_n\le t\le \beta_n}\delta_\Omega\big(\sigma_n(t)\big)=\delta_\Omega\big(\sigma_n(0)\big)$$
    per ogni $n\in\mathbb{N}$. Per la Proposizione \ref{simillip} esiste una costante $C>0$, che dipende solo da $\lambda$, tale che le $(\lambda,\kappa)$-simil-geodetiche di $\Omega$ sono $C$-lipschitziane rispetto alla distanza euclidea. Allora, applicando il teorema di Ascoli-Arzelà e passando a un'opportuna sottosuccessione con un procedimento diagonale, possiamo assumere che:
    \begin{itemize}
        \item si ha $\alpha_n\longrightarrow\alpha\in[-\infty,0]$ e $\beta_n\longrightarrow\beta\in[0,+\infty]$ per $n\longrightarrow+\infty$;
        \item la successione $\{\sigma_n\}_{n\in\mathbb{N}}$ converge uniformemente sui compatti di $(\alpha,\beta)$ a una curva continua $\sigma:(\alpha,\beta)\longrightarrow \overline{B(\xi,\epsilon)}\subseteq U$;
        \item si ha $\sigma_n(\alpha_n)=\xi_n\longrightarrow\xi$ e $\sigma_n(\beta_n)=\eta_n\longrightarrow\eta$ per $n\longrightarrow+\infty$, con $\xi\in\partial\Omega\cap\partial B(p',r)$ e $\eta\in\partial\Omega\cap\partial B(\xi,\epsilon)$.
    \end{itemize}

    Ovviamente dev'essere $\xi\not=\eta$; quindi, dato che $\|\sigma(\alpha_n)-\sigma(\beta_n)\| \le C(\beta_n-\alpha_n)$ per ogni $n\in\mathbb{N}$, si ha che $C(\beta-\alpha)\ge\|\xi-\eta\|>0$, per cui $\beta>\alpha$. \\

    Mostriamo adesso che $\sigma$ è costante. Vediamo innanzitutto che per ogni $n\in\mathbb{N}$ e per quasi ogni $t\in(\alpha_n,\beta_n)$ si ha
    $$\|\sigma_n'(t)\| \le \lambda M_{\Omega,U}\Big(\delta_\Omega\big(\sigma_n(t)\big)\Big).$$
    
    Sia $t\in(\alpha_n,\beta_n)$ tale che $\sigma_n'(t)$ esiste e $\|\sigma_n'(t)\|\not=0$ (altrimenti la disuguaglianza è immediata). Ricordiamo che ogni $\sigma_n$ è una $(\lambda,\kappa)$-simil-geodetica, per cui $K_\Omega\big(\sigma_n(t);\sigma_n'(t)\big)\le\lambda$; quindi, dato che $\sigma_n([\alpha_n,\beta_n])\subseteq U$, si ha
    $$\|\sigma_n'(t)\|\le \frac{\lambda}{K_\Omega\left(\sigma_n(t);\frac{\sigma_n'(t)}{\|\sigma_n'(t)\|}\right)}\le\lambda M_{\Omega,U}\Big(\delta_\Omega\big(\sigma_n(t)\big)\Big),$$
    come voluto. Adesso, poiché $\displaystyle\max_{\alpha_n\le t\le \beta_n}\delta_\Omega\big(\sigma_n(t)\big)\longrightarrow 0$ per $n\longrightarrow+\infty$, dal fatto che $M_{\Omega,U}$ è crescente e dall'ipotesi (2) si ha che $M_{\Omega,U}\Big(\delta_\Omega\big(\sigma_n(t)\big)\Big)\longrightarrow 0$ uniformemente. Ma allora, dati $\alpha<u\le w<\beta$, si ha
    \begin{align*}
        \|\sigma(u)-\sigma(w)\|&\le \lim_{n\longrightarrow+\infty}\|\sigma_n(u)-\sigma_n(w)\| \\
        =&\lim_{n\longrightarrow+\infty}\left\|\int_u^w\sigma_n'(t)\diff t\right\|\le \limsup_{n\longrightarrow+\infty} \int_u^w \|\sigma_n'(t)\|\diff t\\
        &\le \lambda\limsup_{n\longrightarrow+\infty} \int_u^w M_{\Omega,U}\Big(\delta_\Omega\big(\sigma_n(t)\big)\Big)\diff t=0;
    \end{align*}
    dunque $\sigma$ è costante su $(\alpha,\beta)$. \\

    Vogliamo ottenere una contraddizione mostrando anche che $\sigma$ non è costante. Distinguiamo due casi.

    Caso 1: sia $\alpha$ che $\beta$ sono finiti. Per ogni $n\in\mathbb{N}$, definiamo al seguente modo la curva $\tilde{\sigma}_n:[\alpha,\beta]\longrightarrow\Omega$: restringiamo $\sigma_n$ all'intervallo $[\alpha_n,\beta_n]\cap[\alpha,\beta]$ ed estendiamola ad una costante sugli intervalli $[\alpha,\alpha_n]$ e $[\beta_n,\beta]$ se $\alpha<\alpha_n$ o $\beta_n<\beta$. È facile vedere che le $\tilde{\sigma}_n$ sono ancora $C$-lipschitziane, per cui, applicando di nuovo Ascoli-Arzelà, a meno di sottosuccessioni convergono, uniformemente sui compatti, a una curva continua $\tilde{\sigma}:[\alpha,\beta]\longrightarrow\overline{\Omega}$. Questa curva estende in modo continuo $\sigma$ a tutto $[\alpha,\beta]$ e $\tilde{\sigma}(\alpha)=\xi\not=\eta=\tilde{\sigma}(\beta)$, per cui $\tilde{\sigma}$ non è costante, e di conseguenza non lo è neanche $\sigma$.

    Caso 2: $\alpha=-\infty$ o $\beta=+\infty$. Ricordiamo che $\sigma_n$ è una $(\lambda,\kappa)$-simil-geodetica per ogni $n\in\mathbb{N}$; dunque si ha
    \begin{equation}\label{disconf}
        \begin{aligned}
            \frac{1}{\lambda}|t|-\kappa&\le k_\Omega\big(\sigma_n(0),\sigma_n(t)\big)\\
            &\le k_\Omega\big(\sigma_n(0),z_0\big)+k_\Omega\big(z_0,\sigma_n(t)\big) \le 2f\left(\frac{1}{\delta_\Omega\big(\sigma_n(t)\big)}\right),
        \end{aligned}
    \end{equation}
    dove l'ultima disuguaglianza segue dall'ipotesi (1), dal fatto che per ogni $n\in\mathbb{N}$ si ha $\sigma_n([\alpha_n,\beta_n])\subseteq\Omega\cap U$ e da $\displaystyle\max_{\alpha_n\le t\le \beta_n}\delta_\Omega\big(\sigma_n(t)\big)=\delta_\Omega\big(\sigma_n(0)\big)$.

    Consideriamo il caso $\beta=+\infty$. Poiché la successione $\{\beta_n\}_{n\in\mathbb{N}}$ diverge a $+\infty$ e $f$ è continua con $\displaystyle\lim_{n\longrightarrow+\infty}f(t)=+\infty$, esistono un naturale $N\in\mathbb{N}$ e una costante $B>0$ tali che per ogni $n\ge N$ e $t\in(B,\beta_n]$ si ha $\dfrac{|t|}{2\lambda}-\dfrac{\kappa}{2}\in f\big((0,+\infty)\big)$. Usando anche la disuguaglianza \eqref{disconf} e il fatto che $f$ è strettamente crescente, troviamo che
    $$f^{-1}\left(\frac{|t|}{2\lambda}-\frac{\kappa}{2}\right) \le \frac{1}{\delta_\Omega\big(\sigma_n(t)\big)}$$
    per ogni $n\ge N$ e $t\in(B,\beta_n]$. Se $\alpha=-\infty$, ragionando allo stesso modo troviamo un intero $N'$ e una costante $A>0$ tali che
    $$f^{-1}\left(\frac{|t|}{2\lambda}-\frac{\kappa}{2}\right) \le \frac{1}{\delta_\Omega\big(\sigma_n(t)\big)}$$
    per ogni $n\ge N'$ e $t\in[\alpha_n,-A)$. Vediamo il caso $\alpha=-\infty$ e $\beta=+\infty$. Dalle due disuguaglianze appena mostrate, usando anche che $\|\sigma_n'(t)\| \le \lambda M_{\Omega,U}\Big(\delta_\Omega\big(\sigma_n(t)\big)\Big)$ e che $M_{\Omega,U}$ è crescente, troviamo che
    $$\|\sigma_n'(t)\| \le \lambda M_{\Omega,U}\left(\frac{1}{f^{-1}\left(\frac{|t|}{2\lambda}-\frac{\kappa}{2}\right)}\right)$$
    per ogni $n\ge\max\{N,N'\}$ e per quasi ogni $t\in[\alpha_n,-A)\cup(B,\beta_n]$. Usando l'ipotesi (3) e il cambio di variabile $r=\dfrac{1}{f^{-1}\left(\frac{|t|}{2\lambda}-\frac{\kappa}{2}\right)}$, si ha che esistono due costanti $c\in(-\infty,-A)$ e $d\in(B,+\infty)$ tali che
    \begin{align*}
        \lambda\int_{-\infty}^c M_{\Omega,U}\left(\frac{1}{f^{-1}\left(\frac{|t|}{2\lambda}-\frac{\kappa}{2}\right)}\right)\diff t&+\lambda\int_d^{+\infty}M_{\Omega,U}\left(\frac{1}{f^{-1}\left(\frac{|t|}{2\lambda}-\frac{\kappa}{2}\right)}\right)\diff t\\
        &<\|\xi-\eta\|.
    \end{align*}

    Allora, usando le ultime due disuguaglianze, otteniamo
    \begin{align*}
        \|\sigma(d)-\sigma(c)\|&=\lim_{n\longrightarrow+\infty}\|\sigma_n(d)-\sigma_n(c)\|\\
        &\ge\limsup_{n\longrightarrow+\infty}(\|\sigma_n(\beta_n)-\sigma_n(\alpha_n)\|-\|\sigma_n(\alpha_n)-\sigma_n(c)\|\\
        &-\|\sigma_n(\beta_n)-\sigma_n(d)\|)\\
        &\ge\lim_{n\longrightarrow+\infty}\|\sigma_n(\beta_n)-\sigma_n(\alpha_n)\|-\limsup_{n\longrightarrow+\infty}\left\|\int_{\alpha_n}^c \sigma_n'(t)\diff t\right\|\\
        &-\limsup_{n\longrightarrow+\infty}\left\|\int_d^{\beta_n} \sigma_n'(t)\diff t\right\|\\
        &\ge\|\xi-\eta\|-\limsup_{n\longrightarrow+\infty}\int_{\alpha_n}^c \|\sigma_n'(t)\|\diff t-\limsup_{n\longrightarrow+\infty}\int_d^{\beta_n} \|\sigma_n'(t)\|\diff t\\
        &\ge\|\xi-\eta\|-\limsup_{n\longrightarrow+\infty}\lambda\int_{\alpha_n}^c M_{\Omega,U}\left(\frac{1}{f^{-1}\left(\frac{|t|}{2\lambda}-\frac{\kappa}{2}\right)}\right)\diff t\\
        &-\limsup_{n\longrightarrow+\infty}\lambda\int_d^{\beta_n} M_{\Omega,U}\left(\frac{1}{f^{-1}\left(\frac{|t|}{2\lambda}-\frac{\kappa}{2}\right)}\right)\diff t\\
        &=\|\xi-\eta\|-\lambda\int_{-\infty}^c M_{\Omega,U}\left(\frac{1}{f^{-1}\left(\frac{|t|}{2\lambda}-\frac{\kappa}{2}\right)}\right)\diff t\\
        &-\lambda\int_d^{+\infty} M_{\Omega,U}\left(\frac{1}{f^{-1}\left(\frac{|t|}{2\lambda}-\frac{\kappa}{2}\right)}\right)\diff t>0;
    \end{align*}
    dunque in questo caso $\sigma$ non è costante. Se invece $a<-\infty$ e $b=+\infty$ (il caso $a=-\infty$ e $b<+\infty$ è analogo), ragionando come nel caso 1 estendiamo le $\sigma_n$ a delle $\tilde{\sigma}_n$ che, a meno di sottosuccessioni, convergono uniformemente sui compatti a una curva continua $\tilde{\sigma}:[a,+\infty)\longrightarrow\overline{\Omega}$ che estende $\sigma$. Allora basta ripetere la stima precedente con $\tilde{\sigma}_n, \tilde{\sigma}$ e $\alpha$ al posto di $\sigma_n, \sigma$ e $c$ e scegliendo un $d$ opportuno, trovando così che $\tilde{\sigma}$, e di conseguenza $\sigma$, non è costante. \\

    Poiché la nostra assunzione porta a una contraddizione, dev'essere falsa, da cui la tesi.
\end{proof}

Per i domini Caltrops applicheremo il Teorema \ref{extvis} con $U=\Omega$, per cui $M_{\Omega,U}=M_\Omega$, e $S=\emptyset$. Per mostrare che un dominio Caltrop $\Omega$ soddisfa le ipotesi del Teorema \ref{extvis} l'idea, spiegata in \cite[Section 6]{CMS}, è la seguente: si calcola $k_D$ per un dominio planare $D$ che useremo come modello, dopodiché immergeremo copie di $D$ in $\Omega$ in maniera affine, di modo che ogni punti di $\Omega$ sufficientemente vicino al bordo sia contenuto in una di queste copie. A questo punto, useremo la Proposizione \ref{semicontr} per stimare la distanza di Kobayashi su $\Omega$. 

Per essere precisi, useremo una classe di domini con certe proprietà, che adesso andiamo a costruire. Dati $a,h>0$, poniamo
$$S_{a,h}=\{z\in\mathbb{C}\mid\mathfrak{Re}z>a\text{ e }-h<\mathfrak{Im}z<h\}.$$

Indichiamo con $T_{a,h}$ l'immagine di $S_{a,h}$ tramite la mappa $z\longmapsto 1/z$ e notiamo che
$$T_{a,h}=\left(\mathbb{C}\setminus\overline{D\left(\frac{-i}{2h},\frac{1}{2h}\right)}\right)\cap\left(\mathbb{C}\setminus\overline{D\left(\frac{i}{2h},\frac{1}{2h}\right)}\right)\cap D\left(\frac{1}{2a},\frac{1}{2a}\right).$$

Indichiamo con $\mathcal{Q}^{\alpha,a,h}$ l'immagine di $T_{a,h}$ tramite la mappa $\phi_\alpha(z)=z^{\alpha}$, dove $\alpha$ è un reale maggiore di $1$ e $a$ e $h$ sono scelti in modo che $\phi_\alpha$ sia un biolomorfismo.

Osserviamo che $T_{a,h}$ ha una cuspide quadratica in $0$. Dunque esistono due costanti $c_1,c_2>0$ tali che per ogni $z\in\partial T_{a,h}$ si ha
\begin{equation} \label{cusp_estimate}
    c_1(\mathfrak{Re}z)^2 \le |\mathfrak{Im}z| \le c_2(\mathfrak{Re}z)^2
\end{equation}
per $\mathfrak{Re}z$ sufficientemente piccola. Più precisamente, per $\delta>0$ sufficientemente piccolo l'insieme $\partial T_{a,h}\cap\{z\in\mathbb{C}\mid 0 \le \mathfrak{Re}\le\delta\}$ è dato dall'unione dei grafici di $f$ e $-f$, dove, posto $z=x+iy$, si ha $f(x)=\dfrac{1}{2h}-\sqrt{\dfrac{1}{4h^2}-x^2}$, per cui abbiamo che
\begin{equation} \label{cerchio}
    f(x)=hx^2+O(x^4)
\end{equation}
per $x\longrightarrow0^+$.

Il seguente risultato sulle proprietà dei domini $\mathcal{Q}^{\alpha,a,h}$ sarà quello usato per ottenere stime sulla distanza di Kobayashi dei domini Caltrops.

\begin{prop} \label{qaah_biolo}
    Sia $\alpha>1$ e sia $\mathcal{Q}^{\alpha,a,h}$ come sopra, con $a,h>0$ scelti opportunamente. Poniamo $p=(1+\alpha)/\alpha$; allora
    \begin{enumerate}[label={(\arabic*)}]
        \item esistono delle costanti $\epsilon,C_1,C_2>0$ tali che, per ogni $z\in\partial\mathcal{Q}^{\alpha,a,h}$ con $0\le\mathfrak{Re}z\le\epsilon$, si ha che
        $$C_1(\mathfrak{Re}z)^p \le |\mathfrak{Im}z| \le C_2(\mathfrak{Re}z)^p;$$
        \item fissata una costante $M>1$ esiste $\epsilon>0$ sufficientemente piccolo tale che la disuguaglianza al punto (1) vale con $C_2=Mh\alpha$. Inoltre, fissati $\alpha>1$ e $h>0$, tale scelta di $\epsilon$ decresce al crescere di $a$;
        \item fissiamo un punto $x_0\in\mathcal{Q}^{\alpha,a,h}\cap\mathbb{R}$. Esiste una costante $C=C(x_0)>0$ tale che per ogni $x\in(0,x_0)$ si ha che
        $$k_{\mathcal{Q}^{\alpha,a,h}}(x_0,x) \le C+\frac{\pi}{4h}x^{-1/\alpha}.$$
    \end{enumerate}
\end{prop}

\begin{proof}
    Siano $c_1$ e $c_2$ le costanti date in \eqref{cusp_estimate}, e sia $f$ la funzione di \eqref{cerchio}, cioè $f(x)=\dfrac{1}{2h}-\sqrt{\dfrac{1}{4h^2}-x^2}$. Vogliamo studiare l'immagine dei grafici di $f$ e $-f$ tramite $\phi_\alpha$. Per simmetria, ci basterò studiare l'immagine del grafico di $f$. Scriviamo $z$ nel grafico di $f$ sufficientemente vicino a $0$ come $z=x+iy$, con $x \ge 0$ e $c_1x^2\le y\le c_2x^2$. Per $x>0$ sufficientemente piccolo, svolgiamo il seguente conto:
    \begin{align*}
        \phi_\alpha(z)&=(x+iy)^{\alpha}\\
        &=x^{\alpha}\left(1+\sum_{j=1}^{+\infty}\frac{(-1)^j}{(2j)!}\prod_{\nu=0}^{2j-1}(\alpha-\nu)\frac{y^{2j}}{x^{2j}}\right)\\
        &+ix^{\alpha}\left(\sum_{j=0}^{+\infty}\frac{(-1)^j}{(2j+1)!}\prod_{\nu=0}^{2j}(\alpha-\nu)\frac{y^{2j+1}}{x^{2j+1}}\right).
    \end{align*}

    Usando il fatto che $c_1x^2\le y\le c_2x^2$, si vede facilmente che
    \begin{gather*}
        \mathfrak{Re}\big(\phi_\alpha(z)\big)=x^\alpha+O(x^{2+\alpha})\\
        \text{ e }\\
        c_1\alpha x^{1+\alpha}\big(1-O(x^2)\big) \le \mathfrak{Im}\big(\phi_\alpha(z)\big) \le c_2\alpha x^{1+\alpha}\big(1+O(x^2)\big)
    \end{gather*}
    per $z=x+iy$ nel grafico di $f$ e $x>0$ sufficientemente piccolo. Da queste disuguaglianze segue la tesi del punto (1).

    Il punto (2) segue dalle stime con le quali abbiamo dimostrato il punto (1), e da come il dominio di $f$ dipende, per costruzione, da $a$.

    Consideriamo il biolomorfismo $\Phi_{\alpha,a,h}$ da $\mathcal{Q}^{\alpha,a,h}$ in $\mathbb{D}$ dato da
    $$\Phi_{\alpha,a,h}=f_3\circ f_2\circ f_1\circ g\circ(\phi_\alpha\restrict{T_{a,h}})^{-1},$$
    dove
    \begin{align*}
        g(z)&=1/z\text{ per ogni }z\in T_{a,h},\\
        f_1(z)&=\frac{\pi i}{2h}(z-a)\text{ per ogni }z\in S_{a,h},\\
        f_2(z)&=\sin{z}\text{ per ogni }z\in \mathbb{C}\text{ con }-\pi/2<\mathfrak{Re}z<\pi/2\text{ e }\mathfrak{Im}z>0,\\
        f_3(z)&=\frac{z-i}{z+i}\text{ per ogni }z\in\mathbb{C}\text{ con }\mathfrak{Im}z>0.
    \end{align*}

    Osserviamo che $\Phi_{\alpha,a,h}$ manda l'intervallo chiuso e limitato $\overline{\mathcal{Q}^{\alpha,a,h}\cap\mathbb{R}}$ omeomorficamente in $[-1,1]$. Inoltre, manda il punto $o=\dfrac{1}{\left(\frac{2h}{\pi}\log(\sqrt{2}+1)+a\right)^\alpha}$ in $0$, e se $x\in\mathcal{Q}^{\alpha,a,h}\cap\mathbb{R}$ è minore di $o$ allora $\Phi_{\alpha,a,h}(x)\in(0,1)$. Allora per tali $x$ si ha che
    $$k_{\mathcal{Q}^{\alpha,a,h}}(o,x)=k_{\mathbb{D}}\big(0,\Phi_{\alpha,a,h}(x)\big)=\frac{1}{2}\log\left(\frac{1+\Phi_{\alpha,a,h}(x)}{1-\Phi_{\alpha,a,h}(x)}\right).$$

    Calcolando esplicitamente $\Phi_{\alpha,a,h}(x)$, troviamo che
    \begin{align*}
        \frac{1}{2}\log\left(\frac{1+\Phi_{\alpha,a,h}(x)}{1-\Phi_{\alpha,a,h}(x)}\right)&=\frac{1}{2}\log\left(e^{\frac{\pi}{2h}\left(\frac{1}{x^{1/\alpha}}-a\right)}-e^{-\frac{\pi}{2h}\left(\frac{1}{x^{1/\alpha}}-a\right)}\right)-\frac{\log{2}}{2}\\
        &\le \frac{1}{2}\log\left(e^{\frac{\pi}{2h}\left(\frac{1}{x^{1/\alpha}}-a\right)}\right) \le \frac{\pi}{4h}x^{-1/\alpha};
    \end{align*}
    usando anche la disuguaglianza triangolare, otteniamo così la tesi del punto (3).
\end{proof}

I prossimi risultati saranno quelli necessari a immergere affinamente copie di $\mathcal{Q}^{\alpha,a,h}$ in un dominio Caltrop nel modo voluto. Nel seguito, con $o$ indichiamo il punto introdotto nella dimostrazione della Proposizione \ref{qaah_biolo}, associato al dominio $\mathcal{Q}^{\alpha,a,h}$ che staremo trattando.

\begin{lm} \label{superadd}
    Siano $\epsilon>0$ e $\phi:[0,\epsilon)\longrightarrow\mathbb{R}$ una funzione continua, strettamente crescente e derivabile in $(0,\epsilon)$. Supponiamo che $\phi'$ sia strettamente crescente e che $\phi(0)=0$. Allora per ogni $(x,y)\in[0,+\infty)\times[0,+\infty)$ tale che $x+y<\epsilon$ si ha che $\phi(x+y) \ge \phi(x)+\phi(y)$.
\end{lm}

\begin{proof}
    È una banale conseguenza del teorema fondamentale del calcolo integrale.
\end{proof}

\begin{lm} \label{psierre}
    Siano $A>0$ e $\psi:[0,A]\longrightarrow[0,+\infty)$ una funzione continua che sia di classe $C^2$ su $(0,A)$, e sia $p\in(1,2)$. Supponiamo inoltre che:
    \begin{itemize}
        \item esiste una costante $C>1$ tale che $x^p/C\le\psi(x)\le Cx^p$ per ogni $x\in[0,A]$;
        \item si ha che $\psi$ è strettamente crescente;
        \item si ha che $\psi'$ è strettamente crescente su $(0,A)$.
    \end{itemize}

    Poniamo $\mathcal{R}:=\left\{z\in\mathbb{C}\mid 0<\mathfrak{Re}z<A\text{ e }|\mathfrak{Im}z|<\psi\bigl(\mathfrak{Re}z\bigr)\right\}$. Allora esistono una costante $B\in(0,A)$, un compatto $K$ che interseca $\{z\in\mathbb{C}\mid\mathfrak{Re}z=A\}$ e tale che $K\setminus\{z\in\mathbb{C}\mid\mathfrak{Re}z=A\}\subsetneq\mathcal{R}$, e due costanti $a,h>0$ tali che per ogni $x+iy\in\mathcal{R}$ con $x\le B$ si ha che:
    \begin{enumerate}[label={(\arabic*)}]
        \item vale $\bigl(\psi^{-1}(|y|)+iy\bigr)+\mathcal{Q}^{1/(p-1),a,h}\subseteq\mathcal{R}$;
        \item vale $\psi^{-1}(|y|)+o>x$;
        \item vale $\bigl(\psi^{-1}(|y|)+iy\bigr)+o\in K$;
        \item vale $\delta_{\mathcal{R}}(x+iy) \le |\psi^{-1}(|y|)-x|$.
    \end{enumerate}
\end{lm}

\begin{proof}
    Per il punto (2) della Proposizione \ref{qaah_biolo}, possiamo fissare una costante $M>1$ tale che per ogni $\alpha>1$ e ogni $a,h>0$ esiste un $\epsilon=\epsilon(\alpha,a,h)>0$ tale che
    \begin{align*}
        \mathcal{Q}^{\alpha,a,h}\cap\{\mathfrak{Re}w<\epsilon\}&\subseteq \{w\in\mathbb{C}\mid 0<\mathfrak{Re}w<\epsilon\text{ e }|\mathfrak{Im}w|<Mh\alpha(\mathfrak{Re}w)^{(1+\alpha)/\alpha}\}\\
        &=:S^{\alpha,a,h},
    \end{align*}
    e tale che, per $\alpha$ e $h$ fissati, $\epsilon\longrightarrow0$ per $a\longrightarrow+\infty$; notiamo anche che possiamo imporre $\mathcal{Q}^{\alpha,a,h}\cap\{\mathfrak{Re}w<\epsilon\}=\mathcal{Q}^{\alpha,a,h}$ per $a$ sufficientemente grande. Fissiamo adesso $\alpha=1/(p-1)$. Poiché, per costruzione di $\mathcal{Q}^{\alpha,a,h}$, a parte reale fissata di un punto del bordo la parte immaginaria decresce, la costante $\epsilon$ scelta non decresce al decrescere di $h$. Allora possiamo scegliere $a$ sufficientemente grande e $h$ sufficientemente piccolo, in modo che $\mathcal{Q}^{\alpha,a,h}\cap\{\mathfrak{Re}w<\epsilon\}=\mathcal{Q}^{\alpha,a,h}$ e $\epsilon,o<A/2$, e $Mh\alpha<1/C$.
    
    Adesso fissiamo una costante $B\in(0,A)$ tale che $B<\min\{o,\epsilon/2\}$. Sia $z=x+iy\in\mathcal{R}$ con $x\le B$. Consideriamo l'insieme $\bigl(\psi^{-1}(|y|)+iy\bigr)+\mathcal{Q}^{\alpha,a,h}$. Un elemento arbitrario di questo insieme è della forma $\bigl(\psi^{-1}(|y|)+s\bigr)+i(y+t)$, con $s+it\in\mathcal{Q}^{\alpha,a,h}$. Dato che $\mathcal{Q}^{\alpha,a,h}\subseteq S^{\alpha,a,h}$, si ha $0<s<\epsilon$ e $|t|<Mh\alpha s^p$. Il punto $\bigl(\psi^{-1}(|y|)+s\bigr)+i(y+t)$ sta in $\mathcal{R}$ se e solo \begin{gather*}
        0<\psi^{-1}(|y|)+s<A\\
        \text{e}\\
        |y+t|<\psi\big(\psi^{-1}(|y|)+s\big).
    \end{gather*}

    Poiché $x+iy\in\mathcal{R}$ e $\psi$ è strettamente crescente, si ha $0 \le \psi^{-1}(|y|)<x\le B$, per cui $0<\psi^{-1}(|y|)+s<\epsilon/2+\epsilon<A$. Dunque, per mostrare il punto (1), ci resta da dimostrare che $|y+t|<\psi\big(\psi^{-1}(|y|)+s\big)$. Notiamo che $\psi$ soddisfa le ipotesi del Lemma \ref{superadd}, da cui
    $$\psi\big(\psi^{-1}(|y|)+s\big)\ge |y|+\psi(s)\ge |y|+s^p/C,$$
    dove la prima disuguaglianza è data dal Lemma e la seconda è vera per ipotesi. Allora si ha che
    $$|y+t| \le |y|+|t|<|y|+Mh\alpha s^p<|y|+s^p/C \le \psi\big(\psi^{-1}(|y|)+s\big),$$
    come voluto.

    Per ogni $x+iy\in\mathcal{R}$ con $x\le B$ si ha $\psi^{-1}(|y|)+o>B \ge x$ per come è stato scelto $B$, e questo dimostra il punto (2).

    Definiamo $K:=\left\{z\in\mathbb{C}\mid o \le \mathfrak{Re}z\le A\text{ e }|\mathfrak{Im}z| \le \psi\big(\mathfrak{Re}(z)-o\big)\right\}$. Per ogni $x+iy\in\mathcal{R}$ con $x\le B$ si ha
    $$o \le o+\psi^{-1}(|y|) <o+x \le o+B<2o<A.$$

    Inoltre $|y|=\psi\Big(\big(\psi^{-1}(|y|)+o\big)-o\Big)$, per cui $o+\big(\psi^{-1}(|y|)+iy\big)\in K$. Per costruzione, $K$ è un compatto che soddisfa le condizioni richeste, dunque abbiamo mostrato il punto (3).

    Infine, per ogni $x+iy\in\mathcal{R}$ con $x\le B$ si ha che $\psi^{-1}(|y|)+iy\in\partial\mathcal{R}$, per cui $\delta_{\mathcal{R}}(x+iy) \le \left|\big(\psi^{-1}(|y|)+iy\big)-(x+iy)\right|=|\psi^{-1}(|y|)-x|$; questo dimostra il punto (4).
\end{proof}

Il prossimo risultato, che è sostanzialmente una versione parametrizzata del precedente, tratta dell'immersione del dominio modello $\mathcal{Q}^{\alpha,a,h}$ in un dominio Caltrop con una punta. Per brevità, scriveremo $(z_1,\dots,z_{n-1},z_n)\in\mathbb{C}^n$ come $(z',z_n)$.

\begin{lm}
    Siano $A>0$ e $\psi:[0,A]\longrightarrow[0,+\infty)$ una funzione continua che sia di classe $C^2$ su $(0,A)$, e sia $p\in(1,2)$. Supponiamo inoltre che:
    \begin{itemize}
        \item esiste una costante $C>1$ tale che $x^p/C\le\psi(x)\le Cx^p$ per ogni $x\in[0,A]$;
        \item si ha che $\psi$ è strettamente crescente;
        \item si ha che $\psi'$ è strettamente crescente su $(0,A)$.
    \end{itemize}

    Sia
    $$D:=\{z\in\mathbb{C}^n\mid 0<\mathfrak{Re}z_n<A\text{ e }(\mathfrak{Im}z_n)^2+\|z'\|^2<\big(\psi(\mathfrak{Re}z_n)\big)^2\}.$$

    Sia $w'\in\mathbb{C}^{n-1}$ e poniamo
    $$\mathcal{R}_{w'}:=\pi_n\left[\big((w',0)+\{0_{n-1}\}\times\mathbb{C}\big)\cap D\right],$$
    dove $\pi_n$ è la proiezione sull'ultima coordinata. Sia $\alpha=1/(p-1)$. Allora esistono delle costanti $a,h,B>0$ e un compatto $K\subseteq\{z\in\mathbb{C}^n\mid\mathfrak{Re}z_n\le A\}$, che interseca $\{z\in\mathbb{C}^n\mid\mathfrak{Re}z_n=A\}$ e tale che $K\setminus\{z\in\mathbb{C}^n\mid\mathfrak{Re}z_n=A\}\subsetneq D$, tali che per ogni $w'\in\mathbb{C}^{n-1}$ con $\|w'\|<\psi(B/2)$ e ogni $\zeta\in\mathcal{R}_{w'}$ con $\mathfrak{Re}\zeta\le B$ si ha che:
    \begin{enumerate}[label={(\arabic*)}]
        \item vale $\Big(\psi^{-1}\big(S(\zeta,w')\big)+i\mathfrak{Im}\zeta\Big)+\mathcal{Q}^{\alpha,a,h}\subseteq\mathcal{R}_{w'}$;
        \item vale $\psi^{-1}\big(S(\zeta,w')\big)+o>\mathfrak{Re}\zeta$;
        \item vale $\Big(\psi^{-1}\big(S(\zeta,w')\big)+i\mathfrak{Im}\zeta\Big)+o\in\pi_n\left[\big((w',0)+\{0_{n-1}\}\times\mathbb{C}\big)\cap K\right]$;
        \item vale $\delta_D\big((w',\zeta)\big) \le \left|\mathfrak{Re}\zeta-\psi^{-1}\big(S(\zeta,w')\big)\right|$,
    \end{enumerate}
    dove $S(\zeta,w')=\sqrt{(\mathfrak{Im}\zeta)^2+\|w'\|^2}$.
\end{lm}

\begin{proof}
    Notiamo che
    \begin{align*}
        \mathcal{R}_{w'}&=\{\zeta\mid (w',\zeta)\in D\}\\
        &=\{\zeta\mid0<\mathfrak{Re}\zeta<A\text{ e }(\mathfrak{Im}\zeta)^2+\|w'\|^2<\big(\psi(\mathfrak{Re}\zeta)\big)^2\}\\
        &=\{\zeta\mid \psi^{-1}(\|w'\|)<\mathfrak{Re}\zeta<A\text{ e }(\mathfrak{Im}\zeta)^2+\|w'\|^2<\big(\psi(\mathfrak{Re}\zeta)\big)^2\},
    \end{align*}
    per cui $\mathcal{R}_{w'}\not=\emptyset$ se e solo se $\|w'\|<\psi(A)$; poiché la costante $B$ sarà scelta in $(0,A)$, nel nostro caso $\mathcal{R}_{w'}$ sarà sempre non vuoto. Notiamo inoltre che l'insieme $\mathcal{R}_{0_{n-1}}$ coincide con l'insieme $\mathcal{R}$ del Lemma \ref{psierre}. Prendiamo dunque $a,h$ e $B$ date da tale Lemma. Scriviamo per semplicità $c=1/C$, e osserviamo che dalla dimostrazione del Lemma \ref{psierre} discendono le seguenti disuguaglianze:
    \begin{gather}
        B+s<3A/4<A\text{ e }|t|<cs^p\text{ per ogni }s+it\in\mathcal{Q}^{\alpha,a,h};\label{qminoredia}\\
        o>B.\label{obbi}
    \end{gather}

    Supponiamo ora che $w'\not=0_{n-1}$. Sia $\zeta\in\mathcal{R}_{w'}$ tale che $\mathfrak{Re}\zeta\le B$. Un elemento arbitrario dell'insieme $\Big(\psi^{-1}\big(S(\zeta,w')\big)+i\mathfrak{Im}\zeta\Big)+\mathcal{Q}^{\alpha,a,h}$ è della forma $\Big(\psi^{-1}\big(S(\zeta,w')\big)+s\Big)+i(\mathfrak{Im}\zeta+t)$ con $s+it\in\mathcal{Q}^{\alpha,a,h}$. Un tale punto appartiene a $\mathcal{R}_{w'}$ se e solo se:
    \begin{nlist}
        \item si ha $\psi^{-1}\big(S(\zeta,w')\big)+s<A$;
        \item si ha $\|w'\|^2+(\mathfrak{Im}w+t)^2<\bigg(\psi\Big(\psi^{-1}\big(S(\zeta,w'\big)+s)\Big)\bigg)^2$.
    \end{nlist}

    Poiché $\zeta\in\mathcal{R}_{w'}$ e $\mathfrak{Re}\zeta\le B$, abbiamo che
    $$(\mathfrak{Im}\zeta)^2+\|w'\|^2<\big(\psi(\mathfrak{Re}\zeta)\big)^2<\big(\psi(B)\big)^2;$$
    quindi $\psi^{-1}\big(S(\zeta,w')\big)+s<B+s<A$, dove l'ultima disuguaglianza è la prima in \eqref{qminoredia}. È così verificata la condizione (i). Vediamo ora la condizione (ii). Per il Lemma \ref{superadd} e per le ipotesi su $\psi$ si ha che
    $$\psi\Big(\psi^{-1}\big(S(\zeta,w'\big)+s)\Big)\ge S(\zeta,w')+\psi(s) \ge S(\zeta,w')+cs^p;$$
    dunque
    $$\bigg(\psi\Big(\psi^{-1}\big(S(\zeta,w'\big)+s)\Big)\bigg)^2-(\mathfrak{Im}\zeta)^2-\|w'\|^2\ge 2cs^pS(\zeta,w')+c^2s^{2p}.$$

    Allora la condizione (ii) segue se mostriamo che
    $$2t\mathfrak{Im}\zeta+t^2<2cs^pS(\zeta,w')+c^2s^{2p},$$
    ma quest'ultima disuguaglianza segue dalla seconda disuguaglianza in \eqref{qminoredia}. Abbiamo così dimostrato il punto (1).

    Adesso notiamo che
    $$\psi^{-1}\big(S(\zeta,w')\big)+o \ge \psi^{-1}(\mathfrak{Im}\zeta)+o>B\ge\mathfrak{Re}\zeta,$$
    dove la seconda disuguaglianza segue da \eqref{obbi}; questo dimostra il punto (2).

    Poniamo $K:=\{(w',\zeta)\in\mathbb{C}^n\mid o\le\mathfrak{Re}\zeta\le A\text{ e }S(\zeta,w')\le\psi(\mathfrak{Re}\zeta-o)\}$. Si verifica facilmente che $K$ è un compatto con le proprietà richieste. Sia $\zeta\in\mathcal{R}_{w'}$ tale che $\mathfrak{Re}\zeta\le B$ e poniamo $\eta:=\Big(\psi^{-1}\big(S(\zeta,w')\big)+i\mathfrak{Im}\zeta\Big)+o$; allora abbiamo che
    $$o\le\psi^{-1}(\|w'\|)+o\le\mathfrak{Re}\eta<\mathfrak{Re}\zeta+o\le B+o<A.$$

    Inoltre, si ha che $S(\eta,w')=S(\zeta,w')=\psi(\mathfrak{Re}\eta-o)$; quindi
    $$\eta\in\pi_n\left[\big((w',0)+\{0_{n-1}\}\times\mathbb{C}\big)\cap K\right],$$
    e questo dimostra il punto (3).

    Per il punto (4), se $(w',\zeta)$ è preso come sopra, allora
    $$\psi^{-1}\big(S(\zeta,w')\big)+i\mathfrak{Im}\zeta\in\partial\mathcal{R}_{w'};$$
    dunque
    $$\delta_D\big((w',\zeta)\big)\le \delta_{\mathcal{R}_{w'}}(\zeta)\le\left|\mathfrak{Re}\zeta-\psi^{-1}\big(S(\zeta,w')\big)\right|,$$
    come voluto.
\end{proof}