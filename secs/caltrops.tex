Il secondo esempio, introdotto in \cite{BM}, è quello dei domini Caltrops.

\begin{defn}
    Un dominio limitato $\Omega\subseteq\mathbb{C}^n$, con $n\ge 2$, è detto \textit{dominio Caltrop} se esiste un insieme finito di punti $\{q_1,\dots,q_N\}\subseteq\partial\Omega$ tale che:
    \begin{itemize}
        \item il sottoinsieme del bordo $\partial\Omega\setminus\{q_1,\dots,q_N\}$ è $C^2$ e $\Omega$ è strettamente pseudoconvesso in ogni punto di tale insieme;
        \item per ogni $j=1,\dots, N$ esiste un intorno aperto e connesso $V_j\ni q_j$ tale che esistono due costanti $p_j\in(1,3/2)$ e $C_j>1$, una trasformazione unitaria $\mathbb{U}^{(j)}$ e una funzione continua $\psi_j:[0,A_j]\longrightarrow[0,+\infty)$, con $A_j>0$, tali che $\mathbb{U}_j(\Omega\cap V_j)$ è un ``solido di rivoluzione'' dato da
        \begin{align*}
            \mathbb{U}_j(\Omega\cap V_j)=&\Bigg\{(z_1,\dots,z_n)\in\mathbb{C}^n\mid \mathfrak{Re}z_n\in (0,A_j),\\
            &\left.(\mathfrak{Im}z_n)^2+\sum_{j=1}^{n-1}|z_j|^2 \le \psi_j(\mathfrak{Re}z_n)^2\right\}
        \end{align*}
        dove $\mathbb{U}_j(z)=\mathbb{U}^{(j)}(z-q_j)$ per ogni $z\in\mathbb{C}^n$. Inoltre, $\psi_j$ ha le seguenti proprietà:
        \begin{itemize}
            \item è di classe $C^2$ su $(0,A_j)$;
            \item per ogni $x\in[0,A_j]$ si ha $(1/C_j)x^{p_j} \le \psi_j(x) \le C_jx^{p_j}$;
            \item si ha che $\psi_j$ è strettamente crescente e $\psi_j'$ è crescente su $(0,A_j)$;
            \item si ha $\displaystyle\lim_{x\longrightarrow0^+}\psi_j(x)\psi_j''(x)=0$.
        \end{itemize}
    \end{itemize}
\end{defn}

\begin{oss}
    Il nome, che in italiano può essere tradotto come tribolo o ``piede di corvo'' (un'arma da lancio a quattro punte), rimanda al fatto che, vicino ai punti in cui non è liscio, il bordo di tali domini assume una forma simile a quella di una cuspide hölderiana non eccessivamente appuntita.
\end{oss}

Ci occupiamo adesso di mostrare che i domini Caltrops esistono. Vediamo l'esempio di un dominio Caltrop con una sola punta in $\mathbb{C}^2$. Siano $A,\beta>0$ e sia $\psi:[-A,\beta]\longrightarrow[0,+\infty)$ una funzione continua di classe $C^2$ su $(-A,\beta)$ tale che:
\begin{enumerate}[label={(\arabic*)}]
    \item per ogni $t\in(-A,-B)$ si ha $\psi(t)=(t+A)^p$;
    \item per ogni $t\in(0,\beta)$ si ha $\psi(t)=\sqrt{\beta^2-t^2}$,
\end{enumerate}
dove $B\in(0,A)$ e $p\in(1,3/2)$. Consideriamo il ``solido di rivoluzione'' dato da
$$\Omega:=\{(z,w)\in\mathbb{C}^2\mid |z|^2+|\mathfrak{Im}w|^2<C\psi(\mathfrak{Re}w)^2,-A<\mathfrak{Re}w<\beta\},$$
dove $C>0$ è una costante che sceglieremo più avanti. Poniamo inoltre
$$\rho(z,w):=|z|^2+|\mathfrak{Im}w|^2-C\psi(\mathfrak{Re}w)^2,$$
considerata sull'insieme $\{(z,w)\in\mathbb{C}^2\mid -A<\mathfrak{Re}w<\beta+\epsilon\}$, dove $\epsilon>0$ è fissato e $\psi^2$ è estesa nel modo ovvio su $(\beta,\epsilon)$. Si verifica che $\rho$ è una funzione $C^2$ avente l'ipersuperficie reale $\partial\Omega\cap\{(z,w)\in\mathbb{C}^2\mid -A<\mathfrak{Re}w\}$ come luogo di zeri. Calcoliamo le seguenti derivati parziali seconde:
\begin{gather*}
    \partial^2_{z\bar{z}}\rho\equiv 1;\\
    \partial^2_{z\bar{w}}\rho=\partial^2_{\bar{z}w}\rho\equiv 0;\\
    \partial^2_{w\bar{w}}\rho(z,w)=\frac{1}{2}-\frac{C}{2}\big(\psi''(\mathfrak{Re}w)\psi(\mathfrak{Re}w)+\psi'(\mathfrak{Re}w)^2\big).
\end{gather*}

In particolare, si ha che
$$\partial^2_{w\bar{w}}\rho(z,w)-\frac{1}{2}=-\frac{Cp(2p-1)}{2}(\mathfrak{Re}w+A)^{2(p-1)}$$
per $\mathfrak{Re}w$ sufficientemente vicino a $-A$, che tende crescendo a $0$ per $\mathfrak{Re}w$ che tende descrescendo a $-A$. Allora, essendo $\psi$ di classe $C^2$ su $(-A,\beta)$, scegliendo $C$ sufficientemente piccolo possiamo imporre che $\partial^2_{w\bar{w}}\rho(z,w)\ge\dfrac{1}{4}$ per ogni $w$ tale che $-A<\mathfrak{Re}w\le 0$. Segue che $\partial\Omega\cap\{(z,w)\in\mathbb{C}^2\mid -A<\mathfrak{Re}w\le 0\}$ è un sottoinsieme di punti strettamente pseudoconvessi del bordo di $\Omega$. Per la condizione (2) su $\psi$, anche $\partial\Omega\cap\{(z,w)\in\mathbb{C}^2\mid \mathfrak{Re}w>0\}$ lo è. Le altre proprietà di dominio Caltrop seguono dalla condizione (1) su $\psi$; la punta è in $(0,-A)$.

In \cite[Section 3.2]{CMS} vengono costruiti domini Caltrops con un numero arbitrario di punte. \\

Vediamo adesso che i domini Caltrops hanno le proprietà volute. Cominciamo con la condizione di visibilità.