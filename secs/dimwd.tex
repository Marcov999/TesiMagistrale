Andiamo adesso ad enunciare e dimostrare una versione generale di un teorema di tipo ``Wolff-Denjoy'' per varietà Kobayashi-iperboliche. La dimostrazione riportata ricalca quella data in \cite{CMS}, che a sua volta riprende la strategia e le tecniche impiegate da \cite{BZ1} e \cite{BM}. Ognuno di questi articoli ha generalizzato il risultato ottenuto nel precedente.

\begin{thm} \label{wd_type_strong}
    Sia $X$ una sottovarietà complessa e tautly embedded di una varietà Kobayashi-iperbolica $Y$. Supponiamo che $X$ sia taut e che esista un $\kappa_0>0$ tale che $X$ sia $(1,\kappa_0)$-visibile.
    
    Sia $F:X \longrightarrow X$ una funzione olomorfa. Allora vale esattamente una delle seguenti affermazioni:
    \begin{itemize}
        \item le orbite dei punti di $X$ tramite $F$ sono relativamente compatte in $X$; oppure,
        \item esiste un unico punto di $\partial_YX$ tale che la successione delle iterate di $F$ converge, uniformemente sui compatti, a quel punto.
    \end{itemize}
\end{thm}

\begin{proof}
    Poiché $X$ è taut, per il Teorema \ref{dicotomia} o l'orbita di $z$ tramite $F$ è relativamente compatta per ogni $z \in X$, oppure la successione delle iterate $\{F^\nu\}_{\nu\in\mathbb{N}}$ è compattamente divergente. Supponiamo che le orbite di $F$ non siano relativamente compatte in $X$; allora la successione delle iterate di $F$ è compattamente divergente.

    Consideriamo una sottosuccessione di $\{F^\nu\}_{\nu\in\mathbb{N}}$ che converge uniformemente sui compatti a una funzione olomorfa $\tilde{F}:X \rightarrow Y$ con $\tilde{F}(X)\subseteq\overline{X}$. Poiché le iterate di $F$ sono compattamente divergenti, si deve avere che $\tilde{F}(X)\subseteq\partial_YX$. Allora, per il Teorema \ref{lim_is_const}, troviamo che $\tilde{F}$ è costante. Identifichiamo quindi
    $$\Gamma:=\overline{\{F^\nu\}}_{\nu\in\mathbb{N}}\setminus\{F^\nu\}_{\nu\in\mathbb{N}}$$
    come un insieme di punti di $\partial_YX$, dove la chiusura è intesa rispetto alla topologia compatta-aperta. Supponiamo, per assurdo, che $\Gamma$ contenga almeno due punti. \\

    Caso 1: esiste (e quindi per ogni) $o \in X$ tale che
    $$\limsup_{\nu\longrightarrow+\infty} k_X\big(F^\nu(o),o\big)=+\infty.$$

    Possiamo dunque scegliere, in modo simile a quanto fatto all'inizio della dimostrazione della Proposizione \ref{inf_impl_ugu} e applicando la Proposizione \ref{montel}, una sottosuccessione $\{\nu_j\}_{j\in\mathbb{N}}$ tale che:
    \begin{enumerate}[label={(\arabic*)}]
        \item per ogni $j\in\mathbb{N}$ e $k\le\nu_j$ si ha $k_X\big(F^k(o),o\big) \le k_X\big(F^{\nu_j}(o),o\big)$;
        \item si ha $\displaystyle\lim_{j\longrightarrow+\infty}k_X\big(F^{\nu_j}(o),o\big)=+\infty$;
        \item si ha che $\{F^{\nu_j}\}_{j\in\mathbb{N}}$ converge, uniformemente sui compatti di $X$, a $\xi\in\partial_YX$.
    \end{enumerate}

    Adesso, poiché abbiamo assunto che $\Gamma$ contenga almeno due elementi, esiste una sottosuccessione $\{\mu_j\}_{j\in\mathbb{N}}$ tale che $\{F^{\mu_j}\}_{j\in\mathbb{N}}$ converge, uniformemente sui compatti di $X$, a $\eta\in\partial_YX$ con $\eta\not=\xi$. Segue immediatamente dalla Proposizione \ref{inf_impl_ugu} che non possiamo avere $\displaystyle\limsup_{j\longrightarrow+\infty} k_X\big(F^{\mu_j}(o),o\big)=+\infty$. Perciò dev'essere $\displaystyle\limsup_{j\longrightarrow+\infty} k_X\big(F^{\mu_j}(o),o\big)<+\infty$; notiamo che se $X$ fosse Kobayashi-iperbolica completa, allora per il Lemma \ref{comp_is_comp} avremmo subito un assurdo. Altrimenti, si ha
    \begin{equation} \label{daoainfinito}
        \begin{aligned}
            \limsup_{h\longrightarrow+\infty}\limsup_{j\longrightarrow+\infty}\, &k_X\big(F^{\nu_h}(o),F^{\mu_j}(o)\big) \\
            &\ge \limsup_{h\longrightarrow+\infty}\limsup_{j\longrightarrow+\infty} \Big(k_X\big(F^{\nu_h}(o),o\big)-k_X\big(F^{\mu_j}(o),o\big)\Big)=+\infty.
        \end{aligned}
    \end{equation}

    Consideriamo ora un $l\in\mathbb{N}$. Poiché la successione $\{F^{\mu_j-l}\}_{j\in\mathbb{N}}$ converge uniformemente sul compatto $\{F^l(o)\}$ a $\eta$, per il Lemma \ref{lemmasottosotto2} converge, uniformemente su tutti i compatti di $X$, a $\eta$.

    Poniamo
    $$M_l:=\limsup_{j\longrightarrow+\infty}k_X\big(F^{\mu_j-l}(o),o\big);$$
    affermiamo che
    $$\limsup_{l\longrightarrow+\infty}M_l<+\infty.$$
    Supponiamo per assurdo che non sia così; allora esiste una sottosuccessione $\{l_m\}_{m\in\mathbb{N}}$ tale che $M_{l_m}>m$ per ogni $m$. Per definizione di $M_l$ e per quanto appena trovato sulla successione $\{F^{\mu_j-l}\}_{j\in\mathbb{N}}$, abbiamo quindi che esiste una sottosottosuccessione $\{j_m\}_{m\in\mathbb{N}}$ tale che:
    \begin{enumerate}[label={(\arabic*)}]
        \item $d_Y\big(F^{\mu_{j_m}-l_m}(o),\eta\big)<1/m$;
        \item $k_X\big(F^{\mu_{j_m}-l_m}(o),o\big)>m$.
    \end{enumerate}

    Per la Proposizione \ref{inf_impl_ugu} deve dunque essere $\eta=\xi$, contraddizione. Perciò segue che $\displaystyle\limsup_{l\longrightarrow+\infty}M_l<+\infty$. Allora
    \begin{align*}
        \limsup_{h\longrightarrow+\infty}\limsup_{j\longrightarrow+\infty}\, &k_X\big(F^{\nu_h}(o),F^{\mu_j}(o)\big) \\
        &\le \limsup_{h\longrightarrow+\infty}\limsup_{j\longrightarrow+\infty} k_X\big(o,F^{\mu_j-\nu_h}(o)\big)=\limsup_{h\longrightarrow+\infty}M_{\nu_h}<+\infty,
    \end{align*}
    in contraddizione con la \eqref{daoainfinito}; questo conclude il Caso 1.\\

    Caso 2: esiste (e quindi per ogni) $o \in X$ tale che
    $$\limsup_{\nu\longrightarrow+\infty} k_X\big(F^\nu(o),o\big)<+\infty.$$
    
    Ricordiamo che abbiamo assunto che esistano due punti distinti $\xi,\eta\in\Gamma$. Poiché $X$ è $(1,\kappa_0)$-visibile, esistono $V_\xi, V_\eta$ intorni in $\overline{X}$ rispettivamente di $\xi$ e di $\eta$, con $\overline{V}_\xi\cap\overline{V}_\eta=\emptyset$, e un compatto $K$ di $X$ tali che ogni $(1,\kappa_0)$-simil-geodetica in $X$ che collega un punto di $V_\xi$ a un punto di $V_\eta$ interseca $K$.
    Adesso definiamo, per $\delta>0$ arbitrario, la funzione $G_\delta:K\times K\longrightarrow [0,+\infty)$ data da
    $$G_\delta(x_1,x_2):=\inf\big\{k_X\big(F^m(x_1),x_2\big)\mid m\in\mathbb{N}, d_Y\big(F^m(x_1),\xi\big)<\delta\big\}.$$

    Notiamo che $G_\delta$ è ben definita per ogni $\delta>0$ (basta considerare la sottosuccessione delle iterate di $F$ che converge a $\xi$ uniformemente sui compatti di $X$) e che $G_{\delta_1}(x_1,x_2) \ge G_{\delta_2}(x_1,x_2)$ per ogni $x_1,x_2 \in K$ e $\delta_1 \le \delta_2$. Vogliamo dire che, in questo caso, si ha
    $$\sup_{\delta>0,x_1,x_2\in K}G_\delta(x_1,x_2)<+\infty.$$

    Supponiamo per assurdo che non sia così; allora esistono una successione di reali positivi $\{\delta_n\}_{n\in\mathbb{N}}$ e due successioni $\{x_n'\}_{n\in\mathbb{N}}$ e $\{x_n''\}_{n\in\mathbb{N}}$ in $K$ tali che $G_{\delta_n}(x_n',x_n'') \ge n$ per ogni $n\in\mathbb{N}$. Possiamo inoltre supporre che le successioni $\{\delta_n\}_{n\in\mathbb{N}}, \{x_n'\}_{n\in\mathbb{N}}$ e $\{x_n''\}_{n\in\mathbb{N}}$ convergano, rispettivamente, a $\delta_0 \ge 0$ e $x',x''\in K$. Sia $\{\nu_j\}_{j\in\mathbb{N}}$ una sottosuccessione di $\mathbb{N}$ tale che $\{F^{\nu_j}\}_{j\in\mathbb{N}}$ converge, uniformemente sui compatti di $X$, a $\xi$. Allora per ogni $n\in\mathbb{N}$ esiste $j(n)\in\mathbb{N}$ tale che si ha $\displaystyle\sup_{x\in K}d_Y\big(F^{\nu_j}(x),\xi\big)<\delta_n$ per ogni $j \ge j(n)$. In particolare, per ogni $n \in \mathbb{N}$ si ha $d_Y\big(F^{\nu_{j(n)}}(x_n'),\xi\big)<\delta_n$. Segue che
    $$n \le G_{\delta_n}(x_n',x_n'') \le k_X\big(F^{\nu_{j(n)}}(x_n'),x_n''\big);$$
    da ciò discende facilmente che $\displaystyle\lim_{n\longrightarrow+\infty}k_X\big(F^{\nu_{j(n)}}(x'),x'\big)=+\infty$, in contraddizione con l'ipotesi del Caso 2.

    Abbiamo dunque che, per ogni $x_1,x_2 \in K$, è ben definita la funzione data da
    $$G(x_1,x_2):=\lim_{\delta\longrightarrow 0^+}G_\delta(x_1,x_2);$$
    definiamo inoltre
    $$\epsilon:=\liminf_{z \longrightarrow\eta}\inf_{y\in K}k_X(z,y).$$
    
    Per il Corollario \ref{dist_da_sotto} abbiamo che $k_X(z,y) \ge c\cdot d_Y(z,y)$, e quest'ultima quantità è sempre maggiore di una costante positiva per $z$ sufficientemente vicino al bordo e di conseguenza lontano dal compatto $K$; quindi $\epsilon>0$. Scegliamo ora due punti $q_1,q_2 \in K$ tali che
    $$G(q_1,q_2)<\inf_{x_1,x_2\in K}G(x_1,x_2)+\epsilon.$$
    
    Inoltre, dalla definizione di $G_{\delta}$ abbiamo che, per ogni $j\in\mathbb{N}$, esiste $m\in\mathbb{N}$ tale che $d_Y\big(F^m(q_1),\xi\big)<1/j$ e $G_{1/j}(q_1,q_2) \le k_X\big(F^m(q_1),q_2\big) \le G_{1/j}(q_1,q_2)+1/j$. Possiamo dunque trovare una funzione strettamente crescente $\nu:\mathbb{N}\longrightarrow\mathbb{N}$ tale che per ogni $j\in\mathbb{N}$ si ha
    $$d_Y\big(F^{\nu(j)}(q_1),\xi\big)<1/\nu(j)$$
    e
    $$G_{1/\nu(j)}(q_1,q_2) \le k_X\big(F^{\nu(j)}(q_1),q_2\big) \le G_{1/\nu(j)}(q_1,q_2)+1/\nu(j);$$
    poiché $\{F^{\nu(j)}(q_1)\}_{j\in\mathbb{N}}$ converge a $\xi$, per il Lemma \ref{lemmasottosotto2} la successione $\{F^{\nu(j)}\}_{j\in\mathbb{N}}$ converge, uniformemente sui compatti di $X$, a $\xi$. Da come è stata scelta $\nu$, abbiamo anche che $\displaystyle\lim_{j\longrightarrow+\infty}k_X\big(F^{\nu(j)}(q_1),q_2\big)=G(q_1,q_2)$.

    Fissiamo adesso una funzione strettamente crescente $\mu:\mathbb{N}\longrightarrow\mathbb{N}$ tale che la successione $\{F^{\mu(j)}\}_{j\in\mathbb{N}}$ converge, uniformemente sui compatti di $X$, a $\eta$. Dato un compatto $K\subseteq X$, anche $F^{\mu(j)}(K)$ è compatto, in quanto immagine continua di un compatto; allora, poiché la successione $\{F^{\nu(h)}\}_{h\in\mathbb{N}}$ converge a $\xi$ uniformemente sui compatti di $X$, per ogni $j\in\mathbb{N}$ fissato troviamo che la successione $\{F^{\nu(h)+\mu(j)}\}_{h\in\mathbb{N}}$ converge a $\xi$ uniformemente sui compatti di $X$. Ciò implica, considerando un'esaustione di $X$ in compatti, che esiste una funzione strettamente crescente $\tau:\mathbb{N}\longrightarrow\mathbb{N}$ tale che la successione $\{F^{(\nu\circ\tau)(j)+\mu(j)}\}_{j\in\mathbb{N}}$ converge a $\xi$ uniformemente sui compatti di $X$. Ricordando le proprietà di $\nu$, e a meno di rinominare le funzioni, possiamo dunque dire di avere due funzioni strettamente crescenti $\nu,\mu:\mathbb{N}\longrightarrow\mathbb{N}$ tali che:
    \begin{enumerate}[label={(\arabic*)}]
        \item le successioni $\{F^{\nu(j)}\}_{j\in\mathbb{N}}$ e $\{F^{\mu(j)}\}_{j\in\mathbb{N}}$ convergono, uniformemente sui compatti di $X$, a $\xi$ e $\eta$ rispettivamente;
        \item la successione $\{F^{\nu(j)+\mu(j)}\}_{j\in\mathbb{N}}$ converge a $\xi$ uniformemente sui compatti di $X$;
        \item si ha $\displaystyle\lim_{j\longrightarrow+\infty}k_X\big(F^{\nu(j)}(q_1),q_2\big)=G(q_1,q_2)$.
    \end{enumerate}

    Per il Teorema \ref{similgeo_conn}, abbiamo che per ogni $j\in\mathbb{N}$ esiste una $(1,1/j)$-simil-geodetica $\sigma_j:[0,T_j]\longrightarrow X$ con $\sigma_j(0)=F^{\nu(j)+\mu(j)}(q_1)$ e $\sigma_j(T_j)=F^{\mu(j)}(q_2)$. Dato che $\{F^{\nu(j)+\mu(j)}(q_1)\}_{j\in\mathbb{N}}$ e $\{F^{\mu(j)}(q_2)\}_{j\in\mathbb{N}}$ convergono, rispettivamente, a $\xi$ e $\eta$, per $j$ abbastanza grande si ha che $\sigma_j(0)\in V_\xi$ e $\sigma_j(T_j)\in V_\eta$, e $\sigma_j$ è una $(1,\kappa_0)$-simil-geodetica. Dunque $\sigma_j([0,T_j])\cap K\not=\emptyset$ per $j$ abbastanza grande. Per ogni tale $j$ scegliamo un $t_j\in[0,T_j]$ tale che $x_j^*=\sigma_j(t_j)\in K$. Per compattezza di $K$, a meno di passare a una sottosuccessione possiamo supporre che $x_j^* \longrightarrow x^* \in K$ per $j\longrightarrow+\infty$. Poiché per ogni $j\in\mathbb{N}$ la curva $\sigma_j$ è una $(1,1/j)$-simil-geodetica, per il Lemma \ref{quasiovvio} troviamo che
    \begin{equation} \label{trejay}
        k_X\big(F^{\nu(j)+\mu(j)}(q_1),F^{\mu(j)}(q_2)\big) \ge k_X\big(F^{\nu(j)+\mu(j)}(q_1),x_j^*\big)+k_X\big(x_j^*,F^{\mu(j)}(q_2)\big)-3/j.
    \end{equation}

    Adesso, si ha che
    \begin{equation} \label{xstar}
        \begin{aligned}
            \liminf_{j\longrightarrow+\infty}k_X\big(F^{\nu(j)+\mu(j)}(q_1),x_j^*\big) &\ge \liminf_{j\longrightarrow+\infty} \Big(k_X\big(F^{\nu(j)+\mu(j)}(q_1),x^*\big)-k_X(x^*,x_j^*)\Big)\\
            &=\liminf_{j\longrightarrow+\infty}k_X\big(F^{\nu(j)+\mu(j)}(q_1),x^*\big) \ge G(q_1,x^*);
        \end{aligned}
    \end{equation}
    inoltre, per definizione di $\epsilon$ abbiamo che
    \begin{equation} \label{xstarepsilon}
        \liminf_{j\longrightarrow+\infty}k_X\big(x_j^*,F^{\mu(j)}(q_2)\big) \ge \epsilon.
    \end{equation}

    Mettendo assieme la \eqref{trejay}, la \eqref{xstar} e la \eqref{xstarepsilon}, troviamo che
    $$\liminf_{j\longrightarrow+\infty}k_X\big(F^{\nu(j)+\mu(j)}(q_1),F^{\mu(j)}(q_2)\big) \ge G(q_1,x^*)+\epsilon;$$
    d'altra parte, abbiamo che
    $$\limsup_{j\longrightarrow+\infty}k_X\big(F^{\nu(j)+\mu(j)}(q_1),F^{\mu(j)}(q_2)\big) \le \limsup_{j\longrightarrow+\infty}k_X\big(F^{\nu(j)}(q_1),q_2\big)=G(q_1,q_2).$$

    Dunque si ha che $G(q_1,q_2) \ge G(q_1,x^*)+\epsilon$, in contraddizione con la scelta di $q_1$ e $q_2$. \\

    Poiché sia il Caso 1 che il Caso 2 portano a una contraddizione, ne consegue che la supposizione che $\Gamma$ contenga almeno due punti dev'essere sbagliata, da cui segue facilmente la tesi.
\end{proof}

Vogliamo ora ottenere come conseguenza il teorema di tipo ``Wolff-Denjoy'' dimostrato in \cite{CMS}. Per fare ciò, ci servirà il seguente lemma.

\begin{lm} \label{rcitit}
    (\cite[Proposition 2.1.5]{A2}) Sia $X$ una sottovarietà complessa e relativamente compatta di una varietà taut $Y$. Allora $X$ è una sottovarietà tautly embedded di $Y$.
\end{lm}

\begin{proof}
    Poiché $X$ è relativamente compatta in $Y$, nessuna successione in $\text{Hol}(\mathbb{D},X)$ può essere compattamente divergente in $Y$. Dato che $Y$ è taut, ne consegue che ogni tale successione ammette una sottosuccessione convergente, uniformemente sui compatti, a una funzione in $\text{Hol}(\mathbb{D},Y)$. Ciò implica che $\text{Hol}(\mathbb{D},X)$ è relativamente compatto per successioni in $\text{Hol}(\mathbb{D},Y)$. Quindi è relativamente compatto per il Corollario \ref{ciffcps}, cioè $X$ è una sottovarietà tautly embedded di $Y$, come voluto.
\end{proof}

\begin{cor} \label{wd_type}
    (\cite[Theorem 1.15]{CMS}) Sia $X$ una sottovarietà complessa, connessa e limitata di $\mathbb{C}^d$. Supponiamo che $X$ sia taut e che esista un $\kappa_0>0$ tale che $X$ sia $(1,\kappa_0)$-visibile.
    
    Sia $F:X \longrightarrow X$ una funzione olomorfa. Allora vale esattamente una delle seguenti affermazioni:
    \begin{itemize}
        \item le orbite dei punti di $X$ tramite $F$ sono relativamente compatte in $X$; oppure,
        \item esiste un unico punto di $\partial X$ tale che la successione delle iterate di $F$ converge, uniformemente sui compatti, a quel punto.
    \end{itemize}
\end{cor}

\begin{proof}
    Poiché $X$ è limitata, esiste $R>0$ tale che $X$ è una sottovarietà complessa e relativamente compatta della varietà Kobayashi-iperbolica $\mathbb{B}_R^d$. Sia $f$ una funzione nella chiusura di $\text{Hol}(\mathbb{D},\mathbb{B}_R^d)$ in $C^0(\mathbb{D},\mathbb{B}_R^d)$, e consideriamo una successione $\{f_n\}_{n\in\mathbb{N}}\subseteq\text{Hol}(\mathbb{D},\mathbb{B}_R^d)$ che vi converge. Per il teorema di Montel, a meno di sottosuccessioni possiamo supporre che la successione converga a una funzione $\tilde{f}\in\text{Hol}(\mathbb{D},\mathbb{C}^d)$ con $\tilde{f}(\mathbb{D})\subseteq \overline{\mathbb{B}_R^d}$. Supponiamo che esista $z\in\mathbb{D}$ tale che $\tilde{f}(z)\in\partial\mathbb{B}_R^d$. Componendo $\tilde{f}$ con una funzione affine che si annulla in $\tilde{f}(z)$ e ha parte reale negativa nel resto di $\mathbb{B}_R^d$, e applicando il principio del massimo, si ottiene che $\tilde{f}$ è costante; di conseguenza, $f$ è la costante $\infty$. Perciò $\mathbb{B}_R^d$ è taut. Segue dal Lemma \ref{rcitit} che $X$ è una sottovarietà tautly embedded di $\mathbb{B}_R^d$. Essendo anche taut e $(1,\kappa_0)$-visibile, soddisfa le ipotesi del Teorema \ref{wd_type_strong}, da cui la tesi.
\end{proof}

Adesso la dimostrazione del Teorema \ref{abate_wd} è immediata.

\begin{cor}
    Sia $\Omega \subseteq \mathbb{C}^n$ un dominio limitato e strettamente pseudoconvesso, e sia $f \in \textnormal{Hol}(\Omega,\Omega)$. Allora vale esattamente una delle seguenti affermazioni:
    \begin{itemize}
        \item le orbite dei punti di $\Omega$ tramite $f$ sono relativamente compatte in $\Omega$; oppure,
        \item esiste un unico punto di $\partial\Omega$ tale che la successione delle iterate di $f$ converge, uniformemente sui compatti, a quel punto.
    \end{itemize}
\end{cor}

\begin{proof}
    Abbiamo già notato, nell'Osservazione \ref{spscvxisvis}, che i domini limitati e strettamente pseudoconvessi di $\mathbb{C}^n$ sono visibili per le simil-geodetiche. Abbiamo anche visto che sono completi rispetto alla distanza di Kobayashi, dunque per la Proposizione \ref{comp_is_taut} sono taut. Si conclude applicando il Corollario \ref{wd_type}.
\end{proof}

Sotto ipotesi di natura topologica si può ottenere qualcosa di più.

\begin{defn}
    Una varietà complessa $X$ si dice di \textit{tipo topologico finito} se i gruppi di omologia singolare $H_j(\Omega;\mathbb{Z})$ hanno rango finito per ogni $j\in\mathbb{N}$.
\end{defn}

\begin{thm} \label{periodico}
    (\cite[Theorem 0.4]{A4}) Sia $X$ una varietà taut e di tipo topologico finito. Supponiamo che $H^j(X;\mathbb{Q})=0$ per ogni $j>0$, e consideriamo una funzione $F\in\textnormal{Hol}(X,X)$. Allora la successione $\{F^h\}_{h\in\mathbb{N}}$ non è compattamente divergente se e solo se $F$ ha un punto periodico in $X$, cioè esistono $x\in X$ e $h_0\in\mathbb{N}$ tali che $F^{h_0}(x)=x$.
\end{thm}

\begin{cor}
    (\cite[Theorem 1.9]{BM}) Sia $\Omega\subseteq\mathbb{C}^n$ un dominio limitato. Supponiamo che $\Omega$ sia taut e che esista un $\kappa_0>0$ tale che $\Omega$ sia $(1,\kappa_0)$-visibile. Supponiamo inoltre che $\Omega$ sia di tipo topologico finito e che $H^j(X;\mathbb{C})=0$ per ogni $1\le j\le n$.

    Sia $F:\Omega\longrightarrow\Omega$ una funzione olomorfa. Allora vale esattamente una delle seguenti affermazioni:
    \begin{itemize}
        \item le orbite dei punti di $\Omega$ tramite $F$ sono relativamente compatte in $\Omega$, ed esiste un punto periodico in $\Omega$ per $F$; oppure,
        \item esiste un unico punto di $\partial X$ tale che la successione delle iterate di $F$ converge, uniformemente sui compatti, a quel punto.
    \end{itemize}
\end{cor}

\begin{proof}
    Dato che $\Omega$ è un dominio taut limitato, per \cite[Theorem F]{Wu} è pseudoconvesso. È allora ben noto che $H^j(\Omega;\mathbb{C})=0$ per ogni $j>n$ (si veda \cite[Theorem 4.2.7]{H}). Poiché $\Omega$ è di tipo topologico finito, per il teorema dei coefficienti universali si ha che
    $$\dim_{\mathbb{Q}}{H^j(\Omega;\mathbb{Q})}=\rk{H_j(\Omega;\mathbb{Z})}=\dim_{\mathbb{C}}{H^j(\Omega;\mathbb{C})}$$
    per ogni $j\in\mathbb{N}$. Di conseguenza, $H^j(X;\mathbb{Q})=0$ per ogni $j\in\mathbb{N}$. Si conclude facilmente usando il Teorema \ref{periodico} e il Corollario \ref{wd_type}.
\end{proof}

Concludiamo il capitolo citando un teorema che mostra come in più variabili, anche nel caso di domini contraibili che soddisfano le ipotesi dei teoremi visti in questa sezione, esistono funzioni con successione delle iterate non compattamente divergente ma senza punti fissi.
\begin{thm}
    (\cite[Theorem 1]{AH}) Sia $G=\faktor{\mathbb{Z}}{pq\mathbb{Z}}$ con $p$ e $q$ primi fra loro. Allora esiste un dominio limitato, pseudoconvesso, taut e contraibile $\Omega\subseteq\mathbb{C}^n$ tale che $G$ vi agisce tramite funzioni lineari complesse senza punti fissi.
\end{thm}