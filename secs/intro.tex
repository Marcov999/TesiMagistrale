L'obiettivo di questa tesi è dimostrare alcune possibili generalizzazioni, in più variabili complesse o anche per varietà complesse astratte che soddisfano opportune ipotesi, del teorema di Wolff-Denjoy, dimostrato indipendentemente nel 1926 da Denjoy in \cite{D} e da Wolff in \cite{Wo}. Riportiamo l'enunciato di tale teorema.

\begin{thm}
    (Wolff-Denjoy) Sia $f$ una funzione olomorfa nel disco unitario in $\mathbb{C}$ a valori nel disco stesso. Allora vale esattamente una delle seguenti affermazioni:
    \begin{itemize}
        \item la funzione $f$ ha un punto fisso nel disco; oppure,
        \item esiste un unico punto del bordo del disco tale che la successione delle iterate di $f$ converge, uniformemente sui compatti, a quel punto.
    \end{itemize}
\end{thm}

Era già nota da tempo la generalizzazione, dovuta ad Abate (\cite[Theorem 0.5]{A2}), per domini limitati e strettamente pseudoconvessi in più variabili.\\
SCALETTA PER DOMANI (POSSIBILE INIZIO: "Più recentemente, ..."):\\
- PARLARE DI BZ1;\\
- PARLARE DI BM;\\
- PARLARE DI CMS;\\
- RIASSUMERE COSA VIENE FATTO NELLA TESI (CERCARE DI CAPIRE SE VERRANNO FATTI I DOMINI ILLIMITATI E LE ESTENSIONI AL BORDO).