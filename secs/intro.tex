L'obiettivo di questa tesi è dimostrare alcune possibili generalizzazioni, in più variabili complesse o anche per varietà complesse astratte che soddisfano opportune ipotesi, del teorema di Wolff-Denjoy, dimostrato indipendentemente nel 1926 da Denjoy in \cite{D} e da Wolff in \cite{Wo}. Riportiamo l'enunciato di tale teorema.

\begin{thm}
    (Wolff-Denjoy) Sia $f$ una funzione olomorfa nel disco unitario in $\mathbb{C}$ a valori nel disco stesso. Allora vale esattamente una delle seguenti affermazioni:
    \begin{itemize}
        \item la funzione $f$ ha un punto fisso nel disco; oppure,
        \item esiste un unico punto del bordo del disco tale che la successione delle iterate di $f$ converge, uniformemente sui compatti, a quel punto.
    \end{itemize}
\end{thm}

Era già nota da tempo la generalizzazione, dovuta ad Abate (\cite[Theorem 0.5]{A2}), per domini limitati e strettamente pseudoconvessi in più variabili; in questo caso, però, l'affermazione ``la funzione $f$ ha un punto fisso'' è sostituita da ``le orbite dei punti del dominio tramite $f$ sono relativamente compatte nel dominio''. Ci riferiremo a enunciati con tesi simili come teoremi di tipo ``Wolff-Denjoy''.

Più recentemente, in diversi articoli si è cercato di ottenere risultati per domini sempre più generali. Nel 2017 Bharali e Zimmer, nel loro articolo \cite{BZ1}, hanno dimostrato un teorema di tipo ``Wolff-Denjoy'' per una particolare classe di domini, da loro chiamati domini Goldilocks, che siano anche taut. Nel 2021 Bharali e Maitra, in \cite{BM}, hanno osservato che la dimostrazione fatta in \cite{BZ1} può essere estesa a domini taut che soddisfino una certa condizione per i punti del bordo, detta di visibilità; hanno anche definito e costruito esempi di una classe di domini, che chiamano domini Caltrops, che hanno questa proprietà ma che non sono domini Goldilocks. Nel preprint \cite{CMS} del 2021 Chandel, Maitra e Sarkar fanno vedere che non è necessario supporre che la varietà sia un dominio, basta che sia una sottovarietà di $\mathbb{C}^d$; costruiscono inoltre altri esempi di domini che soddisfano la condizione di di visibilità.

Negli articoli citati finora, sono stati considerati domini e varietà limitati. Il caso di domini illimitati è stato studiato nel 2022 da Bharali e Zimmer nel preprint cite\{BZ2\}\marginpar{Aggiungere alla bibliografia e decidere come tradurre end compactification}; considerando il bordo della end compactification al posto del bordo euclideo, riescono così ad ottenere un teorema di tipo ``Wolff-Denjoy'' anche per il caso illimitato.

Notiamo inoltre che in \cite{BZ1}, \cite{CMS} e cite\{BZ2\} vengono anche mostrati dei risultati di estensione al bordo di funzioni olomorfe, isometrie e quasi-isometrie; nei due preprint viene anche evidenziato un legame tra l'estensione al bordo e la condizione di visibilità.\\

In questa tesi ci poniamo l'obiettivo di generalizzare ulteriormente \cite[Theorem 1.15]{CMS}. Osserviamo come le varie ipotesi giocano un ruolo fondamentale nella dimostrazione di un teorema di tipo ``Wolff-Denjoy'':
\begin{nlist}
    \item dall'ipotesi che la varietà sia taut segue la dicotomia tra i due enunciati della tesi, cioè se le orbite non sono relativamente compatte la successione delle iterate dev'essere compattamente divergente;
    \item dall'ipotesi di limitatezza, grazie al teorema di Montel, segue, a meno di sottosuccessioni, la convergenza uniforme sui compatti a una funzione olomorfa a valori nel bordo euclideo della varietà;
    \item dall'ipotesi di visibilità segue che tale limite dev'essere una funzione costante.
\end{nlist}

Nell'enunciato che vogliamo andare a dimostrare, l'ipotesi di che la varietà sia una sottovarietà limitata di $\mathbb{C}^d$ sarà sostituita dall'essere una sottovarietà tautly embedded\marginpar{Anche qui devo decidere come tradurre} di una varietà Kobayashi-iperbolica. \\

Nella sezione \ref{Preliminari} daremo le definizioni di base, come quelle di varietà Kobayashi-iperbolica e di varietà taut, ed enunceremo risultati noti e alla base della teoria della dinamica olomorfa in più variabili; alcuni verranno anche dimostrati, tra cui una versione del teorema di Montel per varietà tautly embedded e il fatto che per varietà taut vale una dicotomia nella dinamica delle funzioni olomorfe.

Nella sezione \ref{Un teorema di tipo ``Wolff-Denjoy'' per varietà taut con visibilità} daremo una definizione formale della condizione di visibilità e vedremo una serie di risultati tecnici, tra cui il fatto che con l'ipotesi di visibilità il limite di una successione convergente uniformemente sui compatti a una funzione a valori nel bordo dev'essere costante. A questo punto, i vari risultati ottenuti verranno utilizzati per dare una dimostrazione di un teorema di tipo ``Wolff-Denjoy''; nella dimostrazione rimarrà da verificare che il limite è lo stesso per ogni sottosuccessione delle iterate, da cui poi segue facilmente che dev'essere il limite di tutta la successione. Otterremo \cite[Theorem 0.5]{A2} come corollario.

Nella sezione \ref{Esempi di domini con visibilità} costruiremo gli esempi citati sopra.

Nella sezione \ref{Ulteriori risultati} vedremo il caso dei domini di tipo finito e dei domini illimitati, e citeremo alcuni dei legami tra la condizione di visibilità e i risultati di estensione al bordo.