L'obiettivo di questa tesi è dimostrare alcune possibili generalizzazioni, in più variabili complesse, o anche per varietà complesse astratte che soddisfano opportune ipotesi, del teorema di Wolff-Denjoy sul comportamento delle iterate di funzioni olomorfe nel disco unitario in $\mathbb{C}$, dimostrato indipendentemente nel 1926 da Denjoy in \cite{D} e da Wolff in \cite{Wo}. Riportiamo l'enunciato di tale teorema.

\begin{thm*}
    (Wolff-Denjoy) Sia $f$ una funzione olomorfa nel disco unitario $\mathbb{D}$ in $\mathbb{C}$ a valori nel disco stesso. Allora vale esattamente una delle seguenti affermazioni:
    \begin{itemize}
        \item la funzione $f$ ha un punto fisso nel disco; oppure,
        \item esiste un unico punto del bordo del disco tale che la successione delle iterate di $f$ converge, uniformemente sui compatti, a quel punto.
    \end{itemize}
\end{thm*}

Era già nota da tempo la generalizzazione, dovuta ad Abate (\cite[Theorem 0.5]{A4}), per domini limitati strettamente pseudoconvessi in più variabili; in questo caso, però, l'affermazione ``la funzione $f$ ha un punto fisso'' dev'essere sostituita da ``le orbite dei punti del dominio tramite $f$ sono relativamente compatte nel dominio''. Ci riferiremo a enunciati con tesi simili come teoremi di tipo ``Wolff-Denjoy''.

Il teorema ottenuto da Abate può essere dimostrato, come vedremo, usando i risultati di \cite{BB} e \cite{Ka}. In particolare, nel primo dei due articoli gli autori dimostrano che i domini limitati strettamente pseudoconvessi sono iperbolici nel senso di Gromov. Sebbene gli approcci meno recenti ai risultati che vedremo si sono sviluppati sotto queste ipotesi, in questa tesi presenteremo solo un breve accenno a come possono essere usate. Ci interesseremo invece di ottenere teoremi di tipo ``Wolff-Denjoy'' sotto ipotesi che verificheremo poi essere soddisfatte anche, ma non solo, dai domini limitati strettamente pseudoconvessi.\\

Più recentemente, in diversi articoli si è cercato di ottenere risultati analoghi a quello di Abate per domini sempre più generali.

Nel 2017 Bharali e Zimmer, nel loro articolo \cite{BZ1}, hanno dimostrato un teorema di tipo ``Wolff-Denjoy'' per una particolare classe di domini, da loro chiamati domini Goldilocks, che siano anche taut.

\begin{defn*}
    Una varietà complessa e connessa $X$ si dice \textit{taut} se ogni funzione nella chiusura di $\text{Hol}(\mathbb{D},X)$ in $C^0(\mathbb{D},X^*)$ è in $\text{Hol}(\mathbb{D},X)$ oppure è la funzione costante $\infty$.
\end{defn*}

Esempi di domini Goldilocks sono i domini limitati, pseudoconvessi e di tipo finito nel senso di D'Angelo.

Nel 2021 Bharali e Maitra, in \cite{BM}, hanno osservato che la dimostrazione fatta in \cite{BZ1} può essere estesa a domini taut che soddisfino una certa condizione per i punti del bordo, detta di visibilità.

\begin{defn*}
    Sia $X$ una varietà complessa; la \textit{pseudometrica di Kobayashi} su $X$ è
    \begin{equation}\begin{split}
        K_X(x;Z)=&\inf\{|v| \mid v \in \mathbb{C}, \text{ esiste }f \in \text{Hol}(\mathbb{D},X) \\
        &\text{ tale che } f(0)=x, \diff_0 f(v)=Z\}
    \end{split}\end{equation}
    per ogni $x \in X$ e $Z \in T_xX$.
\end{defn*}

\begin{defn*}
    Sia $X$ una varietà complessa e connessa; la \textit{pseudodistanza di Kobayashi} su $X$ è $k_X$, la forma integrata di $K_X$.
    
    Se $k_X$ è effettivamente una distanza, $X$ si dice \textit{Kobayashi-iperbolica.}
\end{defn*}

Si noti che ogni varietà taut è Kobayashi-iperbolica (\cite[Proposition 2]{Ki1}).

\begin{defn*}
    Sia $X$ una varietà complessa e connessa; fissiamo due costanti $\lambda \ge 1$ e $\kappa \ge 0$. Sia $I\subseteq \mathbb{R}$ un intervallo; una curva $\sigma:I \longrightarrow X$ è detta una \textit{$(\lambda,\kappa)$-simil-geodetica} se
    \begin{enumerate}
        \item per ogni $s,t \in I$ si ha
        $$\frac{1}{\lambda}|t-s|-\kappa \le k_X\big(\sigma(s),\sigma(t)\big)\le\lambda|t-s|+\kappa;$$
        \item $\sigma$ è assolutamente continua (quindi $\sigma'(t)$ esiste per quasi ogni $t \in I$) e per quasi ogni $t \in I$ si ha
        $$K_X\big(\sigma(t);\sigma'(t)\big) \le \lambda.$$
    \end{enumerate}
\end{defn*}

\begin{defn*}
    Sia $X$ una sottovarietà complessa e connessa di una varietà complessa $Y$, e fissiamo $\lambda \ge 1$ e $\kappa \ge 0$. Diciamo che $X$ è \textit{$(\lambda,\kappa)$-visibile} se
    \begin{enumerate}
        \item ogni due punti distinti di $X$ possono essere collegati da una $(\lambda,\kappa)$-simil-geodetica;
        \item per ogni coppia di punti $p,q\in\partial_YX$ con $p\not=q$, esistono in $\overline{X}$ due intorni $V$ e $W$, di $p$ e $q$ rispettivamente, con chiusura disgiunta, e un compatto $K$ di $X$ tali che  ogni $(\lambda,\kappa)$-simil-geodetica in $X$ che collega un punto di $V$ a un punto di $W$ interseca $K$.
    \end{enumerate}
\end{defn*}

Bharali e Maitra hanno anche dimostrato un teorema di tipo ``Wolff-Denjoy'' per domini di tipo topologico finito; inoltre, hanno definito e costruito esempi di una classe di domini, che chiamano domini Caltrops, che sono taut e hanno la condizione di visibilità, ma non sono domini Goldilocks.

Nel preprint \cite{CMS} del 2021 Chandel, Maitra e Sarkar fanno vedere che non è necessario supporre che la varietà sia un dominio, basta che sia una sottovarietà di $\mathbb{C}^d$; costruiscono inoltre altri esempi di domini che soddisfano la condizione di visibilità.\\

Negli articoli citati finora, sono stati considerati domini e varietà limitati. Il caso di domini illimitati è stato studiato nel 2022 da Bharali e Zimmer nel preprint \cite{BZ2}; considerando il bordo della \textit{end compactification} al posto del bordo euclideo riescono ad ottenere un teorema di tipo ``Wolff-Denjoy'' anche per il caso illimitato.

Notiamo inoltre che in \cite{BZ1}, \cite{CMS} e \cite{BZ2} vengono anche mostrati dei risultati di estensione al bordo di funzioni olomorfe, isometrie e quasi-isometrie; nei due preprint viene anche evidenziato un legame tra l'estensione al bordo e la condizione di visibilità. Si noti che in alcuni dei teoremi di estensione ritroviamo tra le ipotesi la condizione di Gromov-iperbolicità. \\

In questa tesi ci poniamo l'obiettivo di generalizzare ulteriormente il teorema di tipo ``Wolff-Denjoy'' mostrato in \cite{CMS}. Osserviamo come le varie ipotesi giocano un ruolo fondamentale nella dimostrazione di tale teorema:
\begin{nlist}
    \item dall'ipotesi che la varietà sia taut segue la dicotomia tra i due enunciati della tesi, cioè se le orbite non sono relativamente compatte la successione delle iterate dev'essere compattamente divergente;
    \item dall'ipotesi di limitatezza, grazie al teorema di Montel, segue, a meno di sottosuccessioni, la convergenza uniforme sui compatti a una funzione olomorfa a valori nel bordo della varietà;
    \item dall'ipotesi di visibilità segue che tale funzione limite dev'essere costante.
\end{nlist}

Nell'enunciato che vogliamo dimostrare, l'ipotesi che la varietà sia una sottovarietà limitata di $\mathbb{C}^d$ sarà sostituita dall'essere una sottovarietà tautly embedded di una varietà Kobayashi-iperbolica; tale ipotesi può essere interpretata dicendo che per la sottovarietà vale una forma del teorema di Montel.

\begin{defn*}
    Sia $X$ una sottovarietà complessa di una varietà complessa $Y$. Diciamo che $X$ è \textit{tautly embedded} in $Y$ se $\text{Hol}(\mathbb{D},X)$ è relativamente compatto in $\text{Hol}(\mathbb{D},Y)$.
\end{defn*}

Ecco infine il Teorema che ci proponiamo di dimostrare.

\begin{thm*}
    Sia $X$ una sottovarietà complessa e tautly embedded di una varietà Kobayashi-iperbolica $Y$. Supponiamo che $X$ sia taut e che esista un $\kappa_0>0$ tale che $X$ sia $(1,\kappa_0)$-visibile.
    
    Sia $F:X \longrightarrow X$ una funzione olomorfa. Allora vale esattamente una delle seguenti affermazioni:
    \begin{itemize}
        \item le orbite dei punti di $X$ tramite $F$ sono relativamente compatte in $X$; oppure,
        \item esiste un unico punto di $\partial_YX$ tale che la successione delle iterate di $F$ converge, uniformemente sui compatti, a quel punto.
    \end{itemize}
\end{thm*}

Nella sezione \ref{Preliminari} daremo le definizioni ed enunceremo i risultati alla base della teoria della dinamica olomorfa in più variabili; alcuni verranno anche dimostrati, tra cui una versione generalizzata del teorema di Montel per varietà tautly embedded e il fatto che per varietà taut vale una dicotomia nella dinamica delle funzioni olomorfe.

Nella sezione \ref{Un teorema di tipo ``Wolff-Denjoy'' per varietà taut con visibilità} studieremo la condizione di visibilità e vedremo una serie di risultati tecnici, tra cui il fatto che con l'ipotesi di visibilità il limite di una successione convergente uniformemente sui compatti a una funzione a valori nel bordo dev'essere costante. A questo punto, i vari risultati ottenuti verranno utilizzati per dare una dimostrazione di un teorema di tipo ``Wolff-Denjoy''; nella dimostrazione rimarrà da verificare che il limite è lo stesso per ogni sottosuccessione delle iterate, da cui poi segue facilmente che dev'essere il limite di tutta la successione. Otterremo come corollari i teoremi di tipo ``Wolff-Denjoy'' dimostrati in \cite{A4} e in \cite{CMS}, e quello per i domini di tipo topologico finito dimostrato in \cite{BM}.

Nella sezione \ref{Esempi di domini con visibilità} costruiremo gli esempi citati sopra e vedremo quali ipotesi del teorema soddisfano.

Nella sezione \ref{Ulteriori risultati} vedremo il caso dei domini illimitati, e citeremo alcuni dei legami tra la condizione di visibilità e i risultati di estensione al bordo.